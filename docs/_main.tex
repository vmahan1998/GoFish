% Options for packages loaded elsewhere
\PassOptionsToPackage{unicode}{hyperref}
\PassOptionsToPackage{hyphens}{url}
\documentclass[
]{book}
\usepackage{xcolor}
\usepackage{amsmath,amssymb}
\setcounter{secnumdepth}{5}
\usepackage{iftex}
\ifPDFTeX
  \usepackage[T1]{fontenc}
  \usepackage[utf8]{inputenc}
  \usepackage{textcomp} % provide euro and other symbols
\else % if luatex or xetex
  \usepackage{unicode-math} % this also loads fontspec
  \defaultfontfeatures{Scale=MatchLowercase}
  \defaultfontfeatures[\rmfamily]{Ligatures=TeX,Scale=1}
\fi
\usepackage{lmodern}
\ifPDFTeX\else
  % xetex/luatex font selection
\fi
% Use upquote if available, for straight quotes in verbatim environments
\IfFileExists{upquote.sty}{\usepackage{upquote}}{}
\IfFileExists{microtype.sty}{% use microtype if available
  \usepackage[]{microtype}
  \UseMicrotypeSet[protrusion]{basicmath} % disable protrusion for tt fonts
}{}
\makeatletter
\@ifundefined{KOMAClassName}{% if non-KOMA class
  \IfFileExists{parskip.sty}{%
    \usepackage{parskip}
  }{% else
    \setlength{\parindent}{0pt}
    \setlength{\parskip}{6pt plus 2pt minus 1pt}}
}{% if KOMA class
  \KOMAoptions{parskip=half}}
\makeatother
\usepackage{color}
\usepackage{fancyvrb}
\newcommand{\VerbBar}{|}
\newcommand{\VERB}{\Verb[commandchars=\\\{\}]}
\DefineVerbatimEnvironment{Highlighting}{Verbatim}{commandchars=\\\{\}}
% Add ',fontsize=\small' for more characters per line
\usepackage{framed}
\definecolor{shadecolor}{RGB}{248,248,248}
\newenvironment{Shaded}{\begin{snugshade}}{\end{snugshade}}
\newcommand{\AlertTok}[1]{\textcolor[rgb]{0.94,0.16,0.16}{#1}}
\newcommand{\AnnotationTok}[1]{\textcolor[rgb]{0.56,0.35,0.01}{\textbf{\textit{#1}}}}
\newcommand{\AttributeTok}[1]{\textcolor[rgb]{0.13,0.29,0.53}{#1}}
\newcommand{\BaseNTok}[1]{\textcolor[rgb]{0.00,0.00,0.81}{#1}}
\newcommand{\BuiltInTok}[1]{#1}
\newcommand{\CharTok}[1]{\textcolor[rgb]{0.31,0.60,0.02}{#1}}
\newcommand{\CommentTok}[1]{\textcolor[rgb]{0.56,0.35,0.01}{\textit{#1}}}
\newcommand{\CommentVarTok}[1]{\textcolor[rgb]{0.56,0.35,0.01}{\textbf{\textit{#1}}}}
\newcommand{\ConstantTok}[1]{\textcolor[rgb]{0.56,0.35,0.01}{#1}}
\newcommand{\ControlFlowTok}[1]{\textcolor[rgb]{0.13,0.29,0.53}{\textbf{#1}}}
\newcommand{\DataTypeTok}[1]{\textcolor[rgb]{0.13,0.29,0.53}{#1}}
\newcommand{\DecValTok}[1]{\textcolor[rgb]{0.00,0.00,0.81}{#1}}
\newcommand{\DocumentationTok}[1]{\textcolor[rgb]{0.56,0.35,0.01}{\textbf{\textit{#1}}}}
\newcommand{\ErrorTok}[1]{\textcolor[rgb]{0.64,0.00,0.00}{\textbf{#1}}}
\newcommand{\ExtensionTok}[1]{#1}
\newcommand{\FloatTok}[1]{\textcolor[rgb]{0.00,0.00,0.81}{#1}}
\newcommand{\FunctionTok}[1]{\textcolor[rgb]{0.13,0.29,0.53}{\textbf{#1}}}
\newcommand{\ImportTok}[1]{#1}
\newcommand{\InformationTok}[1]{\textcolor[rgb]{0.56,0.35,0.01}{\textbf{\textit{#1}}}}
\newcommand{\KeywordTok}[1]{\textcolor[rgb]{0.13,0.29,0.53}{\textbf{#1}}}
\newcommand{\NormalTok}[1]{#1}
\newcommand{\OperatorTok}[1]{\textcolor[rgb]{0.81,0.36,0.00}{\textbf{#1}}}
\newcommand{\OtherTok}[1]{\textcolor[rgb]{0.56,0.35,0.01}{#1}}
\newcommand{\PreprocessorTok}[1]{\textcolor[rgb]{0.56,0.35,0.01}{\textit{#1}}}
\newcommand{\RegionMarkerTok}[1]{#1}
\newcommand{\SpecialCharTok}[1]{\textcolor[rgb]{0.81,0.36,0.00}{\textbf{#1}}}
\newcommand{\SpecialStringTok}[1]{\textcolor[rgb]{0.31,0.60,0.02}{#1}}
\newcommand{\StringTok}[1]{\textcolor[rgb]{0.31,0.60,0.02}{#1}}
\newcommand{\VariableTok}[1]{\textcolor[rgb]{0.00,0.00,0.00}{#1}}
\newcommand{\VerbatimStringTok}[1]{\textcolor[rgb]{0.31,0.60,0.02}{#1}}
\newcommand{\WarningTok}[1]{\textcolor[rgb]{0.56,0.35,0.01}{\textbf{\textit{#1}}}}
\usepackage{longtable,booktabs,array}
\usepackage{calc} % for calculating minipage widths
% Correct order of tables after \paragraph or \subparagraph
\usepackage{etoolbox}
\makeatletter
\patchcmd\longtable{\par}{\if@noskipsec\mbox{}\fi\par}{}{}
\makeatother
% Allow footnotes in longtable head/foot
\IfFileExists{footnotehyper.sty}{\usepackage{footnotehyper}}{\usepackage{footnote}}
\makesavenoteenv{longtable}
\usepackage{graphicx}
\makeatletter
\newsavebox\pandoc@box
\newcommand*\pandocbounded[1]{% scales image to fit in text height/width
  \sbox\pandoc@box{#1}%
  \Gscale@div\@tempa{\textheight}{\dimexpr\ht\pandoc@box+\dp\pandoc@box\relax}%
  \Gscale@div\@tempb{\linewidth}{\wd\pandoc@box}%
  \ifdim\@tempb\p@<\@tempa\p@\let\@tempa\@tempb\fi% select the smaller of both
  \ifdim\@tempa\p@<\p@\scalebox{\@tempa}{\usebox\pandoc@box}%
  \else\usebox{\pandoc@box}%
  \fi%
}
% Set default figure placement to htbp
\def\fps@figure{htbp}
\makeatother
\setlength{\emergencystretch}{3em} % prevent overfull lines
\providecommand{\tightlist}{%
  \setlength{\itemsep}{0pt}\setlength{\parskip}{0pt}}
\usepackage[]{natbib}
\bibliographystyle{plainnat}
\usepackage{booktabs}
\usepackage{bookmark}
\IfFileExists{xurl.sty}{\usepackage{xurl}}{} % add URL line breaks if available
\urlstyle{same}
\hypersetup{
  pdftitle={GoFish: A Next-Generation ToolKit for Modeling Migratory Fish and Quantifying Environmental Risk in Estuaries},
  pdfauthor={Vanessa Quintana},
  pdfkeywords={ecological modeling, coding library, netlogo, agent-based, pattern-oriented, individual-based, bioaccumulation, estuary contamination, hydrodynamics, mercury, anadromous fish, GoFish, SPM transport, methylmercury, GoFish Toolkit, osmoregulation, estuarine dynamics, coupled models, co-development, fish behavior, migratory fish},
  hidelinks,
  pdfcreator={LaTeX via pandoc}}

\title{GoFish: A Next-Generation ToolKit for Modeling Migratory Fish and Quantifying Environmental Risk in Estuaries}
\author{Vanessa Quintana}
\date{2025-11-23}

\begin{document}
\maketitle

{
\setcounter{tocdepth}{1}
\tableofcontents
}
\chapter*{Preface}\label{preface}
\addcontentsline{toc}{chapter}{Preface}

This library provides a comprehensive, modular framework for developing and documenting agent-based models (ABMs) that simulate the movement, behavior, and environmental interactions of migratory fish in coastal aquatic systems. It is designed to support the standardization and implementation of ABMs in fisheries management, enabling researchers and practitioners to address complex environmental questions and evaluate remediation or restoration scenarios.

\section*{Motivation}\label{motivation}
\addcontentsline{toc}{section}{Motivation}

The goal of this resource is to support students, researchers, and decision-makers by making agent-based modeling of migratory fish more accessible, reproducible, and applicable to real-world fisheries and habitat management challenges. By providing a standardized modular framework for key behavioral processes, this library promotes consistency, transparency, and credibility in ecological forecasting and decision support tools. It also establishes a foundation for critical conversations about the behaviors and functions represented in modeling, while supporting the empirical quantification of ecological relationships that influence movement, survival, and habitat use of migratory fish.

\section*{Application Context}\label{application-context}
\addcontentsline{toc}{section}{Application Context}

This library was originally developed in support of research on the influence of tidal behavior and contaminant exposure on anadromous fish in the Penobscot River Estuary. However, its modular design allows for application to other estuarine and coastal systems where fish respond to environmental changes (i.e., salinity, velocity, and pollutants).

\section*{How to Use This Library}\label{how-to-use-this-library}
\addcontentsline{toc}{section}{How to Use This Library}

Each function or behavior in this library can be combined with others to build a complete agent-based model for anadromous fish. These functions are designed to be adaptable, and easily configured for different species, life stages, or site-specific conditions.

*For questions, feedback, guidance on implementation, or \textbf{interest in adding to the library}, please contact \textbf{Vanessa Quintana} at \href{mailto:mahan.vanessa98@gmail.com}{\textbf{mahan.vanessa98@gmail.com}}\textbf{.}

\chapter{Introduction}\label{introduction}

Anadromous fish species such as river herring, striped bass, salmon, and sturgeon navigate coastal and estuarine systems that are increasingly affected by human activity, climate change, and legacy contaminants. Modeling their movement and behavior at fine spatial and temporal scales requires tools that can integrate physiological stressors, environmental variability, and behavior-based decision-making.

Agent-based models are among the most powerful tools available for ecological forecasting and fisheries management, but they are also among the most complex. Their structure and computational demands can make them difficult to apply in practical management settings. Many biologists and ecologists who hold deep, species-specific expertise often have limited training in advanced programming or systems modeling.

As a result, traditional approaches to simulating anadromous fish frequently oversimplify or exclude key biological functions and environmental interactions such as salinity exposure and contaminant exposure. Existing models also lack standardized representations of these behaviors and interactions, limiting the applicability of results and reducing their usefulness for applied management. This function library was developed to address these limitations by offering modular, empirically grounded components designed for use in modeling migratory fish. Each function is clearly documented and can be applied independently, allowing for transparent testing, modification, and reuse across a wide range of ecological scenarios and sites.

\section{Structure}\label{structure}

Each chapter in this library corresponds to a major behavior, physiological function, or foundational component relevant to migratory fish, beginning with:

\begin{itemize}
\item
  \textbf{Co-Development of Agent-Based Models}
\item
  \textbf{Calculate Metabolism}
\item
  \textbf{Salinity Exposure}
\item
  \textbf{Temperature Exposure}
\item
  \textbf{Contaminant Exposure}
\item
  \textbf{Digestion}
\item
  \textbf{Filter Feeding}
\item
  \textbf{Lipid Catabolism}
\item
  \textbf{Landward Migration}
\item
  \textbf{Seaward Migration}
\item
  \textbf{Schooling}
\item
  \textbf{Selective Tidal Stream Transport}
\item
  \textbf{Predator-prey interactions}
\end{itemize}

The final chapters provide guidance on how to integrate multiple functions into a complete agent-based model, demonstrating how these components work together to simulate fish behavior in dynamic coastal and estuarine systems.

Within each chapter, each function or behavior is documented using the ODD protocol \citep{grimm_pattern-oriented_2005, grimm_odd_2010, grimm_odd_2020}. The ODD (Overview, Design concepts, Details) protocol is a standardized framework for describing agent-based models. It promotes transparency in model development and ensures consistency across implementations, especially when integrating multiple behavioral or ecological functions.

\begin{itemize}
\item
  \textbf{Overview} provides the purpose of the model component, identifies the entities involved (e.g., fish agents, environmental patches), and outlines the general processes.
\item
  \textbf{Design} concepts describe the key theoretical underpinnings such as emergence, adaptation, objectives, sensing, stochasticity, and interaction.
\item
  \textbf{Details} specify initialization steps, input data requirements, and the rules or submodels that govern behavior.
\end{itemize}

By following the ODD protocol, this library ensures that each function is self-contained, interpretable, and ready for adaptation to a wide range of species, sites, or management scenarios.

\chapter{Agent-Based Models}\label{agent-based-models}

Agent-based models (ABMs) simulate the behavior of individual organisms---called ``agents''---and how their interactions with each other and with their environment create larger-scale patterns. Each agent follows a set of simple rules based on physiology, behavior, or environmental cues. When many agents follow these rules simultaneously, complex system-level outcomes emerge, such as migration pathways, schooling patterns, habitat bottlenecks, or exposure hotspots. ABMs are especially useful for ecological systems because they allow researchers and communities to explore how individual decisions scale up to influence population dynamics, habitat use, and risk in changing environments.

ABMs are designed to capture processes that depend on \textbf{behavior}, \textbf{timing}, \textbf{environmental variability}, and \textbf{individual differences}. They allow modelers to explore how organisms respond to their surroundings, make movement decisions, and interact with other individuals or species. Because agents experience the environment at fine spatial and temporal scales, ABMs can reveal patterns that occur only when many individuals interact across a dynamic landscape.

\subsection{What ABMs Can Help Us Understand}\label{what-abms-can-help-us-understand}

ABMs can answer questions related to:

\begin{itemize}
\item
  How individual movement decisions accumulate into migration routes or staging locations
\item
  When and where organisms are most exposed to environmental risks
\item
  How behaviors such as foraging, schooling, predator avoidance, resting, or spawning influence exposure or survival
\item
  How variation in traits (size, energy, age, or species) contributes to different outcomes
\item
  How environmental conditions shape habitat use, movement choices, or interactions between species
\item
  How system-level patterns emerge from local behaviors and simple rules
\item
  How alternative conditions or restoration actions might alter movement, habitat use, or exposure
\end{itemize}

These models are especially useful when management questions depend on \textbf{patterns created by individual behavior}, rather than population totals alone.

\subsection{What ABMs Are Not Designed to Do}\label{what-abms-are-not-designed-to-do}

ABMs cannot provide or replace:

\begin{itemize}
\item
  Exact estimates of population abundance
\item
  Precise contaminant concentrations inside individual organisms
\item
  Direct measurements of chemical or physiological processes
\item
  Predictions of exact numbers of predation or spawning events
\item
  Large-scale demographic projections without additional modeling components
\item
  Replacements for field sampling, monitoring programs, or laboratory measurements
\end{itemize}

Instead, ABMs offer \textbf{process-based insight}, showing how behavior and environmental conditions interact to create observable patterns.

\subsection{Emergent Behaviors}\label{emergent-behaviors}

Emergent behavior refers to patterns that arise when many individual agents follow their own rules, creating outcomes that cannot be predicted from observing a single fish on its own. These outcomes develop only when many individuals interact with each other and respond to their surrounding environment at the same time. Many ecological processes such as migration timing, habitat bottlenecks, exposure hotspots, schooling dynamics, predator and prey encounters, and shifts in group structure are emergent rather than linear. They form through small, individual decisions that accumulate into larger and sometimes unexpected system-level patterns.

By visualizing and simulating emergence, ABMs make it possible to see how these individual behaviors scale up to influence population level outcomes and habitat use patterns. This helps identify the specific conditions, behaviors, and locations that shape system-wide responses. For management, this is valuable because it shows how small-scale decisions made by individual fish can contribute to broader patterns that influence risk, habitat quality, restoration effectiveness, management actions, and ecological resilience. Understanding emergence allows managers and communities to explore why certain patterns occur, identify potential leverage points, and anticipate how changes in environmental conditions may alter system behavior in the future.

\chapter{Co-Development of ABMs: Penobscot Estuary Community Workshop August 6, 2025}\label{co-development-of-abms-penobscot-estuary-community-workshop-august-6-2025}

This page documents how a participatory workshop informed the functions within this coding library, the representation of physiological processes, behaviors of migratory fish, and their interactions within an estuary. The purpose of this page is to showcase the value added by co-development and increase transparency about the development of this coding library.

The Penobscot Estuary Community Modeling Workshop was designed from the beginning as a co-development space for an agent based model to address mercury toxicity dynamics of anadromous fish within the estuary. The day invited Tribal collaborators, community members, managers, and researchers to shape the modeling objectives, including which species, processes, behaviors, and outputs the model should represent, and how those outputs should be represented in relation to contaminant exposure in the Penobscot River Estuary.

\section{Core Co-Development Themes}\label{core-co-development-themes}

Several themes guided the design of the workshop and the way it connects to model development.

\begin{itemize}
\item
  \textbf{Shared problem framing.} The workshop centered a shared concern about mercury contamination in the Penobscot River and its effects on sea-run fish, subsistence access, and recreation use, rather than starting from a predefined modeling objective.
\item
  \textbf{Attention to both internal and external processes.} Breakout discussions were split into two rounds, one for ``internal'' or biological functions of fish such as osmoregulation, thermoregulation, bioaccumulation, and resting, and the second round focused on ``external processes'' or external interactions with intra- or interspecific factors and environmental responses such as swimming, predation, foraging, and spawning. This structure maps directly onto modules detailed within this library.
\item
  \textbf{Participant driven priorities.} Participants were asked which species were most important to understand for management of the system or species that should be prioritized based on community interest, which functions and interactions they believed to be relevant for understanding mercury toxicity in migratory fish, and how those behaviors should be represented.
\item
  \textbf{Outputs that answer real questions.} Each participant had the opportunity to propose a question or hypothesis to test using the developed functions and select preferred outputs from the functions such as migration paths, exposure metrics, or habitat use. These wish-list requests now correspond to specific model output routines and visualizations.
\item
  \textbf{Accessible language and shared vocabulary.} A glossary of terms translated physiological, hydrodynamic, and modeling concepts into accessible language. This glossary can be found at the end of this document and is intended to support a range of readers.
\item
  \textbf{Iterative relationships, not a single event.} The workshop was framed as one step in an ongoing partnership. Feedback forms, follow-up contact options, and explicit invitations for later review are built into the process.
\end{itemize}

\section{Workshop Development}\label{workshop-development}

\subsection{Pre-Workshop Collaboration Steps}\label{pre-workshop-collaboration-steps}

Planning the workshop involved months of coordination with partners, including the Penobscot Nation, faculty and staff at the University of Maine, and federal collaborators. The planning process itself was a first stage of co-development.

Key steps included:

\begin{itemize}
\item
  \textbf{Clarifying objectives with partners.} Early conversations identified contamination in the Penobscot River, migratory fish, and access to safe subsistence resources as the shared focus. Partners helped guide agenda development, identify potential participants of interest, and identify community outputs of interest to support Tribal management. Their suggestions were used when conducting outreach.
\item
  \textbf{Aligning model scope with community priorities.} The agenda was constructed so that introductory talks on Penobscot Nation natural resource management and contaminants of interest, an overview of estuary dynamics (connection of fish + physical processes + contamination), and an introduction to agent based modeling were followed by interactive surveys and break-out discussions. This allowed participants to identify anadromous species, behaviors, and processes of interest, which then feed directly into the model design.
\item
  \textbf{Preparing interactive polls and prompts.} A Mentimeter survey was developed to help participants identify priority species, important environmental drivers, and fish behaviors that shape exposure. These questions now map to the list of behaviors and parameters implemented in the model.
\end{itemize}

\href{materials/Mentimeter\%20Survey_10.docx}{Question Example}

\begin{itemize}
\tightlist
\item
  \textbf{Designing materials for many kinds of learners.} Each participant received a notepad and workshop folder, which included a plain-language glossary, a participant agenda, a hypothesis and output worksheet, and a feedback form. These materials supported participants who learn best through reading, writing, discussion, or visual exploration. Break-out discussions were accompanied by visual demonstrations of function demos, allowing participants to directly address draft model functions for each behavior.
\end{itemize}

Find examples below.\\
Note: these are presentations of functions before participant feedback. Updated functions in this library reflect changes informed by the workshop.

\subsection{Identifying Participants of Interest}\label{identifying-participants-of-interest}

Participant selection was guided by the goal of gathering diverse knowledge about the river and its fish.

\begin{itemize}
\item
  \textbf{Tribal Collaborators.} Representatives from the Penobscot Nation were invited to collaborate because the Penobscot River Estuary is the Ancestral Homeland of the Penobscot Nation and other Wabanaki Peoples. It was necessary to prioritize their concerns and support their management. We wanted to give them a platform to share their work and management perspectives on migratory fish and contamination. Indigenous people have sovereign rights to migratory species, which have been deeply impacted by contamination in the estuary.
\item
  \textbf{Community Experts.} The workshop brought in local managers, fishermen, researchers, engineers, and students whose work focuses on estuarine dynamics, fisheries, remediation, and environmental management in the Penobscot, or who focus on anadromous fish.
\end{itemize}

\subsection{Enticing Participants to Attend}\label{enticing-participants-to-attend}

The invitations and overview materials emphasized that this was a hands-on, collaborative event where participant input would shape real model structure and outputs---not a passive lecture.

\href{materials/Penobscot\%20Workshop\%20Overview.pdf}{Copy of Invitation}

Strategies included:

\begin{itemize}
\item
  \textbf{Clear framing.} The overview highlighted how tidal dynamics, fish behavior, and mercury contamination intersect in the Penobscot River, and described specific opportunities for participants to contribute knowledge during breakout sessions and wish-list activities. The invitation also emphasized that participants did not need any modeling experience---only their expertise and perspectives.
\item
  \textbf{Concrete outcomes.} Participants could request a specific model output or test a hypothesis relevant to their interests, with results attempted to be shared after the workshop (as feasible within model limitations).
\item
  \textbf{Relationship building.} The agenda previewed time for partnership discussions, highlighting opportunities to support Tribal priorities and future collaboration on management and research questions.
\end{itemize}

\subsection{Selecting the Venue}\label{selecting-the-venue}

The workshop was held at the Innovation Media Research Center at the University of Maine, a space that could support immersive presentations and movement between small and large group activities.

The venue allowed for:

\begin{itemize}
\item
  \textbf{A theater-style room with a large screen.} The introductory talks and interactive survey used a dark background, low lighting, and a large screen to create an immersive experience similar to an intimate concert hall, while still leaving space for questions and discussion.
\item
  \textbf{A separate interactive space.} Round tables, easel pads, printed handouts, and discussion areas were set up in a second room for the world-café breakout sessions, allowing participants to interact and face one another in a conversation-style setting as they discussed each function.
\end{itemize}

\section{Facilitating the Workshop}\label{facilitating-the-workshop}

\subsection{Agenda Activities}\label{agenda-activities}

The agenda combined presentations, interactive polls, world-café discussions, and reflection.

Key components included:

\begin{itemize}
\item
  \textbf{Welcome, blessing, and introductions.} The day opened with a welcome and words from the Penobscot Nation, followed by introductions and an icebreaker activity asking participants why they attended and what they hoped to gain.
\item
  \textbf{Map-based activities.} Participants could pin locations on a system-scale map, connecting their lived experience and knowledge to specific reaches of the river and estuary. This place-based interaction showcased lived knowledge and gave participants autonomy over the spatial representation of the system.
\item
  \textbf{Penobscot Nation history and river overview.} A management representative of the Penobscot Nation, Dan Kircheis, provided management context for contaminants and migratory fish, grounding the model in current work conducted within the system.
\item
  \textbf{Modeling overview and agent based model presentation.} Modeling presentations introduced the Penobscot system, key hydrodynamic and contaminant processes, and the rationale for an agent based approach. Examples of emergent behaviors and validation were shown on a large screen to make model dynamics visible and engaging.
\item
  \textbf{Mentimeter priority setting.} Participants completed an interactive survey about priority species, important environmental drivers, and key behaviors. These responses were summarized live and used to frame the breakout sessions.
\item
  \textbf{World café on internal fish processes.} Small rotating groups discussed processes such as osmoregulation, thermoregulation, bioaccumulation, and resting. Discussion leaders presented demos and recorded notes, which later informed how these processes are parameterized and sequenced within this library.
\item
  \textbf{World café on external behaviors and conditions.} A second set of rotations focused on external functions such as foraging, migration, predation, and spawning. These conversations helped identify how fish interact with their environment and where and when the model should track exposure risk.
\item
  \textbf{Wish list outputs.} Participants wrote down specific outputs they wanted from the model and the reasoning behind each request, including migration timing, exposure patterns, energy budgets, and habitat use. The goal is to incorporate as many of these outputs as possible to support subsequent management and research.
\end{itemize}

\href{materials/Workshop_Hypothesis_Model_Form_45}{Hypothesis Form}

\begin{itemize}
\item
  \textbf{Model demo.} A short demonstration of the model interface supported visual learners and connected the function modules to emergent behaviors.
\item
  \textbf{Partnership and next steps discussion.} The workshop closed with a short session on building partnerships and identifying community projects where the model could support ongoing work. This included relationship-building between participants and completion of a follow-up survey.
\end{itemize}

\href{materials/Penobscot\%20Estuary\%20Workshop\%20Feedback\%20FollowUp_35.docx}{Participant Survey}

\begin{itemize}
\tightlist
\item
  \textbf{Facilitation.} Facilitation roles were shared across a team of organizers and volunteers who kept time, prompted discussion, recorded notes, and supported logistics, which allowed the lead modeler to focus on listening and translating insights into model design.
\end{itemize}

\section{After the Workshop}\label{after-the-workshop}

\subsection{Incorporating Feedback}\label{incorporating-feedback}

Immediately following the workshop, participants were invited to fill out feedback forms on what they found most useful, what felt unclear or missing, and how future workshops or model development activities could be improved.

Feedback is being used to:

\begin{itemize}
\item
  Refine explanations of agent based modeling and estuarine processes in the model interface and supporting documents, using more examples, more visuals, and clearer connections between field observations and model rules.
\item
  Adjust the balance between presentation time and discussion time.
\item
  Clarify how participant input has been integrated, so that people can see their contributions in the model structure and outputs rather than only in summary notes.
\end{itemize}

\subsection{Post-Workshop Follow-Up}\label{post-workshop-follow-up}

Several tools support ongoing co-development after the workshop day itself. This work is still ongoing as products are completed and co-development of final materials continues.

\begin{itemize}
\item
  \textbf{Hypothesis and output forms.} The participant hypothesis and output form captured individual questions, preferred outputs, and desired scenarios. These forms now guide scenario design and output routines in the model (as much as feasible within model limitations and scope), and follow-up emails will share results with those who opted in.
\item
  \textbf{Ongoing communication.} Participants who requested follow-up are being contacted with model updates, refinements, and opportunities to review model structure.
\item
  \textbf{Follow-up survey.} The feedback form includes an option to receive a later survey on model results, which will allow participants to review outputs, comment on whether they find them credible and useful, and suggest further refinements.
\item
  \textbf{Tech Transfer Webinar.} A planned webinar will support technology transfer of the completed model and products to workshop participants, allowing them to see how their input made it into the final product.
\item
  \textbf{Integration into model documentation.} This co-development page, the glossary, and workshop-derived figures will be hosted alongside the developed functions to make the participatory process visible to anyone using or reviewing the simulations.
\end{itemize}

\subsection{Acknowledgements}\label{acknowledgements}

The workshop and this co developed model were made possible by the time, knowledge, and care shared by workshop participants and organizers.

\subsubsection{Participants}\label{participants}

Ernie Atkinson\\
Andrea Casey\\
Devon Gleason\\
Heather Hamlin\\
John Hildebrand\\
Dan Kircheis\\
Dianne Kopec\\
John Kocik\\
Dan Kusnierz\\
Chuck Loring\\
Dan McCaw\\
Amelia McAvoy\\
Sasha Milsky\\
Elias Pinilla\\
Darren Ranco\\
Angie Reed\\
Lauren Ross\\
Justin Stevens\\
Joe Zydlewski

\subsubsection{Volunteers}\label{volunteers}

\textbf{Nicolas Cyr}\\
University of Maine

\textbf{Katherine Daza}\\
University of Maine

\textbf{Joseph Dello Russo}\\
University of Maine

\textbf{Nalika Lakmali}\\
University of Maine

\textbf{Saba Molaei}\\
University of Maine

\textbf{Cristian Rojas}\\
University of Maine

\textbf{Malavika Sudhakaran}\\
University of Maine

\textbf{Kaylyn Zipp}\\
University of Maine

\subsubsection{Organizers:}\label{organizers}

\textbf{Vanessa Quintana}\\
University of Maine; USACE Engineering Research and Development Center

\textbf{Gayle Zydlewski}\\
University of Maine; NOAA Sea Grant

\textbf{Katrina Armstrong}\\
University of Maine

\textbf{Todd Swannack}\\
Texas State University; USACE Engineering Research and Development Center

\chapter{Metabolism}\label{metabolism}

\section{Overview}\label{overview}

Metabolism determines the energy required for fish to maintain basic
physiological function, support movement, regulate ions, digest food,
and sustain long term migration. This submodel computes metabolic rate
based on size, temperature, age scaling, and allometric relationships.
Metabolic efficiency, based on divergence from an agent's optimal
temperature, is then scaled toward multiple internal processes including
swimming, osmoregulation, and lipid catabolism.

\section{Purpose}\label{purpose}

To simulate energetically realistic metabolic costs for migratory fishes
in coastal and estuarine environments by accounting for size scaling,
temperature dependence, age effects, and dynamic energy partitioning
across physiological processes.

\section{Entities, State Variables, and Scales}\label{entities-state-variables-and-scales}

\subsection{Spatial and Temporal Scales}\label{spatial-and-temporal-scales}

\begin{itemize}
\tightlist
\item
  \textbf{Spatial Unit}: Patch (3 m x 3 m resolution)
\item
  \textbf{Temporal Unit}: 5-minute time steps (\texttt{tick})
\end{itemize}

\subsection{Global Variables}\label{global-variables}

\begin{longtable}[]{@{}
  >{\raggedright\arraybackslash}p{(\linewidth - 2\tabcolsep) * \real{0.5862}}
  >{\raggedright\arraybackslash}p{(\linewidth - 2\tabcolsep) * \real{0.4138}}@{}}
\toprule\noalign{}
\begin{minipage}[b]{\linewidth}\raggedright
Global Variable
\end{minipage} & \begin{minipage}[b]{\linewidth}\raggedright
Definition
\end{minipage} \\
\midrule\noalign{}
\endhead
\bottomrule\noalign{}
\endlastfoot
\textbf{Q10} & Temperature coefficient used for metabolic scaling. \\
\textbf{base-constant} & Global constant for baseline metabolic scaling (0.1). \\
\textbf{age-sensitivity} & Scalar controlling how metabolism changes with age (0.5). \\
\end{longtable}

\subsection{Patch Variables}\label{patch-variables}

\begin{longtable}[]{@{}
  >{\raggedright\arraybackslash}p{(\linewidth - 2\tabcolsep) * \real{0.5556}}
  >{\raggedright\arraybackslash}p{(\linewidth - 2\tabcolsep) * \real{0.4444}}@{}}
\toprule\noalign{}
\begin{minipage}[b]{\linewidth}\raggedright
Variable Name
\end{minipage} & \begin{minipage}[b]{\linewidth}\raggedright
Definition
\end{minipage} \\
\midrule\noalign{}
\endhead
\bottomrule\noalign{}
\endlastfoot
\textbf{temperature} & The water temperature of the current patch used to calculate temperature dependent metabolic rate. \\
\end{longtable}

\subsection{Agent Variables}\label{agent-variables}

\begin{longtable}[]{@{}
  >{\raggedright\arraybackslash}p{(\linewidth - 2\tabcolsep) * \real{0.5556}}
  >{\raggedright\arraybackslash}p{(\linewidth - 2\tabcolsep) * \real{0.4444}}@{}}
\toprule\noalign{}
\begin{minipage}[b]{\linewidth}\raggedright
Variable Name
\end{minipage} & \begin{minipage}[b]{\linewidth}\raggedright
Definition
\end{minipage} \\
\midrule\noalign{}
\endhead
\bottomrule\noalign{}
\endlastfoot
\textbf{size} & Body mass or size index used for allometric scaling. \\
\textbf{age-num} & Age used in scaling metabolic rate. \\
\textbf{optimal-temperature} & Preferred temperature at which baseline metabolism is defined. \\
\textbf{max-survival-temp} & Maximum temperature at which the agent can survive and function. \\
\textbf{Met-base} & Baseline metabolic constant calculated during initialization. \\
\textbf{metabolism-rate} & Metabolic rate for the current tick. \\
\textbf{total-metabolism} & Accumulated metabolism over time. \\
\textbf{swim-efficiency} & Efficiency applied to movement when metabolic rate is lower or higher than baseline. \\
\textbf{osmoregulation-efficiency} & Efficiency applied to osmoregultion when metabolic rate is lower or higher than baseline. \\
\textbf{lipid-catabolism-efficiency} & Efficiency applied to lipid catabolism when metabolic rate is lower or higher than baseline. \\
\textbf{swim-base} & Baseline allocation of metabolism toward movement. \\
\textbf{E-base} & Base metabolic maintenance cost. \\
\textbf{E-creation} & Energy cost for creating new chloride cells. \\
\textbf{E-osmo} & Total energy spent on osmoregulation. \\
\textbf{motion-ratio} & Fraction of metabolism allocated to swimming. \\
\textbf{osmo-ratio} & Fraction of metabolism allocated to osmoregulation. \\
\textbf{M-max} & Maximum size for scaling across the population. \\
\textbf{max-age} & Maximum age for scaling across the population. \\
\end{longtable}

\section{Process Overview and Scheduling}\label{process-overview-and-scheduling}

\begin{enumerate}
\def\labelenumi{\arabic{enumi}.}
\tightlist
\item
  Retrieve patch temperature and verify safe environmental inputs.
\item
  Compute allometric, temperature, and age scalars.
\item
  Calculate metabolic rate for the current tick.
\item
  Update metabolic accumulation.
\item
  Compute a reference baseline metabolic rate.
\item
  Apply efficiency multipliers to swimming, digestion, osmoregulation,
  and lipid catabolism.
\end{enumerate}

\section{Design Concepts}\label{design-concepts}

\textbf{Basic Principles:}\\
Metabolic rate combines allometric scaling, temperature responses, and
age scaling consistent with metabolic ecology theory. The model uses a
Q10 temperature function, Kleiber type allometry, and age dependent
reduction in metabolic intensity.

\textbf{Emergence:}\\
Energetic constraints emerge from interactions among size, temperature,
and age. These constraints then shape movement, digestion, and
osmoregulation.

\textbf{Adaptation:}\\
Agents adapt by adjusting internal efficiencies that influence
downstream behavioral and physiological submodels.

\textbf{Objectives:}\\
Agents maintain adequate energy for survival and migration and balance
energetic cost across processes.

\textbf{Sensing:}\\
Agents sense the temperature of the patch they occupy.

\textbf{Stochasticity:}\\
The function itself is deterministic but stochastic variation in
individual size and age creates emergent metabolic variation.

\textbf{Collectives:}\\
Scaling uses the maximum size and age of the entire population to
normalize differences among individuals.

\textbf{Observation:}\\
Outputs include metabolism-rate, total-metabolism, and efficiency
multipliers.

\section{Initialization}\label{initialization}

\begin{longtable}[]{@{}
  >{\centering\arraybackslash}p{(\linewidth - 4\tabcolsep) * \real{0.2500}}
  >{\centering\arraybackslash}p{(\linewidth - 4\tabcolsep) * \real{0.4167}}
  >{\centering\arraybackslash}p{(\linewidth - 4\tabcolsep) * \real{0.3333}}@{}}
\toprule\noalign{}
\begin{minipage}[b]{\linewidth}\centering
Variable
\end{minipage} & \begin{minipage}[b]{\linewidth}\centering
Initialized Value
\end{minipage} & \begin{minipage}[b]{\linewidth}\centering
Justification
\end{minipage} \\
\midrule\noalign{}
\endhead
\bottomrule\noalign{}
\endlastfoot
Met-base & Computed using size, age, and temperature factors & Establishes a physiologically realistic baseline. \\
swim-base & Met-base times movement ratio & Allocates baseline metabolism for locomotion. \\
E-base & Met-base times 0.02 & Maintenance cost of cellular metabolic activity. \\
E-creation & Met-base times 0.05 & Cost of forming new chloride cells. \\
motion-ratio & 0.7 normalized & Movement is energetically dominant. \\
osmo-ratio & 0.3 normalized & Osmoregulation is energetically significant. \\
E-osmo & 0 & No osmoregulation cost at initialization. \\
metabolism-rate & Met-base & Baseline metabolism in the absence of environmental variation. \\
total-metabolism & 0 & Starts accumulation at zero. \\
\end{longtable}

\section{Submodels}\label{submodels}

\subsection{Initialize Metabolism}\label{initialize-metabolism}

Baseline metabolic constant is computed using:

\$\$Met-base = base-constant \emph{(size / M-max) \^{} 0.75} {[}1 +
age-sensitivity * (1 - age / max-age){]} * (optimal-temperature /
max-survival-temp)\$\$

Where:

\begin{itemize}
\tightlist
\item
  base-constant is 0.1
\item
  age-sensitivity is 0.5
\end{itemize}

\subsection{Energy Partitions}\label{energy-partitions}

Baseline energy partitions are computed from the initialized metabolic base:

\[
\text{swim-base} = \text{Met-base} \cdot \text{motion-ratio}
\]

\[
E_{\text{base}} = \text{Met-base} \cdot 0.02
\]

\[
E_{\text{creation}} = \text{Met-base} \cdot 0.05
\]

\subsection{Metabolic Rate Calculation}\label{metabolic-rate-calculation}

At each tick, metabolism is recalculated using allometric, temperature, and age scaling:

\[
\text{size-factor} = \left(\frac{\text{weight}}{\text{M-max}}\right)^{0.75}
\]

\[
\text{temp-factor} = Q_{10}^{\left(\frac{T_{\text{env}} - T_{\text{opt}}}{10}\right)}
\]

\[
\text{age-factor} = 1 + \left(1 - \frac{\text{age}}{\text{max-age}}\right)\cdot 0.5
\]

Total metabolic rate:

\[
M_{\text{total}} = \text{Met-base} \cdot 
\text{size-factor} \cdot 
\text{temp-factor} \cdot 
\text{age-factor}
\]

Metabolic accumulation:

\[
\text{total-metabolism} = \text{total-metabolism} + M_{\text{total}}
\]

\subsection{Efficiency Multipliers}\label{efficiency-multipliers}

A baseline metabolic reference is defined using:

\begin{itemize}
\tightlist
\item
  baseline size-factor = 1\\
\item
  baseline age-factor = 1.5\\
\item
  baseline temp-factor = 1
\end{itemize}

Thus:

\[
M_{\text{baseline}} =
\text{Met-base} \cdot 1 \cdot 1 \cdot 1.5
\]

Efficiency is computed as:

\[
\text{efficiency} = 
\frac{M_{\text{total}}}{M_{\text{baseline}}}
\]

This value is applied to:

\begin{itemize}
\tightlist
\item
  \textbf{swim-efficiency}\\
\item
  \textbf{osmoregulation-efficiency}\\
\item
  \textbf{lipid-catabolism-efficiency}
\end{itemize}

Efficiency dynamically scales the cost of movement, ion regulation, and lipid catabolism.

\section{Netlogo Implementation}\label{netlogo-implementation}

\begin{Shaded}
\begin{Highlighting}[]

\NormalTok{;; ============================}
\NormalTok{;;   METABOLISM SUBMODEL}
\NormalTok{;; ============================}

\NormalTok{globals [}
\NormalTok{  ;; global physiological constants}
\NormalTok{  Q10}
\NormalTok{]}

\NormalTok{patches{-}own [}
\NormalTok{  temperature}
\NormalTok{]}

\NormalTok{fish{-}own [}
\NormalTok{  ;; core state variables}
\NormalTok{  weight}
\NormalTok{  age{-}num}
\NormalTok{  optimal{-}temperature}
\NormalTok{  max{-}survival{-}temp}

\NormalTok{  ;; metabolic scalars}
\NormalTok{  Met{-}base}
\NormalTok{  metabolism{-}rate}
\NormalTok{  total{-}metabolism}

\NormalTok{  ;; efficiency multipliers}
\NormalTok{  swim{-}efficiency}
\NormalTok{  osmoregulation{-}efficiency}
\NormalTok{  lipid{-}catabolism{-}efficiency}

\NormalTok{  ;; energy partitioning}
\NormalTok{  swim{-}base}
\NormalTok{  E{-}base}
\NormalTok{  E{-}creation}
\NormalTok{  E{-}osmo}
\NormalTok{  motion{-}ratio}
\NormalTok{  osmo{-}ratio}

\NormalTok{  ;; population normalization}
\NormalTok{  M{-}max}
\NormalTok{  max{-}age}
\NormalTok{]}

\NormalTok{;; ======================================}
\NormalTok{;;       INITIALIZE METABOLISM}
\NormalTok{;; ======================================}

\NormalTok{to initialize{-}metabolism}
\NormalTok{  let beta 0.75}
\NormalTok{  let age{-}sensitivity 0.5}
\NormalTok{  let base{-}constant 0.1}
  
\NormalTok{  ;; normalize population scaling}
\NormalTok{  set M{-}max max [weight] of fish}
\NormalTok{  set max{-}age max [age{-}num] of fish}
\NormalTok{  if M{-}max \textless{}= 0 [ set M{-}max 1 ]}
\NormalTok{  if max{-}age \textless{}= 0 [ set max{-}age 1 ]}

\NormalTok{  ;; scaling factors}
\NormalTok{  let size{-}factor (weight / M{-}max) \^{} beta}
\NormalTok{  let age{-}factor 1 + age{-}sensitivity * (1 {-} (age{-}num / max{-}age))}
\NormalTok{  let temp{-}factor (optimal{-}temperature / max{-}survival{-}temp)}

\NormalTok{  ;; baseline metabolic constant}
\NormalTok{  set Met{-}base base{-}constant * size{-}factor * age{-}factor * temp{-}factor}

\NormalTok{  ;; energy partitions}
\NormalTok{  let motion{-}ratio{-}local 0.7}
\NormalTok{  let osmo{-}ratio{-}local   0.3}
\NormalTok{  let total{-}ratio (motion{-}ratio{-}local + osmo{-}ratio{-}local)}

\NormalTok{  set motion{-}ratio (motion{-}ratio{-}local / total{-}ratio)}
\NormalTok{  set osmo{-}ratio   (osmo{-}ratio{-}local / total{-}ratio)}

\NormalTok{  set swim{-}base (Met{-}base * motion{-}ratio)}
\NormalTok{  set E{-}osmo 0}

\NormalTok{  ;; fixed fractions}
\NormalTok{  set E{-}base     (Met{-}base * 0.02)}
\NormalTok{  set E{-}creation (Met{-}base * 0.05)}

\NormalTok{  ;; initial metabolic state}
\NormalTok{  set metabolism{-}rate Met{-}base}
\NormalTok{  set total{-}metabolism 0}
\NormalTok{end}


\NormalTok{;; ======================================}
\NormalTok{;;        CALCULATE METABOLISM EACH TICK}
\NormalTok{;; ======================================}

\NormalTok{to calculate{-}metabolism}
\NormalTok{  let beta 0.75}

\NormalTok{  ;; repopulate scaling values}
\NormalTok{  set M{-}max max [weight] of fish}
\NormalTok{  set max{-}age max [age{-}num] of fish}
\NormalTok{  if M{-}max \textless{}= 0 [ set M{-}max 1 ]}
\NormalTok{  if max{-}age \textless{}= 0 [ set max{-}age 1 ]}

\NormalTok{  ;; get patch temperature safely}
\NormalTok{  let T{-}env [temperature] of patch{-}here}
\NormalTok{  if not is{-}number? T{-}env [ set T{-}env optimal{-}temperature ]}

\NormalTok{  ;; scaling factors}
\NormalTok{  let size{-}factor (weight / M{-}max) \^{} beta}
\NormalTok{  let temp{-}factor Q10 \^{} ((T{-}env {-} optimal{-}temperature) / 10)}
\NormalTok{  let age{-}factor 1 + (1 {-} (age{-}num / max{-}age)) * 0.5}

\NormalTok{  ;; metabolic rate (per tick)}
\NormalTok{  let M{-}total Met{-}base * size{-}factor * temp{-}factor * age{-}factor}
\NormalTok{  set metabolism{-}rate M{-}total}
\NormalTok{  set total{-}metabolism total{-}metabolism + M{-}total}

\NormalTok{  ;; baseline reference metabolism}
\NormalTok{  let baseline{-}size{-}factor 1}
\NormalTok{  let baseline{-}age{-}factor 1.5}
\NormalTok{  let baseline{-}temp{-}factor 1}

\NormalTok{  let baseline{-}metabolism Met{-}base *}
\NormalTok{      baseline{-}size{-}factor *}
\NormalTok{      baseline{-}temp{-}factor *}
\NormalTok{      baseline{-}age{-}factor}

\NormalTok{  ;; efficiency multipliers}
\NormalTok{  let efficiency M{-}total / baseline{-}metabolism}

\NormalTok{  set swim{-}efficiency              efficiency}
\NormalTok{  set osmoregulation{-}efficiency    efficiency}
\NormalTok{  set lipid{-}catabolism{-}efficiency  efficiency}
\NormalTok{end}
\end{Highlighting}
\end{Shaded}

\chapter{Digestion}\label{digestion}

\section{Overview}\label{overview-1}

Digestion converts consumed prey mass into usable energy and transfers associated contaminants such as inorganic mercury and methylmercury from stomach contents into internal body burden pools. This submodel simulates the breakdown of ingested prey at a rate proportional to the agent's current metabolic rate. As stomach contents are digested, a fraction is converted into energy while mercury and methylmercury are incorporated into the agent's contaminant loads. When stomach contents reach zero, agents transition into foraging mode and begin drawing on lipid reserves.

\section{Purpose}\label{purpose-1}

To simulate how migratory fish process ingested prey into energy and contaminants while linking digestive performance to metabolic rate, foraging behavior, and exposure risk. Digestion provides the energetic foundation required for movement, migration, and physiological processes while directly influencing contaminant accumulation.

\section{Entities, State Variables, and Scales}\label{entities-state-variables-and-scales-1}

\subsection{Spatial and Temporal Scales}\label{spatial-and-temporal-scales-1}

\textbf{Spatial Unit:} Patch (3 m x 3 m resolution)\\
\textbf{Temporal Unit:} 5 minute time steps (tick)

\subsection{Global Variables}\label{global-variables-1}

\begin{longtable}[]{@{}
  >{\raggedright\arraybackslash}p{(\linewidth - 2\tabcolsep) * \real{0.5862}}
  >{\raggedright\arraybackslash}p{(\linewidth - 2\tabcolsep) * \real{0.4138}}@{}}
\toprule\noalign{}
\begin{minipage}[b]{\linewidth}\raggedright
Global Variable
\end{minipage} & \begin{minipage}[b]{\linewidth}\raggedright
Definition
\end{minipage} \\
\midrule\noalign{}
\endhead
\bottomrule\noalign{}
\endlastfoot
\textbf{none required} & Digestion uses internal agent state + metabolism-rate from metabolism submodel. \\
\end{longtable}

\subsection{Patch Variables}\label{patch-variables-1}

\begin{longtable}[]{@{}
  >{\raggedright\arraybackslash}p{(\linewidth - 2\tabcolsep) * \real{0.5556}}
  >{\raggedright\arraybackslash}p{(\linewidth - 2\tabcolsep) * \real{0.4444}}@{}}
\toprule\noalign{}
\begin{minipage}[b]{\linewidth}\raggedright
Variable Name
\end{minipage} & \begin{minipage}[b]{\linewidth}\raggedright
Definition
\end{minipage} \\
\midrule\noalign{}
\endhead
\bottomrule\noalign{}
\endlastfoot
None required & Digestion is an internal physiological process and does not depend on patch attributes. \\
\end{longtable}

\subsection{Agent Variables}\label{agent-variables-1}

\begin{longtable}[]{@{}
  >{\raggedright\arraybackslash}p{(\linewidth - 2\tabcolsep) * \real{0.5556}}
  >{\raggedright\arraybackslash}p{(\linewidth - 2\tabcolsep) * \real{0.4444}}@{}}
\toprule\noalign{}
\begin{minipage}[b]{\linewidth}\raggedright
Variable Name
\end{minipage} & \begin{minipage}[b]{\linewidth}\raggedright
Definition
\end{minipage} \\
\midrule\noalign{}
\endhead
\bottomrule\noalign{}
\endlastfoot
\textbf{stomach-contents} & Total mass of prey currently held in the stomach. \\
\textbf{stomach-contents-Hg} & Inorganic mercury contained within the stomach contents. \\
\textbf{stomach-contents-MeHg} & Methylmercury contained within the stomach contents. \\
\textbf{digestion-rate} & Rate of digestion equal to the current metabolic rate. \\
\textbf{digestion-efficiency} & Fraction of digested mass converted into usable energy. \\
\textbf{energy} & Total energetic reserves available to the agent. \\
\textbf{gained-energy} & Energy gained during the current tick from digestion. \\
\textbf{mehg-foraging} & Methylmercury released from stomach contents and absorbed during digestion. \\
\textbf{hg-foraging} & Inorganic mercury released from stomach contents during digestion. \\
\textbf{mehg-foraging-total} & Cumulative methylmercury risk through feeding. \\
\textbf{hg-foraging-total} & Cumulative inorganic mercury risk via digestion. \\
\textbf{mehg-total} & Total body burden of methylmercury risk. \\
\textbf{hg-total} & Total body burden of inorganic mercury risk. \\
\textbf{foraging} & Boolean indicating whether the agent is actively searching for prey. \\
\textbf{lipid-loss} & Boolean indicating when lipid reserves are used due to lack of stomach content. \\
\end{longtable}

\section{Process Overview and Scheduling}\label{process-overview-and-scheduling-1}

\begin{enumerate}
\def\labelenumi{\arabic{enumi}.}
\tightlist
\item
  Set digestion rate equal to the current metabolic rate.\\
\item
  Calculate the mass digested during the current tick.\\
\item
  Compute the corresponding amounts of inorganic and methylmercury digested.\\
\item
  Remove digested prey mass and associated contaminants from stomach contents.\\
\item
  Convert a fraction of digested biomass into usable energy.\\
\item
  Add mercury and methylmercury to cumulative body burden pools.\\
\item
  When stomach contents reach zero, set foraging to true and begin lipid use.
\end{enumerate}

\section{Design Concepts}\label{design-concepts-1}

\textbf{Basic Principles:}\\
Digestion is modeled as a metabolic rate dependent process, where higher metabolic demand results in greater digestive throughput. Contaminants are transferred into the agent's body burden proportionally to the fraction of mass digested.

\textbf{Emergence:}\\
Patterns of energy gain and contaminant uptake emerge from interactions among foraging success, metabolism, digestion efficiency, and prey availability.

\textbf{Adaptation:}\\
Agents respond to depleted stomach contents by transitioning into foraging mode and initiating lipid catabolism when necessary.

\textbf{Objectives:}\\
Agents maximize energetic intake to support migration, physiological function, and survival while minimizing energetic deficits.

\textbf{Sensing:}\\
Agents sense their own energetic state through stomach contents and energy levels.

\textbf{Stochasticity:}\\
Stochasticity arises indirectly from variation in prey encounters or foraging success, although digestion itself is deterministic.

\textbf{Collectives:}\\
Digestion occurs independently for each agent and does not require group level dynamics.

\textbf{Observation:}\\
Outputs of interest include digested mass per tick, gained energy, contaminant uptake, and cumulative contaminant burdens.

\section{Initialization}\label{initialization-1}

\begin{longtable}[]{@{}
  >{\centering\arraybackslash}p{(\linewidth - 4\tabcolsep) * \real{0.2500}}
  >{\centering\arraybackslash}p{(\linewidth - 4\tabcolsep) * \real{0.4167}}
  >{\centering\arraybackslash}p{(\linewidth - 4\tabcolsep) * \real{0.3333}}@{}}
\toprule\noalign{}
\begin{minipage}[b]{\linewidth}\centering
Variable
\end{minipage} & \begin{minipage}[b]{\linewidth}\centering
Initialized Value
\end{minipage} & \begin{minipage}[b]{\linewidth}\centering
Justification
\end{minipage} \\
\midrule\noalign{}
\endhead
\bottomrule\noalign{}
\endlastfoot
stomach-contents & 0 & Agents begin without prey unless defined by scenario. \\
stomach-contents-Hg & 0 & No initial inorganic mercury from digestion. \\
stomach-contents-MeHg & 0 & No initial methylmercury from digestion. \\
digestion-efficiency & Species specific & Determines fraction of biomass converted to energy. \\
foraging & true & Agents begin in search mode when stomach is empty. \\
\end{longtable}

\section{Submodels}\label{submodels-1}

\subsection{Digestion Rate and Breakdown}\label{digestion-rate-and-breakdown}

Digestive throughput is driven by metabolic rate:

\[
digestion\text{-}rate = metabolism\text{-}rate
\]

Digested mass per tick:

\[
digested = stomach\text{-}contents \cdot digestion\text{-}rate
\]

Proportional mercury assimilation:

\[
digested\text{-}Hg = \left( \frac{digested}{stomach\text{-}contents} \right) \cdot stomach\text{-}contents\text{-}Hg
\]

\[
digested\text{-}MeHg = \left( \frac{digested}{stomach\text{-}contents} \right) \cdot stomach\text{-}contents\text{-}MeHg
\]

Updated stomach contents:

\[
stomach\text{-}contents = stomach\text{-}contents - digested
\]

\subsection{Energy Conversion}\label{energy-conversion}

Energy gained from digestion:

\[
gained\text{-}energy = digested \cdot digestion\text{-}efficiency \cdot 40
\]

Updated energy pool:

\[
energy = energy + gained\text{-}energy
\]

\subsection{Contaminant Assimilation}\label{contaminant-assimilation}

Contaminants absorbed during digestion:

\[
mehg\text{-}foraging = digested\text{-}MeHg
\]

\[
hg\text{-}foraging = digested\text{-}Hg
\]

Accumulated body burden:

\[
mehg\text{-}total = mehg\text{-}total + digested\text{-}MeHg
\]

\[
hg\text{-}total = hg\text{-}total + digested\text{-}Hg
\]

\subsection{Transition to Foraging}\label{transition-to-foraging}

When stomach contents reach zero:

\begin{itemize}
\tightlist
\item
  lipid-loss is activated
\end{itemize}

This switches the prey to burn fat energy reserves instead of relying on stomach contents.

\section{Netlogo Implementation}\label{netlogo-implementation-1}

\begin{Shaded}
\begin{Highlighting}[]
\NormalTok{;; ============================================}
\NormalTok{;; DIGESTION SUBMODEL}
\NormalTok{;; ============================================}

\NormalTok{fish{-}own [}
\NormalTok{  ;; digestion state}
\NormalTok{  stomach{-}contents}
\NormalTok{  stomach{-}contents{-}Hg}
\NormalTok{  stomach{-}contents{-}MeHg}
\NormalTok{  digestion{-}rate}
\NormalTok{  digestion{-}efficiency}

\NormalTok{  ;; contaminant tracking}
\NormalTok{  mehg{-}foraging}
\NormalTok{  hg{-}foraging}
\NormalTok{  mehg{-}foraging{-}total}
\NormalTok{  hg{-}foraging{-}total}
\NormalTok{  mehg{-}total}
\NormalTok{  hg{-}total}

\NormalTok{  ;; energy + biomass}
\NormalTok{  energy}
\NormalTok{  weight}
\NormalTok{  lipid{-}loss}
\NormalTok{  lipid{-}catabolism{-}efficiency}
\NormalTok{]}

\NormalTok{to digest}
\NormalTok{  ;; digestion tied to metabolic rate}
\NormalTok{  set digestion{-}rate metabolism{-}rate}

\NormalTok{  ;; nothing to digest}
\NormalTok{  if stomach{-}contents \textless{}= 0 [}
\NormalTok{    set foraging true}
\NormalTok{    set lipid{-}loss true}
\NormalTok{    stop}
\NormalTok{  ]}

\NormalTok{  ;; ============================================}
\NormalTok{  ;; DIGESTIVE BREAKDOWN}
\NormalTok{  ;; ============================================}

\NormalTok{  ;; amount of prey broken down this tick}
\NormalTok{  let digested (stomach{-}contents * digestion{-}rate)}

\NormalTok{  ;; numerical guard to avoid divide{-}by{-}zero}
\NormalTok{  if stomach{-}contents \textless{}= 0.000001 [ set digested 0 ]}

\NormalTok{  let digested{-}Hg   0}
\NormalTok{  let digested{-}MeHg 0}

\NormalTok{  if digested \textgreater{} 0 and stomach{-}contents \textgreater{} 0 [}
\NormalTok{    set digested{-}Hg   ((digested / stomach{-}contents) * stomach{-}contents{-}Hg)}
\NormalTok{    set digested{-}MeHg ((digested / stomach{-}contents) * stomach{-}contents{-}MeHg)}
\NormalTok{  ]}

\NormalTok{  ;; ============================================}
\NormalTok{  ;; UPDATE STOMACH CONTENTS}
\NormalTok{  ;; ============================================}

\NormalTok{  set stomach{-}contents     (max list 0 (stomach{-}contents {-} digested))}
\NormalTok{  set stomach{-}contents{-}Hg  (max list 0 (stomach{-}contents{-}Hg {-} digested{-}Hg))}
\NormalTok{  set stomach{-}contents{-}MeHg (max list 0 (stomach{-}contents{-}MeHg {-} digested{-}MeHg))}

\NormalTok{  ;; ============================================}
\NormalTok{  ;; ENERGY CONVERSION}
\NormalTok{  ;; ============================================}

\NormalTok{  let usable{-}energy (digested * digestion{-}efficiency * 40)}
\NormalTok{  set energy energy + usable{-}energy}

\NormalTok{  ;; ============================================}
\NormalTok{  ;; STORE LIPID IF ENERGY SURPLUS}
\NormalTok{  ;; ============================================}

\NormalTok{  if energy \textgreater{} 100 [}
\NormalTok{    let surplus (energy {-} 100)}

\NormalTok{    ;; convert surplus to lipid}
\NormalTok{    let lipid{-}added (surplus / 40)}

\NormalTok{    ;; apply storage efficiency}
\NormalTok{    set lipid{-}added (lipid{-}added * lipid{-}catabolism{-}efficiency)}

\NormalTok{    ;; update biomass}
\NormalTok{    set weight weight + lipid{-}added}

\NormalTok{    ;; subtract energy equivalent}
\NormalTok{    set energy energy {-} (lipid{-}added * 40)}
\NormalTok{  ]}

\NormalTok{  ;; ============================================}
\NormalTok{  ;; CONTAMINANT ASSIMILATION}
\NormalTok{  ;; ============================================}

\NormalTok{  set mehg{-}foraging          digested{-}MeHg}
\NormalTok{  set hg{-}foraging            digested{-}Hg}
\NormalTok{  set mehg{-}foraging{-}total    mehg{-}foraging{-}total + digested{-}MeHg}
\NormalTok{  set hg{-}foraging{-}total      hg{-}foraging{-}total + digested{-}Hg}
\NormalTok{  set mehg{-}total             mehg{-}total + digested{-}MeHg}
\NormalTok{  set hg{-}total               hg{-}total + digested{-}Hg}

\NormalTok{  ;; ============================================}
\NormalTok{  ;; SWITCH TO Lipid Loss IF EMPTY}
\NormalTok{  ;; ============================================}

\NormalTok{  if stomach{-}contents \textless{}= 0 [}
\NormalTok{    set lipid{-}loss true}
\NormalTok{  ]}
\NormalTok{end}
\end{Highlighting}
\end{Shaded}

\chapter{Salinity Exposure}\label{salinity-exposure}

\section{Overview}\label{overview-2}

Osmoregulation allows migratory fish to maintain homeostasis by regulating internal ion concentrations in response to varying environmental salinities. This function simulates salinity exposure through ion-regulatory stress, chloride cell expression, and the metabolic energy cost of to acclimate to this exposure internally in a spatiotemporal explicit context.

\section{Purpose}\label{purpose-2}

To simulate stress response to salinity changes for migratory fish in coastal systems by regulating chloride cell density and allocating energy toward ion-regulatory processes.

\section{Entities, State Variables, and Scales}\label{entities-state-variables-and-scales-2}

\subsection{Spatial and Temporal Scales}\label{spatial-and-temporal-scales-2}

\begin{itemize}
\tightlist
\item
  \textbf{Spatial Unit}: Patch (3 m x 3 m resolution)
\item
  \textbf{Temporal Unit}: 5-minute time steps (\texttt{tick})
\end{itemize}

\subsection{Global Variables}\label{global-variables-2}

\begin{longtable}[]{@{}ll@{}}
\toprule\noalign{}
Global Variable & Definition \\
\midrule\noalign{}
\endhead
\bottomrule\noalign{}
\endlastfoot
\textbf{none required} & Salinity exposure uses internal agent state \\
\end{longtable}

\subsection{Patch Variables}\label{patch-variables-2}

\begin{longtable}[]{@{}
  >{\raggedright\arraybackslash}p{(\linewidth - 2\tabcolsep) * \real{0.2455}}
  >{\raggedright\arraybackslash}p{(\linewidth - 2\tabcolsep) * \real{0.7545}}@{}}
\toprule\noalign{}
\begin{minipage}[b]{\linewidth}\raggedright
Variable Name
\end{minipage} & \begin{minipage}[b]{\linewidth}\raggedright
Definition
\end{minipage} \\
\midrule\noalign{}
\endhead
\bottomrule\noalign{}
\endlastfoot
\textbf{Salinity} \(S_{patch}\) & The salt concentration of a given patch, derived from hydrodynamic model inputs. \\
\end{longtable}

\subsection{Agent Variables}\label{agent-variables-2}

\begin{longtable}[]{@{}
  >{\raggedright\arraybackslash}p{(\linewidth - 2\tabcolsep) * \real{0.3510}}
  >{\raggedright\arraybackslash}p{(\linewidth - 2\tabcolsep) * \real{0.6490}}@{}}
\toprule\noalign{}
\begin{minipage}[b]{\linewidth}\raggedright
Variable Name
\end{minipage} & \begin{minipage}[b]{\linewidth}\raggedright
Definition
\end{minipage} \\
\midrule\noalign{}
\endhead
\bottomrule\noalign{}
\endlastfoot
\textbf{acclimated-salinity} \(S_{agent}\) & The salinity level the agent is currently acclimated to. \\
\textbf{ionregulatory-stress} \(I_{stress}\) & The level of stress an agent experiences when regulating ion balance due to osmotic difference. \\
\textbf{chloride-density-min} \(C_{min}\) & Minimum level of chloride cells, present even in low-stress conditions. \\
\textbf{chloride-density-max} \(C_{max}\) & Maximum level of chloride cells at high stress. \\
\textbf{chloride-cell-density} \(C\) & The current number of chloride cells expressed by the agent. \\
\textbf{chloride-max-proliferation} \(R_{proliferation}\) & The max number of chloride cells that can be expressed per time step. \\
\textbf{chloride-cells-this-tick} \(C_{tick}\) & The number of chloride cells created (or destroyed) in the current time step. \\
\textbf{acclimation-rate} \(\alpha\) & The rate at which chloride cell density increases over time. \\
\textbf{C-mid} \(C_{mid}\) & The chloride cell density at which stress buffering is 50\% effective. \\
\textbf{time-since-last-osmoregulation} \(t_{osmo}\) & The time elapsed since the last chloride cell regulation event. \\
\textbf{Energy} \(E_{agent}\) & The agent's total available energy for physiological functions. \\
\textbf{E-osmo} \(E_{osmo}\) & Total energy used for ion regulation (osmoregulation). \\
\textbf{E-base} \(E_{base}\) & The base energy cost per chloride cell. \\
\textbf{E-creation} \(E_{creation}\) & The energy cost for producing new chloride cells. \\
\textbf{metabolic-max} \(Met_{max}\) & Maximum metabolic cost for chloride cell creation. \\
\end{longtable}

\section{Process Overview and Scheduling}\label{process-overview-and-scheduling-2}

\begin{enumerate}
\def\labelenumi{\arabic{enumi}.}
\item
  Compute osmotic stress based on difference between \(S_{patch}\) and \(S_{agent}\).
\item
  Adjust chloride cell density depending on time since last osmoregulation.
\item
  Compute energy cost of osmoregulation.
\item
  Deduct energy expenditure from agent's energy pool.
\end{enumerate}

\section{Design Concepts}\label{design-concepts-2}

\textbf{Basic Principles:} The model is based on principles of physiological ecology and osmoregulatory energetics in teleost and apterygian species. It draws from empirical findings (e.g., Allen et al., 2009; Little et al., 2023) and includes size scaling, stress buffering, and energy constraints. These principles are implemented at the submodel level (e.g., chloride proliferation, stress calculation) to simulate realistic physiological feedbacks to changes in environmental salinity.

\textbf{Emergence:} Ion-regulatory stress, chloride cell expression, and energy expenditure emerge from an agent's interaction with temporally and spatially variable salinity environments. These patterns are not pre-specified but arise dynamically through adaptive physiological responses.

\textbf{Adaptation}: Agents respond to osmotic stress by adjusting chloride cell density, a trait that buffers stress. This process allows individuals to reduce internal-external salinity gradients and maintain ion homeostasis.

\textbf{Objectives:} Agents seek to support survival by reducing stress and avoiding excessive energy loss through regulating chloride cell expression.

\textbf{Sensing}: Agents sense local salinity (\(S_{patch}\)) and compare it with their acclimated salinity (\(S_{agent}\)). They also track their own energy state and time since last osmoregulation.

\textbf{Stochasticity}: Acclimation may vary with \(\alpha\), which can be drawn from a defined range per individual to reflect physiological variation across the population.

\textbf{Observation:} Outputs include \(I_{stress}\), \(C\), \(E_{osmo}\), and \(E_{agent}\), all tracked per individual and exportable for analysis or visualization.

\section{Initialization}\label{initialization-2}

\begin{longtable}[]{@{}
  >{\centering\arraybackslash}p{(\linewidth - 4\tabcolsep) * \real{0.1034}}
  >{\centering\arraybackslash}p{(\linewidth - 4\tabcolsep) * \real{0.2069}}
  >{\centering\arraybackslash}p{(\linewidth - 4\tabcolsep) * \real{0.6828}}@{}}
\toprule\noalign{}
\begin{minipage}[b]{\linewidth}\centering
Variable
\end{minipage} & \begin{minipage}[b]{\linewidth}\centering
Initialized Value
\end{minipage} & \begin{minipage}[b]{\linewidth}\centering
Justification
\end{minipage} \\
\midrule\noalign{}
\endhead
\bottomrule\noalign{}
\endlastfoot
\(S_{patch}\) & user-defined for data input & This input can be user-defined realistic data values or known spatial data. \\
\(S_{agent}\) & 35 (psu) & Assumes agents start acclimated to marine environment. \\
\(I_{stress}\) & 1 & Acclimated agents have minimal stress levels. \\
\(C\) & 50\% & Starts with partial cell density, allowing for regulation depending on environmental conditions. \\
\(C_{min}\) & 10\% & A baseline level of chloride cells is necessary for basic osmoregulatory functions. \\
\(C_{max}\) & 100\% & Agents can't express more than 100\% of cells. \\
\(\alpha\) & 0.0017 - 0.002 & Osmolarity stabilization from Figure 3. in (Allen et al., 2009). \\
\(C_{mid}\) & 50\% & When cells are 50\% density, stress buffering is 50\% effective (Allen et al., 2009). \\
\(E_{agent}\) & 100\% & Agent starts with limited energy before migration. \\
\(E_{base}\) & Teleost (4\%)

Aptoerygian () & Based on the \textbf{branchial cost} (Little et al., 2023; Kirschner, 1993). \\
\(Met_{max}\) & Teleost (3.5\%) & Based on the intestinal and renal cost \& size of agent (Little et al., 2023; Kirschner, 1993). \\
\(k\) & -0.75 & Scaling component for body mass is negative (Kirschner, 1993) and follows Kleiber's Law. \\
\end{longtable}

\section{Submodels}\label{submodels-2}

\subsection{Osmotic Stress}\label{osmotic-stress}

Ion-regulatory stress (\(I_{stress}\)) is calculated based on the difference between an agent's acclimated salinity and the ambient patch salinity, adjusted by the chloride cell buffering effect:

\[
I_{stress} = \frac{\log_{10}(1 + |S_{agent} - S_{patch}|) \cdot 10}{1 + e^{-2 \cdot (C / C_{mid})}}
\]

Stress is capped within the range {[}1, 10{]}, and may be reduced slightly over time if salinity remains stable and chloride density is sufficient:

\[
I_{stress} = I_{stress} \cdot 0.98 \quad \text{if conditions are stable and } C > C_{min}
\]

Agents also slowly shift their acclimated salinity toward ambient salinity when conditions have been stable for several time steps:

\[
S_{agent} = S_{agent} + (S_{patch} - S_{agent}) \cdot 0.02
\]

Where:

\begin{itemize}
\tightlist
\item
  \(I_{stress}\) is ion-regulatory (osmotic) stress, scaled between 1 and 10.
\item
  \(S_{agent}\) is the agent's acclimated salinity.
\item
  \(S_{patch}\) is the environmental salinity at the current patch.
\item
  \(C\) is the chloride cell density (percent of maximum).
\item
  \(C_{mid}\) is the density at which buffering is 50\% effective.
\end{itemize}

\subsection{Chloride Cell Density}\label{chloride-cell-density}

Chloride cell proliferation is driven by the level of ion-regulatory stress the agent experiences when encountering a difference in salinity. The greater the stress, the higher the target chloride density the agent attempts to reach, up to a maximum threshold. Agents adjust their chloride cell density based on their current ion-regulatory stress and acclimation status. Chloride cells are not adjusted unless the agent's energy exceeds 25\%.

The chloride cell density is based on stress:

\[
C_{target} = C_{min} + (C_{max} - C_{min}) \cdot \left(\frac{I_{stress}}{10}\right)
\]

If salinity conditions have remained stable for an extended period (e.g., 288 ticks), \textbackslash(C\_\{target\}\textbackslash) is slightly reduced to reflect partial downregulation of chloride cells due to long-term acclimation:

\[
C_{target} = C_{target} \cdot 0.99 \quad \text{if stable}
\]

Chloride cell density then approaches the target using a double-rate adjustment and capped maximum rate of change:

\[
\Delta C = \left(C_{target} - C_{current}\right) \cdot \left(2 \cdot R_{proliferation}\right)
\]

\[
\Delta C = \max\left(-R_{max}, \min(R_{max}, \Delta C)\right)
\]

If the agent has low energy (\(\leq 50\%\)), the adjustment rate is halved:

\[
\Delta C = \Delta C \cdot 0.5 \quad \text{if energy is low}
\]

Finally, the chloride cell density is updated and constrained between \(C_{min}\) and \(C_{max}\):

\[
C_{new} = \max(C_{min}, \min(C_{max}, C_{current} + \Delta C))
\]

This ensures that the agent does not overshoot the physiologically realistic limit of chloride cell density, while still responding to osmotic stress.

Chloride density is only recalculated after a given acclimation interval:

\[
t_{osmo} \geq \alpha^{-1}
\]

After updating, the acclimation timer is reset:

\[
t_{osmo} = 0
\]

This prevents agents from recalculating chloride density every time step and allows for controlled, realistic responses to prolonged stress and salinity changes.

Where:

\begin{itemize}
\tightlist
\item
  \(I_{stress}\) is the ion-regulatory stress, scaled from 1 to 10.
\item
  \(C_{target}\) is the desired chloride cell density based on stress level.
\item
  \(C_{min}\) and \(C_{max}\) are the bounds for chloride cell density.
\item
  \(R_{proliferation}\) determines the \textbf{maximum allowable increase} per time step.
\item
  \(\Delta C\) is the rate of change in chloride cell expression.
\item
  \(R_{max} = (C_{max} - C_{min}) \cdot R_{proliferation}\)
\item
  \(C_{new}\) is the percent of new chloride cell expression.
\item
  \(\alpha\) is the acclimation rate constant.
\item
  \(t_{osmo}\) represents time since the last osmoregulation event.
\end{itemize}

\subsection*{Osmoregulation Energy}\label{osmoregulation-energy}
\addcontentsline{toc}{subsection}{Osmoregulation Energy}

Metabolic cost related to size:

\[
E_{creation} = Met_{max} * (\frac{M}{M_{max}})^k
\]

Where:

\begin{itemize}
\item
  \(E_{creation}\) is the energy cost of chloride cell creation
\item
  \(Met_{max}\) is the maximum metabolic cost of the agent
\item
  \(M\) is equal to the agent's size, where smaller fish spend proportionally more energy on osmoregulation (Little et al., 2023)
\item
  \(M_{max}\) is the maximum mass of an agent within the population
\item
  \(k\) follows size-dependent variation in energy allocation, consistent with a negative scaling exponent.
\end{itemize}

Energy required for ion regulation:

\[
E_{osmo} = (E_{base} \cdot C) + (E_{creation} \cdot C_{tick})
\]

Where:

\begin{itemize}
\item
  \(E_{base}\) represents the energy cost per chloride cell for maintenance.
\item
  \(C\) is the current chloride density.
\item
  \(E_{creation}\) represents the cost of producing new chloride cells.
\item
  \(C_{tick}\) is the number of newly created chloride cells in the current time step.
\end{itemize}

\subsection{Energy Balance}\label{energy-balance}

Agents balance energy to osmoregulate with total energy allowance:

\[
E_{agent} = E_{agent} - E_{osmo}
\]

Where:

\begin{itemize}
\item
  \(E_{agent}\) is the total energy of the agent.
\item
  \(E_{osmo}\) is the energy consumed during osmoregulation.
\end{itemize}

\begin{Shaded}
\begin{Highlighting}[]
\NormalTok{;; ================================================================}
\NormalTok{;; SALINITY EXPOSURE AND OSMOREGULATION SUBMODEL}
\NormalTok{;; ================================================================}

\NormalTok{globals [}
\NormalTok{  stable{-}time                ;; tracks duration of stable salinity}
\NormalTok{  salinity{-}threshold         ;; threshold difference to count as “stable”}
\NormalTok{]}

\NormalTok{patches{-}own [}
\NormalTok{  salinity                   ;; environmental salinity (psu)}
\NormalTok{]}

\NormalTok{fish{-}own [}
\NormalTok{  ;; {-}{-}{-} Salinity Physiology {-}{-}{-}}
\NormalTok{  acclimated{-}salinity}
\NormalTok{  ionregulatory{-}stress       ;; 1–10 scale}

\NormalTok{  ;; {-}{-}{-} Chloride Cell Physiology {-}{-}{-}}
\NormalTok{  chloride{-}density{-}min}
\NormalTok{  chloride{-}density{-}max}
\NormalTok{  chloride{-}cell{-}density}
\NormalTok{  chloride{-}max{-}proliferation}
\NormalTok{  chloride{-}cells{-}this{-}tick}
\NormalTok{  C{-}mid}

\NormalTok{  ;; {-}{-}{-} Acclimation Timing {-}{-}{-}}
\NormalTok{  acclimation{-}rate}
\NormalTok{  time{-}since{-}last{-}osmoregulation}

\NormalTok{  ;; {-}{-}{-} Energetics {-}{-}{-}}
\NormalTok{  energy}
\NormalTok{  E{-}osmo}
\NormalTok{  E{-}base}
\NormalTok{  E{-}creation}
\NormalTok{  Met{-}max}

\NormalTok{  ;; {-}{-}{-} Size Scaling {-}{-}{-}}
\NormalTok{  weight}
\NormalTok{]}


\NormalTok{;; ================================================================}
\NormalTok{;; MAIN OSMOREGULATION PROCEDURE}
\NormalTok{;; ================================================================}

\NormalTok{to osmoregulate}
\NormalTok{  calculate{-}ionregulatory{-}stress}
\NormalTok{  regulate{-}chloride{-}cell{-}density}
\NormalTok{  calculate{-}osmoregulation{-}energy}
\NormalTok{end}


\NormalTok{;; ================================================================}
\NormalTok{;; STEP 1 – CALCULATE ION{-}REGULATORY STRESS}
\NormalTok{;; ================================================================}

\NormalTok{to calculate{-}ionregulatory{-}stress}
\NormalTok{  let prev{-}sal [salinity] of previous{-}patch}
\NormalTok{  let patch{-}sal [salinity] of patch{-}here}
\NormalTok{  let agent{-}sal acclimated{-}salinity}

\NormalTok{  ;; {-}{-}{-} Base salinity difference {-}{-}{-}}
\NormalTok{  let sal{-}diff abs (agent{-}sal {-} patch{-}sal)}

\NormalTok{  ;; {-}{-}{-} Core salinity{-}based stress function {-}{-}{-}}
\NormalTok{  let salinity{-}stress salinity{-}to{-}stress sal{-}diff chloride{-}cell{-}density ionregulatory{-}stress}

\NormalTok{  ;; {-}{-}{-} Salinity stability, used for acclimation {-}{-}{-}}
\NormalTok{  let sal{-}change abs (patch{-}sal {-} prev{-}sal)}
  
\NormalTok{  if sal{-}change \textless{} salinity{-}threshold [}
\NormalTok{    set stable{-}time stable{-}time + 1}
\NormalTok{  ]}
\NormalTok{  if sal{-}change \textgreater{}= salinity{-}threshold [}
\NormalTok{    set stable{-}time 0}
\NormalTok{  ]}

\NormalTok{  ;; {-}{-}{-} Slow acclimation when salinity is stable {-}{-}{-}}
\NormalTok{  if stable{-}time \textgreater{} 4 and chloride{-}cell{-}density \textgreater{} chloride{-}density{-}min [}
\NormalTok{    set salinity{-}stress salinity{-}stress * 0.98}
\NormalTok{    set acclimated{-}salinity acclimated{-}salinity +}
\NormalTok{      ((patch{-}sal {-} acclimated{-}salinity) * 0.02)}
\NormalTok{  ]}

\NormalTok{  ;; {-}{-}{-} Chloride buffering {-}{-}{-}}
\NormalTok{  let buffer{-}reduct calculate{-}stress{-}reduction{-}chloride{-}function chloride{-}cell{-}density}

\NormalTok{  ;; Buffering only applies above minimum chloride levels}
\NormalTok{  if chloride{-}cell{-}density \textgreater{} chloride{-}density{-}min [}
\NormalTok{    set ionregulatory{-}stress max (list 1 (salinity{-}stress {-} buffer{-}reduct * 0.8))}
\NormalTok{  ]}

\NormalTok{  if chloride{-}cell{-}density \textless{}= chloride{-}density{-}min [}
\NormalTok{    set ionregulatory{-}stress salinity{-}stress}
\NormalTok{  ]}
\NormalTok{end}


\NormalTok{;; ================================================================}
\NormalTok{;; STEP 2 – REGULATE CHLORIDE CELL DENSITY}
\NormalTok{;; ================================================================}

\NormalTok{to regulate{-}chloride{-}cell{-}density}
\NormalTok{  ;; energetic constraint}
\NormalTok{  if energy \textless{}= 25 [ stop ]}

\NormalTok{  set time{-}since{-}last{-}osmoregulation time{-}since{-}last{-}osmoregulation + 1}

\NormalTok{  ;; wait until acclimation interval}
\NormalTok{  if time{-}since{-}last{-}osmoregulation \textless{} acclimation{-}rate [ stop ]}

\NormalTok{  ;; {-}{-}{-} Target chloride cell density {-}{-}{-}}
\NormalTok{  let target{-}dens chloride{-}density{-}min +}
\NormalTok{    ((chloride{-}density{-}max {-} chloride{-}density{-}min) *}
\NormalTok{     (ionregulatory{-}stress / 10))}

\NormalTok{  ;; down{-}regulate if long{-}term stability}
\NormalTok{  if stable{-}time \textgreater{} 288 [}
\NormalTok{    set target{-}dens (target{-}dens * 0.99)}
\NormalTok{  ]}

\NormalTok{  ;; constrain target to bounds}
\NormalTok{  set target{-}dens max (list chloride{-}density{-}min}
\NormalTok{                       min (list chloride{-}density{-}max target{-}dens))}

\NormalTok{  ;; {-}{-}{-} Adjustment rate {-}{-}{-}}
\NormalTok{  let change (target{-}dens {-} chloride{-}cell{-}density) *}
\NormalTok{             (2 * chloride{-}max{-}proliferation)}

\NormalTok{  let max{-}change ((chloride{-}density{-}max {-} chloride{-}density{-}min) *}
\NormalTok{                   chloride{-}max{-}proliferation)}

\NormalTok{  ;; clamp change}
\NormalTok{  set change max (list ({-}1 * max{-}change) (min (list max{-}change change)))}

\NormalTok{  ;; energy{-}dependent reduction}
\NormalTok{  if energy \textless{}= 50 [}
\NormalTok{    set change (change * 0.5)}
\NormalTok{  ]}

\NormalTok{  ;; apply change}
\NormalTok{  set chloride{-}cell{-}density chloride{-}cell{-}density + change}
\NormalTok{  set chloride{-}cell{-}density max (list chloride{-}density{-}min}
\NormalTok{                                  min (list chloride{-}density{-}max chloride{-}cell{-}density))}

\NormalTok{  set chloride{-}cells{-}this{-}tick abs change}
\NormalTok{  set time{-}since{-}last{-}osmoregulation 0}
\NormalTok{end}


\NormalTok{;; ================================================================}
\NormalTok{;; STEP 3 – OSMOREGULATION ENERGY COST}
\NormalTok{;; ================================================================}

\NormalTok{to calculate{-}osmoregulation{-}energy}
\NormalTok{  let maintenance (E{-}base * chloride{-}cell{-}density)}
\NormalTok{  let creation    (E{-}creation * chloride{-}cells{-}this{-}tick)}

\NormalTok{  set E{-}osmo (maintenance + creation)}
\NormalTok{  set energy max (list 0 (energy {-} E{-}osmo))}
\NormalTok{end}


\NormalTok{;; ================================================================}
\NormalTok{;; CHLORIDE BUFFERING FUNCTION}
\NormalTok{;; ================================================================}

\NormalTok{to{-}report calculate{-}stress{-}reduction{-}chloride{-}function [C]}
\NormalTok{  ;; logistic{-}like buffering curve}
\NormalTok{  let reduction 5 * (1 {-} exp(C * {-}0.5 / C{-}mid))}
\NormalTok{  report reduction}
\NormalTok{end}


\NormalTok{;; ================================================================}
\NormalTok{;; SALINITY → STRESS FUNCTION}
\NormalTok{;; ================================================================}

\NormalTok{to{-}report salinity{-}to{-}stress [diff C prev{-}stress]}
\NormalTok{  let scaled (log (1 + abs diff) 10) * 10}
\NormalTok{  let buffer   (1 + exp({-}2 * (C / C{-}mid)))}
\NormalTok{  let result   (scaled / buffer)}

\NormalTok{  ;; constrain to 1–10}
\NormalTok{  set result max (list 1 (min (list 10 result)))}

\NormalTok{  ;; acclimation effect}
\NormalTok{  if stable{-}time \textgreater{} 288 [}
\NormalTok{    set result (result * 0.99)}
\NormalTok{  ]}

\NormalTok{  report result}
\NormalTok{end}


\NormalTok{;; ================================================================}
\NormalTok{;; SIZE{-}SCALED CREATION COST}
\NormalTok{;; ================================================================}

\NormalTok{to calculate{-}E{-}creation}
\NormalTok{  let M{-}max max [weight] of fish}
\NormalTok{  if M{-}max \textless{}= 0 [ set M{-}max weight ]}

\NormalTok{  let beta {-}0.75}
\NormalTok{  set E{-}creation Met{-}max * ((weight / M{-}max) \^{} beta)}
\NormalTok{end}
\end{Highlighting}
\end{Shaded}

\chapter{Migration Cues}\label{migration-cues}

\section{Overview}\label{overview-3}

This submodel simulates how migratory fish detect environmental cues that signal the onset of migration. The mechanism is grounded in empirical evidence that \textbf{simultaneous declines in daylength and temperature} reliably precede the initiation of seaward migration in several anadromous species. In this model, migration begins only when a fish experiences environmental conditions consistent with an autumnal cue window defined by:

\begin{itemize}
\item
  A \textbf{decline in daylength}\\
  \[
  \Delta L < 0
  \]
\item
  A \textbf{decline in temperature}\\
  \[
  \Delta T < 0
  \]
\item
  And these declines occurring \textbf{on the same day}, indicating a \textbf{coincident decline}
\end{itemize}

When this occurs, the day is labeled a \emph{cue day}. Upon reaching cue days, individuals begin stochastically accumulating the probability of initiating migration. For predators (striped bass), migration probability additionally depends on alewife presence, linking species interactions to migration onset.

\section{Purpose}\label{purpose-3}

The purpose of this submodel is to mechanistically trigger migration using \textbf{real, observed environmental phenology} rather than arbitrary or date-based rules. This framework captures how:

\begin{itemize}
\tightlist
\item
  \textbf{Changes in ΔTemperature and ΔDaylength structure seasonality}
\item
  \textbf{Migration timing emerges pragmatically rather than being fixed}
\item
  \textbf{Fish respond to external conditions and species interactions}
\item
  \textbf{Environmental preprocessing in R feeds into NetLogo behavioral rules}
\end{itemize}

This allows migration in the model to originate from measurable ecological drivers instead of hard-coded thresholds.

\section{Entities, State Variables, and Scales}\label{entities-state-variables-and-scales-3}

\subsection{Spatial and Temporal Scales}\label{spatial-and-temporal-scales-3}

\begin{itemize}
\tightlist
\item
  \textbf{Spatial Unit}: Patch (3 m x 3 m resolution)
\item
  \textbf{Temporal Unit}: 5-minute time steps (\texttt{tick})
\end{itemize}

\subsection{Global Variables}\label{global-variables-3}

\begin{longtable}[]{@{}
  >{\raggedright\arraybackslash}p{(\linewidth - 2\tabcolsep) * \real{0.5833}}
  >{\raggedright\arraybackslash}p{(\linewidth - 2\tabcolsep) * \real{0.4167}}@{}}
\toprule\noalign{}
\begin{minipage}[b]{\linewidth}\raggedright
Global Variable
\end{minipage} & \begin{minipage}[b]{\linewidth}\raggedright
Definition
\end{minipage} \\
\midrule\noalign{}
\endhead
\bottomrule\noalign{}
\endlastfoot
\textbf{cue-table} & Lookup table containing ΔTemperature, ΔDaylength, their smoothed values, and Boolean coincident-decline flags (from R preprocessing) \\
\textbf{cue-days} & List of day-of-year values where ΔT \textless{} 0 and ΔL \textless{} 0 simultaneously (identified in R) \\
\textbf{ΔT} & Daily rate of change in temperature imported from R (\texttt{dT\_per\_day\_f} or smoothed \texttt{dT\_smooth}) \\
\textbf{ΔL} & Daily rate of change in photoperiod imported from R (\texttt{dL\_per\_day\_min} or smoothed \texttt{dL\_smooth}) \\
\textbf{cue-active?} & Whether the current day-of-year corresponds to a coincident decline day \\
\textbf{migration-trigger?} & Whether environmental signals are strong enough to begin building migration probability \\
\end{longtable}

\subsection{Patch Variables}\label{patch-variables-3}

Migration cues do not depend on patch-level conditions. No patch-scale variables are required.

\subsection{Agent Variables}\label{agent-variables-3}

\begin{longtable}[]{@{}
  >{\raggedright\arraybackslash}p{(\linewidth - 2\tabcolsep) * \real{0.4583}}
  >{\raggedright\arraybackslash}p{(\linewidth - 2\tabcolsep) * \real{0.5417}}@{}}
\toprule\noalign{}
\begin{minipage}[b]{\linewidth}\raggedright
Variable
\end{minipage} & \begin{minipage}[b]{\linewidth}\raggedright
Definition
\end{minipage} \\
\midrule\noalign{}
\endhead
\bottomrule\noalign{}
\endlastfoot
\textbf{migration-trigger?} & Whether the agent has entered a window where migration probability begins to accumulate \\
\textbf{migration-probability} & The cumulative probability of initiating migration, updated each tick once the trigger is active \\
\textbf{start-migration?} & Whether the individual has begun active migration \\
\end{longtable}

\section{Process Overview and Scheduling}\label{process-overview-and-scheduling-3}

\subsection{\texorpdfstring{1. \textbf{Environmental Inputs (from R)}}{1. Environmental Inputs (from R)}}\label{environmental-inputs-from-r}

External R preprocessing computes:

\begin{itemize}
\tightlist
\item
  ΔTemperature per day\\
\item
  ΔDaylength per day\\
\item
  Smoothed 3-day rolling means\\
\item
  Boolean \texttt{both\_negative} indicating coincident decline\\
\item
  Exports full dataset as \texttt{"cue\_dataset\_full.csv"}
\end{itemize}

This dataset defines the environmental basis for triggering migration.

\subsection{\texorpdfstring{2. \textbf{Identify Cue Days}}{2. Identify Cue Days}}\label{identify-cue-days}

Cue days are those where:

\[
\Delta T < 0 \quad \land \quad \Delta L < 0
\]

The model loads the CSV at setup and extracts all days meeting this criterion into \texttt{cue-days}.

\subsection{\texorpdfstring{3. \textbf{Activate Cue}}{3. Activate Cue}}\label{activate-cue}

Each simulation day checks whether:

\[
\text{day-of-year} \in cue\text{-}days
\]

If true, then:

\begin{itemize}
\tightlist
\item
  \textbf{cue-active? = true}
\item
  \textbf{migration-trigger? = true}
\end{itemize}

This marks the environmental onset of migration preparedness.

\subsection{\texorpdfstring{4. \textbf{Alewife Migration Probability Accumulation}}{4. Alewife Migration Probability Accumulation}}\label{alewife-migration-probability-accumulation}

Once the migration trigger is active, each tick adds:

\[
migration\_probability_{t+1} =
migration\_probability_t +
\left(0.00001 + U(0, 0.00004)\right)
\]

Capped at:

\[
migration\_probability \le 1
\]

This produces:

\begin{itemize}
\tightlist
\item
  Individual variation in migration timing\\
\item
  Realistic ramping into migration\\
\item
  Stochastic onset probabilities
\end{itemize}

\subsection{\texorpdfstring{5. \textbf{Initiation of Migration}}{5. Initiation of Migration}}\label{initiation-of-migration}

An agent begins migration when:

\[
\text{random-float}(1) < migration\_probability
\]

At this moment:

\begin{itemize}
\tightlist
\item
  \textbf{start-migration? = true}
\item
  Movement functions shift from staging to directed migration
\end{itemize}

\section{Design Concepts}\label{design-concepts-3}

\textbf{Basic Principles}\\
The mechanism reflects literature on photoperiod- and temperature-driven migration cues.

\textbf{Emergence}\\
Migration timing varies among individuals because probability increments differ slightly with stochasticity.

\textbf{Observation}\\
Key outputs: cue day activation, probability trajectories, proportion of migrants through time.

\section{Initialization}\label{initialization-3}

\begin{longtable}[]{@{}
  >{\raggedright\arraybackslash}p{(\linewidth - 4\tabcolsep) * \real{0.2603}}
  >{\raggedright\arraybackslash}p{(\linewidth - 4\tabcolsep) * \real{0.3836}}
  >{\raggedright\arraybackslash}p{(\linewidth - 4\tabcolsep) * \real{0.3562}}@{}}
\toprule\noalign{}
\begin{minipage}[b]{\linewidth}\raggedright
Variable
\end{minipage} & \begin{minipage}[b]{\linewidth}\raggedright
Initial Value
\end{minipage} & \begin{minipage}[b]{\linewidth}\raggedright
Justification
\end{minipage} \\
\midrule\noalign{}
\endhead
\bottomrule\noalign{}
\endlastfoot
\textbf{cue-days} & Extracted from CSV & Based on empirical environmental data \\
\textbf{migration-probability} & 0 & Ensures probability is accumulated only within cue window \\
\textbf{migration-trigger?} & false & No cue until ΔT and ΔL decline \\
\textbf{cue-active?} & false & Depends on DOY \\
\textbf{start-migration?} & false & Fish begin in staging mode \\
\end{longtable}

\section{Submodels}\label{submodels-3}

\section{1. Identifying Cue Days (Preprocessed in R)}\label{identifying-cue-days-preprocessed-in-r}

\[
\Delta T < 0, \;\; \Delta L < 0
\]

The preprocessing script:

\begin{enumerate}
\def\labelenumi{\arabic{enumi}.}
\tightlist
\item
  Computes ΔT and ΔL
\item
  Applies smoothing
\item
  Flags coincident declines
\end{enumerate}

\section{2. Alewife Migration Probability}\label{alewife-migration-probability}

\[
migration\_probability_{t+1}
= migration\_probability_t +
\left(0.00001 + U(0, 0.00004)\right)
\]

\[
migration\_probability \le 1
\]

\section{Netlogo Implementation}\label{netlogo-implementation-2}

\begin{Shaded}
\begin{Highlighting}[]
\NormalTok{;; {-}{-}{-}{-}{-}{-}{-}{-}{-}{-}{-}{-}{-}{-}{-}{-}{-}{-}{-}{-}{-}{-}{-}{-}{-}{-}{-}{-}{-}{-}{-}{-}{-}{-}{-}{-}{-}{-}{-}{-}{-}{-}{-}{-}{-}{-}{-}{-}{-}{-}{-}{-}{-}{-}{-}{-}{-}{-}{-}{-}{-}{-}{-}{-}{-}{-}{-}}
\NormalTok{;;  GLOBALS}
\NormalTok{;; {-}{-}{-}{-}{-}{-}{-}{-}{-}{-}{-}{-}{-}{-}{-}{-}{-}{-}{-}{-}{-}{-}{-}{-}{-}{-}{-}{-}{-}{-}{-}{-}{-}{-}{-}{-}{-}{-}{-}{-}{-}{-}{-}{-}{-}{-}{-}{-}{-}{-}{-}{-}{-}{-}{-}{-}{-}{-}{-}{-}{-}{-}{-}{-}{-}{-}{-}}
\NormalTok{globals [}
\NormalTok{  cue{-}table}
\NormalTok{  cue{-}days}

\NormalTok{  migration{-}trigger?}
\NormalTok{  cue{-}active?}
\NormalTok{]}

\NormalTok{;; {-}{-}{-}{-}{-}{-}{-}{-}{-}{-}{-}{-}{-}{-}{-}{-}{-}{-}{-}{-}{-}{-}{-}{-}{-}{-}{-}{-}{-}{-}{-}{-}{-}{-}{-}{-}{-}{-}{-}{-}{-}{-}{-}{-}{-}{-}{-}{-}{-}{-}{-}{-}{-}{-}{-}{-}{-}{-}{-}{-}{-}{-}{-}{-}{-}{-}{-}}
\NormalTok{;;  TURTLE VARIABLES}
\NormalTok{;; {-}{-}{-}{-}{-}{-}{-}{-}{-}{-}{-}{-}{-}{-}{-}{-}{-}{-}{-}{-}{-}{-}{-}{-}{-}{-}{-}{-}{-}{-}{-}{-}{-}{-}{-}{-}{-}{-}{-}{-}{-}{-}{-}{-}{-}{-}{-}{-}{-}{-}{-}{-}{-}{-}{-}{-}{-}{-}{-}{-}{-}{-}{-}{-}{-}{-}{-}}
\NormalTok{turtles{-}own [}
\NormalTok{  migration{-}probability}
\NormalTok{  start{-}migration?}
\NormalTok{]}

\NormalTok{;; {-}{-}{-}{-}{-}{-}{-}{-}{-}{-}{-}{-}{-}{-}{-}{-}{-}{-}{-}{-}{-}{-}{-}{-}{-}{-}{-}{-}{-}{-}{-}{-}{-}{-}{-}{-}{-}{-}{-}{-}{-}{-}{-}{-}{-}{-}{-}{-}{-}{-}{-}{-}{-}{-}{-}{-}{-}{-}{-}{-}{-}{-}{-}{-}{-}{-}{-}}
\NormalTok{;;  LOAD CUE DATA (CSV FROM R)}
\NormalTok{;; {-}{-}{-}{-}{-}{-}{-}{-}{-}{-}{-}{-}{-}{-}{-}{-}{-}{-}{-}{-}{-}{-}{-}{-}{-}{-}{-}{-}{-}{-}{-}{-}{-}{-}{-}{-}{-}{-}{-}{-}{-}{-}{-}{-}{-}{-}{-}{-}{-}{-}{-}{-}{-}{-}{-}{-}{-}{-}{-}{-}{-}{-}{-}{-}{-}{-}{-}}
\NormalTok{to load{-}cue{-}csv}
\NormalTok{  set cue{-}table table:make}
\NormalTok{  let raw csv:from{-}file "inputs/cue\_dataset\_full.csv"}

\NormalTok{  let header first raw}
\NormalTok{  let rows but{-}first raw}

\NormalTok{  foreach rows [}
\NormalTok{    r {-}\textgreater{}}
\NormalTok{      let doy     item 7  r}
\NormalTok{      let dTsm    item 21 r}
\NormalTok{      let dLsm    item 22 r}
\NormalTok{      let negflag item 24 r}

\NormalTok{      table:put cue{-}table doy (list dTsm dLsm negflag)}
\NormalTok{  ]}
\NormalTok{end}

\NormalTok{;; {-}{-}{-}{-}{-}{-}{-}{-}{-}{-}{-}{-}{-}{-}{-}{-}{-}{-}{-}{-}{-}{-}{-}{-}{-}{-}{-}{-}{-}{-}{-}{-}{-}{-}{-}{-}{-}{-}{-}{-}{-}{-}{-}{-}{-}{-}{-}{-}{-}{-}{-}{-}{-}{-}{-}{-}{-}{-}{-}{-}{-}{-}{-}{-}{-}{-}{-}}
\NormalTok{;;  IDENTIFY CUE DAYS (ΔT \textless{} 0 AND ΔL \textless{} 0)}
\NormalTok{;; {-}{-}{-}{-}{-}{-}{-}{-}{-}{-}{-}{-}{-}{-}{-}{-}{-}{-}{-}{-}{-}{-}{-}{-}{-}{-}{-}{-}{-}{-}{-}{-}{-}{-}{-}{-}{-}{-}{-}{-}{-}{-}{-}{-}{-}{-}{-}{-}{-}{-}{-}{-}{-}{-}{-}{-}{-}{-}{-}{-}{-}{-}{-}{-}{-}{-}{-}}
\NormalTok{to find{-}migration{-}days}
\NormalTok{  set cue{-}days []}

\NormalTok{  foreach sort table:keys cue{-}table [}
\NormalTok{    d {-}\textgreater{}}
\NormalTok{      let vals table:get cue{-}table d}
\NormalTok{      let negflag item 2 vals}

\NormalTok{      if (negflag = "TRUE" or negflag = true) [}
\NormalTok{        set cue{-}days lput d cue{-}days}
\NormalTok{      ]}
\NormalTok{  ]}
\NormalTok{end}

\NormalTok{;; {-}{-}{-}{-}{-}{-}{-}{-}{-}{-}{-}{-}{-}{-}{-}{-}{-}{-}{-}{-}{-}{-}{-}{-}{-}{-}{-}{-}{-}{-}{-}{-}{-}{-}{-}{-}{-}{-}{-}{-}{-}{-}{-}{-}{-}{-}{-}{-}{-}{-}{-}{-}{-}{-}{-}{-}{-}{-}{-}{-}{-}{-}{-}{-}{-}{-}{-}}
\NormalTok{;;  UPDATE MIGRATION CUE (DAILY)}
\NormalTok{;; {-}{-}{-}{-}{-}{-}{-}{-}{-}{-}{-}{-}{-}{-}{-}{-}{-}{-}{-}{-}{-}{-}{-}{-}{-}{-}{-}{-}{-}{-}{-}{-}{-}{-}{-}{-}{-}{-}{-}{-}{-}{-}{-}{-}{-}{-}{-}{-}{-}{-}{-}{-}{-}{-}{-}{-}{-}{-}{-}{-}{-}{-}{-}{-}{-}{-}{-}}
\NormalTok{to update{-}migration{-}cue}
\NormalTok{  ifelse member? day cue{-}days [}
\NormalTok{    set cue{-}active? true}
\NormalTok{    set migration{-}trigger? true}
\NormalTok{  ] [}
\NormalTok{    set cue{-}active? false}
\NormalTok{  ]}
\NormalTok{end}

\NormalTok{;; {-}{-}{-}{-}{-}{-}{-}{-}{-}{-}{-}{-}{-}{-}{-}{-}{-}{-}{-}{-}{-}{-}{-}{-}{-}{-}{-}{-}{-}{-}{-}{-}{-}{-}{-}{-}{-}{-}{-}{-}{-}{-}{-}{-}{-}{-}{-}{-}{-}{-}{-}{-}{-}{-}{-}{-}{-}{-}{-}{-}{-}{-}{-}{-}{-}{-}{-}}
\NormalTok{;;  ALEWIFE MIGRATION PROBABILITY ACCUMULATION}
\NormalTok{;; {-}{-}{-}{-}{-}{-}{-}{-}{-}{-}{-}{-}{-}{-}{-}{-}{-}{-}{-}{-}{-}{-}{-}{-}{-}{-}{-}{-}{-}{-}{-}{-}{-}{-}{-}{-}{-}{-}{-}{-}{-}{-}{-}{-}{-}{-}{-}{-}{-}{-}{-}{-}{-}{-}{-}{-}{-}{-}{-}{-}{-}{-}{-}{-}{-}{-}{-}}
\NormalTok{to calc{-}migration{-}probability}
\NormalTok{  if migration{-}trigger? [}
\NormalTok{    set migration{-}probability}
\NormalTok{      migration{-}probability}
\NormalTok{      + (0.00001 + random{-}float 0.00004)}

\NormalTok{    if migration{-}probability \textgreater{} 1 [}
\NormalTok{      set migration{-}probability 1}
\NormalTok{    ]}
\NormalTok{  ]}
\NormalTok{end}

\NormalTok{to migrate}
\NormalTok{if random{-}float 1 \textless{}= migration{-}probability }
\NormalTok{set start{-}migration? true}
\NormalTok{end}
\end{Highlighting}
\end{Shaded}

\chapter{Contaminant Exposure}\label{contaminant-exposure}

\section{Overview}\label{overview-4}

The contaminant exposure submodel simulates mercury and methylmercury exposure risk across gills for migratory fish moving through spatially and temporally variable contamination fields. Exposure risk depends on contaminant concentration, suspended particulate matter, and metabolic rate. Concentrations of mercury and methylmercury are normalized across the spatial domain, and depth averaged SPM values modulate risk to reflect particle bound contaminant transport. Agents accumulate both instantaneous and cumulative exposure, allowing estimation of bioaccumulation risk over the migration season.

\section{Purpose}\label{purpose-4}

To evaluate mercury and methylmercury exposure and bioaccumulation risk for migratory fishes by integrating patch level contamination, ionregulatory stress, depth averaged suspended particulate matter, and metabolism driven uptake processes in estuarine and coastal systems.

\section{Entities, State Variables, and Scales}\label{entities-state-variables-and-scales-4}

\subsection{Spatial and Temporal Scales}\label{spatial-and-temporal-scales-4}

\textbf{Spatial Unit:} Patch (3 m x 3 m resolution)\\
\textbf{Temporal Unit:} 5 minute time steps (tick)\\

\subsection{Global Variables}\label{global-variables-4}

\begin{longtable}[]{@{}
  >{\raggedright\arraybackslash}p{(\linewidth - 4\tabcolsep) * \real{0.2500}}
  >{\raggedright\arraybackslash}p{(\linewidth - 4\tabcolsep) * \real{0.4167}}
  >{\raggedright\arraybackslash}p{(\linewidth - 4\tabcolsep) * \real{0.3333}}@{}}
\toprule\noalign{}
\begin{minipage}[b]{\linewidth}\raggedright
Variable
\end{minipage} & \begin{minipage}[b]{\linewidth}\raggedright
Initialized Value
\end{minipage} & \begin{minipage}[b]{\linewidth}\raggedright
Justification
\end{minipage} \\
\midrule\noalign{}
\endhead
\bottomrule\noalign{}
\endlastfoot
\textbf{MeHg-threshold} & 15 ug per kg & Screening threshold for methylmercury \citep{noaa_national_oceanic_and_atmospheric_administration_sediment_1990, gaudet_canadian_1995} \\
\textbf{Hg-threshold} & 150 ug per kg & NOAA guideline for mercury contamination \citep{noaa_national_oceanic_and_atmospheric_administration_sediment_1990, gaudet_canadian_1995} \\
\textbf{min-Hg, max-Hg} & Calculated from input layers & Required for Hg normalization across domain \\
\textbf{min-MeHg, max-MeHg} & Calculated from input layers & Required for MeHg normalization across domain \\
\end{longtable}

\subsection{Patch Variables}\label{patch-variables-4}

\begin{longtable}[]{@{}
  >{\raggedright\arraybackslash}p{(\linewidth - 2\tabcolsep) * \real{0.5694}}
  >{\raggedright\arraybackslash}p{(\linewidth - 2\tabcolsep) * \real{0.4306}}@{}}
\toprule\noalign{}
\begin{minipage}[b]{\linewidth}\raggedright
Variable Name
\end{minipage} & \begin{minipage}[b]{\linewidth}\raggedright
Definition
\end{minipage} \\
\midrule\noalign{}
\endhead
\bottomrule\noalign{}
\endlastfoot
\textbf{mercury} & Mercury concentration at this patch. \\
\textbf{methylmercury} & Methylmercury concentration at this patch. \\
\textbf{SPM} & Depth averaged suspended particulate matter at this patch. \\
\textbf{Hg-exp-risk-alewife} & Patch level mercury exposure risk from alewives. \\
\textbf{Hg-exp-risk-stripedbass} & Patch level mercury exposure risk from striped bass. \\
\textbf{MeHg-exp-risk-alewife} & Patch level methylmercury risk from alewives. \\
\textbf{MeHg-exp-risk-stripedbass} & Patch level methylmercury risk from striped bass. \\
\end{longtable}

\subsection{Agent Variables}\label{agent-variables-4}

\begin{longtable}[]{@{}
  >{\raggedright\arraybackslash}p{(\linewidth - 2\tabcolsep) * \real{0.5694}}
  >{\raggedright\arraybackslash}p{(\linewidth - 2\tabcolsep) * \real{0.4306}}@{}}
\toprule\noalign{}
\begin{minipage}[b]{\linewidth}\raggedright
Variable Name
\end{minipage} & \begin{minipage}[b]{\linewidth}\raggedright
Definition
\end{minipage} \\
\midrule\noalign{}
\endhead
\bottomrule\noalign{}
\endlastfoot
\textbf{metabolism-rate} & Current metabolic rate used to scale uptake risk. \\
\textbf{hg-exposure-duration} & Number of ticks spent above the mercury threshold. \\
\textbf{mehg-exposure-duration} & Number of ticks spent above the MeHg threshold. \\
\textbf{hg-uptake-risk} & Instantaneous mercury uptake risk per tick. \\
\textbf{mehg-uptake-risk} & Instantaneous methylmercury uptake risk per tick. \\
\textbf{hg-total} & Cumulative mercury uptake. \\
\textbf{mehg-total} & Cumulative methylmercury uptake. \\
\textbf{hg-exposure-total} & Total non normalized Hg exposure. \\
\textbf{mehg-exposure-total} & Total non normalized MeHg exposure. \\
\textbf{hg-exposure-total-normalized} & Sum of normalized Hg exposure values. \\
\textbf{mehg-exposure-total-normalized} & Sum of normalized MeHg exposure values. \\
\end{longtable}

\section{Process Overview and Scheduling}\label{process-overview-and-scheduling-4}

\begin{enumerate}
\def\labelenumi{\arabic{enumi}.}
\tightlist
\item
  Retrieve mercury, methylmercury, and depth averaged SPM values from the current patch.\\
\item
  Normalize patch contaminants between 0 and 1.\\
\item
  Compare patch levels to stress thresholds.\\
\item
  Update exposure durations.\\
\item
  Compute uptake risk using metabolism, ionregulatory stress, and SPM.\\
\item
  Add uptake risk to cumulative body burden.\\
\item
  Update patch level species specific risk metrics.
\end{enumerate}

\section{Design Concepts}\label{design-concepts-4}

\textbf{Basic Principles:}\\
Contaminant uptake follows toxicokinetic principles where absorption increases with concentration, metabolic rate, and suspended particulate matter (SPM). Depth averaged SPM reflects hydrodynamic transport and particle bound contaminant processes for fish typically found near the middle of the water column.

\textbf{Emergence:}\\
Exposure and risk patterns emerge from individual movement across heterogeneous contamination fields coupled with metabolic and stress driven physiology.

\textbf{Adaptation:}\\
Agents do not actively avoid contamination, but their internal physiology modulates uptake in response to metabolic changes.

\textbf{Objectives:}\\
This model does not give agents risk minimization behavior. Instead, risk is assessed to evaluate means of exposure pathways and cumulative toxic exposure.

\textbf{Sensing:}\\
Agents sense patch level mercury, methylmercury, SPM, and internal metabolic state.

\textbf{Stochasticity:}\\
Variation in spatial contaminant fields or individual characteristics and state creates stochastic variation in exposure.

\textbf{Observation:}\\
Outputs include instantaneous exposure risk, cumulative exposure risk, normalized exposure totals, and patch level species specific contributions.

\section{Initialization}\label{initialization-4}

\begin{longtable}[]{@{}
  >{\raggedright\arraybackslash}p{(\linewidth - 4\tabcolsep) * \real{0.2500}}
  >{\raggedright\arraybackslash}p{(\linewidth - 4\tabcolsep) * \real{0.4167}}
  >{\raggedright\arraybackslash}p{(\linewidth - 4\tabcolsep) * \real{0.3333}}@{}}
\toprule\noalign{}
\begin{minipage}[b]{\linewidth}\raggedright
Variable
\end{minipage} & \begin{minipage}[b]{\linewidth}\raggedright
Initialized Value
\end{minipage} & \begin{minipage}[b]{\linewidth}\raggedright
Justification
\end{minipage} \\
\midrule\noalign{}
\endhead
\bottomrule\noalign{}
\endlastfoot
mercury & Input layer & Hydrodynamic or field derived contaminant layer. \\
methylmercury & Input layer & Same as above. \\
SPM & Depth averaged input & Reflects hydrodynamic particle transport. \\
\end{longtable}

\section{Submodels: Mercury Exposure and Uptake}\label{submodels-mercury-exposure-and-uptake}

\subsection{Normalization}\label{normalization}

\[
Hg_{normalized} = \frac{Hg_{current} - minHg}{maxHg - minHg}
\]

Bounded in the range 0 to 1.

\subsection{Exposure Duration}\label{exposure-duration}

\[
Hg_{exp} = Hg_{exp} + 1 \quad \text{if } Hg_{current} > Hg_{threshold}
\]

\subsection{Uptake Risk}\label{uptake-risk}

\[
Hg_{risk} = metabolism \cdot Hg_{normalized} \cdot (1 + S) \cdot (1 + SPM)
\]

\subsection{Cumulative Uptake}\label{cumulative-uptake}

\[
Hg_{total} = Hg_{total} + Hg_{risk}
\]

Patch level storage based on species for cumulative duration risk experience in space:

\begin{itemize}
\tightlist
\item
  \textbf{Hg-exp-risk}\\
\item
  \textbf{Hg-exp-risk}
\end{itemize}

\section{Submodels: Methylmercury Exposure and Uptake}\label{submodels-methylmercury-exposure-and-uptake}

\subsection{Normalization}\label{normalization-1}

\[
MeHg_{normalized} = \frac{MeHg_{current} - minMeHg}{maxMeHg - minMeHg}
\]

Clamped to the range 0 to 1.

\subsection{Exposure Duration}\label{exposure-duration-1}

\[
MeHg_{exp} = MeHg_{exp} + 1 \quad \text{if } MeHg_{current} > MeHg_{threshold}
\]

\subsection{Uptake Risk}\label{uptake-risk-1}

\[
MeHg_{risk} = metabolism \cdot MeHg_{normalized} \cdot (1 + S) \cdot (1 + SPM)
\]

\subsection{Cumulative Uptake}\label{cumulative-uptake-1}

\[
MeHg_{total} = MeHg_{total} + MeHg_{risk}
\]

Patch level storage based on species for cumulative duration risk experience in space:

\begin{itemize}
\tightlist
\item
  \textbf{MeHg-exp-risk}\\
\item
  \textbf{MeHg-exp-risk}
\end{itemize}

\begin{Shaded}
\begin{Highlighting}[]

\NormalTok{globals [}
\NormalTok{  ;; Thresholds}
\NormalTok{  MeHg{-}threshold}
\NormalTok{  Hg{-}threshold}
  
\NormalTok{  ;; Domain{-}wide normalization bounds}
\NormalTok{  min{-}Hg}
\NormalTok{  max{-}Hg}
\NormalTok{  min{-}MeHg}
\NormalTok{  max{-}MeHg}
\NormalTok{]}

\NormalTok{patches{-}own [}
\NormalTok{  ;; Contaminant inputs}
\NormalTok{  mercury}
\NormalTok{  methylmercury}
\NormalTok{  SPM}

\NormalTok{  ;; Species{-}specific instantaneous risk}
\NormalTok{  Hg{-}exp{-}risk{-}alewife}
\NormalTok{  Hg{-}exp{-}risk{-}stripedbass}
\NormalTok{  MeHg{-}exp{-}risk{-}alewife}
\NormalTok{  MeHg{-}exp{-}risk{-}stripedbass}

\NormalTok{  ;; Species{-}specific cumulative patch totals}
\NormalTok{  Hg{-}exp{-}risk{-}total{-}alewife}
\NormalTok{  Hg{-}exp{-}risk{-}total{-}stripedbass}
\NormalTok{  MeHg{-}exp{-}risk{-}total{-}alewife}
\NormalTok{  MeHg{-}exp{-}risk{-}total{-}stripedbass}
\NormalTok{]}

\NormalTok{turtles{-}own [}
\NormalTok{  ;; Physiology}
\NormalTok{  metabolism{-}rate}
\NormalTok{  ionregulatory{-}stress}

\NormalTok{  ;; Exposure durations}
\NormalTok{  hg{-}exposure{-}duration}
\NormalTok{  mehg{-}exposure{-}duration}

\NormalTok{  ;; Instantaneous uptake}
\NormalTok{  hg{-}uptake{-}risk}
\NormalTok{  mehg{-}uptake{-}risk}

\NormalTok{  ;; Cumulative uptake}
\NormalTok{  hg{-}total}
\NormalTok{  mehg{-}total}

\NormalTok{  ;; Cumulative exposure raw}
\NormalTok{  hg{-}exposure{-}total}
\NormalTok{  mehg{-}exposure{-}total}

\NormalTok{  ;; Cumulative normalized exposure}
\NormalTok{  hg{-}exposure{-}total{-}normalized}
\NormalTok{  mehg{-}exposure{-}total{-}normalized}
\NormalTok{]}

\NormalTok{;; ================================================================}
\NormalTok{;; MERCURY EXPOSURE}
\NormalTok{;; ================================================================}

\NormalTok{to mercury{-}contamination}
  
\NormalTok{  ;; {-}{-}{-}{-}{-}{-}{-}{-}{-}{-}{-}{-}{-}{-}{-}{-}{-}{-}{-}{-}{-}{-}{-}{-}{-}{-}{-}{-}{-}{-}{-}{-}{-}{-}{-}{-}{-}{-}{-}{-}{-}{-}{-}{-}{-}}
\NormalTok{  ;; Retrieve patch contaminant and SPM values}
\NormalTok{  ;; {-}{-}{-}{-}{-}{-}{-}{-}{-}{-}{-}{-}{-}{-}{-}{-}{-}{-}{-}{-}{-}{-}{-}{-}{-}{-}{-}{-}{-}{-}{-}{-}{-}{-}{-}{-}{-}{-}{-}{-}{-}{-}{-}{-}{-}}
\NormalTok{  let Hg{-}current [mercury] of patch{-}here}
\NormalTok{  let SPM{-}current [SPM] of patch{-}here}
  
\NormalTok{  if not is{-}number? Hg{-}current [ set Hg{-}current 0 ]}
\NormalTok{  if not is{-}number? SPM{-}current [ set SPM{-}current 0 ]}

\NormalTok{  ;; {-}{-}{-}{-}{-}{-}{-}{-}{-}{-}{-}{-}{-}{-}{-}{-}{-}{-}{-}{-}{-}{-}{-}{-}{-}{-}{-}{-}{-}{-}{-}{-}{-}{-}{-}{-}{-}{-}{-}{-}{-}{-}{-}{-}{-}}
\NormalTok{  ;; Normalize Hg concentration (0–1)}
\NormalTok{  ;; {-}{-}{-}{-}{-}{-}{-}{-}{-}{-}{-}{-}{-}{-}{-}{-}{-}{-}{-}{-}{-}{-}{-}{-}{-}{-}{-}{-}{-}{-}{-}{-}{-}{-}{-}{-}{-}{-}{-}{-}{-}{-}{-}{-}{-}}
\NormalTok{  let Hg{-}normalized 0}
\NormalTok{  if (max{-}Hg {-} min{-}Hg) \textgreater{} 0 [}
\NormalTok{    set Hg{-}normalized (Hg{-}current {-} min{-}Hg) / (max{-}Hg {-} min{-}Hg)}
\NormalTok{  ]}
  
\NormalTok{  ;; clamp}
\NormalTok{  set Hg{-}normalized max list 0 (min list 1 Hg{-}normalized)}
  
\NormalTok{  ;; {-}{-}{-}{-}{-}{-}{-}{-}{-}{-}{-}{-}{-}{-}{-}{-}{-}{-}{-}{-}{-}{-}{-}{-}{-}{-}{-}{-}{-}{-}{-}{-}{-}{-}{-}{-}{-}{-}{-}{-}{-}{-}{-}{-}{-}}
\NormalTok{  ;; Exposure duration: threshold exceedance}
\NormalTok{  ;; {-}{-}{-}{-}{-}{-}{-}{-}{-}{-}{-}{-}{-}{-}{-}{-}{-}{-}{-}{-}{-}{-}{-}{-}{-}{-}{-}{-}{-}{-}{-}{-}{-}{-}{-}{-}{-}{-}{-}{-}{-}{-}{-}{-}{-}}
\NormalTok{  if Hg{-}current \textgreater{} Hg{-}threshold [}
\NormalTok{    set hg{-}exposure{-}duration hg{-}exposure{-}duration + 1}
\NormalTok{  ]}
  
\NormalTok{  ;; {-}{-}{-}{-}{-}{-}{-}{-}{-}{-}{-}{-}{-}{-}{-}{-}{-}{-}{-}{-}{-}{-}{-}{-}{-}{-}{-}{-}{-}{-}{-}{-}{-}{-}{-}{-}{-}{-}{-}{-}{-}{-}{-}{-}{-}}
\NormalTok{  ;; Uptake risk (toxicokinetic rule)}
\NormalTok{  ;; {-}{-}{-}{-}{-}{-}{-}{-}{-}{-}{-}{-}{-}{-}{-}{-}{-}{-}{-}{-}{-}{-}{-}{-}{-}{-}{-}{-}{-}{-}{-}{-}{-}{-}{-}{-}{-}{-}{-}{-}{-}{-}{-}{-}{-}}
\NormalTok{  let stress{-}factor (1 + ionregulatory{-}stress)}
\NormalTok{  let spm{-}factor    (1 + SPM{-}current)}

\NormalTok{  set hg{-}uptake{-}risk (metabolism{-}rate * Hg{-}normalized * stress{-}factor * spm{-}factor)}

\NormalTok{  ;; no negative risk ever}
\NormalTok{  if hg{-}uptake{-}risk \textless{} 0 [ set hg{-}uptake{-}risk 0 ]}

\NormalTok{  ;; {-}{-}{-}{-}{-}{-}{-}{-}{-}{-}{-}{-}{-}{-}{-}{-}{-}{-}{-}{-}{-}{-}{-}{-}{-}{-}{-}{-}{-}{-}{-}{-}{-}{-}{-}{-}{-}{-}{-}{-}{-}{-}{-}{-}{-}}
\NormalTok{  ;; Add to cumulative body burden}
\NormalTok{  ;; {-}{-}{-}{-}{-}{-}{-}{-}{-}{-}{-}{-}{-}{-}{-}{-}{-}{-}{-}{-}{-}{-}{-}{-}{-}{-}{-}{-}{-}{-}{-}{-}{-}{-}{-}{-}{-}{-}{-}{-}{-}{-}{-}{-}{-}}
\NormalTok{  set hg{-}total hg{-}total + hg{-}uptake{-}risk}
  
\NormalTok{  ;; {-}{-}{-}{-}{-}{-}{-}{-}{-}{-}{-}{-}{-}{-}{-}{-}{-}{-}{-}{-}{-}{-}{-}{-}{-}{-}{-}{-}{-}{-}{-}{-}{-}{-}{-}{-}{-}{-}{-}{-}{-}{-}{-}{-}{-}}
\NormalTok{  ;; Track cumulative raw \& normalized exposure}
\NormalTok{  ;; {-}{-}{-}{-}{-}{-}{-}{-}{-}{-}{-}{-}{-}{-}{-}{-}{-}{-}{-}{-}{-}{-}{-}{-}{-}{-}{-}{-}{-}{-}{-}{-}{-}{-}{-}{-}{-}{-}{-}{-}{-}{-}{-}{-}{-}}
\NormalTok{  set hg{-}exposure{-}total            hg{-}exposure{-}total + Hg{-}current}
\NormalTok{  set hg{-}exposure{-}total{-}normalized hg{-}exposure{-}total{-}normalized + Hg{-}normalized}
  
\NormalTok{  ;; {-}{-}{-}{-}{-}{-}{-}{-}{-}{-}{-}{-}{-}{-}{-}{-}{-}{-}{-}{-}{-}{-}{-}{-}{-}{-}{-}{-}{-}{-}{-}{-}{-}{-}{-}{-}{-}{-}{-}{-}{-}{-}{-}{-}{-}}
\NormalTok{  ;; Patch{-}level risk recording (species specific)}
\NormalTok{  ;; {-}{-}{-}{-}{-}{-}{-}{-}{-}{-}{-}{-}{-}{-}{-}{-}{-}{-}{-}{-}{-}{-}{-}{-}{-}{-}{-}{-}{-}{-}{-}{-}{-}{-}{-}{-}{-}{-}{-}{-}{-}{-}{-}{-}{-}}
\NormalTok{  let patch{-}risk hg{-}uptake{-}risk}
  
\NormalTok{  if breed = alewives [}
\NormalTok{    ask patch{-}here [}
\NormalTok{      set Hg{-}exp{-}risk{-}alewife       patch{-}risk}
\NormalTok{      set Hg{-}exp{-}risk{-}total{-}alewife (Hg{-}exp{-}risk{-}total{-}alewife + patch{-}risk)}
\NormalTok{    ]}
\NormalTok{  ]}
  
\NormalTok{  if breed = stripedbass [}
\NormalTok{    ask patch{-}here [}
\NormalTok{      set Hg{-}exp{-}risk{-}stripedbass       patch{-}risk}
\NormalTok{      set Hg{-}exp{-}risk{-}total{-}stripedbass (Hg{-}exp{-}risk{-}total{-}stripedbass + patch{-}risk)}
\NormalTok{    ]}
\NormalTok{  ]}

\NormalTok{end}


\NormalTok{;; ================================================================}
\NormalTok{;; METHYLMERCURY EXPOSURE}
\NormalTok{;; ================================================================}

\NormalTok{to methylmercury{-}contamination}
  
\NormalTok{  ;; {-}{-}{-}{-}{-}{-}{-}{-}{-}{-}{-}{-}{-}{-}{-}{-}{-}{-}{-}{-}{-}{-}{-}{-}{-}{-}{-}{-}{-}{-}{-}{-}{-}{-}{-}{-}{-}{-}{-}{-}{-}{-}{-}{-}{-}}
\NormalTok{  ;; Retrieve patch contaminant and SPM values}
\NormalTok{  ;; {-}{-}{-}{-}{-}{-}{-}{-}{-}{-}{-}{-}{-}{-}{-}{-}{-}{-}{-}{-}{-}{-}{-}{-}{-}{-}{-}{-}{-}{-}{-}{-}{-}{-}{-}{-}{-}{-}{-}{-}{-}{-}{-}{-}{-}}
\NormalTok{  let MeHg{-}current [methylmercury] of patch{-}here}
\NormalTok{  let SPM{-}current  [SPM] of patch{-}here}

\NormalTok{  if not is{-}number? MeHg{-}current [ set MeHg{-}current 0 ]}
\NormalTok{  if not is{-}number? SPM{-}current  [ set SPM{-}current 0 ]}

\NormalTok{  ;; {-}{-}{-}{-}{-}{-}{-}{-}{-}{-}{-}{-}{-}{-}{-}{-}{-}{-}{-}{-}{-}{-}{-}{-}{-}{-}{-}{-}{-}{-}{-}{-}{-}{-}{-}{-}{-}{-}{-}{-}{-}{-}{-}{-}{-}}
\NormalTok{  ;; Normalize MeHg concentration (0–1)}
\NormalTok{  ;; {-}{-}{-}{-}{-}{-}{-}{-}{-}{-}{-}{-}{-}{-}{-}{-}{-}{-}{-}{-}{-}{-}{-}{-}{-}{-}{-}{-}{-}{-}{-}{-}{-}{-}{-}{-}{-}{-}{-}{-}{-}{-}{-}{-}{-}}
\NormalTok{  let MeHg{-}normalized 0}
\NormalTok{  if (max{-}MeHg {-} min{-}MeHg) \textgreater{} 0 [}
\NormalTok{    set MeHg{-}normalized (MeHg{-}current {-} min{-}MeHg) / (max{-}MeHg {-} min{-}MeHg)}
\NormalTok{  ]}
  
\NormalTok{  ;; clamp}
\NormalTok{  set MeHg{-}normalized max list 0 (min list 1 MeHg{-}normalized)}
  
\NormalTok{  ;; {-}{-}{-}{-}{-}{-}{-}{-}{-}{-}{-}{-}{-}{-}{-}{-}{-}{-}{-}{-}{-}{-}{-}{-}{-}{-}{-}{-}{-}{-}{-}{-}{-}{-}{-}{-}{-}{-}{-}{-}{-}{-}{-}{-}{-}}
\NormalTok{  ;; Exposure duration: threshold exceedance}
\NormalTok{  ;; {-}{-}{-}{-}{-}{-}{-}{-}{-}{-}{-}{-}{-}{-}{-}{-}{-}{-}{-}{-}{-}{-}{-}{-}{-}{-}{-}{-}{-}{-}{-}{-}{-}{-}{-}{-}{-}{-}{-}{-}{-}{-}{-}{-}{-}}
\NormalTok{  if MeHg{-}current \textgreater{} MeHg{-}threshold [}
\NormalTok{    set mehg{-}exposure{-}duration mehg{-}exposure{-}duration + 1}
\NormalTok{  ]}
  
\NormalTok{  ;; {-}{-}{-}{-}{-}{-}{-}{-}{-}{-}{-}{-}{-}{-}{-}{-}{-}{-}{-}{-}{-}{-}{-}{-}{-}{-}{-}{-}{-}{-}{-}{-}{-}{-}{-}{-}{-}{-}{-}{-}{-}{-}{-}{-}{-}}
\NormalTok{  ;; Uptake risk}
\NormalTok{  ;; {-}{-}{-}{-}{-}{-}{-}{-}{-}{-}{-}{-}{-}{-}{-}{-}{-}{-}{-}{-}{-}{-}{-}{-}{-}{-}{-}{-}{-}{-}{-}{-}{-}{-}{-}{-}{-}{-}{-}{-}{-}{-}{-}{-}{-}}
\NormalTok{  let stress{-}factor (1 + ionregulatory{-}stress)}
\NormalTok{  let spm{-}factor    (1 + SPM{-}current)}

\NormalTok{  set mehg{-}uptake{-}risk (metabolism{-}rate * MeHg{-}normalized * stress{-}factor * spm{-}factor)}

\NormalTok{  if mehg{-}uptake{-}risk \textless{} 0 [ set mehg{-}uptake{-}risk 0 ]}
  
\NormalTok{  ;; {-}{-}{-}{-}{-}{-}{-}{-}{-}{-}{-}{-}{-}{-}{-}{-}{-}{-}{-}{-}{-}{-}{-}{-}{-}{-}{-}{-}{-}{-}{-}{-}{-}{-}{-}{-}{-}{-}{-}{-}{-}{-}{-}{-}{-}}
\NormalTok{  ;; Add to cumulative body burden}
\NormalTok{  ;; {-}{-}{-}{-}{-}{-}{-}{-}{-}{-}{-}{-}{-}{-}{-}{-}{-}{-}{-}{-}{-}{-}{-}{-}{-}{-}{-}{-}{-}{-}{-}{-}{-}{-}{-}{-}{-}{-}{-}{-}{-}{-}{-}{-}{-}}
\NormalTok{  set mehg{-}total mehg{-}total + mehg{-}uptake{-}risk}
  
\NormalTok{  ;; {-}{-}{-}{-}{-}{-}{-}{-}{-}{-}{-}{-}{-}{-}{-}{-}{-}{-}{-}{-}{-}{-}{-}{-}{-}{-}{-}{-}{-}{-}{-}{-}{-}{-}{-}{-}{-}{-}{-}{-}{-}{-}{-}{-}{-}}
\NormalTok{  ;; Track cumulative raw \& normalized exposure}
\NormalTok{  ;; {-}{-}{-}{-}{-}{-}{-}{-}{-}{-}{-}{-}{-}{-}{-}{-}{-}{-}{-}{-}{-}{-}{-}{-}{-}{-}{-}{-}{-}{-}{-}{-}{-}{-}{-}{-}{-}{-}{-}{-}{-}{-}{-}{-}{-}}
\NormalTok{  set mehg{-}exposure{-}total            mehg{-}exposure{-}total + MeHg{-}current}
\NormalTok{  set mehg{-}exposure{-}total{-}normalized mehg{-}exposure{-}total{-}normalized + MeHg{-}normalized}
  
\NormalTok{  ;; {-}{-}{-}{-}{-}{-}{-}{-}{-}{-}{-}{-}{-}{-}{-}{-}{-}{-}{-}{-}{-}{-}{-}{-}{-}{-}{-}{-}{-}{-}{-}{-}{-}{-}{-}{-}{-}{-}{-}{-}{-}{-}{-}{-}{-}}
\NormalTok{  ;; Patch{-}level risk recording (species specific)}
\NormalTok{  ;; {-}{-}{-}{-}{-}{-}{-}{-}{-}{-}{-}{-}{-}{-}{-}{-}{-}{-}{-}{-}{-}{-}{-}{-}{-}{-}{-}{-}{-}{-}{-}{-}{-}{-}{-}{-}{-}{-}{-}{-}{-}{-}{-}{-}{-}}
\NormalTok{  let patch{-}risk mehg{-}uptake{-}risk}
  
\NormalTok{  if breed = alewives [}
\NormalTok{    ask patch{-}here [}
\NormalTok{      set MeHg{-}exp{-}risk{-}alewife       patch{-}risk}
\NormalTok{      set MeHg{-}exp{-}risk{-}total{-}alewife (MeHg{-}exp{-}risk{-}total{-}alewife + patch{-}risk)}
\NormalTok{    ]}
\NormalTok{  ]}
  
\NormalTok{  if breed = stripedbass [}
\NormalTok{    ask patch{-}here [}
\NormalTok{      set MeHg{-}exp{-}risk{-}stripedbass       patch{-}risk}
\NormalTok{      set MeHg{-}exp{-}risk{-}total{-}stripedbass (MeHg{-}exp{-}risk{-}total{-}stripedbass + patch{-}risk)}
\NormalTok{    ]}
\NormalTok{  ]}

\NormalTok{end}
\end{Highlighting}
\end{Shaded}

\chapter{Filter Feeding}\label{filter-feeding}

\section{Overview}\label{overview-5}

Filter feeding is a foraging behavior in which fish consume suspended particulate matter as they migrate through estuarine and coastal systems. In this submodel, agents evaluate neighboring patches based on suspended particulate matter concentration and selectively filter feed where energetic return is highest. Filtration capacity increases with body size, metabolic rate, and digestion efficiency, and is reduced under competition from other filter feeding fish. Contaminant loads in the water column directly influence mercury and methylmercury assimilation during feeding, including biomagnification effects.

\section{Purpose and Patterns}\label{purpose-and-patterns}

The purpose of this submodel is to simulate ecologically realistic filter feeding behavior in migratory fish using suspended particulate matter as a food resource. This behavior links hydrodynamic particle fields to energy gain, contaminant exposure, and movement decisions. Filter feeding only occurs when spatial conditions and energetic returns are favorable, and incorporates competition, biomagnification, and digestion constraints.

This submodel represents the following ecological patterns:

\begin{itemize}
\item
  \textbf{Filtration capacity increases with body size, metabolism, and digestive efficiency}\\
  Larger individuals with higher metabolism filter more water and assimilate more prey mass.
\item
  \textbf{Gaussian SPM preference centered on optimal SPM}\\
  Feeding intensity is highest near an optimal SPM range and decreases away from that peak.
\item
  \textbf{Crowding penalty from con-specifics}\\
  Filtration benefit decreases when multiple individuals feed in the same patch.
\item
  \textbf{Movement toward highest benefit}\\
  Fish move toward the patch with the highest filtration benefit based on SPM and competitive pressure.
\item
  \textbf{Contaminant assimilation tied directly to filtration}\\
  Mercury and methylmercury are assimilated proportionally to SPM concentration and biomagnification factors derived from empirical trophic magnification relationships.
\end{itemize}

\section{Entities, State Variables, and Scales}\label{entities-state-variables-and-scales-5}

\subsection{Spatial and Temporal Scales}\label{spatial-and-temporal-scales-5}

\textbf{Spatial Unit:} Patch (3 m x 3 m resolution)\\
\textbf{Temporal Unit:} 5 minute time steps (tick)

\subsection{\texorpdfstring{\textbf{Global Variables}}{Global Variables}}\label{global-variables-5}

\begin{longtable}[]{@{}
  >{\raggedright\arraybackslash}p{(\linewidth - 2\tabcolsep) * \real{0.5000}}
  >{\raggedright\arraybackslash}p{(\linewidth - 2\tabcolsep) * \real{0.5000}}@{}}
\toprule\noalign{}
\begin{minipage}[b]{\linewidth}\raggedright
Global Variable
\end{minipage} & \begin{minipage}[b]{\linewidth}\raggedright
Definition
\end{minipage} \\
\midrule\noalign{}
\endhead
\bottomrule\noalign{}
\endlastfoot
\textbf{spm-sd} & Standard deviation for Gaussian SPM preference curve (controls width of feeding preference) \\
\textbf{max-filter-distance} & Maximum distance within which patches can be evaluated (optional, model dependent) \\
\textbf{BMF-MeHg} & Biomagnification factor for methylmercury (≈ 1.74; after Lavoie et al.~2013) \\
\textbf{BMF-Hg} & Biomagnification factor for inorganic Hg (≈ 1.38; after Lavoie et al.~2013) \\
\textbf{previous-x} & X-coordinate of agent's previous location (used to compute travel distance and energy cost) \\
\textbf{previous-y} & Y-coordinate of agent's previous location (used with previous-x) \\
\end{longtable}

\subsection{\texorpdfstring{\textbf{Patch Variables}}{Patch Variables}}\label{patch-variables-5}

\begin{longtable}[]{@{}
  >{\raggedright\arraybackslash}p{(\linewidth - 2\tabcolsep) * \real{0.5000}}
  >{\raggedright\arraybackslash}p{(\linewidth - 2\tabcolsep) * \real{0.5000}}@{}}
\toprule\noalign{}
\begin{minipage}[b]{\linewidth}\raggedright
Variable Name
\end{minipage} & \begin{minipage}[b]{\linewidth}\raggedright
Definition
\end{minipage} \\
\midrule\noalign{}
\endhead
\bottomrule\noalign{}
\endlastfoot
\textbf{SPM} & Suspended particulate matter concentration in the water column \\
\textbf{mercury} & Inorganic mercury concentration at the patch \\
\textbf{methylmercury} & Methylmercury concentration at the patch \\
\textbf{patch-terrain} & Identifies water or land patches \\
\end{longtable}

\subsection{\texorpdfstring{\textbf{Agent Variables}}{Agent Variables}}\label{agent-variables-5}

\begin{longtable}[]{@{}
  >{\raggedright\arraybackslash}p{(\linewidth - 2\tabcolsep) * \real{0.5000}}
  >{\raggedright\arraybackslash}p{(\linewidth - 2\tabcolsep) * \real{0.5000}}@{}}
\toprule\noalign{}
\begin{minipage}[b]{\linewidth}\raggedright
Variable Name
\end{minipage} & \begin{minipage}[b]{\linewidth}\raggedright
Definition
\end{minipage} \\
\midrule\noalign{}
\endhead
\bottomrule\noalign{}
\endlastfoot
\textbf{size} & Body size used to scale filtration capacity \\
\textbf{weight} & Biomass used in base filtration scaling (from metabolism submodel) \\
\textbf{metabolism-rate} & Metabolic rate that modulates filtration intensity \\
\textbf{digestion-efficiency} & Fraction of filtered biomass digested \\
\textbf{optimal-SPM} & SPM concentration where filtration is most efficient \\
\textbf{filter-feed?} & Boolean indicating whether the agent engages in filter feeding \\
\textbf{energy} & Current energy reserves \\
\textbf{E-swim} & Energy cost of swimming per meter \\
\textbf{swim-efficiency} & Efficiency of movement based on metabolic state \\
\textbf{speed} & Swimming speed \\
\textbf{stomach-contents} & Biomass currently held in the stomach \\
\textbf{stomach-capacity} & Maximum biomass that can be held \\
\textbf{stomach-contents-Hg} & Inorganic mercury within stomach contents \\
\textbf{stomach-contents-MeHg} & Methylmercury within stomach contents \\
\textbf{hg-total} & Total cumulative inorganic mercury body burden \\
\textbf{mehg-total} & Total cumulative methylmercury body burden \\
\textbf{hg-exposure-total} & Cumulative Hg exposure from all pathways \\
\textbf{mehg-exposure-total} & Cumulative MeHg exposure from all pathways \\
\textbf{hg-exposure-total-normalized} & Normalized inorganic mercury exposure \\
\textbf{mehg-exposure-total-normalized} & Normalized methylmercury exposure \\
\textbf{previous-x} \emph{(if not global)} & Previous x-location for travel-distance calculation \\
\textbf{previous-y} \emph{(if not global)} & Previous y-location \\
\end{longtable}

\section{Process Overview and Scheduling}\label{process-overview-and-scheduling-5}

\begin{enumerate}
\def\labelenumi{\arabic{enumi}.}
\tightlist
\item
  Compute filtration capacity based on size, metabolism, and digestion efficiency.\\
\item
  Evaluate neighboring patches using a Gaussian suitability curve around optimal SPM.\\
\item
  Apply a competition penalty based on the number of con-specifics feeding on each patch.\\
\item
  Select the patch with the highest filtration benefit.\\
\item
  Move toward the chosen patch with travel distance reduced by swimming efficiency.\\
\item
  Apply swimming energy cost.\\
\item
  Filter feed by assimilating SPM scaled biomass from the chosen patch.\\
\item
  Assimilate mercury and methylmercury using biomagnification factors from Lavoie et al (2013).\\
\item
  Add assimilated biomass and contaminants to stomach contents.
\end{enumerate}

\section{Design Concepts}\label{design-concepts-5}

\textbf{Basic Principles}\\
Filter feeding reflects how pelagic and estuarine fish forage on suspended particles while migrating. Filtration scales with size, metabolism, and digestive efficiency. SPM determines food availability and contaminant exposure.

\textbf{Emergence}\\
Feeding hotspots emerge where hydrodynamic SPM fields align with optimal SPM and competition is minimal. Contaminant loads emerge from local mercury and methylmercury concentrations and biomagnification.

\textbf{Adaptation}\\
Agents adjust their direction and feeding behavior based on SPM quality, filtration capacity, and local competition.

\textbf{Sensing}\\
Agents sense SPM, mercury, methylmercury, competition from conspecifics, stomach fullness, and their own metabolic and digestive state.

\textbf{Stochasticity}\\
Variation in SPM quality, contaminant concentrations, and local competition create variability in feeding and assimilation outcomes.

\textbf{Interaction}\\
Competition reduces filtration benefit when more fish feed on the same patch. Mercury and methylmercury assimilation depends on local contaminant levels and feeding intensity.

\textbf{Observation}\\
The model tracks biomass intake, mercury and methylmercury assimilation, stomach contents, and contaminant load added per patch.

\section{Initialization}\label{initialization-5}

\begin{longtable}[]{@{}
  >{\raggedright\arraybackslash}p{(\linewidth - 4\tabcolsep) * \real{0.3333}}
  >{\raggedright\arraybackslash}p{(\linewidth - 4\tabcolsep) * \real{0.3333}}
  >{\raggedright\arraybackslash}p{(\linewidth - 4\tabcolsep) * \real{0.3333}}@{}}
\toprule\noalign{}
\begin{minipage}[b]{\linewidth}\raggedright
Variable
\end{minipage} & \begin{minipage}[b]{\linewidth}\raggedright
Initialized Value
\end{minipage} & \begin{minipage}[b]{\linewidth}\raggedright
Justification
\end{minipage} \\
\midrule\noalign{}
\endhead
\bottomrule\noalign{}
\endlastfoot
stomach-contents & 0 & No biomass in the stomach at model start \\
stomach-contents-Hg & 0 & No initial inorganic mercury from feeding \\
stomach-contents-MeHg & 0 & No initial methylmercury \\
digestion-efficiency & Species specific & Determines fraction of filtered biomass assimilated \\
optimal-SPM & Species specific & Represents preferred SPM concentration \\
\end{longtable}

\section{Submodels}\label{submodels-4}

\subsection{Filtration Capacity}\label{filtration-capacity}

Filtration scales with size, metabolism, and digestion efficiency:

\[
Filtration_{base} = \left(\frac{size}{100}\right) \cdot metabolism\_{rate} \cdot digestion\_{efficiency}
\]

\subsection{SPM Interest (Gaussian Suitability)}\label{spm-interest-gaussian-suitability}

Suitability peaks at the optimal SPM and declines with distance:

\[
Interest = e^{-\frac{(SPM - optimalSPM)^2}{2 \cdot \sigma^2}}
\]

Where \(\sigma\) is the SPM standard deviation tolerance (here 25).

\subsection{Benefit Calculation}\label{benefit-calculation}

Benefit for each neighboring patch:

\[
Benefit = SPM \cdot Interest \cdot Filtration_{base}
\]

If con-specifics are present:

\[
Benefit = \frac{Benefit}{N_{conspecifics}}
\]

\subsection{Movement}\label{movement}

Travel distance is reduced by swimming efficiency:

\[
Distance = min(DistanceToPatch,\ speed \cdot swim\_{efficiency})
\]

Energy cost of swimming:

\[
Energy = Energy - (E\_{swim} \cdot Distance)
\]

\subsection{Feeding and Stomach Loading}\label{feeding-and-stomach-loading}

Assimilated biomass:

\[
Intake = min(PreyAssimilated,\ StomachCapacity - StomachContents)
\]

\subsection{Contaminant Assimilation and Biomagnification}\label{contaminant-assimilation-and-biomagnification}

Biomagnification factors follow \citet{lavoie_biomagnification_2013}:

\[
BMF_{MeHg} = 10^{(0.24 \cdot \Delta TP)}
\]

\[
BMF_{Hg} = 10^{(0.14 \cdot \Delta TP)}
\]

With \(\Delta TP = 1\):

\[
BMF_{MeHg} = 10^{0.24} \approx 1.74
\]

\[
BMF_{Hg} = 10^{0.14} \approx 1.38
\]

Assimilation equations:

\[
MeHg_{assim} = SPM \cdot MeHg_{patch} \cdot BMF_{MeHg} \cdot PreyAssimilated
\]

\[
Hg_{assim} = SPM \cdot Hg_{patch} \cdot BMF_{Hg} \cdot PreyAssimilated
\]

Stomach contaminant update:

\[
StomachMeHg = StomachMeHg + MeHg_{assim}
\]

\[
StomachHg = StomachHg + Hg_{assim}
\]

\begin{Shaded}
\begin{Highlighting}[]

\NormalTok{globals [}
\NormalTok{  ;; Filter feeding parameters}
\NormalTok{  spm{-}sd                    ;; standard deviation for Gaussian SPM preference curve (≈25)}
\NormalTok{  max{-}filter{-}distance       ;; how far fish can evaluate patches (optional)}

\NormalTok{  ;; Biomagnification constants from Lavoie et al. 2013}
\NormalTok{  BMF{-}MeHg                  ;; ≈ 1.74 (MeHg biomagnification factor)}
\NormalTok{  BMF{-}Hg                    ;; ≈ 1.38 (Hg biomagnification factor)}
\NormalTok{]}

\NormalTok{patches{-}own [}
\NormalTok{  ;; physical environment}
\NormalTok{  SPM                         ;; suspended particulate matter}
\NormalTok{  patch{-}terrain               ;; "water" or "land"}

\NormalTok{  ;; contaminant concentrations}
\NormalTok{  mercury                     ;; inorganic Hg}
\NormalTok{  methylmercury               ;; MeHg}
\NormalTok{]}

\NormalTok{fish{-}own [}

\NormalTok{  ;; {-}{-}{-}{-}{-}{-}{-}{-}{-}{-}{-}{-}{-}{-}{-}{-}{-}{-}{-}{-}{-}{-}{-}{-}{-}{-}{-}}
\NormalTok{  ;; Filter feeding physiology}
\NormalTok{  ;; {-}{-}{-}{-}{-}{-}{-}{-}{-}{-}{-}{-}{-}{-}{-}{-}{-}{-}{-}{-}{-}{-}{-}{-}{-}{-}{-}}
\NormalTok{  size                          ;; body size scaling filtration}
\NormalTok{  metabolism{-}rate               ;; modulates filtration intensity}
\NormalTok{  digestion{-}efficiency          ;; fraction of filtered biomass digested}
\NormalTok{  optimal{-}SPM                   ;; preferred SPM concentration}
\NormalTok{  filter{-}feed?                  ;; Boolean: whether agent filter feeds}

\NormalTok{  ;; {-}{-}{-}{-}{-}{-}{-}{-}{-}{-}{-}{-}{-}{-}{-}{-}{-}{-}{-}{-}{-}{-}{-}{-}{-}{-}{-}}
\NormalTok{  ;; Movement \& energy}
\NormalTok{  ;; {-}{-}{-}{-}{-}{-}{-}{-}{-}{-}{-}{-}{-}{-}{-}{-}{-}{-}{-}{-}{-}{-}{-}{-}{-}{-}{-}}
\NormalTok{  speed                         ;; movement rate}
\NormalTok{  swim{-}efficiency               ;; metabolism{-}adjusted efficiency}
\NormalTok{  E{-}swim                        ;; energy cost per unit distance}
\NormalTok{  energy                        ;; energetic budget}

\NormalTok{  ;; {-}{-}{-}{-}{-}{-}{-}{-}{-}{-}{-}{-}{-}{-}{-}{-}{-}{-}{-}{-}{-}{-}{-}{-}{-}{-}{-}}
\NormalTok{  ;; Stomach contents}
\NormalTok{  ;; {-}{-}{-}{-}{-}{-}{-}{-}{-}{-}{-}{-}{-}{-}{-}{-}{-}{-}{-}{-}{-}{-}{-}{-}{-}{-}{-}}
\NormalTok{  stomach{-}contents}
\NormalTok{  stomach{-}capacity}
\NormalTok{  stomach{-}contents{-}Hg           ;; inorganic Hg from filter feeding}
\NormalTok{  stomach{-}contents{-}MeHg         ;; MeHg from filter feeding}

\NormalTok{  ;; {-}{-}{-}{-}{-}{-}{-}{-}{-}{-}{-}{-}{-}{-}{-}{-}{-}{-}{-}{-}{-}{-}{-}{-}{-}{-}{-}}
\NormalTok{  ;; Contamination physiology}
\NormalTok{  ;; (needed for risk integration)}
\NormalTok{  ;; {-}{-}{-}{-}{-}{-}{-}{-}{-}{-}{-}{-}{-}{-}{-}{-}{-}{-}{-}{-}{-}{-}{-}{-}{-}{-}{-}}
\NormalTok{  hg{-}total                      ;; cumulative Hg body burden}
\NormalTok{  mehg{-}total                    ;; cumulative MeHg body burden}
\NormalTok{  hg{-}exposure{-}total}
\NormalTok{  mehg{-}exposure{-}total}
\NormalTok{  hg{-}exposure{-}total{-}normalized}
\NormalTok{  mehg{-}exposure{-}total{-}normalized}
\NormalTok{]}

\NormalTok{to filter{-}feed}
\NormalTok{if filter{-}feed? [}
\NormalTok{  let spm{-}sd 25}
\NormalTok{  ;; Base filtration capacity scales with body size *and* metabolism}
\NormalTok{  let base{-}filtration ((weight / 100) * metabolism{-}rate)}
\NormalTok{  set base{-}filtration (base{-}filtration * digestion{-}efficiency)}
\NormalTok{  let neighbors{-}list sort (neighbors with [patch{-}terrain = "water"])}
\NormalTok{  let best{-}patch nobody}
\NormalTok{  let best{-}score {-}999}

\NormalTok{  foreach neighbors{-}list [p {-}\textgreater{}}
\NormalTok{    let spm{-}patch [SPM] of p}
\NormalTok{    let interest exp ({-} ((spm{-}patch {-} optimal{-}SPM) \^{} 2) / (2 * (spm{-}sd \^{} 2)))}
\NormalTok{    let benefit (spm{-}patch * interest * base{-}filtration)}
\NormalTok{    let comp count alewives{-}on p}
\NormalTok{    if comp \textgreater{} 0 [ set benefit benefit / comp ]}
\NormalTok{    if benefit \textgreater{} best{-}score [}
\NormalTok{      set best{-}score benefit}
\NormalTok{      set best{-}patch p}
\NormalTok{    ]}
\NormalTok{  ]}

\NormalTok{  if best{-}patch != nobody and [patch{-}terrain] of best{-}patch = "water" [}
\NormalTok{    face best{-}patch}
\NormalTok{    face best{-}patch}
\NormalTok{    move{-}to best{-}patch}
      
\NormalTok{    ;; calculate ditance between best patch and previous patch }
\NormalTok{    let travel{-}distance distancexy previous{-}x previous{-}y}
      
\NormalTok{    ;; {-}{-}{-} Swimming energy cost (scaled by efficiency) {-}{-}{-}}
\NormalTok{    set energy max list 0 (energy {-} (E{-}swim * travel{-}distance))}

\NormalTok{    ;; {-}{-}{-} Feeding and digestion (temperature{-}dependent) {-}{-}{-}}
\NormalTok{    let patch{-}spm [SPM] of best{-}patch}
\NormalTok{    let prey{-}assimilated patch{-}spm * base{-}filtration * 300}
      
\NormalTok{    ;; {-}{-}{-} Contaminant uptake (SPM × Hg and MeHg) {-}{-}{-}}
\NormalTok{    let spm{-}hg [mercury] of best{-}patch}
\NormalTok{    let spm{-}mehg [methylmercury] of best{-}patch}
      
\NormalTok{    ;; {-}{-}{-} Biomagnification {-}{-}{-}}
\NormalTok{    let biomag{-}factor (random{-}float 30) + 60  ;; random 60–90}
\NormalTok{    let hg{-}assimilated (patch{-}spm * spm{-}mehg * biomag{-}factor * prey{-}assimilated)}
\NormalTok{    let mehg{-}assimilated (patch{-}spm * spm{-}hg * (biomag{-}factor / 10) * prey{-}assimilated)}
      
\NormalTok{    ;; Remaining capacity}
\NormalTok{    let space{-}available (stomach{-}capacity {-} stomach{-}contents)}

\NormalTok{    ;; You can only eat as much as fits}
\NormalTok{    let intake (min (list prey{-}assimilated space{-}available))}

\NormalTok{    ;; Add biomass to stomach}
\NormalTok{    set stomach{-}contents (stomach{-}contents + intake)}
\NormalTok{    set stomach{-}contents{-}Hg (stomach{-}contents{-}Hg + hg{-}assimilated);; mercury stomach amount}
\NormalTok{    set stomach{-}contents{-}MeHg (stomach{-}contents{-}MeHg + mehg{-}assimilated);; methylmercury stomach amount}

\NormalTok{  ]}
\NormalTok{end}
\end{Highlighting}
\end{Shaded}

\section{Overview}\label{overview-6}

Lipid catabolism provides a secondary energy pathway that fish use when immediate food resources are unavailable or when energy falls below a functional threshold. In this submodel, agents metabolize stored lipid reserves to restore energy, maintain migration, and avoid starvation. Lipid use is influenced by body condition, metabolic efficiency, and the magnitude of energetic deficit. Lipid depletion also reduces body weight over time, creating a biologically meaningful cost to sustained lipid reliance. Contaminant uptake does not occur during lipid catabolism, reflecting the absence of feeding.

\section{Purpose}\label{purpose-5}

The purpose of this submodel is to simulate endogenous energy mobilization through lipid reserves when fish cannot meet energetic needs through feeding. Lipid catabolism allows migratory fish to continue movement and survive temporary periods of low food availability. This mechanism reflects empirical observations of lipid use during spawning migrations, overwintering, and other energetically demanding life history periods.

This submodel is best used in tandpm with an energy recovery stratedge like filter feelding and digestion, or pref=dation and digesiton

\section{Entities, State Variables, and Scales}\label{entities-state-variables-and-scales-6}

\subsection{Spatial and Temporal Scales}\label{spatial-and-temporal-scales-6}

\textbf{Spatial Unit:} Patch (3 m x 3 m resolution)\\
\textbf{Temporal Unit:} 5 minute time steps (tick)

\subsection{\texorpdfstring{\textbf{Global Variables}}{Global Variables}}\label{global-variables-6}

\begin{longtable}[]{@{}
  >{\raggedright\arraybackslash}p{(\linewidth - 2\tabcolsep) * \real{0.5000}}
  >{\raggedright\arraybackslash}p{(\linewidth - 2\tabcolsep) * \real{0.5000}}@{}}
\toprule\noalign{}
\begin{minipage}[b]{\linewidth}\raggedright
Global Variable
\end{minipage} & \begin{minipage}[b]{\linewidth}\raggedright
Definition
\end{minipage} \\
\midrule\noalign{}
\endhead
\bottomrule\noalign{}
\endlastfoot
\textbf{full-energy-level} & Target energy level used to compute energetic deficit during lipid use (default = 100). \\
\textbf{energy-per-lipid} & Energy returned per unit lipid mass mobilized (default = 40), used to convert lipid into energy. \\
\textbf{lipid-refill-fraction} & Fraction of body mass available to mobilize as lipid during fasting (default = 0.5). \\
\textbf{lipid-max-fraction} \emph{(optional)} & Upper bound on lipid fraction that can be lost before mortality or physiological collapse (not currently enforced but biologically relevant). \\
\textbf{debug-lipid?} \emph{(optional)} & Toggles printing of intermediate lipid catabolism calculations for debugging. \\
\end{longtable}

\subsection{Patch Variables}\label{patch-variables-6}

Lipid catabolism does not directly depend on patch level variables, since no external resources are consumed.

\subsection{Agent Variables}\label{agent-variables-6}

\begin{longtable}[]{@{}
  >{\raggedright\arraybackslash}p{(\linewidth - 2\tabcolsep) * \real{0.5000}}
  >{\raggedright\arraybackslash}p{(\linewidth - 2\tabcolsep) * \real{0.5000}}@{}}
\toprule\noalign{}
\begin{minipage}[b]{\linewidth}\raggedright
Variable Name
\end{minipage} & \begin{minipage}[b]{\linewidth}\raggedright
Definition
\end{minipage} \\
\midrule\noalign{}
\endhead
\bottomrule\noalign{}
\endlastfoot
\textbf{energy} & Current energy level of the agent \\
\textbf{weight} & Current body mass of the fish \\
\textbf{original-weight} & Initial body mass used to scale maximum lipid availability \\
\textbf{lipid-catabolism-efficiency} & Efficiency with which lipids are converted into usable energy. This value is equal to the current metabolic rate divided by the metabolic rate at optimal temperature, allowing lipid use to scale with physiological performance \\
\textbf{lipid-used} & Amount of lipid converted to energy during the current step \\
\textbf{metabolism-rate} & Current metabolic rate, which influences efficiency scaling \\
baseline-metabolism-rate & Metabolic rate at optimal temperature \\
\textbf{mehg-foraging} & Methylmercury acquired during feeding (set to zero when lipids are used) \\
\textbf{hg-foraging} & Inorganic mercury acquired during feeding (set to zero when lipids are used) \\
\textbf{mehg-foraging-total} & Total methylmercury obtained from foraging \\
\textbf{hg-foraging-total} & Total inorganic mercury obtained from foraging \\
\textbf{mehg-total} & Total methylmercury accumulated \\
\textbf{hg-total} & Total inorganic mercury accumulated \\
\end{longtable}

\section{Process Overview and Scheduling}\label{process-overview-and-scheduling-6}

\begin{enumerate}
\def\labelenumi{\arabic{enumi}.}
\tightlist
\item
  Compute energy deficit relative to a target energy level of one hundred.\\
\item
  Determine the lipid amount needed to fill the deficit.\\
\item
  Limit lipid use by the maximum lipid available for mobilization.\\
\item
  Apply lipid catabolism efficiency to determine usable lipid energy.\\
\item
  Convert mobilized lipid to energy.\\
\item
  Reduce weight by the amount of lipid burned.\\
\item
  Set contaminant uptake to zero during this process, since no feeding occurs.
\end{enumerate}

\section{Design Concepts}\label{design-concepts-6}

\textbf{Basic Principles}\\
Lipid catabolism models endogenous energy mobilization in the absence of food. Fish draw upon lipid stores when energy is low, which is a well documented physiological process during migration, overwintering, and fasting.

\textbf{Emergence}\\
Patterns of weight loss, energy cycling, and recovery emerge from repeated lipid use during periods of limited food access.

\textbf{Adaptation}\\
Agents adaptively mobilize lipids based on the severity of their energy deficit and on efficiency scaling from the metabolism submodel.

\textbf{Objectives}\\
Agents seek to maintain enough energy to migrate, forage, avoid predation, and prevent starvation.

\textbf{Sensing}\\
Agents sense their own internal energy level and lipid reserve availability.

\textbf{Interaction}\\
Lipid catabolism does not require environmental interaction and therefore represents a purely endogenous process.

\textbf{Stochasticity}\\
This submodel is deterministic. However, variation in weight, metabolism, and previous feeding creates emergent differences among individuals.

\textbf{Observation}\\
Outputs include energy restored through lipid catabolism, lipid used, and body mass loss over time.

\section{Initialization}\label{initialization-6}

\begin{longtable}[]{@{}
  >{\raggedright\arraybackslash}p{(\linewidth - 4\tabcolsep) * \real{0.3333}}
  >{\raggedright\arraybackslash}p{(\linewidth - 4\tabcolsep) * \real{0.3333}}
  >{\raggedright\arraybackslash}p{(\linewidth - 4\tabcolsep) * \real{0.3333}}@{}}
\toprule\noalign{}
\begin{minipage}[b]{\linewidth}\raggedright
Variable
\end{minipage} & \begin{minipage}[b]{\linewidth}\raggedright
Initialized Value
\end{minipage} & \begin{minipage}[b]{\linewidth}\raggedright
Justification
\end{minipage} \\
\midrule\noalign{}
\endhead
\bottomrule\noalign{}
\endlastfoot
weight & species specific & Establishes initial body condition \\
original-weight & equal to initial weight & Defines total lipid availability \\
energy & 100 & Full energy at initialization \\
lipid-catabolism-efficiency & from metabolism submodel & Reflects physiological scaling of lipid use \\
lipid-used & 0 & No lipid used at start \\
\end{longtable}

\section{Submodels}\label{submodels-5}

\subsection{Lipid Mobilization}\label{lipid-mobilization}

Maximum lipid available for mobilization depends on relative body mass:

\[
Lipid_{refill} = 0.5 \times \left(\frac{weight}{original\_weight}\right)
\]

Energy deficit relative to full energy:

\[
Energy_{deficit} = 100 - Energy
\]

Lipid required to refill the deficit:

\[
Lipid_{needed} = \frac{Energy_{deficit}}{40}
\]

Actual lipid mobilized:

\[
Lipid_{used} = \min(Lipid_{needed},\ Lipid_{refill}) \times lipid\_{catabolism\_efficiency}
\]

\subsection{Energy Conversion}\label{energy-conversion-1}

Mobilized lipid is converted into energy:

\[
Energy_{added} = Lipid_{used} \times 40
\]

Energy is capped at one hundred:

\[
Energy = \min(100,\ Energy + Energy_{added})
\]

\subsection{Weight Loss}\label{weight-loss}

Weight decreases proportionally to lipid used:

\[
Weight = Weight - Lipid_{used}
\]

\subsection{Contaminant Bookkeeping}\label{contaminant-bookkeeping}

No mercury or methylmercury is acquired during lipid catabolism:

\[
MeHg_{digested} = 0
\]

\[
Hg_{digested} = 0
\]

These values are added to cumulative totals to maintain bookkeeping consistency, but do not increase contaminant burden.

\section{Netlogo Implementation}\label{netlogo-implementation-3}

\begin{Shaded}
\begin{Highlighting}[]

\NormalTok{;; ===================================================================}
\NormalTok{;;  LIPID CATABOLISM SUBMODEL DECLARATIONS}
\NormalTok{;; ===================================================================}

\NormalTok{turtles{-}own [}

\NormalTok{  ;; Energetics \& body condition}
\NormalTok{  energy}
\NormalTok{  weight}
\NormalTok{  original{-}weight}

\NormalTok{  ;; Lipid metabolism}
\NormalTok{  lipid{-}catabolism{-}efficiency}
\NormalTok{  lipid{-}used}

\NormalTok{  ;; Metabolic rate inputs}
\NormalTok{  metabolism{-}rate}
\NormalTok{  baseline{-}metabolism{-}rate}

\NormalTok{  ;; Contaminant bookkeeping during fasting}
\NormalTok{  mehg{-}foraging}
\NormalTok{  hg{-}foraging}
  
\NormalTok{  mehg{-}foraging{-}total}
\NormalTok{  hg{-}foraging{-}total}
  
\NormalTok{  mehg{-}total}
\NormalTok{  hg{-}total}
\NormalTok{]}

\NormalTok{;; ===================================================================}
\NormalTok{;;  LIPID CATABOLISM SUBMODEL }
\NormalTok{;; ===================================================================}

\NormalTok{ to lipid{-}loss}
\NormalTok{   ;; Maximum lipid metabolized this step, adjusted by efficiency}
\NormalTok{  let lipid{-}refill 0.5 * (weight / original{-}weight)}
\NormalTok{  ;print (word "lipid{-}refill: " lipid{-}refill)}
\NormalTok{  ;print (word "current energy: " energy)}
\NormalTok{  ;; Calculate deficit and lipid need}
\NormalTok{  let energy{-}deficit (100 {-} energy)}
\NormalTok{  ;print (word "energy{-}deficit: " energy{-}deficit)}
\NormalTok{  let lipid{-}needed (energy{-}deficit / 40)}
\NormalTok{  ;print (word "lipid{-}needed: " lipid{-}needed)}
  
\NormalTok{  ;; Actual lipid use depends on efficiency and availability}
\NormalTok{  let lipid{-}used (min list lipid{-}needed lipid{-}refill)}
\NormalTok{  set lipid{-}used (lipid{-}used * lipid{-}catabolism{-}efficiency)}
\NormalTok{  ;print (word "lipid{-}used: " lipid{-}used)}
  
\NormalTok{  ;; Convert lipid to energy (scaled by efficiency)}
\NormalTok{  let energy{-}change (lipid{-}used * 40)}
\NormalTok{  ;print (word "energy added: " energy{-}change)}
\NormalTok{  set energy (energy + energy{-}change)}
\NormalTok{  ;print (word "energy after lipid conversion: " energy)}
\NormalTok{  if energy \textgreater{} 100 [ }
\NormalTok{    set energy 100 }
\NormalTok{    print "energy capped at 100"}
\NormalTok{  ]}
  
\NormalTok{  ;; Weight loss proportional to lipid burned}
\NormalTok{  set weight weight {-} (lipid{-}used)}
\NormalTok{  ;print (word "weight after lipid loss: " weight)}
  
\NormalTok{  ;; When relying on lipids, Hg mercury foraging risk stops}
    
\NormalTok{  ;; contaminant bookkeeping}
\NormalTok{  let digested{-}MeHg 0}
\NormalTok{  let digested{-}Hg 0}
\NormalTok{  set mehg{-}foraging digested{-}MeHg}
\NormalTok{  set hg{-}foraging digested{-}Hg}

\NormalTok{  set mehg{-}foraging{-}total mehg{-}foraging{-}total + digested{-}MeHg}
\NormalTok{  set hg{-}foraging{-}total hg{-}foraging{-}total + digested{-}Hg}
\NormalTok{  set mehg{-}total mehg{-}total + digested{-}MeHg}
\NormalTok{  set hg{-}total hg{-}total + digested{-}Hg}
\NormalTok{ end}
 
\end{Highlighting}
\end{Shaded}

\chapter{Landward Migration}\label{landward-migration}

\section{Overview}\label{overview-7}

This function simulates landward migration for fish moving upstream through riverine and estuarine systems. Agents adjust their movement based on local water velocity, energetic condition, and the relative difficulty of swimming in a given flow. Movement occurs along a least cost path that is recalculated according to hydrodynamic conditions.

\section{Purpose}\label{purpose-6}

The purpose of this submodel is to represent upstream migration under realistic spatial hydrodynamic resistance. Agents evaluate flow velocity, movement difficulty, body condition, and patch level travel costs before advancing along a least cost path toward a home location.

\section{Entities, State Variables, and Scales}\label{entities-state-variables-and-scales-7}

\subsection{Spatial and Temporal Scales}\label{spatial-and-temporal-scales-7}

\textbf{Spatial Unit:} Patch (3 m x 3 m resolution)\\
\textbf{Temporal Unit:} 5 minute time steps (tick)

\subsection{Global Variables}\label{global-variables-7}

\begin{longtable}[]{@{}
  >{\centering\arraybackslash}p{(\linewidth - 2\tabcolsep) * \real{0.5000}}
  >{\centering\arraybackslash}p{(\linewidth - 2\tabcolsep) * \real{0.5000}}@{}}
\toprule\noalign{}
\begin{minipage}[b]{\linewidth}\centering
Variable
\end{minipage} & \begin{minipage}[b]{\linewidth}\centering
Definition
\end{minipage} \\
\midrule\noalign{}
\endhead
\bottomrule\noalign{}
\endlastfoot
\textbf{max-seaward-velocity} & Maximum seaward directed velocity in the domain, used for normalization. \\
\textbf{max-landward-velocity} & Maximum landward directed velocity in the domain. \\
\end{longtable}

\subsection{Patch Variables}\label{patch-variables-7}

\begin{longtable}[]{@{}
  >{\centering\arraybackslash}p{(\linewidth - 2\tabcolsep) * \real{0.5000}}
  >{\centering\arraybackslash}p{(\linewidth - 2\tabcolsep) * \real{0.5000}}@{}}
\toprule\noalign{}
\begin{minipage}[b]{\linewidth}\centering
Variable
\end{minipage} & \begin{minipage}[b]{\linewidth}\centering
Definition
\end{minipage} \\
\midrule\noalign{}
\endhead
\bottomrule\noalign{}
\endlastfoot
\textbf{velocity} & Depth averaged hydrodynamic velocity at the patch. \\
\textbf{depth} & Water depth at the patch. \\
\textbf{patch-terrain} & Indicates whether the patch is water or land. \\
\textbf{cost-to-home} & Patch level cost field producing a least cost corridor. \\
\textbf{visits-by-alewife} & Number of times the agent visits the patch. \\
\textbf{ticks-spent-alewife} & Time spent on the patch. \\
\end{longtable}

\subsection{Agent Variables}\label{agent-variables-7}

\begin{longtable}[]{@{}
  >{\centering\arraybackslash}p{(\linewidth - 2\tabcolsep) * \real{0.5000}}
  >{\centering\arraybackslash}p{(\linewidth - 2\tabcolsep) * \real{0.5000}}@{}}
\toprule\noalign{}
\begin{minipage}[b]{\linewidth}\centering
Variable
\end{minipage} & \begin{minipage}[b]{\linewidth}\centering
Definition
\end{minipage} \\
\midrule\noalign{}
\endhead
\bottomrule\noalign{}
\endlastfoot
\textbf{weight} & Body mass of the fish. \\
\textbf{speed} & Current swimming speed. \\
\textbf{prev-speed} & Previous time step swimming speed. \\
\textbf{max-speed} & Maximum sustained speed. \\
\textbf{min-speed} & Minimum allowable speed. \\
\textbf{swim-efficiency} & Scaling factor for acceleration and deceleration. \\
\textbf{difficulty-factor} & Difficulty of movement under local velocity conditions. \\
\textbf{planned-path} & Sequence of patches in the least cost route. \\
\textbf{home-patch} & Destination patch for migration. \\
\textbf{trail} & Record of patches visited by the agent. \\
\end{longtable}

\section{Process Overview and Scheduling}\label{process-overview-and-scheduling-7}

\begin{enumerate}
\def\labelenumi{\arabic{enumi}.}
\tightlist
\item
  Compute swimming difficulty from hydrodynamic velocity and body size.\\
\item
  Calculate swimming speed using desired-speed, velocity influence, and acceleration limits.\\
\item
  Evaluate the least cost field and construct a path to the home patch.\\
\item
  Move along the path for a distance proportional to swimming speed.\\
\item
  Record patch visitation and time spent.
\end{enumerate}

\section{Design Concepts}\label{design-concepts-7}

\textbf{Basic Principles}\\
Hydrodynamic constraints, difficulty scaling, and body condition influence agent movement.

\textbf{Emergence}\\
Migration routes, temporal occupancy, and movement speed emerge from repeated interactions of flow and difficulty.

\textbf{Objectives}\\
Agents attempt to reach the home patch but do not optimize globally. Movement follows the least cost route at each step.

\textbf{Sensing}\\
Agents sense velocity and depth of the local and neighboring patches.

\textbf{Observation}\\
Paths, timing, and residency can be monitored chronologically.

\section{Initialization}\label{initialization-7}

\begin{longtable}[]{@{}
  >{\centering\arraybackslash}p{(\linewidth - 4\tabcolsep) * \real{0.3333}}
  >{\centering\arraybackslash}p{(\linewidth - 4\tabcolsep) * \real{0.3333}}
  >{\centering\arraybackslash}p{(\linewidth - 4\tabcolsep) * \real{0.3333}}@{}}
\toprule\noalign{}
\begin{minipage}[b]{\linewidth}\centering
Variable
\end{minipage} & \begin{minipage}[b]{\linewidth}\centering
Initial Value
\end{minipage} & \begin{minipage}[b]{\linewidth}\centering
Justification
\end{minipage} \\
\midrule\noalign{}
\endhead
\bottomrule\noalign{}
\endlastfoot
\textbf{velocity} & user-defined & Hydrodynamic input data. \\
\textbf{weight} & species-specific & Required for difficulty scaling. \\
\textbf{max-speed} & species-specific & Physiological constraint. \\
\textbf{min-speed} & species-specific & Prevents speed collapse. \\
\textbf{swim-efficiency} & species or constant & Controls acceleration rate. \\
\textbf{difficulty-factor} & 1 & Neutral starting difficulty. \\
\textbf{prev-speed} & starting speed & Smooth initial movement. \\
\end{longtable}

\section{Submodels}\label{submodels-6}

\subsection{Swimming Difficulty}\label{swimming-difficulty}

Swimming difficulty reflects the hydrodynamic effort required for the agent to move through its current patch. Velocity is normalized between the maximum observed landward and seaward velocities:

\[
V_{\text{norm}} = \frac{V_{\text{patch}} - V_{\min}}{V_{\max} - V_{\min}}
\]

This normalized velocity is scaled by relative body size \(M_{\text{agent}} / M_{\max}\) and raised to a scaling exponent \(k\):

\[
Df_{\text{raw}} = 
\left(
\frac{V_{\text{norm}}}{M_{\text{agent}} / M_{\max}}
\right)^k
\]

Difficulty is then mapped onto the one to ten range:

\[
D_f = 1 + 9 \cdot Df_{\text{raw}}
\]

and constrained:

\[
D_f \in [1, 10]
\]

Difficulty increases when:

\begin{itemize}
\tightlist
\item
  \(M_{\text{agent}}\) is small\\
\item
  \(|V_{\text{patch}}|\) approaches its maximum value\\
\item
  flow opposes movement
\end{itemize}

\begin{center}\rule{0.5\linewidth}{0.5pt}\end{center}

\subsection{Swimming Speed}\label{swimming-speed}

Swimming speed depends on energetic condition, hydrodynamic difficulty, and the effect of local flow. The agent first computes an energy factor:

\[
E_{\text{factor}} = \frac{E_{\text{agent}}}{100}
\]

Velocity impact from hydrodynamics is:

\[
V_{\text{impact}} = k \cdot V_{\text{patch}} \cdot 300
\]

The desired swimming speed is:

\[
V_{\text{desired}} =
\frac{V_{\max} \cdot E_{\text{factor}}}{D_f}
+ V_{\text{impact}}
\]

Speed is constrained between a minimum and maximum:

\[
V_{\text{desired}} 
= \min(V_{\max}, \max(V_{\min}, V_{\text{desired}}))
\]

Acceleration is smoothed using the previous speed:

\[
\Delta V_{\max} = 0.5 \cdot \text{swim\_efficiency}
\]

If \(|V_{\text{desired}} - V_{\text{prev}}| > \Delta V_{\max}\), then speed moves gradually toward the desired value. Otherwise speed is set directly.

\subsection{Path Based Movement}\label{path-based-movement}

Landward migration uses a least cost routing strategy instead of direct vector movement. The number of patches traversed in a tick is based on swimming speed:

\[
\text{travel\_distance} = \frac{V_{\text{current}}}{3}
\]

Movement follows:

\begin{enumerate}
\def\labelenumi{\arabic{enumi}.}
\tightlist
\item
  compute cost-to-home\\
\item
  generate a path using decreasing cost\\
\item
  move through the first \(\lceil \text{travel\_distance} \rceil\) patches\\
\item
  record patch visits and time spent\\
\item
  append patch-here to the trail if new
\end{enumerate}

\subsection{Travel Cost}\label{travel-cost}

Travel cost for a candidate patch incorporates velocity magnitude, slope, and difficulty:

\[
\text{cost}_{\text{raw}}
= |V_{\text{patch}}|
+ 0.5 \cdot \max(0, \Delta \text{depth})
\]

Difficulty multiplies the base cost:

\[
\text{cost}
= \text{cost}_{\text{raw}} \cdot D_f
\]

This produces a cost-to-home field representing hydrodynamic resistance.

\subsection{Swimming Energy}\label{swimming-energy}

Swimming incurs a metabolic cost that scales with base swimming cost, difficulty, and swim efficiency.

Let \(\beta = 0.75\).\\
Let the effective multiplier be:

\[
M = \frac{D_f}{\text{swim\_efficiency}}
\]
Where lower efficiency results in higher energetic cost.

Energy expenditure is:

\[
E_{\text{swim}} = \text{swim\_base} \cdot M^\beta
\]

Total energy is reduced accordingly:

\[
E_{\text{agent}} = \max(0, E_{\text{agent}} - E_{\text{swim}})
\]

Higher difficulty and lower swim efficiency increase metabolic cost.

\begin{Shaded}
\begin{Highlighting}[]
\NormalTok{;; ===================================================================}
\NormalTok{;;  LANDWARD MIGRATION + HYDRODYNAMIC SWIMMING SUBMODEL DECLARATIONS}
\NormalTok{;; ===================================================================}

\NormalTok{globals [}

\NormalTok{  ;; Velocity normalization bounds}
\NormalTok{  max{-}seaward{-}velocity}
\NormalTok{  max{-}landward{-}velocity}
\NormalTok{]}

\NormalTok{patches{-}own [}

\NormalTok{  ;; Environmental attributes}
\NormalTok{  velocity}
\NormalTok{  depth}
\NormalTok{  patch{-}terrain   ;; "water" or "land"}

\NormalTok{  ;; Least{-}cost routing fields}
\NormalTok{  cost{-}to{-}home}
\NormalTok{  cost{-}to{-}sea}
\NormalTok{  cost{-}to{-}prey}

\NormalTok{  ;; Tracking fish patch{-}level behavior}
\NormalTok{  visits{-}by{-}fish}
\NormalTok{  ticks{-}spent{-}fish}
\NormalTok{]}

\NormalTok{turtles{-}own [}

\NormalTok{  ;; Body condition}
\NormalTok{  weight}

\NormalTok{  ;; Swimming behavior}
\NormalTok{  speed}
\NormalTok{  prev{-}speed}
\NormalTok{  max{-}speed}
\NormalTok{  min{-}speed}
\NormalTok{  swim{-}efficiency}

\NormalTok{  ;; Hydrodynamic resistance}
\NormalTok{  difficulty{-}factor}

\NormalTok{  ;; Energetics}
\NormalTok{  E{-}swim}
\NormalTok{  swim{-}base}

\NormalTok{  ;; Migration targets}
\NormalTok{  home{-}patch}
\NormalTok{  migration{-}patch}

\NormalTok{  ;; Movement memory}
\NormalTok{  planned{-}path}
\NormalTok{  trail}
\NormalTok{  previous{-}patch}
\NormalTok{  previous{-}x}
\NormalTok{  previous{-}y}
\NormalTok{]}


\NormalTok{;; ===================================================================}
\NormalTok{;;  SWIMMING DIFFICULTY (LANDWARD)}
\NormalTok{;; ===================================================================}

\NormalTok{to calculate{-}difficulty{-}landward}
  
\NormalTok{  let M{-}max max [weight] of breed      ;; largest fish in population}
\NormalTok{  let M{-}agent weight                   ;; focal individual}
\NormalTok{  let V{-}max max{-}seaward{-}velocity}
\NormalTok{  let V{-}min max{-}landward{-}velocity}
\NormalTok{  let k 0.75                           ;; scaling exponent}

\NormalTok{  ;; Normalize velocity}
\NormalTok{  let normalized{-}velocity (velocity {-} V{-}min) / (V{-}max {-} V{-}min)}

\NormalTok{  ;; Difficulty from normalized velocity + size ratio}
\NormalTok{  let Df{-}raw (normalized{-}velocity / (M{-}agent / M{-}max)) \^{} k}
  
\NormalTok{  ;; Scale to range 1–10}
\NormalTok{  set difficulty{-}factor (1 + 9 * Df{-}raw)}
\NormalTok{  set difficulty{-}factor max list 1 (min list 10 difficulty{-}factor)}
\NormalTok{end}


\NormalTok{;; ===================================================================}
\NormalTok{;;  SWIMMING SPEED (LANDWARD)}
\NormalTok{;; ===================================================================}

\NormalTok{to calculate{-}swimming{-}speed{-}landward}
  
\NormalTok{  let k {-}0.75                                    ;; velocity influence}
\NormalTok{  let energy{-}factor (energy / 100)               ;; 0–1 scale}
\NormalTok{  let velocity{-}impact (k * velocity * 300)       ;; 5{-}min timestep scaling}

\NormalTok{  ;; Preliminary target speed}
\NormalTok{  let desired{-}speed }
\NormalTok{    (max{-}speed * energy{-}factor / difficulty{-}factor)}
\NormalTok{      + velocity{-}impact}

\NormalTok{  ;; Clamp desired speed to biologically realistic bounds}
\NormalTok{  set desired{-}speed min list max{-}speed (max list min{-}speed desired{-}speed)}

\NormalTok{  ;; Smooth acceleration / deceleration}
\NormalTok{  let max{-}speed{-}change 0.5}
\NormalTok{  set max{-}speed{-}change (max{-}speed{-}change * swim{-}efficiency)}

\NormalTok{  if abs(desired{-}speed {-} prev{-}speed) \textgreater{} max{-}speed{-}change [}
\NormalTok{    if desired{-}speed \textgreater{} prev{-}speed [}
\NormalTok{      set speed prev{-}speed + max{-}speed{-}change}
\NormalTok{    ]}
\NormalTok{    if desired{-}speed \textless{} prev{-}speed [}
\NormalTok{      set speed prev{-}speed {-} max{-}speed{-}change}
\NormalTok{    ]}
\NormalTok{  ]}

\NormalTok{  if abs(desired{-}speed {-} prev{-}speed) \textless{}= max{-}speed{-}change [}
\NormalTok{    set speed desired{-}speed}
\NormalTok{  ]}

\NormalTok{  set prev{-}speed speed}
\NormalTok{end}


\NormalTok{;; ===================================================================}
\NormalTok{;;  LEAST{-}COST ROUTING TO HOME (LANDWARD)}
\NormalTok{;; ===================================================================}

\NormalTok{to calculate{-}cost{-}to{-}home [homing{-}patch]}

\NormalTok{  ;; initialize routing fields}
\NormalTok{  ask home{-}patch  [ set cost{-}to{-}home 0 ]}
\NormalTok{  ask patch{-}here  [ set cost{-}to{-}home 1e9 ]}

\NormalTok{  let frontier (list home{-}patch)}
\NormalTok{  let df difficulty{-}factor                 ;; hydrodynamic penalty}

\NormalTok{  while [not empty? frontier] [}
\NormalTok{    let current first frontier}
\NormalTok{    set frontier but{-}first frontier}

\NormalTok{    ask current [}
\NormalTok{      let current{-}cost cost{-}to{-}home}
      
\NormalTok{      ask neighbors4 with [patch{-}terrain = "water"] [}
        
\NormalTok{        let travel{-}cost compute{-}travel{-}cost current self df}
\NormalTok{        let new{-}cost current{-}cost + travel{-}cost}

\NormalTok{        if new{-}cost \textless{} cost{-}to{-}home [}
\NormalTok{          set cost{-}to{-}home new{-}cost}
\NormalTok{          set frontier lput self frontier}
\NormalTok{        ]}
\NormalTok{      ]}
\NormalTok{    ]}
\NormalTok{  ]}
\NormalTok{end}


\NormalTok{;; ===================================================================}
\NormalTok{;;  BUILD OPTIMAL PATH ENTIRELY FROM COST{-}FIELD}
\NormalTok{;; ===================================================================}

\NormalTok{to{-}report build{-}path [current{-}patch homing{-}patch]}

\NormalTok{  let path (list current{-}patch)}
\NormalTok{  let cur current{-}patch}
\NormalTok{  let max{-}steps 2000}

\NormalTok{  while [cur != homing{-}patch and max{-}steps \textgreater{} 0] [}
\NormalTok{    set max{-}steps max{-}steps {-} 1}

\NormalTok{    let candidates [neighbors4] of cur}
\NormalTok{    set candidates candidates with [}
\NormalTok{      patch{-}terrain = "water" and}
\NormalTok{      cost{-}to{-}home \textless{} [cost{-}to{-}home] of cur}
\NormalTok{    ]}

\NormalTok{    if not any? candidates [}
\NormalTok{      report path}
\NormalTok{    ]}

\NormalTok{    let next{-}patch min{-}one{-}of candidates [cost{-}to{-}home]}
\NormalTok{    set path lput next{-}patch path}
\NormalTok{    set cur next{-}patch}
\NormalTok{  ]}

\NormalTok{  report path}
\NormalTok{end}


\NormalTok{;; ===================================================================}
\NormalTok{;;  TRAVEL COST UNDER HYDRODYNAMIC RESISTANCE}
\NormalTok{;; ===================================================================}

\NormalTok{to{-}report compute{-}travel{-}cost [from{-}patch to{-}patch df]}

\NormalTok{  ;; velocity resistance}
\NormalTok{  let raw{-}vel [velocity] of to{-}patch}
\NormalTok{  let velocity{-}resistance }
\NormalTok{      ifelse{-}value is{-}number? raw{-}vel [abs raw{-}vel] [0]}

\NormalTok{  ;; depth slope component}
\NormalTok{  let from{-}level [depth] of from{-}patch}
\NormalTok{  let to{-}level   [depth] of to{-}patch}
\NormalTok{  let slope 0}

\NormalTok{  if (is{-}number? from{-}level and is{-}number? to{-}level) [}
\NormalTok{    set slope max list 0 (to{-}level {-} from{-}level)}
\NormalTok{  ]}

\NormalTok{  ;; hydrodynamic cost + slope penalty}
\NormalTok{  let base{-}cost velocity{-}resistance + (0.5 * slope)}

\NormalTok{  ;; difficulty multiplier}
\NormalTok{  let df2 ifelse{-}value is{-}number? df [df] [1]}

\NormalTok{  report max list 0 (base{-}cost * df2)}
\NormalTok{end}


\NormalTok{;; ===================================================================}
\NormalTok{;;  LANDWARD MIGRATION — PATH EXECUTION}
\NormalTok{;; ===================================================================}

\NormalTok{to migrate{-}landward}
  
\NormalTok{  let travel{-}distance (speed / 3)}
\NormalTok{  let path build{-}path patch{-}here home{-}patch}
\NormalTok{  set planned{-}path path}
  
\NormalTok{  if not empty? path [}

\NormalTok{    let steps{-}to{-}move ceiling travel{-}distance}
\NormalTok{    let move{-}patches sublist path 0 (min list steps{-}to{-}move length path)}
    
\NormalTok{    let old{-}patch patch{-}here}

\NormalTok{    foreach move{-}patches [ p {-}\textgreater{}}

\NormalTok{      ;; Record patch visitation}
\NormalTok{      if p != old{-}patch [}
\NormalTok{        ask p [ set visits{-}by{-}alewife visits{-}by{-}alewife + 1 ]}
\NormalTok{        set old{-}patch p}
\NormalTok{      ]}

\NormalTok{      ;; Move into patch}
\NormalTok{      face p}
\NormalTok{      move{-}to p}

\NormalTok{      ask p [}
\NormalTok{        set ticks{-}spent{-}alewife ticks{-}spent{-}alewife + 1}
\NormalTok{      ]}
\NormalTok{    ]}
\NormalTok{  ]}

\NormalTok{  ;; Add patch to historical trail}
\NormalTok{  if not member? patch{-}here trail [}
\NormalTok{    set trail lput patch{-}here trail}
\NormalTok{  ]}
\NormalTok{end}


\NormalTok{;; ===================================================================}
\NormalTok{;;  UNIVERSAL SWIMMING ENERGY COST}
\NormalTok{;; ===================================================================}

\NormalTok{to calculate{-}swim{-}energy}
  
\NormalTok{  let beta 0.75}

\NormalTok{  let energy{-}multiplier }
\NormalTok{      (difficulty{-}factor * (1 / swim{-}efficiency))}

\NormalTok{  set E{-}swim }
\NormalTok{      (swim{-}base * (energy{-}multiplier \^{} beta))}

\NormalTok{  set energy max list 0 (energy {-} E{-}swim)}
\NormalTok{end}
\end{Highlighting}
\end{Shaded}

\chapter{Seaward Migration}\label{seaward-migration}

\section{Overview}\label{overview-8}

This function simulates seaward migration for fish moving downstream through riverine and estuarine systems. Agents respond to hydrodynamic conditions by adjusting swimming difficulty, swimming speed, and movement along a least cost pathway toward a downstream migration patch.

\section{Purpose}\label{purpose-7}

The purpose of this submodel is to represent downstream migration under realistic hydrodynamic resistance. Agents evaluate flow velocity, movement difficulty, energetic condition, and travel costs before advancing along a least cost route toward a migration destination.

\section{Entities, State Variables, and Scales}\label{entities-state-variables-and-scales-8}

\subsection{Spatial and Temporal Scales}\label{spatial-and-temporal-scales-8}

\textbf{Spatial Unit:} Patch (3 m x 3 m resolution)\\
\textbf{Temporal Unit:} 5 minute time steps (tick)

\subsection{Global Variables}\label{global-variables-8}

\begin{longtable}[]{@{}
  >{\centering\arraybackslash}p{(\linewidth - 2\tabcolsep) * \real{0.5000}}
  >{\centering\arraybackslash}p{(\linewidth - 2\tabcolsep) * \real{0.5000}}@{}}
\toprule\noalign{}
\begin{minipage}[b]{\linewidth}\centering
Variable
\end{minipage} & \begin{minipage}[b]{\linewidth}\centering
Definition
\end{minipage} \\
\midrule\noalign{}
\endhead
\bottomrule\noalign{}
\endlastfoot
\textbf{max-seaward-velocity} & Maximum observed seaward velocity in the domain. \\
\textbf{max-landward-velocity} & Maximum observed landward velocity in the domain. \\
\end{longtable}

\subsection{Patch Variables}\label{patch-variables-8}

\begin{longtable}[]{@{}cc@{}}
\toprule\noalign{}
Variable & Definition \\
\midrule\noalign{}
\endhead
\bottomrule\noalign{}
\endlastfoot
\textbf{velocity} & Depth averaged hydrodynamic velocity. \\
\textbf{depth} & Water depth at the patch. \\
\textbf{patch-terrain} & Indicates whether the patch is land or water. \\
\textbf{cost-to-sea} & Patch-level travel cost for seaward movement. \\
\textbf{visits-by-alewife} & Number of times the agent visits the patch. \\
\textbf{ticks-spent-alewife} & Time spent on the patch. \\
\end{longtable}

\subsection{Agent Variables}\label{agent-variables-8}

\begin{longtable}[]{@{}cc@{}}
\toprule\noalign{}
Variable & Definition \\
\midrule\noalign{}
\endhead
\bottomrule\noalign{}
\endlastfoot
\textbf{weight} & Body mass of the fish. \\
\textbf{speed} & Current swimming speed. \\
\textbf{prev-speed} & Previous time step swimming speed. \\
\textbf{max-speed} & Maximum sustained speed. \\
\textbf{min-speed} & Minimum allowable speed. \\
\textbf{swim-efficiency} & Swimming acceleration and energy scaling. \\
\textbf{difficulty-factor} & Hydrodynamic difficulty. \\
\textbf{planned-path} & Sequence of patches in least cost route. \\
\textbf{migration-patch} & Destination patch. \\
\textbf{trail} & Record of visited patches. \\
\end{longtable}

\section{Process Overview and Scheduling}\label{process-overview-and-scheduling-8}

\begin{enumerate}
\def\labelenumi{\arabic{enumi}.}
\tightlist
\item
  Calculate swimming difficulty using normalized velocity and body size.\\
\item
  Update swimming speed using energy reserves, difficulty, and flow velocity.\\
\item
  Calculate cost-to-sea and construct least cost path.\\
\item
  Move along the first part of the path based on swimming speed.\\
\item
  Record patch visits and time spent.
\end{enumerate}

\section{Design Concepts}\label{design-concepts-8}

\textbf{Basic Principles}\\
Hydrodynamic forcing, body size, and energy availability determine downstream movement.

\textbf{Emergence}\\
Route choice, travel duration, and energetic depletion emerge from repeated state updates.

\textbf{Objectives}\\
Agents attempt to reach the seaward migration patch following a hydrodynamically efficient route.

\textbf{Sensing}\\
Agents sense local velocity, slope, difficulty, and patch cost gradients.

\textbf{Observation}\\
Movement trajectories, energy decline, and convergence routes can be monitored.

\section{Initialization}\label{initialization-8}

\begin{longtable}[]{@{}
  >{\centering\arraybackslash}p{(\linewidth - 4\tabcolsep) * \real{0.3333}}
  >{\centering\arraybackslash}p{(\linewidth - 4\tabcolsep) * \real{0.3333}}
  >{\centering\arraybackslash}p{(\linewidth - 4\tabcolsep) * \real{0.3333}}@{}}
\toprule\noalign{}
\begin{minipage}[b]{\linewidth}\centering
Variable
\end{minipage} & \begin{minipage}[b]{\linewidth}\centering
Initial Value
\end{minipage} & \begin{minipage}[b]{\linewidth}\centering
Justification
\end{minipage} \\
\midrule\noalign{}
\endhead
\bottomrule\noalign{}
\endlastfoot
\textbf{velocity} & Hydrodynamic input & Required environmental forcing. \\
\textbf{weight} & Species-specific & Required for difficulty scaling. \\
\textbf{max-speed} & Species-specific & Physiological constraint. \\
\textbf{min-speed} & Species-specific & Prevents collapse to zero speed. \\
\textbf{swim-efficiency} & Species or constant & Controls acceleration and deceleration. \\
\textbf{difficulty-factor} & 1 & Neutral start value. \\
\textbf{prev-speed} & Starting speed & Smooth initial movement. \\
\end{longtable}

\section{Submodels}\label{submodels-7}

\subsection{Swimming Difficulty}\label{swimming-difficulty-1}

Velocity is normalized between minimum and maximum observed values:

\[
V_{\text{norm}}
= 
\frac{V_{\text{patch}} - V_{\min}}
{V_{\max} - V_{\min}}
\]

Difficulty scales with body size:

\[
Df_{\text{raw}}
=
\left(
\frac{V_{\text{norm}}}
{M_{\text{agent}} / M_{\max}}
\right)^k
\]

Difficulty is mapped onto the one-to-ten range:

\[
D_f = 1 + 9 \cdot Df_{\text{raw}}
\]

and constrained:

\[
D_f \in [1, 10]
\]

\subsection{Swimming Speed}\label{swimming-speed-1}

Energy factor:

\[
E_{\text{factor}} = \frac{E_{\text{agent}}}{100}
\]

Flow velocity effect:

\[
V_{\text{impact}} = k \cdot V_{\text{patch}} \cdot 300
\]

Desired swimming speed:

\[
V_{\text{desired}}
=
\frac{V_{\max} \cdot E_{\text{factor}}}{D_f}
+
V_{\text{impact}}
\]

Speed constraints:

\[
V_{\text{desired}}
=
\min(V_{\max}, \max(V_{\min}, V_{\text{desired}}))
\]

Acceleration smoothing:

\[
\Delta V_{\max} = 0.5 \cdot \text{swim\_efficiency}
\]

\subsection{Path-Based Movement}\label{path-based-movement-1}

Travel distance:

\[
\text{travel\_distance}
=
\frac{V_{\text{current}}}{3}
\]

Movement follows:

\begin{enumerate}
\def\labelenumi{\arabic{enumi}.}
\tightlist
\item
  compute cost-to-sea\\
\item
  build path using decreasing cost\\
\item
  traverse first \(\lceil \text{travel\_distance} \rceil\) patches\\
\item
  record visitation and time spent\\
\item
  append to the movement trail
\end{enumerate}

\subsection{Travel Cost}\label{travel-cost-1}

Raw cost:

\[
\text{cost}_{\text{raw}}
=
|V_{\text{patch}}|
+
0.5 \cdot \max(0, \Delta \text{depth})
\]

Difficulty scaling:

\[
\text{cost}
=
\text{cost}_{\text{raw}} \cdot D_f
\]

\subsection{Swimming Energy}\label{swimming-energy-1}

Let \(\beta = 0.75\).

Energy multiplier:

\[
M = \frac{D_f}{\text{swim\_efficiency}}
\]

Energy cost:

\[
E_{\text{swim}}
=
\text{swim\_base} \cdot M^{\beta}
\]

Energy update:

\[
E_{\text{agent}}
=
\max \left(0,\, E_{\text{agent}} - E_{\text{swim}} \right)
\]

\section{NetLogo Implementation}\label{netlogo-implementation-4}

\begin{Shaded}
\begin{Highlighting}[]

\NormalTok{;; ===================================================================}
\NormalTok{;;  LANDWARD MIGRATION + HYDRODYNAMIC SWIMMING SUBMODEL DECLARATIONS}
\NormalTok{;; ===================================================================}

\NormalTok{globals [}

\NormalTok{  ;; Velocity normalization bounds}
\NormalTok{  max{-}seaward{-}velocity}
\NormalTok{  max{-}landward{-}velocity}
\NormalTok{]}

\NormalTok{patches{-}own [}

\NormalTok{  ;; Environmental attributes}
\NormalTok{  velocity}
\NormalTok{  depth}
\NormalTok{  patch{-}terrain   ;; "water" or "land"}

\NormalTok{  ;; Least{-}cost routing fields}
\NormalTok{  cost{-}to{-}home}
\NormalTok{  cost{-}to{-}sea}
\NormalTok{  cost{-}to{-}prey}

\NormalTok{  ;; Tracking fish patch{-}level behavior}
\NormalTok{  visits{-}by{-}fish}
\NormalTok{  ticks{-}spent{-}fish}
\NormalTok{]}

\NormalTok{turtles{-}own [}

\NormalTok{  ;; Body condition}
\NormalTok{  weight}

\NormalTok{  ;; Swimming behavior}
\NormalTok{  speed}
\NormalTok{  prev{-}speed}
\NormalTok{  max{-}speed}
\NormalTok{  min{-}speed}
\NormalTok{  swim{-}efficiency}

\NormalTok{  ;; Hydrodynamic resistance}
\NormalTok{  difficulty{-}factor}

\NormalTok{  ;; Energetics}
\NormalTok{  E{-}swim}
\NormalTok{  swim{-}base}

\NormalTok{  ;; Migration targets}
\NormalTok{  home{-}patch}
\NormalTok{  migration{-}patch}

\NormalTok{  ;; Movement memory}
\NormalTok{  planned{-}path}
\NormalTok{  trail}
\NormalTok{  previous{-}patch}
\NormalTok{  previous{-}x}
\NormalTok{  previous{-}y}
\NormalTok{]}

\NormalTok{;; ===================================================================}
\NormalTok{;;  SEAWARD MIGRATION + HYDRODYNAMIC SWIMMING SUBMODEL}
\NormalTok{;; ===================================================================}

\NormalTok{;; {-}{-}{-}{-}{-}{-}{-}{-}{-}{-}{-}{-}{-}{-}{-}{-}{-}{-}{-}{-}{-}{-}{-}{-}{-}{-}{-}{-}{-}{-}{-}{-}{-}{-}{-}{-}{-}{-}{-}{-}{-}{-}{-}{-}{-}{-}{-}{-}{-}{-}{-}{-}{-}{-}{-}{-}{-}{-}{-}{-}{-}{-}{-}{-}{-}{-}{-}}
\NormalTok{;;  Swimming Difficulty (Seaward)}
\NormalTok{;; {-}{-}{-}{-}{-}{-}{-}{-}{-}{-}{-}{-}{-}{-}{-}{-}{-}{-}{-}{-}{-}{-}{-}{-}{-}{-}{-}{-}{-}{-}{-}{-}{-}{-}{-}{-}{-}{-}{-}{-}{-}{-}{-}{-}{-}{-}{-}{-}{-}{-}{-}{-}{-}{-}{-}{-}{-}{-}{-}{-}{-}{-}{-}{-}{-}{-}{-}}
\NormalTok{to calculate{-}difficulty{-}seaward}
  
\NormalTok{  let M{-}max max [weight] of breed}
\NormalTok{  let M{-}agent weight}
\NormalTok{  let V{-}max max{-}landward{-}velocity}
\NormalTok{  let V{-}min max{-}seaward{-}velocity}
\NormalTok{  let k 0.75}

\NormalTok{  let normalized{-}velocity (velocity {-} V{-}min) / (V{-}max {-} V{-}min)}

\NormalTok{  let Df{-}raw (normalized{-}velocity / (M{-}agent / M{-}max)) \^{} k}

\NormalTok{  set difficulty{-}factor (1 + 9 * Df{-}raw)}
\NormalTok{  set difficulty{-}factor max list 1 (min list 10 difficulty{-}factor)}
\NormalTok{end}


\NormalTok{;; {-}{-}{-}{-}{-}{-}{-}{-}{-}{-}{-}{-}{-}{-}{-}{-}{-}{-}{-}{-}{-}{-}{-}{-}{-}{-}{-}{-}{-}{-}{-}{-}{-}{-}{-}{-}{-}{-}{-}{-}{-}{-}{-}{-}{-}{-}{-}{-}{-}{-}{-}{-}{-}{-}{-}{-}{-}{-}{-}{-}{-}{-}{-}{-}{-}{-}{-}}
\NormalTok{;;  Swimming Speed (Seaward)}
\NormalTok{;; {-}{-}{-}{-}{-}{-}{-}{-}{-}{-}{-}{-}{-}{-}{-}{-}{-}{-}{-}{-}{-}{-}{-}{-}{-}{-}{-}{-}{-}{-}{-}{-}{-}{-}{-}{-}{-}{-}{-}{-}{-}{-}{-}{-}{-}{-}{-}{-}{-}{-}{-}{-}{-}{-}{-}{-}{-}{-}{-}{-}{-}{-}{-}{-}{-}{-}{-}}
\NormalTok{to calculate{-}swimming{-}speed{-}seaward}
  
\NormalTok{  let k 0.75}
\NormalTok{  let energy{-}factor (energy / 100)}

\NormalTok{  let velocity{-}impact (k * velocity * 300)}

\NormalTok{  let desired{-}speed }
\NormalTok{    (max{-}speed * energy{-}factor / difficulty{-}factor)}
\NormalTok{      + velocity{-}impact}

\NormalTok{  set desired{-}speed }
\NormalTok{      min list max{-}speed (max list min{-}speed desired{-}speed)}

\NormalTok{  let max{-}speed{-}change 0.5}
\NormalTok{  set max{-}speed{-}change (max{-}speed{-}change * swim{-}efficiency)}

\NormalTok{  if abs(desired{-}speed {-} prev{-}speed) \textgreater{} max{-}speed{-}change [}
\NormalTok{    if desired{-}speed \textgreater{} prev{-}speed [}
\NormalTok{      set speed prev{-}speed + max{-}speed{-}change}
\NormalTok{    ]}
\NormalTok{    if desired{-}speed \textless{} prev{-}speed [}
\NormalTok{      set speed prev{-}speed {-} max{-}speed{-}change}
\NormalTok{    ]}
\NormalTok{  ]}

\NormalTok{  if abs(desired{-}speed {-} prev{-}speed) \textless{}= max{-}speed{-}change [}
\NormalTok{    set speed desired{-}speed}
\NormalTok{  ]}

\NormalTok{  set prev{-}speed speed}
\NormalTok{end}


\NormalTok{;; {-}{-}{-}{-}{-}{-}{-}{-}{-}{-}{-}{-}{-}{-}{-}{-}{-}{-}{-}{-}{-}{-}{-}{-}{-}{-}{-}{-}{-}{-}{-}{-}{-}{-}{-}{-}{-}{-}{-}{-}{-}{-}{-}{-}{-}{-}{-}{-}{-}{-}{-}{-}{-}{-}{-}{-}{-}{-}{-}{-}{-}{-}{-}{-}{-}{-}{-}}
\NormalTok{;;  Cost{-}to{-}Sea (Least Cost Routing)}
\NormalTok{;; {-}{-}{-}{-}{-}{-}{-}{-}{-}{-}{-}{-}{-}{-}{-}{-}{-}{-}{-}{-}{-}{-}{-}{-}{-}{-}{-}{-}{-}{-}{-}{-}{-}{-}{-}{-}{-}{-}{-}{-}{-}{-}{-}{-}{-}{-}{-}{-}{-}{-}{-}{-}{-}{-}{-}{-}{-}{-}{-}{-}{-}{-}{-}{-}{-}{-}{-}}
\NormalTok{to calculate{-}cost{-}to{-}sea [sea{-}patch]}

\NormalTok{  ask migration{-}patch [ set cost{-}to{-}sea 0 ]}
\NormalTok{  ask patch{-}here     [ set cost{-}to{-}sea 1e9 ]}

\NormalTok{  let frontier (list migration{-}patch)}
\NormalTok{  let df difficulty{-}factor}

\NormalTok{  while [not empty? frontier] [}

\NormalTok{    let current first frontier}
\NormalTok{    set frontier but{-}first frontier}

\NormalTok{    ask current [}

\NormalTok{      let current{-}cost cost{-}to{-}sea}

\NormalTok{      ask neighbors4 with [patch{-}terrain = "water"] [}

\NormalTok{        let travel{-}cost compute{-}travel{-}cost current self df}
\NormalTok{        let new{-}cost current{-}cost + travel{-}cost}

\NormalTok{        if new{-}cost \textless{} cost{-}to{-}sea [}
\NormalTok{          set cost{-}to{-}sea new{-}cost}
\NormalTok{          set frontier lput self frontier}
\NormalTok{        ]}
\NormalTok{      ]}
\NormalTok{    ]}
\NormalTok{  ]}
\NormalTok{end}


\NormalTok{;; {-}{-}{-}{-}{-}{-}{-}{-}{-}{-}{-}{-}{-}{-}{-}{-}{-}{-}{-}{-}{-}{-}{-}{-}{-}{-}{-}{-}{-}{-}{-}{-}{-}{-}{-}{-}{-}{-}{-}{-}{-}{-}{-}{-}{-}{-}{-}{-}{-}{-}{-}{-}{-}{-}{-}{-}{-}{-}{-}{-}{-}{-}{-}{-}{-}{-}{-}}
\NormalTok{;;  Build Path to Sea}
\NormalTok{;; {-}{-}{-}{-}{-}{-}{-}{-}{-}{-}{-}{-}{-}{-}{-}{-}{-}{-}{-}{-}{-}{-}{-}{-}{-}{-}{-}{-}{-}{-}{-}{-}{-}{-}{-}{-}{-}{-}{-}{-}{-}{-}{-}{-}{-}{-}{-}{-}{-}{-}{-}{-}{-}{-}{-}{-}{-}{-}{-}{-}{-}{-}{-}{-}{-}{-}{-}}
\NormalTok{to{-}report build{-}path{-}sea [current{-}patch sea{-}patch]}

\NormalTok{  let path (list current{-}patch)}
\NormalTok{  let cur current{-}patch}
\NormalTok{  let max{-}steps 2000}

\NormalTok{  while [cur != migration{-}patch and max{-}steps \textgreater{} 0] [}

\NormalTok{    set max{-}steps max{-}steps {-} 1}

\NormalTok{    let candidates [neighbors] of cur}
\NormalTok{    set candidates candidates with [}
\NormalTok{      patch{-}terrain = "water" and }
\NormalTok{      cost{-}to{-}sea \textless{} [cost{-}to{-}sea] of cur}
\NormalTok{    ]}

\NormalTok{    if not any? candidates [}
\NormalTok{      report path}
\NormalTok{    ]}

\NormalTok{    let next{-}patch min{-}one{-}of candidates [cost{-}to{-}sea]}
\NormalTok{    set path lput next{-}patch path}
\NormalTok{    set cur next{-}patch}
\NormalTok{  ]}

\NormalTok{  report path}
\NormalTok{end}


\NormalTok{;; {-}{-}{-}{-}{-}{-}{-}{-}{-}{-}{-}{-}{-}{-}{-}{-}{-}{-}{-}{-}{-}{-}{-}{-}{-}{-}{-}{-}{-}{-}{-}{-}{-}{-}{-}{-}{-}{-}{-}{-}{-}{-}{-}{-}{-}{-}{-}{-}{-}{-}{-}{-}{-}{-}{-}{-}{-}{-}{-}{-}{-}{-}{-}{-}{-}{-}{-}}
\NormalTok{;;  Seaward Migration – Movement Execution}
\NormalTok{;; {-}{-}{-}{-}{-}{-}{-}{-}{-}{-}{-}{-}{-}{-}{-}{-}{-}{-}{-}{-}{-}{-}{-}{-}{-}{-}{-}{-}{-}{-}{-}{-}{-}{-}{-}{-}{-}{-}{-}{-}{-}{-}{-}{-}{-}{-}{-}{-}{-}{-}{-}{-}{-}{-}{-}{-}{-}{-}{-}{-}{-}{-}{-}{-}{-}{-}{-}}
\NormalTok{to migrate{-}seaward}
  
\NormalTok{  ;; Schooling logic (excluded in documentation)}
\NormalTok{  find{-}schoolmates}
\NormalTok{  if any? schoolmates [}
\NormalTok{    find{-}nearest{-}neighbor}
\NormalTok{    ifelse distance nearest{-}neighbor \textless{} minimum{-}separation [}
\NormalTok{      separate}
\NormalTok{    ][}
\NormalTok{      cohere}
\NormalTok{      align}
\NormalTok{    ]}
\NormalTok{    adjust{-}speed}
\NormalTok{  ]}

\NormalTok{  ;; Path{-}based movement}
\NormalTok{  let travel{-}distance (speed / 3)}
\NormalTok{  let path build{-}path{-}sea patch{-}here migration{-}patch}
\NormalTok{  set planned{-}path path}

\NormalTok{  if not empty? path [}

\NormalTok{    let steps{-}to{-}move ceiling travel{-}distance}
\NormalTok{    let move{-}patches sublist path 0 (min list steps{-}to{-}move length path)}

\NormalTok{    let old{-}patch patch{-}here}

\NormalTok{    foreach move{-}patches [ p {-}\textgreater{}}

\NormalTok{      if p != old{-}patch [}
\NormalTok{        ask p [ set visits{-}by{-}alewife visits{-}by{-}alewife + 1 ]}
\NormalTok{        set old{-}patch p}
\NormalTok{      ]}

\NormalTok{      face p}
\NormalTok{      move{-}to p}

\NormalTok{      ask p [}
\NormalTok{        set ticks{-}spent{-}alewife ticks{-}spent{-}alewife + 1}
\NormalTok{      ]}
\NormalTok{    ]}
\NormalTok{  ]}

\NormalTok{  if not member? patch{-}here trail [}
\NormalTok{    set trail lput patch{-}here trail}
\NormalTok{  ]}
\NormalTok{end}


\NormalTok{;; {-}{-}{-}{-}{-}{-}{-}{-}{-}{-}{-}{-}{-}{-}{-}{-}{-}{-}{-}{-}{-}{-}{-}{-}{-}{-}{-}{-}{-}{-}{-}{-}{-}{-}{-}{-}{-}{-}{-}{-}{-}{-}{-}{-}{-}{-}{-}{-}{-}{-}{-}{-}{-}{-}{-}{-}{-}{-}{-}{-}{-}{-}{-}{-}{-}{-}{-}}
\NormalTok{;;  Universal Swimming Energy Cost}
\NormalTok{;; {-}{-}{-}{-}{-}{-}{-}{-}{-}{-}{-}{-}{-}{-}{-}{-}{-}{-}{-}{-}{-}{-}{-}{-}{-}{-}{-}{-}{-}{-}{-}{-}{-}{-}{-}{-}{-}{-}{-}{-}{-}{-}{-}{-}{-}{-}{-}{-}{-}{-}{-}{-}{-}{-}{-}{-}{-}{-}{-}{-}{-}{-}{-}{-}{-}{-}{-}}
\NormalTok{to calculate{-}swim{-}energy}
  
\NormalTok{  let beta 0.75}
\NormalTok{  let energy{-}multiplier (difficulty{-}factor * (1 / swim{-}efficiency))}

\NormalTok{  set E{-}swim (swim{-}base * (energy{-}multiplier \^{} beta))}
\NormalTok{  set energy max list 0 (energy {-} E{-}swim)}
\NormalTok{end}
\end{Highlighting}
\end{Shaded}

\chapter{Schooling}\label{schooling}

\section{Overview}\label{overview-9}

Schooling is an emergent collective behavior in which fish align, cohere, and maintain spacing with nearby con-specifics. In this model, schooling increases hydrodynamic efficiency, predator avoidance, and coordinated migration. The submodel follows the classic behavioral rules from agent based schooling and BOIDS frameworks (separation, alignment, and cohesion).

The schooling behavior in this model is adapted from the publicly available NetLogo model \href{https://modelingcommons.org/browse/one_model/5709\#model_tabs_browse_info}{Fish Schooling For Predator Avoidance} created by Sebastian Dobon and shared through the NetLogo Modeling Commons. That model implements classic schooling rules from the original BOIDS framework developed by Craig Reynolds in 1987 in his paper \emph{Flocks, Herds, and Schools: A Distributed Behavioral Model} \citep{reynolds_flocks_1987}.

\section{Purpose}\label{purpose-8}

The purpose of this submodel is to simulate collective movement patterns observed in migratory fish where individuals adjust their position and speed based on nearby schoolmates. This mechanism enhances group cohesion during landward or seaward migration and produces emergent school structure, directional alignment, and spacing.

\section{Entities, State Variables, and Scales}\label{entities-state-variables-and-scales-9}

\subsection{Spatial and Temporal Scales}\label{spatial-and-temporal-scales-9}

\textbf{Spatial Unit:} Patch (3 m x 3 m resolution)\\
\textbf{Temporal Unit:} 5 minute time steps (tick)

\subsection{Patch Variables}\label{patch-variables-9}

\begin{longtable}[]{@{}ll@{}}
\toprule\noalign{}
Variable Name & Definition \\
\midrule\noalign{}
\endhead
\bottomrule\noalign{}
\endlastfoot
patch-terrain & Identifies water or land patches \\
\end{longtable}

\subsection{Agent Variables}\label{agent-variables-9}

\begin{longtable}[]{@{}
  >{\raggedright\arraybackslash}p{(\linewidth - 2\tabcolsep) * \real{0.2558}}
  >{\raggedright\arraybackslash}p{(\linewidth - 2\tabcolsep) * \real{0.7442}}@{}}
\toprule\noalign{}
\begin{minipage}[b]{\linewidth}\raggedright
Variable Name
\end{minipage} & \begin{minipage}[b]{\linewidth}\raggedright
Definition
\end{minipage} \\
\midrule\noalign{}
\endhead
\bottomrule\noalign{}
\endlastfoot
breed & Species identity used to restrict schooling to con-specifics \\
speed & Current swimming speed \\
min-speed & Lowest allowable swimming speed \\
start-migration? & Indicates whether the agent has begun migration \\
landward-migration? & Indicates landward migration mode \\
seaward-migration? & Indicates seaward migration mode \\
schoolmates & Set of nearby con-specifics that meet schooling criteria \\
nearest-neighbor & Closest schoolmate used for spacing and cohesion \\
minimum-separation & Desired minimum inter-fish spacing \\
\end{longtable}

\section{Process Overview and Scheduling}\label{process-overview-and-scheduling-9}

\begin{enumerate}
\def\labelenumi{\arabic{enumi}.}
\tightlist
\item
  Identify schoolmates based on species identity and shared migration state.
\item
  Detect the nearest neighbor among schoolmates.
\item
  Apply separation when neighbors are too close.
\item
  Apply cohesion and alignment when spacing is adequate.
\item
  Adjust speed to match group speed and maintain coordination.
\end{enumerate}

\section{Design Concepts}\label{design-concepts-9}

\textbf{Basic Principles}\\
Schooling arises from separation, cohesion, and alignment. These rules reflect collective behaviors observed in migrating fishes.

\textbf{Emergence}\\
Cohesive schools, spacing patterns, and synchronized direction arise without centralized control.

\textbf{Adaptation}\\
Agents adjust heading and speed in response to nearby schoolmates.

\textbf{Objectives}\\
Fish do not attempt to form schools explicitly. Schooling emerges from local rules.

\textbf{Learning}\\
There is no internal learning. Behavior is reactive and rule based.

\textbf{Prediction}\\
Agents do not predict future states of neighbors. They respond to current spacing and direction.

\textbf{Sensing}\\
Agents sense species identity, distance to schoolmates, spacing, and local movement direction.

\textbf{Interaction}\\
Agents change speed and heading based on interactions with neighbors.

\textbf{Stochasticity}\\
Small random turning angles elsewhere in the model introduce variability in movement.

\textbf{Collectives}\\
Schools form as dynamic collectives with shared direction and spacing rules.

\textbf{Observation}\\
School size, spacing, direction, and cohesion may be tracked to evaluate emergent schooling patterns.

\section{Initialization}\label{initialization-9}

\begin{longtable}[]{@{}
  >{\raggedright\arraybackslash}p{(\linewidth - 4\tabcolsep) * \real{0.2386}}
  >{\raggedright\arraybackslash}p{(\linewidth - 4\tabcolsep) * \real{0.2159}}
  >{\raggedright\arraybackslash}p{(\linewidth - 4\tabcolsep) * \real{0.5455}}@{}}
\toprule\noalign{}
\begin{minipage}[b]{\linewidth}\raggedright
Variable
\end{minipage} & \begin{minipage}[b]{\linewidth}\raggedright
Initialized Value
\end{minipage} & \begin{minipage}[b]{\linewidth}\raggedright
Justification
\end{minipage} \\
\midrule\noalign{}
\endhead
\bottomrule\noalign{}
\endlastfoot
speed & species specific & Starting swimming speed \\
min-speed & species specific & Prevents schools from stopping entirely \\
schoolmates & empty set & No neighbors until movement begins \\
nearest-neighbor & none & Determined dynamically during the simulation \\
minimum-separation & species specific & Prevents collisions and maintains spacing \\
\end{longtable}

\section{Submodels}\label{submodels-8}

\subsection{Identifying Schoolmates}\label{identifying-schoolmates}

Fish select schoolmates from nearby con-specifics that share the same migration state, including the start of migration and direction (landward or seaward).

\subsection{Nearest Neighbor}\label{nearest-neighbor}

The closest schoolmate is used to determine whether spacing rules or cohesion rules should be applied.

\subsection{Separation}\label{separation}

If the nearest neighbor is closer than the minimum desired separation, the agent turns away to maintain spacing.

\subsection{Cohesion}\label{cohesion}

When spacing is appropriate, the agent turns toward the group centroid to maintain cohesive structure.

\subsection{Alignment}\label{alignment}

The agent aligns its heading with the mean direction of nearby con-specifics to maintain synchronized movement.

\subsection{Speed Adjustment}\label{speed-adjustment}

Speed is regulated to preserve cohesion within the school:

\begin{itemize}
\tightlist
\item
  The agent accelerates if any schoolmate is significantly faster.\\
\item
  Otherwise, it gradually slows toward the minimum speed.
\end{itemize}

This produces emergent schooling speeds and prevents fragmentation.

\section{Netlogo Implementation}\label{netlogo-implementation-5}

\begin{Shaded}
\begin{Highlighting}[]
\NormalTok{;; {-}{-}{-}{-}{-}{-}{-}{-}{-}{-}{-}{-}{-}{-}{-}{-}{-}{-}{-}{-}{-}{-}{-}{-}{-}{-}{-}{-}{-}{-}{-}{-}{-}{-}{-}{-}{-}{-}{-}{-}{-}{-}{-}{-}{-}{-}{-}{-}{-}{-}{-}{-}{-}{-}{-}{-}{-}{-}{-}{-}{-}{-}{-}{-}{-}{-}{-}}
\NormalTok{;;  SCHOOLING SUBMODEL — VARIABLE DECLARATIONS}
\NormalTok{;; {-}{-}{-}{-}{-}{-}{-}{-}{-}{-}{-}{-}{-}{-}{-}{-}{-}{-}{-}{-}{-}{-}{-}{-}{-}{-}{-}{-}{-}{-}{-}{-}{-}{-}{-}{-}{-}{-}{-}{-}{-}{-}{-}{-}{-}{-}{-}{-}{-}{-}{-}{-}{-}{-}{-}{-}{-}{-}{-}{-}{-}{-}{-}{-}{-}{-}{-}}

\NormalTok{patches{-}own [}
\NormalTok{  ;; Terrain classification}
\NormalTok{  patch{-}terrain          ;; "water" or "land"}
\NormalTok{]}

\NormalTok{turtles{-}own [}
\NormalTok{  ;; {-}{-}{-}{-}{-}{-}{-}{-}{-}{-}{-}{-}{-}{-}{-}{-}{-}{-}{-}{-}{-}{-}{-}{-}{-}{-}{-}{-}{-}{-}{-}{-}{-}{-}{-}{-}{-}{-}{-}{-}{-}{-}{-}{-}{-}{-}{-}{-}{-}{-}{-}{-}{-}{-}{-}{-}{-}{-}{-}{-}{-}{-}{-}{-}{-}}
\NormalTok{  ;; Schooling state variables}
\NormalTok{  ;; {-}{-}{-}{-}{-}{-}{-}{-}{-}{-}{-}{-}{-}{-}{-}{-}{-}{-}{-}{-}{-}{-}{-}{-}{-}{-}{-}{-}{-}{-}{-}{-}{-}{-}{-}{-}{-}{-}{-}{-}{-}{-}{-}{-}{-}{-}{-}{-}{-}{-}{-}{-}{-}{-}{-}{-}{-}{-}{-}{-}{-}{-}{-}{-}{-}}
\NormalTok{  schoolmates            ;; agentset of nearby con{-}specifics}
\NormalTok{  nearest{-}neighbor       ;; closest fish in schoolmates}
\NormalTok{  minimum{-}separation     ;; desired distance to avoid collisions}

\NormalTok{  ;; {-}{-}{-}{-}{-}{-}{-}{-}{-}{-}{-}{-}{-}{-}{-}{-}{-}{-}{-}{-}{-}{-}{-}{-}{-}{-}{-}{-}{-}{-}{-}{-}{-}{-}{-}{-}{-}{-}{-}{-}{-}{-}{-}{-}{-}{-}{-}{-}{-}{-}{-}{-}{-}{-}{-}{-}{-}{-}{-}{-}{-}{-}{-}{-}{-}}
\NormalTok{  ;; Migration gating used for schooling eligibility}
\NormalTok{  ;; (schoolmates only form among agents sharing these states)}
\NormalTok{  ;; {-}{-}{-}{-}{-}{-}{-}{-}{-}{-}{-}{-}{-}{-}{-}{-}{-}{-}{-}{-}{-}{-}{-}{-}{-}{-}{-}{-}{-}{-}{-}{-}{-}{-}{-}{-}{-}{-}{-}{-}{-}{-}{-}{-}{-}{-}{-}{-}{-}{-}{-}{-}{-}{-}{-}{-}{-}{-}{-}{-}{-}{-}{-}{-}{-}}
\NormalTok{  start{-}migration?       ;; has the agent initiated migration?}
\NormalTok{  landward{-}migration?    ;; migrating upstream?}
\NormalTok{  seaward{-}migration?     ;; migrating downstream?}

\NormalTok{  ;; {-}{-}{-}{-}{-}{-}{-}{-}{-}{-}{-}{-}{-}{-}{-}{-}{-}{-}{-}{-}{-}{-}{-}{-}{-}{-}{-}{-}{-}{-}{-}{-}{-}{-}{-}{-}{-}{-}{-}{-}{-}{-}{-}{-}{-}{-}{-}{-}{-}{-}{-}{-}{-}{-}{-}{-}{-}{-}{-}{-}{-}{-}{-}{-}{-}}
\NormalTok{  ;; Swimming parameters affected by schooling}
\NormalTok{  ;; {-}{-}{-}{-}{-}{-}{-}{-}{-}{-}{-}{-}{-}{-}{-}{-}{-}{-}{-}{-}{-}{-}{-}{-}{-}{-}{-}{-}{-}{-}{-}{-}{-}{-}{-}{-}{-}{-}{-}{-}{-}{-}{-}{-}{-}{-}{-}{-}{-}{-}{-}{-}{-}{-}{-}{-}{-}{-}{-}{-}{-}{-}{-}{-}{-}}
\NormalTok{  speed                  ;; current swimming speed}
\NormalTok{  min{-}speed              ;; lower speed limit to prevent stall}
\NormalTok{]}

\NormalTok{;; {-}{-}{-}{-}{-}{-}{-}{-}{-}{-}{-}{-}{-}{-}{-}{-}{-}{-}{-}{-}{-}{-}{-}{-}{-}{-}{-}{-}{-}{-}{-}{-}{-}{-}{-}{-}{-}{-}{-}{-}{-}{-}{-}{-}{-}{-}{-}{-}{-}{-}{-}{-}{-}{-}{-}{-}{-}{-}{-}{-}{-}{-}{-}{-}{-}{-}{-}}
\NormalTok{;;  SCHOOLING: MAIN CONTROLLER}
\NormalTok{;; {-}{-}{-}{-}{-}{-}{-}{-}{-}{-}{-}{-}{-}{-}{-}{-}{-}{-}{-}{-}{-}{-}{-}{-}{-}{-}{-}{-}{-}{-}{-}{-}{-}{-}{-}{-}{-}{-}{-}{-}{-}{-}{-}{-}{-}{-}{-}{-}{-}{-}{-}{-}{-}{-}{-}{-}{-}{-}{-}{-}{-}{-}{-}{-}{-}{-}{-}}

\NormalTok{to school}
\NormalTok{  find{-}schoolmates}

\NormalTok{  if any? schoolmates [}
\NormalTok{    find{-}nearest{-}neighbor}

\NormalTok{    ;; Separation rule}
\NormalTok{    ifelse distance nearest{-}neighbor \textless{} minimum{-}separation}
\NormalTok{    [}
\NormalTok{      separate}
\NormalTok{    ]}
\NormalTok{    [}
\NormalTok{      ;; Cohesion + alignment rules}
\NormalTok{      cohere}
\NormalTok{      align}
\NormalTok{    ]}

\NormalTok{    ;; Adjust speed to maintain cohesion}
\NormalTok{    adjust{-}speed}
\NormalTok{  ]}
\NormalTok{end}


\NormalTok{;; {-}{-}{-}{-}{-}{-}{-}{-}{-}{-}{-}{-}{-}{-}{-}{-}{-}{-}{-}{-}{-}{-}{-}{-}{-}{-}{-}{-}{-}{-}{-}{-}{-}{-}{-}{-}{-}{-}{-}{-}{-}{-}{-}{-}{-}{-}{-}{-}{-}{-}{-}{-}{-}{-}{-}{-}{-}{-}{-}{-}{-}{-}{-}{-}{-}{-}{-}}
\NormalTok{;;  FIND SCHOOLMATES (CON{-}SPECIFICS + SAME MIGRATION MODE)}
\NormalTok{;; {-}{-}{-}{-}{-}{-}{-}{-}{-}{-}{-}{-}{-}{-}{-}{-}{-}{-}{-}{-}{-}{-}{-}{-}{-}{-}{-}{-}{-}{-}{-}{-}{-}{-}{-}{-}{-}{-}{-}{-}{-}{-}{-}{-}{-}{-}{-}{-}{-}{-}{-}{-}{-}{-}{-}{-}{-}{-}{-}{-}{-}{-}{-}{-}{-}{-}{-}}

\NormalTok{to find{-}schoolmates}
\NormalTok{  set schoolmates other turtles with [}
\NormalTok{    breed = [breed] of myself and}
\NormalTok{    start{-}migration? and}
\NormalTok{    landward{-}migration? = [landward{-}migration?] of myself and}
\NormalTok{    seaward{-}migration? = [seaward{-}migration?] of myself}
\NormalTok{  ]}
\NormalTok{end}


\NormalTok{;; {-}{-}{-}{-}{-}{-}{-}{-}{-}{-}{-}{-}{-}{-}{-}{-}{-}{-}{-}{-}{-}{-}{-}{-}{-}{-}{-}{-}{-}{-}{-}{-}{-}{-}{-}{-}{-}{-}{-}{-}{-}{-}{-}{-}{-}{-}{-}{-}{-}{-}{-}{-}{-}{-}{-}{-}{-}{-}{-}{-}{-}{-}{-}{-}{-}{-}{-}}
\NormalTok{;;  FIND NEAREST NEIGHBOR}
\NormalTok{;; {-}{-}{-}{-}{-}{-}{-}{-}{-}{-}{-}{-}{-}{-}{-}{-}{-}{-}{-}{-}{-}{-}{-}{-}{-}{-}{-}{-}{-}{-}{-}{-}{-}{-}{-}{-}{-}{-}{-}{-}{-}{-}{-}{-}{-}{-}{-}{-}{-}{-}{-}{-}{-}{-}{-}{-}{-}{-}{-}{-}{-}{-}{-}{-}{-}{-}{-}}

\NormalTok{to find{-}nearest{-}neighbor}
\NormalTok{  set nearest{-}neighbor min{-}one{-}of schoolmates [ distance myself ]}
\NormalTok{end}


\NormalTok{;; {-}{-}{-}{-}{-}{-}{-}{-}{-}{-}{-}{-}{-}{-}{-}{-}{-}{-}{-}{-}{-}{-}{-}{-}{-}{-}{-}{-}{-}{-}{-}{-}{-}{-}{-}{-}{-}{-}{-}{-}{-}{-}{-}{-}{-}{-}{-}{-}{-}{-}{-}{-}{-}{-}{-}{-}{-}{-}{-}{-}{-}{-}{-}{-}{-}{-}{-}}
\NormalTok{;;  SEPARATION RULE (AVOID COLLISION)}
\NormalTok{;; {-}{-}{-}{-}{-}{-}{-}{-}{-}{-}{-}{-}{-}{-}{-}{-}{-}{-}{-}{-}{-}{-}{-}{-}{-}{-}{-}{-}{-}{-}{-}{-}{-}{-}{-}{-}{-}{-}{-}{-}{-}{-}{-}{-}{-}{-}{-}{-}{-}{-}{-}{-}{-}{-}{-}{-}{-}{-}{-}{-}{-}{-}{-}{-}{-}{-}{-}}

\NormalTok{to separate}
\NormalTok{  ;; Turn away from nearest neighbor}
\NormalTok{  let dx (dx{-}of nearest{-}neighbor)}
\NormalTok{  let dy (dy{-}of nearest{-}neighbor)}
\NormalTok{  rt (random 20 + 160)}
\NormalTok{  fd 0.1}
\NormalTok{end}


\NormalTok{;; {-}{-}{-}{-}{-}{-}{-}{-}{-}{-}{-}{-}{-}{-}{-}{-}{-}{-}{-}{-}{-}{-}{-}{-}{-}{-}{-}{-}{-}{-}{-}{-}{-}{-}{-}{-}{-}{-}{-}{-}{-}{-}{-}{-}{-}{-}{-}{-}{-}{-}{-}{-}{-}{-}{-}{-}{-}{-}{-}{-}{-}{-}{-}{-}{-}{-}{-}}
\NormalTok{;;  COHESION RULE (TURN TOWARD GROUP)}
\NormalTok{;; {-}{-}{-}{-}{-}{-}{-}{-}{-}{-}{-}{-}{-}{-}{-}{-}{-}{-}{-}{-}{-}{-}{-}{-}{-}{-}{-}{-}{-}{-}{-}{-}{-}{-}{-}{-}{-}{-}{-}{-}{-}{-}{-}{-}{-}{-}{-}{-}{-}{-}{-}{-}{-}{-}{-}{-}{-}{-}{-}{-}{-}{-}{-}{-}{-}{-}{-}}

\NormalTok{to cohere}
\NormalTok{  if any? schoolmates [}
\NormalTok{    face mean{-}heading{-}of schoolmates}
\NormalTok{  ]}
\NormalTok{end}


\NormalTok{;; {-}{-}{-}{-}{-}{-}{-}{-}{-}{-}{-}{-}{-}{-}{-}{-}{-}{-}{-}{-}{-}{-}{-}{-}{-}{-}{-}{-}{-}{-}{-}{-}{-}{-}{-}{-}{-}{-}{-}{-}{-}{-}{-}{-}{-}{-}{-}{-}{-}{-}{-}{-}{-}{-}{-}{-}{-}{-}{-}{-}{-}{-}{-}{-}{-}{-}{-}}
\NormalTok{;;  ALIGNMENT RULE (MATCH GROUP HEADING)}
\NormalTok{;; {-}{-}{-}{-}{-}{-}{-}{-}{-}{-}{-}{-}{-}{-}{-}{-}{-}{-}{-}{-}{-}{-}{-}{-}{-}{-}{-}{-}{-}{-}{-}{-}{-}{-}{-}{-}{-}{-}{-}{-}{-}{-}{-}{-}{-}{-}{-}{-}{-}{-}{-}{-}{-}{-}{-}{-}{-}{-}{-}{-}{-}{-}{-}{-}{-}{-}{-}}

\NormalTok{to align}
\NormalTok{  if any? schoolmates [}
\NormalTok{    set heading (heading + (mean [heading] of schoolmates {-} heading) * 0.1)}
\NormalTok{  ]}
\NormalTok{end}


\NormalTok{;; {-}{-}{-}{-}{-}{-}{-}{-}{-}{-}{-}{-}{-}{-}{-}{-}{-}{-}{-}{-}{-}{-}{-}{-}{-}{-}{-}{-}{-}{-}{-}{-}{-}{-}{-}{-}{-}{-}{-}{-}{-}{-}{-}{-}{-}{-}{-}{-}{-}{-}{-}{-}{-}{-}{-}{-}{-}{-}{-}{-}{-}{-}{-}{-}{-}{-}{-}}
\NormalTok{;;  SPEED ADJUSTMENT (MAINTAIN COHESION)}
\NormalTok{;; {-}{-}{-}{-}{-}{-}{-}{-}{-}{-}{-}{-}{-}{-}{-}{-}{-}{-}{-}{-}{-}{-}{-}{-}{-}{-}{-}{-}{-}{-}{-}{-}{-}{-}{-}{-}{-}{-}{-}{-}{-}{-}{-}{-}{-}{-}{-}{-}{-}{-}{-}{-}{-}{-}{-}{-}{-}{-}{-}{-}{-}{-}{-}{-}{-}{-}{-}}

\NormalTok{to adjust{-}speed}
\NormalTok{  ifelse max [speed] of schoolmates \textgreater{} speed + 0.1}
\NormalTok{  [}
\NormalTok{    ;; Accelerate to keep up with group}
\NormalTok{    set speed speed + 0.1}
\NormalTok{  ]}
\NormalTok{  [}
\NormalTok{    ;; Slow down toward minimum speed}
\NormalTok{    if speed \textgreater{} min{-}speed [}
\NormalTok{      set speed speed {-} 0.2}
\NormalTok{    ]}
\NormalTok{  ]}
\NormalTok{end}
\end{Highlighting}
\end{Shaded}

\chapter{Selective Tidal Stream Transport}\label{selective-tidal-stream-transport}

\section{Overview}\label{overview-10}

Selective Tidal Stream Transport (STST) is a behavioral strategy that enables agents to conserve energy by passively drifting with the current. It is triggered when the along-channel velocity of the patch exceeds the agent's effective swimming speed, and that speed is below a species-specific minimum threshold. Once engaged, agents align with the tidal current and are carried downstream or upstream, depending on flow velocity. STST reduces the metabolic cost of movement by substituting active swimming with passive transport. This behavior persists for a limited duration or until swimming ability improves, after which agents resume directional migration.

\section{Purpose}\label{purpose-9}

To simulate a passive energy-conserving behavior in migratory fish that allows them to use tidal currents to move when swimming capacity is insufficient to overcome flow velocities.

\section{Entities, State Variables, and Scales}\label{entities-state-variables-and-scales-10}

\subsection{Spatial and Temporal Scales}\label{spatial-and-temporal-scales-10}

\textbf{Spatial Unit:} Patch (3 m x 3 m resolution)\\
\textbf{Temporal Unit:} 5 minute time steps (tick)

\subsection{\texorpdfstring{\textbf{Global Variables}}{Global Variables}}\label{global-variables-9}

\begin{longtable}[]{@{}
  >{\raggedright\arraybackslash}p{(\linewidth - 2\tabcolsep) * \real{0.5000}}
  >{\raggedright\arraybackslash}p{(\linewidth - 2\tabcolsep) * \real{0.5000}}@{}}
\toprule\noalign{}
\begin{minipage}[b]{\linewidth}\raggedright
Variable Name
\end{minipage} & \begin{minipage}[b]{\linewidth}\raggedright
Definition
\end{minipage} \\
\midrule\noalign{}
\endhead
\bottomrule\noalign{}
\endlastfoot
\textbf{max-seaward-velocity} & Highest magnitude seaward-directed hydrodynamic velocity across the spatial domain. \\
\textbf{max-landward-velocity} & Highest magnitude landward-directed hydrodynamic velocity across the spatial domain. \\
\end{longtable}

\subsection{Patch Variables}\label{patch-variables-10}

\begin{longtable}[]{@{}
  >{\centering\arraybackslash}p{(\linewidth - 2\tabcolsep) * \real{0.5000}}
  >{\centering\arraybackslash}p{(\linewidth - 2\tabcolsep) * \real{0.5000}}@{}}
\toprule\noalign{}
\begin{minipage}[b]{\linewidth}\centering
Variable Name
\end{minipage} & \begin{minipage}[b]{\linewidth}\centering
Definition
\end{minipage} \\
\midrule\noalign{}
\endhead
\bottomrule\noalign{}
\endlastfoot
\textbf{Velocity} \(V_{patch}\) & The along-channel velocity of a given patch, derived from hydrodynamic model inputs, where positive values are in the landward direction and negative values are in the seaward direction. \\
\textbf{tidal-transport-in-patch} & Count of agents exhibiting tidal stream transport within a patch (for habitat quality analysis). \\
\end{longtable}

\subsection{Agent Variables}\label{agent-variables-10}

\begin{longtable}[]{@{}
  >{\centering\arraybackslash}p{(\linewidth - 2\tabcolsep) * \real{0.5000}}
  >{\centering\arraybackslash}p{(\linewidth - 2\tabcolsep) * \real{0.5000}}@{}}
\toprule\noalign{}
\endhead
\bottomrule\noalign{}
\endlastfoot
\textbf{Variable Name} & \textbf{Definition} \\
\textbf{energy} \(E_{agent}\) & Total energy available to the agent. \\
\textbf{swimming energy} \(E_{swim}\) & Energy expenditure from movement per time step. \\
\textbf{base swim energy} \(swim_{base}\) & Baseline energy cost of movement. \\
\textbf{swimming difficulty} \(D_f\) & Velocity-based proxy representing hydrodynamic resistance. \\
\textbf{in-STST?} \(STST_{?}\) & Boolean value indicating if the agent is actively in STST. \\
\textbf{swimming speed} \(V_{agent}\) & The effective swimming speed of the agent. \\
\textbf{minimum threshold speed} \(Speed_{min}\) & The minimum speed at which an agent will move. \\
\end{longtable}

\section{Process Overview and Scheduling}\label{process-overview-and-scheduling-10}

\begin{enumerate}
\def\labelenumi{\arabic{enumi}.}
\item
  Compare swimming speed (\(V_{agent}\)) with flow speed (\(V_{patch}\)).
\item
  If \(|V_{patch}| > V_{agent}\) and \(V_{agent} \leq Speed_{min}\), enter STST.
\item
  In STST: align with current, update position via drift, apply reduced energy cost.
\item
  If \(V_{agent} > Speed_{min}\), exit STST and resume active swimming.
\end{enumerate}

\section{Design Concepts}\label{design-concepts-10}

\textbf{Basic Principles:} Selective tidal stream transport is based on behavioral ecology and energetics, simulating the tradeoff between active swimming and energy conservation through passive transport.

\textbf{Emergence:} Passive drift behavior and resulting migration paths emerge from agent-flow interactions and individual swimming limitations.

\textbf{Adaptation:} Agents adapt their mode of movement based on their swimming ability relative to environmental flow, dynamically choosing energy-efficient strategies.

\textbf{Objectives:} Agents seek to minimizing energy loss in strong flows.

\textbf{Sensing:} Agents sense their own \(V_{agent}\) and the \(V_{patch}\) to determine whether passive drift is needed.

\textbf{Observation:} Records STST patch events , energy expenditure (\(E_{agent}\)), and displacement are logged to analyze behavior across flow regimes.

\section{Initialization}\label{initialization-10}

\begin{longtable}[]{@{}
  >{\centering\arraybackslash}p{(\linewidth - 4\tabcolsep) * \real{0.3333}}
  >{\centering\arraybackslash}p{(\linewidth - 4\tabcolsep) * \real{0.3333}}
  >{\centering\arraybackslash}p{(\linewidth - 4\tabcolsep) * \real{0.3333}}@{}}
\toprule\noalign{}
\begin{minipage}[b]{\linewidth}\centering
Variable
\end{minipage} & \begin{minipage}[b]{\linewidth}\centering
Initialized Value
\end{minipage} & \begin{minipage}[b]{\linewidth}\centering
Justification
\end{minipage} \\
\midrule\noalign{}
\endhead
\bottomrule\noalign{}
\endlastfoot
\(V_{agent}\) & Based on size, energy, and difficulty factor & Reflects agent's swimming capability based on metabolic limits. \\
\(swim_{min}\) & Species-specific parameter & Represents the minimum sustained swimming velocity of the agent. \\
\(swim_{max}\) & Species-specific parameter & Represents the maximum sustained swimming velocity of the agent. \\
\(E_{agent}\) & 100 & Assumes full energy at the start of simulation or at spawning. \\
\(swim_{base}\) & \(0.02 \cdot \frac{M_{agent}}{M_{max}}^{k}\) & Scales locomotion cost nonlinearly with size; can be calibrated. \\
\end{longtable}

\section{Submodels}\label{submodels-9}

\subsection{Trigger Conditions for STST}\label{trigger-conditions-for-stst}

Agents compare their swimming ability to the flow conditions. If local flow exceeds their capability and their effort is below a defined threshold, they enter STST:

\[
|V_{patch}| > V_{agent} \quad \text{and} \quad V_{agent} \leq Speed_{min}
\]

While in STST:

Heading aligns with the current (drift vector) \& swimming speed is set to:

\[
V_{agent} = |V_{patch}|
\]

Energy is set as:

\[
E_{swim} = swim_{base}
\]

Where:

\begin{itemize}
\item
  \(V_{agent}\) is the current swimming speed of the agent.
\item
  \(V_{patch}\) is the along-channel velocity at the agent's current patch.
\item
  \(Speed_{min}\) is the minimum sustainable swimming speed of the agent.
\item
  \(swim_{base}\) is the base swimming cost based on agent size.
\item
  \(E_{swim}\) is the total energy cost during passive movement.
\end{itemize}

\subsection{Behavior During STST}\label{behavior-during-stst}

While in STST, agents align with the current (either landward or seaward) and are passively transported:

\[
\vec{Y}_{t+1} = \vec{Y}_t + |V_{patch}| \cdot \hat{u}
\]

Swimming speed is overwritten:

\[
V_{agent} = |V_{patch}|
\]

Energy cost is minimized:

\[
E_{swim} = swim_{base}
\]

Where:

\begin{itemize}
\item
  \(\vec{Y}_t\) is the agent's current spatial position.
\item
  \(\vec{Y}_{t+1}\) is the position after drifting.
\item
  \(V_{patch}\) is the along-channel velocity at the agent's current patch.
\item
  \(\hat{u}_{patch}\) is the direction of the patch velocity (unit vector).
\item
  \(swim_{base}\) is the base swimming cost based on agent size.
\item
  \(E_{swim}\) is the total energy cost during passive movement.
\end{itemize}

\subsection{Stop Conditions for STST}\label{stop-conditions-for-stst}

Agents exit STST when they regain sufficient swimming capacity to exceed threshold:

\[
V_{agent} > Speed_{min}
\]

Where:

\begin{itemize}
\item
  \(V_{agent}\) is the current swimming speed of the agent.
\item
  \(Speed_{min}\) is the minimum sustainable swimming speed of the agent.
\end{itemize}

This triggers a return to active migratory movement and deactivates \(STST_{?}\).

\section{Netlogo Implementation}\label{netlogo-implementation-6}

\begin{Shaded}
\begin{Highlighting}[]
\NormalTok{;; {-}{-}{-}{-}{-}{-}{-}{-}{-}{-}{-}{-}{-}{-}{-}{-}{-}{-}{-}{-}{-}{-}{-}{-}{-}{-}{-}{-}{-}{-}{-}{-}{-}{-}{-}{-}{-}{-}{-}{-}{-}{-}{-}{-}{-}{-}{-}{-}{-}{-}{-}{-}{-}{-}{-}{-}{-}{-}{-}{-}{-}{-}{-}{-}{-}{-}{-}}
\NormalTok{;;  GLOBALS}
\NormalTok{;; {-}{-}{-}{-}{-}{-}{-}{-}{-}{-}{-}{-}{-}{-}{-}{-}{-}{-}{-}{-}{-}{-}{-}{-}{-}{-}{-}{-}{-}{-}{-}{-}{-}{-}{-}{-}{-}{-}{-}{-}{-}{-}{-}{-}{-}{-}{-}{-}{-}{-}{-}{-}{-}{-}{-}{-}{-}{-}{-}{-}{-}{-}{-}{-}{-}{-}{-}}
\NormalTok{globals [}
\NormalTok{  max{-}seaward{-}velocity}
\NormalTok{  max{-}landward{-}velocity}
  
\NormalTok{  landward{-}migration?}
\NormalTok{  seaward{-}migration?}

\NormalTok{  selective{-}tidal{-}transport?}
\NormalTok{  STST{-}start{-}tick}
\NormalTok{]}

\NormalTok{;; {-}{-}{-}{-}{-}{-}{-}{-}{-}{-}{-}{-}{-}{-}{-}{-}{-}{-}{-}{-}{-}{-}{-}{-}{-}{-}{-}{-}{-}{-}{-}{-}{-}{-}{-}{-}{-}{-}{-}{-}{-}{-}{-}{-}{-}{-}{-}{-}{-}{-}{-}{-}{-}{-}{-}{-}{-}{-}{-}{-}{-}{-}{-}{-}{-}{-}{-}}
\NormalTok{;;  PATCH VARIABLES}
\NormalTok{;; {-}{-}{-}{-}{-}{-}{-}{-}{-}{-}{-}{-}{-}{-}{-}{-}{-}{-}{-}{-}{-}{-}{-}{-}{-}{-}{-}{-}{-}{-}{-}{-}{-}{-}{-}{-}{-}{-}{-}{-}{-}{-}{-}{-}{-}{-}{-}{-}{-}{-}{-}{-}{-}{-}{-}{-}{-}{-}{-}{-}{-}{-}{-}{-}{-}{-}{-}}
\NormalTok{patches{-}own [}
\NormalTok{  velocity}
\NormalTok{  depth}
\NormalTok{  patch{-}terrain}
  
\NormalTok{  tidal{-}transport{-}in{-}patch}
\NormalTok{]}

\NormalTok{;; {-}{-}{-}{-}{-}{-}{-}{-}{-}{-}{-}{-}{-}{-}{-}{-}{-}{-}{-}{-}{-}{-}{-}{-}{-}{-}{-}{-}{-}{-}{-}{-}{-}{-}{-}{-}{-}{-}{-}{-}{-}{-}{-}{-}{-}{-}{-}{-}{-}{-}{-}{-}{-}{-}{-}{-}{-}{-}{-}{-}{-}{-}{-}{-}{-}{-}{-}}
\NormalTok{;;  TURTLE VARIABLES}
\NormalTok{;; {-}{-}{-}{-}{-}{-}{-}{-}{-}{-}{-}{-}{-}{-}{-}{-}{-}{-}{-}{-}{-}{-}{-}{-}{-}{-}{-}{-}{-}{-}{-}{-}{-}{-}{-}{-}{-}{-}{-}{-}{-}{-}{-}{-}{-}{-}{-}{-}{-}{-}{-}{-}{-}{-}{-}{-}{-}{-}{-}{-}{-}{-}{-}{-}{-}{-}{-}}
\NormalTok{turtles{-}own [}
\NormalTok{  speed}
\NormalTok{  min{-}speed}
\NormalTok{  swim{-}efficiency}
\NormalTok{  swim{-}base}
\NormalTok{  difficulty{-}factor}
\NormalTok{  energy}
\NormalTok{  E{-}swim}
  
\NormalTok{  selective{-}tidal{-}transport?}
\NormalTok{  STST{-}start{-}tick}
\NormalTok{]}

\NormalTok{;; {-}{-}{-}{-}{-}{-}{-}{-}{-}{-}{-}{-}{-}{-}{-}{-}{-}{-}{-}{-}{-}{-}{-}{-}{-}{-}{-}{-}{-}{-}{-}{-}{-}{-}{-}{-}{-}{-}{-}{-}{-}{-}{-}{-}{-}{-}{-}{-}{-}{-}{-}{-}{-}{-}{-}{-}{-}{-}{-}{-}{-}{-}{-}{-}{-}{-}{-}}
\NormalTok{;;  STST — LANDWARD}
\NormalTok{;; {-}{-}{-}{-}{-}{-}{-}{-}{-}{-}{-}{-}{-}{-}{-}{-}{-}{-}{-}{-}{-}{-}{-}{-}{-}{-}{-}{-}{-}{-}{-}{-}{-}{-}{-}{-}{-}{-}{-}{-}{-}{-}{-}{-}{-}{-}{-}{-}{-}{-}{-}{-}{-}{-}{-}{-}{-}{-}{-}{-}{-}{-}{-}{-}{-}{-}{-}}
\NormalTok{to selective{-}tidal{-}stream{-}transport{-}landward}
\NormalTok{  let V{-}patch [velocity] of patch{-}here}
\NormalTok{  let S{-}swim (speed / 300) ;; 5 minute speed}

\NormalTok{  ;; Continue drifting (1 tick cooldown)}
\NormalTok{  if selective{-}tidal{-}transport? and (ticks {-} STST{-}start{-}tick \textless{} 1) [}
\NormalTok{    let drift{-}distance abs V{-}patch * (1 {-} swim{-}efficiency)}
\NormalTok{    let drift{-}target one{-}of neighbors with [patch{-}terrain = "water"]}
\NormalTok{    if drift{-}target != nobody [ safe{-}move drift{-}target drift{-}distance ]}
\NormalTok{    set tidal{-}transport{-}in{-}patch tidal{-}transport{-}in{-}patch + 1}
\NormalTok{    stop}
\NormalTok{  ]}

\NormalTok{  ;; Enter STST}
\NormalTok{  if abs V{-}patch \textgreater{} S{-}swim and S{-}swim \textless{}= min{-}speed [}
\NormalTok{    set selective{-}tidal{-}transport? true}
\NormalTok{    set STST{-}start{-}tick ticks}
\NormalTok{    let drift{-}distance abs V{-}patch * (1 {-} swim{-}efficiency)}
\NormalTok{    let drift{-}target one{-}of neighbors with [patch{-}terrain = "water"]}
\NormalTok{    if drift{-}target != nobody [ safe{-}move drift{-}target drift{-}distance ]}
\NormalTok{    set tidal{-}transport{-}in{-}patch tidal{-}transport{-}in{-}patch + 1}
\NormalTok{    stop}
\NormalTok{  ]}

\NormalTok{  ;; Exit STST → resume active migration}
\NormalTok{  if S{-}swim \textgreater{} min{-}speed or (ticks {-} STST{-}start{-}tick \textgreater{}= 1) [}
\NormalTok{    set selective{-}tidal{-}transport? false}
\NormalTok{    calculate{-}swim{-}energy}
\NormalTok{    set energy max list 0 (energy {-} E{-}swim)}
\NormalTok{    migrate{-}landward}
\NormalTok{  ]}
\NormalTok{end}

\NormalTok{;; {-}{-}{-}{-}{-}{-}{-}{-}{-}{-}{-}{-}{-}{-}{-}{-}{-}{-}{-}{-}{-}{-}{-}{-}{-}{-}{-}{-}{-}{-}{-}{-}{-}{-}{-}{-}{-}{-}{-}{-}{-}{-}{-}{-}{-}{-}{-}{-}{-}{-}{-}{-}{-}{-}{-}{-}{-}{-}{-}{-}{-}{-}{-}{-}{-}{-}{-}}
\NormalTok{;;  STST — SEAWARD}
\NormalTok{;; {-}{-}{-}{-}{-}{-}{-}{-}{-}{-}{-}{-}{-}{-}{-}{-}{-}{-}{-}{-}{-}{-}{-}{-}{-}{-}{-}{-}{-}{-}{-}{-}{-}{-}{-}{-}{-}{-}{-}{-}{-}{-}{-}{-}{-}{-}{-}{-}{-}{-}{-}{-}{-}{-}{-}{-}{-}{-}{-}{-}{-}{-}{-}{-}{-}{-}{-}}
\NormalTok{to selective{-}tidal{-}stream{-}transport{-}seaward}
\NormalTok{  let V{-}patch [velocity] of patch{-}here}
\NormalTok{  let S{-}swim (speed / 300)}

\NormalTok{  ;; Continue STST drift}
\NormalTok{  if selective{-}tidal{-}transport? and (ticks {-} STST{-}start{-}tick \textless{} 1) [}
\NormalTok{    let drift{-}distance abs V{-}patch * (1 {-} swim{-}efficiency)}
\NormalTok{    let drift{-}target one{-}of neighbors with [patch{-}terrain = "water"]}
\NormalTok{    if drift{-}target != nobody [ safe{-}move drift{-}target drift{-}distance ]}
\NormalTok{    set tidal{-}transport{-}in{-}patch tidal{-}transport{-}in{-}patch + 1}
\NormalTok{    stop}
\NormalTok{  ]}

\NormalTok{  ;; Enter STST}
\NormalTok{  if abs V{-}patch \textgreater{} S{-}swim and S{-}swim \textless{}= min{-}speed [}
\NormalTok{    set selective{-}tidal{-}transport? true}
\NormalTok{    set STST{-}start{-}tick ticks}
\NormalTok{    let drift{-}distance abs V{-}patch * (1 {-} swim{-}efficiency)}
\NormalTok{    let drift{-}target one{-}of neighbors with [patch{-}terrain = "water"]}
\NormalTok{    if drift{-}target != nobody [ safe{-}move drift{-}target drift{-}distance ]}
\NormalTok{    set tidal{-}transport{-}in{-}patch tidal{-}transport{-}in{-}patch + 1}
\NormalTok{    stop}
\NormalTok{  ]}

\NormalTok{  ;; Exit STST → resume seaward migration}
\NormalTok{  if S{-}swim \textgreater{} min{-}speed or (ticks {-} STST{-}start{-}tick \textgreater{}= 1) [}
\NormalTok{    set selective{-}tidal{-}transport? false}
\NormalTok{    calculate{-}swim{-}energy}
\NormalTok{    set energy max list 0 (energy {-} E{-}swim)}
\NormalTok{    migrate{-}seaward}
\NormalTok{  ]}
\NormalTok{end}

\NormalTok{;; {-}{-}{-}{-}{-}{-}{-}{-}{-}{-}{-}{-}{-}{-}{-}{-}{-}{-}{-}{-}{-}{-}{-}{-}{-}{-}{-}{-}{-}{-}{-}{-}{-}{-}{-}{-}{-}{-}{-}{-}{-}{-}{-}{-}{-}{-}{-}{-}{-}{-}{-}{-}{-}{-}{-}{-}{-}{-}{-}{-}{-}{-}{-}{-}{-}{-}{-}}
\NormalTok{;;  SAFE MOVEMENT HANDLER}
\NormalTok{;; {-}{-}{-}{-}{-}{-}{-}{-}{-}{-}{-}{-}{-}{-}{-}{-}{-}{-}{-}{-}{-}{-}{-}{-}{-}{-}{-}{-}{-}{-}{-}{-}{-}{-}{-}{-}{-}{-}{-}{-}{-}{-}{-}{-}{-}{-}{-}{-}{-}{-}{-}{-}{-}{-}{-}{-}{-}{-}{-}{-}{-}{-}{-}{-}{-}{-}{-}}
\NormalTok{to safe{-}move [target{-}patch move{-}speed]}
\NormalTok{  face target{-}patch}
\NormalTok{  let next{-}patch patch{-}ahead move{-}speed}

\NormalTok{  ifelse [patch{-}terrain] of next{-}patch = "water" [}
\NormalTok{    fd move{-}speed}
\NormalTok{  ] [}
\NormalTok{    let nearby{-}water one{-}of patches in{-}radius 1 with [patch{-}terrain = "water"]}
\NormalTok{    if nearby{-}water != nobody [ move{-}to nearby{-}water ]}
\NormalTok{  ]}
\NormalTok{end}
\end{Highlighting}
\end{Shaded}

\chapter{Predation and Fleeing Behavior}\label{predation-and-fleeing-behavior}

\section{Overview}\label{overview-11}

This module simulates predator--prey interactions between migratory fish species, specifically predators (i.e., striped bass) and prey (i.e., alewife), during key migratory and staging periods. Predators detect, pursue, and consume prey based on visibility conditions and prey size constraints. Prey detect approaching predators through visual and social cues and respond with directional fleeing behaviors, which vary based on their size, age, energy level, and water clarity. These behaviors reflect biologically observed trade-offs between escape speed, reaction time, and group-based vigilance. The model also tracks predator energy gain and prey energy depletion and recovery, allowing for analysis of individual fitness and risk exposure under different environmental and behavioral conditions.

\section{Purpose and Patterns}\label{purpose-and-patterns-1}

This submodel captures ecologically grounded predator--prey dynamics by simulating behaviorally realistic rules of pursuit and evasion observed in estuarine fishes.

\begin{itemize}
\item
  Prey initiate fleeing based on predator proximity, visual detection, and turbidity-adjusted reaction time, reflecting sensory limitations in murky estuarine water.
\item
  Predators only pursue prey that are within their gape limit and provide a favorable energy return, consistent with optimal foraging theory and size-selective predation.
\item
  Fleeing speed is scaled by energy and body size, capturing the biologically observed trade-off between fatigue and escape performance.
\item
  Directional fleeing (left, right, up, or down) is based on the predator's relative position, aligning with empirical studies on spatial escape responses.
\item
  Patch-level alarm cues propagate fleeing behavior across prey groups, mimicking the benefits of schooling and collective vigilance in fish.
\item
  Energetic constraints influence both predator foraging frequency and prey escape success, contributing to variability in encounter outcomes.
\item
  Spatial tracking of predation events allows identification of high-risk zones, offering insight into how environmental conditions shape survival landscapes.
\end{itemize}

\section{Entities, State Variables, and Scales}\label{entities-state-variables-and-scales-11}

\subsection{Spatial and Temporal Scales}\label{spatial-and-temporal-scales-11}

\textbf{Spatial Unit:} Patch (3 m x 3 m resolution)\\
\textbf{Temporal Unit:} 5 minute time steps (tick)

\subsection{Patch Variables}\label{patch-variables-11}

\begin{longtable}[]{@{}
  >{\raggedright\arraybackslash}p{(\linewidth - 2\tabcolsep) * \real{0.5000}}
  >{\raggedright\arraybackslash}p{(\linewidth - 2\tabcolsep) * \real{0.5000}}@{}}
\toprule\noalign{}
\begin{minipage}[b]{\linewidth}\raggedright
Variable Name
\end{minipage} & \begin{minipage}[b]{\linewidth}\raggedright
Definition
\end{minipage} \\
\midrule\noalign{}
\endhead
\bottomrule\noalign{}
\endlastfoot
\textbf{Suspended-particulate-matter} \(SPM_{t}\) & Level of suspended material in patch (exponential difficulty increase between min and max observed in system) worse vision at max, best vision at min \\
\textbf{Prey-Eaten-in-Patch} & Counts how much prey are consumed in a patch \\
\textbf{Prey-in-Patch} \(prey-in-patch_{t}\) & The amount of prey currently in the patch \\
\textbf{prey-alarmed?} & Boolean variable set to true if any prey in the patch initiates fleeing behavior, triggering collective escape. \\
\end{longtable}

\subsection{Prey: Agent Variables}\label{prey-agent-variables}

\begin{longtable}[]{@{}
  >{\raggedright\arraybackslash}p{(\linewidth - 2\tabcolsep) * \real{0.5000}}
  >{\raggedright\arraybackslash}p{(\linewidth - 2\tabcolsep) * \real{0.5000}}@{}}
\toprule\noalign{}
\begin{minipage}[b]{\linewidth}\raggedright
Variable Name
\end{minipage} & \begin{minipage}[b]{\linewidth}\raggedright
Definition
\end{minipage} \\
\midrule\noalign{}
\endhead
\bottomrule\noalign{}
\endlastfoot
vision & Radius (in patches) within which the prey can detect predators. Decreases with high SPM. \\
fleeing? & Boolean indicator of whether the prey is actively escaping from a predator. \\
energy & Energy levels of an agent \\
swimming speed & The speed at which a prey agent is moving \\
max-speed & Maximum normal swim speed (not in a flee state). \\
max-flee & Maximum achievable fleeing speed, calculated as a function of max-speed, swimming-speed, size, and max-rate-of-speed-change. \\
reaction-time & Time delay in response to predator presence. Increases with smaller size, higher SPM, larger/faster predators, and lower flee-ability. \\
size & Body length (size) of the prey; influences visibility and escape ability. \\
age & Agent's age class; mid-range age groups tend to have better escape success (a normal distribution across age classes). \\
flee-ability & Learning metric based on age; moderate-aged fish have best reflexes and performance. \\
\end{longtable}

\subsection{Predator: Agent Variables}\label{predator-agent-variables}

\begin{longtable}[]{@{}
  >{\raggedright\arraybackslash}p{(\linewidth - 2\tabcolsep) * \real{0.5000}}
  >{\raggedright\arraybackslash}p{(\linewidth - 2\tabcolsep) * \real{0.5000}}@{}}
\toprule\noalign{}
\begin{minipage}[b]{\linewidth}\raggedright
Variable Name
\end{minipage} & \begin{minipage}[b]{\linewidth}\raggedright
Definition
\end{minipage} \\
\midrule\noalign{}
\endhead
\bottomrule\noalign{}
\endlastfoot
vision & Radius (in patches) within which predator can visually detect prey. Decreases with high SPM or low prey size. \\
swimming-speed & The speed at which a predator agent is moving \\
max-speed & Maximum normal swim speed (not in a burst state). \\
bursting? & Boolean indicating whether the predator is currently in a high-speed chase or attack behavior. \\
prey-in-vision & Agentset of prey within vision range \\
daytime-prey-eaten & Count of prey eaten that day \\
time-since-full & Time since last feeding (when will predator be hungry again?) \\
reaction-time & Delay between prey detection and initiation of chase. Influenced by predator size/age, prey density, prey speed, water clarity (SPM), and predation ability. \\
limit-daily-prey-allowance & Max prey allowed per day (predator becomes full) \\
gape-limit & Maximum size of prey the predator can successfully capture and ingest. \\
handling-effort & Time and energy cost required to subdue and consume a prey item. Depends on distance, prey size, and prey density. Used in patch and prey selection decisions. \\
size & Body length (size) of the prey; influences visibility and escape ability. \\
age & Age class of the predator agent. Predation success follows a normal distribution across age classes. \\
predation-ability & Agent's age class; mid-range age groups tend to have better predation success (a normal distribution across age classes). \\
prey-species-eaten & Species identifiers of consumed prey; enables diet tracking and multispecies modeling. \\
total-prey-eaten & Cumulative prey consumed across all days or scenarios. \\
digestion-time & Time it takes to digest prey \\
\end{longtable}

\section{Process Overview and Scheduling}\label{process-overview-and-scheduling-11}

\subsection{Prey Behavior (per agent)}\label{prey-behavior-per-agent}

\begin{enumerate}
\def\labelenumi{\arabic{enumi}.}
\item
  \textbf{Fleeing Trigger}\\
  Each prey evaluates whether a predator is in its visual cone or whether \texttt{prey-alarmed?\ =\ true} in the patch. If so, the prey calculates reaction time based on its own size, age, flee-ability, local SPM, and the predator's size and speed.
\item
  \textbf{Directional Escape}\\
  If the reaction time threshold is met, the prey executes a directional fleeing behavior (\texttt{scare-left}, \texttt{scare-right}, \texttt{scare-down}, or \texttt{scare-up}), and sets the patch to \texttt{prey-alarmed?\ =\ true}.
\item
  \textbf{Schooling Response}\\
  All other prey in the same patch automatically begin fleeing when \texttt{prey-alarmed?\ =\ true}, simulating social alarm propagation.
\item
  \textbf{Speed and Energy Update}\\
  If the agent is fleeing, its speed increases toward \texttt{max-flee}, and energy is reduced based on movement cost. If not fleeing, the agent moves normally and may recover energy depending on its resting status.
\end{enumerate}

\subsection{Predator Behavior (per agent)}\label{predator-behavior-per-agent}

\begin{enumerate}
\def\labelenumi{\arabic{enumi}.}
\item
  \textbf{Visual Detection}\\
  The predator identifies prey within its in-cone vision, adjusted based on local SPM concentration, prey size, and predator size.
\item
  \textbf{Prey Filtering}\\
  Prey that exceed the predator's gape limit or whose handling effort exceeds the predator's available energy are excluded.
\item
  \textbf{Prey Selection}\\
  Among remaining prey, the predator evaluates candidates using an optimal foraging approach that considers size, handling effort, and net energy gain. One prey is selected.
\item
  \textbf{Pursuit Decision}\\
  If a prey is selected, the predator enters a bursting state, increasing its speed toward max-speed. Reaction time is computed based on predator age, size, SPM, and prey proximity.
\item
  \textbf{Scare Prey}\\
  The predator triggers fleeing in visible prey by calling the \texttt{scare-prey} procedure. If any prey flees, the patch is flagged as \texttt{prey-alarmed?\ =\ true}.
\item
  \textbf{Consume Prey}\\
  If within striking range, the predator consumes the prey and updates variables: \texttt{daytime-prey-eaten}, \texttt{prey-species-eaten}, \texttt{total-prey-eaten}, and \texttt{Prey-Eaten-in-Patch}.
\item
  \textbf{Digestion}\\
  If no prey are eaten, \texttt{time-since-eaten} increases. If the predator has reached its daily feeding limit, it enters a digestive phase. Digestion ends when both \texttt{time-since-eaten} and \texttt{time-since-full} exceed the digestion time
\end{enumerate}

\section{Design Concepts}\label{design-concepts-11}

\textbf{Basic Principles:} Simulates predator and prey interactions in a spatial estuarine system. Behaviors such as chasing, fleeing, and feeding are determined by individual traits, patch conditions, and environmental clarity due to suspended particulate matter.

\textbf{Emergence:} Predation hotspots develop in areas with higher prey density and better visibility. Prey movement and group alarm responses create shifting patterns of refuge and risk across the landscape.

\textbf{Adaptation}: Prey modify their direction and intensity of escape based on the predator's location and their own size, age, and escape ability. Predators filter out prey that are too large to consume and shift to resting behavior after reaching their feeding limit.

\textbf{Objectives:} Predators aim to select prey that provide the highest net energy gain while minimizing pursuit and handling costs. Prey aim to avoid detection or flee successfully using both individual detection and social alarm cues.

\textbf{Learning:} Learning is represented through Mid-aged agents generally perform better than very young or very old individuals.

\textbf{Prediction:}

\textbf{Sensing}: Agents rely on in-radius and in-cone vision to detect others. Detection range is influenced by water clarity, body size, and prey movement. Predators exclude prey that are too large to handle even if they fall within the visible field.

\textbf{Interaction}: Predators and prey interact through visual detection, spatial proximity, and energy transfer through feeding. Social interactions occur through alarm cue propagation at the patch level and school level.

\textbf{Stochasticity}: Prey selection, fleeing direction, and reaction timing include probabilistic variation. Randomness is used to reflect natural variability in movement and perception.

\textbf{Collectives:} When one prey detects a predator and flees, all prey in the same patch respond. This collective behavior improves predator detection and increases the likelihood of survival through schooling.

\textbf{Observation:} Key outputs include prey consumption, alarm responses, spatial predation distribution, and energy use of individual agents.

\section{Initialization}\label{initialization-11}

\begin{longtable}[]{@{}
  >{\centering\arraybackslash}p{(\linewidth - 4\tabcolsep) * \real{0.3333}}
  >{\centering\arraybackslash}p{(\linewidth - 4\tabcolsep) * \real{0.3333}}
  >{\centering\arraybackslash}p{(\linewidth - 4\tabcolsep) * \real{0.3333}}@{}}
\toprule\noalign{}
\begin{minipage}[b]{\linewidth}\centering
Variable
\end{minipage} & \begin{minipage}[b]{\linewidth}\centering
Initialized Value
\end{minipage} & \begin{minipage}[b]{\linewidth}\centering
Justification
\end{minipage} \\
\midrule\noalign{}
\endhead
\bottomrule\noalign{}
\endlastfoot
\textbf{energy} & 100 & Prey begin fully energized to allow immediate fleeing or normal movement \\
\textbf{fleeing?} & false & Prey are not actively escaping at the start \\
\textbf{prey-alarmed?} & false & Alarm cue inactive at start of simulation \\
\textbf{daytime-prey-eaten} & 0 & daily counter of prey eaten starts at zero \\
\textbf{time-since-full} & 0 & predator begins simulation hungry \\
\textbf{gape-limit} & function of size & Large predators have wider gape limits \\
\textbf{handling-effort} & calculated per encounter & Depends on prey traits and local patch conditions \\
\textbf{reaction-time} & size-, age-, and SPM-based & Prey and predator values dynamically computed each tick \\
\textbf{flee-ability} & based on age & Represents individual escape competence across age classes \\
\textbf{predation-ability} & based on age & Mid-age predators have fastest and most accurate prey responses \\
\(swim_{max}\) & \(1.5-3 \frac{body lengths}{sec}\) & Typical value for sustained swimming speed in small pelagic fish (refer to Videler, 1993). \\
\end{longtable}

\section{Submodels}\label{submodels-10}

\subsection{Prey Detection Probability}\label{prey-detection-probability}

Visual detection of prey by predators is scaled by local water clarity, prey size, and prey movement. The effective detection radius decreases as SPM increases and increases as prey size increases.

\[
P_{detect} = \left( \frac{S_{prey}}{S_{max}} \right) \cdot e^{-SPM_t / \tau}
\]

Where:

\begin{itemize}
\item
  \(P_{detect}\) is the probability of detecting a prey agent.
\item
  \(S_{prey}\) is the size of the prey agent.
\item
  \(S_{max}\) is the maximum size of any prey in the system.
\item
  \(SPM_t\) is the suspended particulate matter concentration in the current patch.
\item
  \(\tau\) is a turbidity scaling constant.
\end{itemize}

Biological Justification: Larger prey are more visible, but high turbidity (SPM) reduces contrast and visual range. The exponential function reflects rapid degradation of visibility with increased turbidity.

\subsection{Reaction Time}\label{reaction-time}

Reaction time determines how quickly an agent initiates a behavior in response to a threat (for prey) or opportunity (for predators). It is influenced by SPM, size, age, and ability.

\[
RT_{agent} = \left( \frac{1}{F_{ability}} \cdot \frac{S_{opp}}{S_{agent}} \right) \cdot (1 + \alpha \cdot SPM_t)
\]

Where:

\begin{itemize}
\item
  \(RT_{agent}\) is the reaction time of the prey or predator.
\item
  \(F_{ability}\) is flee-ability (for prey) or predation-ability (for predators).
\item
  \(S_{opp}\) is the size of the opposing agent (predator or prey).
\item
  \(S_{agent}\) is the size of the focal agent.
\item
  \(SPM_t\) is the suspended particulate matter in the patch.
\item
  \(\alpha\) is the turbidity penalty factor.
\end{itemize}

Biological Justification: Smaller or younger fish tend to react slower. High turbidity delays response time. Age-related ability improves reaction time in middle-aged agents.

\subsection{Alarm Response Propagation}\label{alarm-response-propagation}

\[
P_{alarm} = 1 - e^{\frac{-n_{flex}}{k}}
\]

Where:

\begin{itemize}
\item
  \(P_{alarm}\) is the probability that non-detecting prey will flee.
\item
  \(n_{flee}\) is the number of fleeing fish in the patch.
\item
  \(\kappa\) is a sensitivity constant governing social responsiveness.
\end{itemize}

\textbf{Biological Justification:} Prey benefit from collective vigilance. The more individuals that flee in a patch, the more likely others will follow.

\subsection{Fleeing Speed}\label{fleeing-speed}

The maximum speed a prey agent can reach when fleeing is scaled by body size, energy, and physical acceleration limits.

\[
V_{flee} = V_{max} \cdot \left( \frac{E_{agent}}{100} \right) + \delta \cdot S_{agent}
\]

Where:

\begin{itemize}
\item
  \(V_{flee}\) is the maximum fleeing speed.
\item
  \(V_{max}\) is the prey's maximum speed.
\item
  \(E_{agent}\) is the energy level of the prey agent.
\item
  \(S_{agent}\) is the size of the prey.
\item
  \(\delta\) is a scaling factor for size-based speed increase.
\end{itemize}

Biological Justification: Larger prey generally swim faster, but energy limits how much of that speed can be used. Exhausted fish flee more slowly.

\subsection{Catch Effort}\label{catch-effort}

Predators calculate the cost of pursuing and consuming prey using the prey's size and density, and the distance to the target.

\[
H_{effort} = \gamma \cdot S_{prey} \cdot D_{patch} \cdot \left(1 + \frac{1}{\rho_{patch}} \right)
\]

Where:

\begin{itemize}
\item
  \(H_{effort}\) is the handling effort for the predator.
\item
  \(S_{prey}\) is the size of the prey.
\item
  \(D_{patch}\) is the distance between predator and prey.
\item
  \(\rho_{patch}\) is the prey density in the patch.
\item
  \(\gamma\) is a handling cost coefficient.
\end{itemize}

Biological Justification: Prey that are farther away or more dispersed are harder to catch. Denser prey clusters reduce handling time by enabling rapid repeat captures.

\subsection{Gape Filtering}\label{gape-filtering}

Predators apply a binary filter before pursuit, excluding prey that exceed their maximum ingestible size.

\[
P_{pursue} =\begin{cases}1 & \text{if } S_{prey} \leq G_{pred} \\0 & \text{otherwise}\end{cases}
\]

Where:

\begin{itemize}
\item
  \(P_{pursue}\) is the pursuit decision (1 = pursue, 0 = ignore).
\item
  \(S_{prey}\) is the size of the prey.
\item
  \(G_{pred}\) is the predator's gape limit.
\end{itemize}

Biological Justification: Predators cannot capture or ingest prey that are too large, so these individuals are ignored even if they are detected.

\subsection{Energy Gain by Predators}\label{energy-gain-by-predators}

\[
E_{agent} = E_{agent} - E_{burst}
\]

Where:

\begin{itemize}
\item
  \(E_{agent}\) is the current energy of the predator.
\item
  \(E_{burst}\) is the energy cost per unit of bursting behavior.
\end{itemize}

\textbf{Biological Justification:} Bursting reduces available energy. Energy cost is a limiting factor for predation.

\subsection{Energy Depletion by Prey}\label{energy-depletion-by-prey}

\[
E_{agent} = E_{agent} - E_{flee}
\]

Where:

\begin{itemize}
\item
  \(E_{agent}\) is the current energy of the prey.
\item
  \(E_{flee}\) is the energy cost per unit of fleeing behavior.
\end{itemize}

\textbf{Biological Justification:} Fleeing reduces available energy and increases recovery time. Energy loss is a limiting factor for repeated escape attempts.

(no need to account for swimming energy in addition if you are using the migration functions)

\chapter{Model Building Tutorial: Staging \& Schooling}\label{model-building-tutorial-staging-schooling}

\subsection{Module Integration}\label{module-integration}

\begin{itemize}
\tightlist
\item
  Brief explanation of how functions (e.g., staging, schooling) interact.
\item
  Clarify temporal structure (e.g., tick-based sequence) and spatial scale.
\item
  Describe coupling logic
\end{itemize}

\subsection{Function Dependencies}\label{function-dependencies}

\begin{itemize}
\item
  What variables are required as inputs for each function?
\item
  What functions must be called before/after? (e.g., must calculate stress before checking staging triggers)
\item
  Dependency table showing variable flow between submodels.
\end{itemize}

\begin{longtable}[]{@{}
  >{\raggedright\arraybackslash}p{(\linewidth - 6\tabcolsep) * \real{0.1375}}
  >{\raggedright\arraybackslash}p{(\linewidth - 6\tabcolsep) * \real{0.2875}}
  >{\raggedright\arraybackslash}p{(\linewidth - 6\tabcolsep) * \real{0.3500}}
  >{\raggedright\arraybackslash}p{(\linewidth - 6\tabcolsep) * \real{0.2250}}@{}}
\toprule\noalign{}
\begin{minipage}[b]{\linewidth}\raggedright
Function
\end{minipage} & \begin{minipage}[b]{\linewidth}\raggedright
Required Inputs
\end{minipage} & \begin{minipage}[b]{\linewidth}\raggedright
Output Variables
\end{minipage} & \begin{minipage}[b]{\linewidth}\raggedright
Dependent On
\end{minipage} \\
\midrule\noalign{}
\endhead
\bottomrule\noalign{}
\endlastfoot
\texttt{staging} & \texttt{E\_agent}, \texttt{I\_stress} & \texttt{E\_agent}, \texttt{patch\ records} & \texttt{osmoregulation} \\
schooling & & & \\
\end{longtable}

\section{Implementation in NetLogo:}\label{implementation-in-netlogo}

Programming details

example code

\section{Implementation in R:}\label{implementation-in-r}

Programming details

example code

\chapter{Model Building Tutorial: Landward Migration \& Selective Tidal Stream Transport}\label{model-building-tutorial-landward-migration-selective-tidal-stream-transport}

\subsection{Module Integration}\label{module-integration-1}

\begin{itemize}
\tightlist
\item
  Brief explanation of how functions (e.g., staging, schooling) interact.
\item
  Clarify temporal structure (e.g., tick-based sequence) and spatial scale.
\item
  Describe coupling logic
\end{itemize}

\subsection{Function Dependencies}\label{function-dependencies-1}

\begin{itemize}
\item
  What variables are required as inputs for each function?
\item
  What functions must be called before/after? (e.g., must calculate stress before checking staging triggers)
\item
  Dependency table showing variable flow between submodels.
\end{itemize}

\begin{longtable}[]{@{}
  >{\raggedright\arraybackslash}p{(\linewidth - 6\tabcolsep) * \real{0.2500}}
  >{\raggedright\arraybackslash}p{(\linewidth - 6\tabcolsep) * \real{0.2500}}
  >{\raggedright\arraybackslash}p{(\linewidth - 6\tabcolsep) * \real{0.2500}}
  >{\raggedright\arraybackslash}p{(\linewidth - 6\tabcolsep) * \real{0.2500}}@{}}
\toprule\noalign{}
\begin{minipage}[b]{\linewidth}\raggedright
Function
\end{minipage} & \begin{minipage}[b]{\linewidth}\raggedright
Required Inputs
\end{minipage} & \begin{minipage}[b]{\linewidth}\raggedright
Output Variables
\end{minipage} & \begin{minipage}[b]{\linewidth}\raggedright
Dependent On
\end{minipage} \\
\midrule\noalign{}
\endhead
\bottomrule\noalign{}
\endlastfoot
\texttt{osmoregulation} & \texttt{S\_patch}, \texttt{S\_agent}, \texttt{C} & \texttt{I\_stress}, \texttt{C\_new}, \texttt{E\_osmo} & \texttt{environmental-sensing} \\
\texttt{staging} & \texttt{E\_agent}, \texttt{I\_stress} & \texttt{E\_agent}, \texttt{patch\ records} & \texttt{osmoregulation} \\
\texttt{migration} & \texttt{E\_agent}, \texttt{V\_patch} & \texttt{Y\_t}, \texttt{E\_agent} & \texttt{staging}, \texttt{STST} \\
\texttt{STST} & \texttt{V\_patch}, \texttt{V\_agent} & \texttt{Y\_t+1}, \texttt{E\_agent} & \texttt{migration\ logic} \\
\end{longtable}

\section{Implementation in NetLogo:}\label{implementation-in-netlogo-1}

Programming details

example code

\section{Implementation in R:}\label{implementation-in-r-1}

Programming details

example code

\chapter{Complex Model Building Tutorial: One-Way Migration}\label{complex-model-building-tutorial-one-way-migration}

\subsection{Module Integration}\label{module-integration-2}

\begin{itemize}
\tightlist
\item
  Brief explanation of how functions (e.g., staging, schooling) interact.
\item
  Clarify temporal structure (e.g., tick-based sequence) and spatial scale.
\item
  Describe coupling logic
\end{itemize}

\subsection{Function Dependencies}\label{function-dependencies-2}

\begin{itemize}
\item
  What variables are required as inputs for each function?
\item
  What functions must be called before/after? (e.g., must calculate stress before checking staging triggers)
\item
  Dependency table showing variable flow between submodels.
\end{itemize}

\section{Implementation in NetLogo:}\label{implementation-in-netlogo-2}

Programming details

example code

\section{Implementation in R:}\label{implementation-in-r-2}

Programming details

example code

\chapter{Modeling Toolkit}\label{modeling-toolkit}

\section{Learning Resources}\label{learning-resources}

\begin{itemize}
\tightlist
\item
  \textbf{NetLogo Library}: Brief note on structure (e.g., model categories), how you used it (e.g., behavior ideas, calibration).
\item
  \textbf{NetLogo User Manual}: Link to official documentation.
\item
  \textbf{NetLogo Modeling Commons}: Peer-contributed models, code sharing, and idea sourcing.
\item
  \textbf{NetLogo Forum}: Where to get help or search issues.
\item
  \textbf{Book References}:
\end{itemize}

\begin{verbatim}
-    Railsback & Grimm (2019) *Agent-Based and Individual-Based Modeling*

-    Grimm & Railsback (2005) *Individual-Based Modeling and Ecology*

-    Add others specific to modeling.
\end{verbatim}

\section{Best Practices}\label{best-practices}

\begin{itemize}
\item
  \textbf{ODD Protocol (Overview, Design concepts, Details)}: Follow for transparent and structured documentation of agent-based models. Ensures clarity across entities, processes, and assumptions.
\item
  \textbf{Modular Design}: Structure behavioral functions into separate procedures (e.g., swimming, osmoregulation, staging) to support testing, integration, and reuse.
\item
  \textbf{Version Control}: Use GitHub or other tools to track changes and link code to research outputs.
\item
  \textbf{Model File structure}:
\item
  \textbf{Journal References}:
\end{itemize}

\section{Code Repositories \& Examples}\label{code-repositories-examples}

\begin{longtable}[]{@{}
  >{\raggedright\arraybackslash}p{(\linewidth - 4\tabcolsep) * \real{0.3333}}
  >{\raggedright\arraybackslash}p{(\linewidth - 4\tabcolsep) * \real{0.3333}}
  >{\raggedright\arraybackslash}p{(\linewidth - 4\tabcolsep) * \real{0.3333}}@{}}
\toprule\noalign{}
\begin{minipage}[b]{\linewidth}\raggedright
Project
\end{minipage} & \begin{minipage}[b]{\linewidth}\raggedright
Repository
\end{minipage} & \begin{minipage}[b]{\linewidth}\raggedright
Description
\end{minipage} \\
\midrule\noalign{}
\endhead
\bottomrule\noalign{}
\endlastfoot
\textbf{Penobscot River} & GitHub link & Simulates agent-based landward \& seaward migration using STST. \\
\textbf{Martha's Vineyard} & GitHub link & Chloride cell regulation and stress-energy tradeoffs. \\
\end{longtable}

\chapter*{References}\label{references}
\addcontentsline{toc}{chapter}{References}

\chapter*{Glossary}\label{glossary}
\addcontentsline{toc}{chapter}{Glossary}

This glossary defines key terms used throughout this book.\\
Terms are organized alphabetically for clarity.

\begin{center}\rule{0.5\linewidth}{0.5pt}\end{center}

\section*{A}\label{a}
\addcontentsline{toc}{section}{A}

\subsection{\texorpdfstring{\textbf{Acclimated Salinity}}{Acclimated Salinity}}\label{acclimated-salinity}

The salinity level a fish has gradually adjusted to and can tolerate without stress.

\subsection{\texorpdfstring{\textbf{Agent}}{Agent}}\label{agent}

A virtual organism (e.g., fish) in the agent-based model that makes decisions and moves through space.

\subsection{\texorpdfstring{\textbf{Allometric}}{Allometric}}\label{allometric}

A biological scaling relationship where physiological processes such as metabolism, growth, or filtration capacity scale nonlinearly with body size and/or age rather than proportionally (linearly).

\subsection{\texorpdfstring{\textbf{Anadromous}}{Anadromous}}\label{anadromous}

Fish that hatch in freshwater, migrate to the ocean to grow, and return to freshwater to spawn.

\subsection{\texorpdfstring{\textbf{Base Energy}}{Base Energy}}\label{base-energy}

The minimum energy required for basic physiological maintenance under resting conditions.

\begin{center}\rule{0.5\linewidth}{0.5pt}\end{center}

\section*{B}\label{b}
\addcontentsline{toc}{section}{B}

\subsection{\texorpdfstring{\textbf{Benthic}}{Benthic}}\label{benthic}

Related to the bottom of a water body, where some fish feed or rest.

\subsection{\texorpdfstring{\textbf{Bioaccumulation}}{Bioaccumulation}}\label{bioaccumulation}

The gradual buildup of contaminants like mercury in a fish over time.

\subsection{\texorpdfstring{\textbf{Biomagnification}}{Biomagnification}}\label{biomagnification}

Increase in contaminant concentration as it moves up the food chain from prey to predator.

\subsection{\texorpdfstring{\textbf{Bottom Feeding}}{Bottom Feeding}}\label{bottom-feeding}

A feeding strategy where fish forage along the riverbed or estuary floor.

\subsection{\texorpdfstring{\textbf{Broadcast Spawning}}{Broadcast Spawning}}\label{broadcast-spawning}

A reproductive strategy where males and females release sperm and eggs freely into the water column.

\begin{center}\rule{0.5\linewidth}{0.5pt}\end{center}

\section*{C}\label{c}
\addcontentsline{toc}{section}{C}

\subsection{\texorpdfstring{\textbf{Chloride Cells}}{Chloride Cells}}\label{chloride-cells}

Specialized gill cells that regulate salt balance during transitions between fresh and saltwater.

\subsection{\texorpdfstring{\textbf{Contamination}}{Contamination}}\label{contamination}

Presence of toxins or pollutants (e.g., mercury) in water or sediment.

\subsection{\texorpdfstring{\textbf{Current Velocity}}{Current Velocity}}\label{current-velocity}

Speed and direction of water movement at a location.

\begin{center}\rule{0.5\linewidth}{0.5pt}\end{center}

\section*{D}\label{d}
\addcontentsline{toc}{section}{D}

\subsection{\texorpdfstring{\textbf{Diadromous}}{Diadromous}}\label{diadromous}

Fish that migrate between freshwater and saltwater as part of their life cycle.

\subsection{\texorpdfstring{\textbf{Digestion}}{Digestion}}\label{digestion-1}

The physiological process by which ingested biomass is broken down and converted into usable energy.

\subsection{\texorpdfstring{\textbf{Depth (Bathymetry)}}{Depth (Bathymetry)}}\label{depth-bathymetry}

Vertical distance from water surface to the bottom; influences habitat and hydrodynamics.

\begin{center}\rule{0.5\linewidth}{0.5pt}\end{center}

\section*{E}\label{e}
\addcontentsline{toc}{section}{E}

\subsection{\texorpdfstring{\textbf{Ebb}}{Ebb}}\label{ebb}

Outgoing tide when water flows toward the sea.

\subsection{\texorpdfstring{\textbf{Environmental Stress}}{Environmental Stress}}\label{environmental-stress}

Physiological stress caused by environmental conditions such as salinity or temperature extremes.

\begin{center}\rule{0.5\linewidth}{0.5pt}\end{center}

\section*{F}\label{f}
\addcontentsline{toc}{section}{F}

\subsection{\texorpdfstring{\textbf{Filter Feeding}}{Filter Feeding}}\label{filter-feeding-1}

Feeding method where fish strain tiny particles (plankton, detritus, SPM) from water using gill structures.

\subsection{\texorpdfstring{\textbf{Flood}}{Flood}}\label{flood}

Incoming tide when water flows inland.

\subsection{\texorpdfstring{\textbf{Flood Limit}}{Flood Limit}}\label{flood-limit}

The maximum upstream reach of tidal influence.

\begin{center}\rule{0.5\linewidth}{0.5pt}\end{center}

\section*{H}\label{h}
\addcontentsline{toc}{section}{H}

\subsection{\texorpdfstring{\textbf{Homing}}{Homing}}\label{homing}

The ability of fish to return to their birthplace to spawn.

\subsection{\texorpdfstring{\textbf{Homeostasis}}{Homeostasis}}\label{homeostasis}

Maintenance of stable internal conditions such as temperature and ion balance.

\begin{center}\rule{0.5\linewidth}{0.5pt}\end{center}

\section*{I}\label{i}
\addcontentsline{toc}{section}{I}

\subsection{\texorpdfstring{\textbf{Iteroparous}}{Iteroparous}}\label{iteroparous}

Species that can spawn multiple times across their lifespan.

\begin{center}\rule{0.5\linewidth}{0.5pt}\end{center}

\section*{L}\label{l}
\addcontentsline{toc}{section}{L}

\subsection{\texorpdfstring{\textbf{Landward Migration}}{Landward Migration}}\label{landward-migration-1}

Movement from the ocean into estuaries or rivers, often for spawning.

\subsection{\texorpdfstring{\textbf{Lipid Catabolism}}{Lipid Catabolism}}\label{lipid-catabolism}

The mobilization and breakdown of stored lipids to meet energetic demands when food intake is insufficient.

\begin{center}\rule{0.5\linewidth}{0.5pt}\end{center}

\section*{M}\label{m}
\addcontentsline{toc}{section}{M}

\subsection{\texorpdfstring{\textbf{Mercury}}{Mercury}}\label{mercury}

A toxic metal that accumulates in aquatic systems and organisms.

\subsection{\texorpdfstring{\textbf{Metabolism}}{Metabolism}}\label{metabolism-1}

The total energy expenditure required for physiological maintenance, swimming, digestion, osmoregulation, and other biological processes.

\subsection{\texorpdfstring{\textbf{Methylmercury}}{Methylmercury}}\label{methylmercury}

A highly toxic, bioaccumulative form of mercury.

\begin{center}\rule{0.5\linewidth}{0.5pt}\end{center}

\section*{O}\label{o}
\addcontentsline{toc}{section}{O}

\subsection{\texorpdfstring{\textbf{Optimal Temperature}}{Optimal Temperature}}\label{optimal-temperature}

Temperature range where fish perform best physiologically.

\subsection{\texorpdfstring{\textbf{Osmoregulation}}{Osmoregulation}}\label{osmoregulation}

Physiological process of regulating internal salt and water balance.

\subsection{\texorpdfstring{\textbf{Osmoregulatory Energy}}{Osmoregulatory Energy}}\label{osmoregulatory-energy}

Energy required to maintain ion balance under varying salinities.

\begin{center}\rule{0.5\linewidth}{0.5pt}\end{center}

\section*{P}\label{p}
\addcontentsline{toc}{section}{P}

\subsection{\texorpdfstring{\textbf{Pairwise Spawning}}{Pairwise Spawning}}\label{pairwise-spawning}

A reproductive strategy requiring a male and female to occupy the same location for fertilization.

\subsection{\texorpdfstring{\textbf{Patch}}{Patch}}\label{patch}

A discrete spatial cell in an agent-based model with its own environmental attributes.

\subsection{\texorpdfstring{\textbf{Pelagic}}{Pelagic}}\label{pelagic}

Relating to open-water habitats away from the bottom.

\begin{center}\rule{0.5\linewidth}{0.5pt}\end{center}

\section*{S}\label{s}
\addcontentsline{toc}{section}{S}

\subsection{\texorpdfstring{\textbf{Salinity}}{Salinity}}\label{salinity}

Concentration of dissolved salts in water.

\subsection{\texorpdfstring{\textbf{Salinity Stress}}{Salinity Stress}}\label{salinity-stress}

Energy strain caused when salinity deviates from a fish's preferred range.

\subsection{\texorpdfstring{\textbf{Salt Wedge}}{Salt Wedge}}\label{salt-wedge}

A dense layer of seawater intruding beneath freshwater in an estuary.

\subsection{\texorpdfstring{\textbf{Seaward Migration}}{Seaward Migration}}\label{seaward-migration-1}

Movement from rivers or estuaries toward the ocean.

\subsection{\texorpdfstring{\textbf{Selective Tidal Stream Transport (STST)}}{Selective Tidal Stream Transport (STST)}}\label{selective-tidal-stream-transport-stst}

Behavior where fish exploit tidal currents to assist migration.

\subsection{\texorpdfstring{\textbf{Semelparous}}{Semelparous}}\label{semelparous}

Species that spawn once and then die.

\subsection{\texorpdfstring{\textbf{Spawning}}{Spawning}}\label{spawning}

Release of eggs and sperm for reproduction.

\subsection{\texorpdfstring{\textbf{Staging}}{Staging}}\label{staging}

A temporary resting or holding phase in migration.

\subsection{\texorpdfstring{\textbf{Suspended Particulate Matter (SPM)}}{Suspended Particulate Matter (SPM)}}\label{suspended-particulate-matter-spm}

Fine particles (sediment, organic matter) suspended in the water column.

\subsection{\texorpdfstring{\textbf{Swimming Difficulty}}{Swimming Difficulty}}\label{swimming-difficulty-2}

Effort required to swim based on flow direction and strength.

\subsection{\texorpdfstring{\textbf{Swimming Energy}}{Swimming Energy}}\label{swimming-energy-2}

Energy expenditure required for locomotion.

\subsection{\texorpdfstring{\textbf{Swimming Speed (Velocity)}}{Swimming Speed (Velocity)}}\label{swimming-speed-velocity}

Rate of movement relative to water.

\begin{center}\rule{0.5\linewidth}{0.5pt}\end{center}

\section*{T}\label{t}
\addcontentsline{toc}{section}{T}

\subsection{\texorpdfstring{\textbf{Temperature}}{Temperature}}\label{temperature}

Water temperature; major driver of metabolism and behavior.

\subsection{\texorpdfstring{\textbf{Temperature Stress}}{Temperature Stress}}\label{temperature-stress}

Stress caused when water is too warm or too cold for optimal performance.

\subsection{\texorpdfstring{\textbf{Thermal Tolerance}}{Thermal Tolerance}}\label{thermal-tolerance}

Range of temperatures within which a fish can survive and function.

\subsection{\texorpdfstring{\textbf{Thermoregulation}}{Thermoregulation}}\label{thermoregulation}

Behavioral regulation of temperature by moving into warmer or cooler habitats.

\bibliography{Penobscot.bib}

\end{document}
