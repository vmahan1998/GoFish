% Options for packages loaded elsewhere
\PassOptionsToPackage{unicode}{hyperref}
\PassOptionsToPackage{hyphens}{url}
\documentclass[
]{book}
\usepackage{xcolor}
\usepackage{amsmath,amssymb}
\setcounter{secnumdepth}{5}
\usepackage{iftex}
\ifPDFTeX
  \usepackage[T1]{fontenc}
  \usepackage[utf8]{inputenc}
  \usepackage{textcomp} % provide euro and other symbols
\else % if luatex or xetex
  \usepackage{unicode-math} % this also loads fontspec
  \defaultfontfeatures{Scale=MatchLowercase}
  \defaultfontfeatures[\rmfamily]{Ligatures=TeX,Scale=1}
\fi
\usepackage{lmodern}
\ifPDFTeX\else
  % xetex/luatex font selection
\fi
% Use upquote if available, for straight quotes in verbatim environments
\IfFileExists{upquote.sty}{\usepackage{upquote}}{}
\IfFileExists{microtype.sty}{% use microtype if available
  \usepackage[]{microtype}
  \UseMicrotypeSet[protrusion]{basicmath} % disable protrusion for tt fonts
}{}
\makeatletter
\@ifundefined{KOMAClassName}{% if non-KOMA class
  \IfFileExists{parskip.sty}{%
    \usepackage{parskip}
  }{% else
    \setlength{\parindent}{0pt}
    \setlength{\parskip}{6pt plus 2pt minus 1pt}}
}{% if KOMA class
  \KOMAoptions{parskip=half}}
\makeatother
\usepackage{longtable,booktabs,array}
\usepackage{multirow}
\usepackage{calc} % for calculating minipage widths
% Correct order of tables after \paragraph or \subparagraph
\usepackage{etoolbox}
\makeatletter
\patchcmd\longtable{\par}{\if@noskipsec\mbox{}\fi\par}{}{}
\makeatother
% Allow footnotes in longtable head/foot
\IfFileExists{footnotehyper.sty}{\usepackage{footnotehyper}}{\usepackage{footnote}}
\makesavenoteenv{longtable}
\usepackage{graphicx}
\makeatletter
\newsavebox\pandoc@box
\newcommand*\pandocbounded[1]{% scales image to fit in text height/width
  \sbox\pandoc@box{#1}%
  \Gscale@div\@tempa{\textheight}{\dimexpr\ht\pandoc@box+\dp\pandoc@box\relax}%
  \Gscale@div\@tempb{\linewidth}{\wd\pandoc@box}%
  \ifdim\@tempb\p@<\@tempa\p@\let\@tempa\@tempb\fi% select the smaller of both
  \ifdim\@tempa\p@<\p@\scalebox{\@tempa}{\usebox\pandoc@box}%
  \else\usebox{\pandoc@box}%
  \fi%
}
% Set default figure placement to htbp
\def\fps@figure{htbp}
\makeatother
\setlength{\emergencystretch}{3em} % prevent overfull lines
\providecommand{\tightlist}{%
  \setlength{\itemsep}{0pt}\setlength{\parskip}{0pt}}
\usepackage[]{natbib}
\bibliographystyle{plainnat}
\usepackage{booktabs}
\usepackage{bookmark}
\IfFileExists{xurl.sty}{\usepackage{xurl}}{} % add URL line breaks if available
\urlstyle{same}
\hypersetup{
  pdftitle={GoFish: A Next-Generation Toolkit for Modeling Migratory Fish and Environmental Risk in Estuaries},
  pdfauthor={Vanessa Quintana},
  hidelinks,
  pdfcreator={LaTeX via pandoc}}

\title{GoFish: A Next-Generation Toolkit for Modeling Migratory Fish and Environmental Risk in Estuaries}
\author{Vanessa Quintana}
\date{2025-07-04}

\begin{document}
\maketitle

{
\setcounter{tocdepth}{1}
\tableofcontents
}
\chapter*{Preface}\label{preface}
\addcontentsline{toc}{chapter}{Preface}

This library provides a comprehensive, modular framework for developing and documenting agent-based models (ABMs) that simulate the movement, behavior, and environmental interactions of migratory fish in coastal aquatic systems. It is designed to support the standardization and implementation of ABMs in fisheries management, enabling researchers and practitioners to address complex environmental questions and evaluate remediation or restoration scenarios.

\section*{Motivation}\label{motivation}
\addcontentsline{toc}{section}{Motivation}

The goal of this resource is to support students, researchers, and decision-makers by making agent-based modeling of migratory fish more accessible, reproducible, and applicable to real-world fisheries and habitat management challenges. By providing a standardized modular framework for key behavioral processes, this library promotes consistency, transparency, and credibility in ecological forecasting and decision support tools. It also establishes a foundation for critical conversations about the behaviors and functions represented in modelling, while supporting the empirical quantification of ecological relationships that influence movement, survival, and habitat use of migratory fish.

\chapter{Overview}\label{overview}

\section{Background}\label{background}

Anadromous fish species such as river herring, striped bass, and sturgeon navigate coastal and estuarine systems that are increasingly affected by human activity, climate change, and legacy contaminants. Modeling their movement and behavior at fine spatial and temporal scales requires tools that can integrate physiological stressors, environmental variability, and behavior-based decision-making.

Agent-based models are among the most powerful tools available for ecological forecasting and fisheries management, but they are also among the most complex. Their structure and computational demands can make them difficult to apply in practical management settings. Many biologists and ecologists who hold deep, species-specific expertise often have limited training in advanced programming or systems modeling. This is partly due to gaps in secondary and post-secondary education, where exposure to high-level mathematics and coding is often minimal, despite the fact that many ecological processes are governed by nonlinear systems and feedback.

As a result, traditional approaches to modeling anadromous fish frequently oversimplify or exclude key biological functions such as osmoregulation, schooling behavior, and contaminant exposure. Many existing models also lack standardized representations of these behaviors, limiting the applicability of results and reducing their usefulness for applied management. This function library was developed to address these limitations by offering modular, empirically grounded components designed for use in modeling anadromous fish. Each function is clearly documented and can be applied independently, allowing for transparent testing, modification, and reuse across a wide range of ecological scenarios and sites.

\section{Introduction to Agent-Based Models}\label{introduction-to-agent-based-models}

provide basics of ABM

\section{Structure}\label{structure}

Each chapter in this library corresponds to a major behavior or physiological function relevant to migratory fish, including:

\begin{itemize}
\item
  \textbf{Osmoregulation}
\item
  \textbf{Bioaccumulation of contaminants}
\item
  \textbf{Directional migration (landward and seaward)}
\item
  \textbf{Schooling and Staging}
\item
  \textbf{Selective Tidal Stream Transport}
\item
  \textbf{Homing behavior}
\item
  \textbf{Foraging}
\item
  \textbf{Predator-prey interactions}
\item
  \textbf{Spawning}
\end{itemize}

The final chapter provides guidance on how to integrate multiple functions into a complete agent-based model, demonstrating how these components work together to simulate fish behavior in dynamic coastal and estuarine systems.

Within each chapter, each function or behavior is documented using the ODD protocol (Grimm et al., 2006; 2010; 2020). The ODD (Overview, Design concepts, Details) protocol is a standardized framework for describing agent-based models. It promotes transparency in model development and ensures consistency across implementations, especially when integrating multiple behavioral or ecological functions.

\begin{itemize}
\item
  \textbf{Overview} provides the purpose of the model component, identifies the entities involved (e.g., fish agents, environmental patches), and outlines the general processes.
\item
  \textbf{Design} concepts describe the key theoretical underpinnings such as emergence, adaptation, objectives, sensing, stochasticity, and interaction.
\item
  \textbf{Details} specify initialization steps, input data requirements, and the rules or submodels that govern behavior.
\end{itemize}

By following the ODD protocol, this library ensures that each function is self-contained, interpretable, and ready for adaptation to a wide range of species, sites, or management scenarios.

\section{Application Context}\label{application-context}

This library was originally developed in support of research on the influence of tidal behavior and contaminant exposure on anadromous fish in the Penobscot River Estuary. However, its modular design allows for application to other estuarine and coastal systems where fish respond to environmental changes (i.e., salinity, velocity, and pollutants).

\emph{Can include addition project or model links here*}

\section{How to Use This Library}\label{how-to-use-this-library}

Each function or behavior in this library can be combined with others to build a complete agent-based model for anadromous fish. These functions are designed to be adaptable, and easily configured for different species, life stages, or site-specific conditions.

\emph{For questions, feedback, guidance on implementation, or \textbf{interest in adding to the library}, please contact \textbf{Vanessa Quintana} at \textbf{\href{mailto:mahan.vanessa98@gmail.com}{\nolinkurl{mahan.vanessa98@gmail.com}}.}}

\chapter{Osmoregulation Function}\label{osmoregulation-function}

\section{Overview}\label{overview-1}

Osmoregulation allows migratory fish to maintain homeostasis by regulating internal ion concentrations in response to varying environmental salinities. This function simulates osmotic or ion-regulatory stress, chloride cell expression, and the metabolic energy cost of osmoregulation in a spatially explicit context.

\section{Purpose}\label{purpose}

To simulate stress response to salinity changes for migratory fish in coastal systems by regulating chloride cell density and allocating energy toward ion-regulatory processes.

\section{Entities, State Variables, and Scales}\label{entities-state-variables-and-scales}

\subsection{Patch Variables}\label{patch-variables}

\begin{longtable}[]{@{}
  >{\raggedright\arraybackslash}p{(\linewidth - 2\tabcolsep) * \real{0.5000}}
  >{\raggedright\arraybackslash}p{(\linewidth - 2\tabcolsep) * \real{0.5000}}@{}}
\toprule\noalign{}
\begin{minipage}[b]{\linewidth}\raggedright
Variable Name
\end{minipage} & \begin{minipage}[b]{\linewidth}\raggedright
Definition
\end{minipage} \\
\midrule\noalign{}
\endhead
\bottomrule\noalign{}
\endlastfoot
\textbf{Salinity} \(S_{patch}\) & The salt concentration of a given patch, derived from hydrodynamic model inputs. \\
\end{longtable}

\subsection{Agent Variables}\label{agent-variables}

\begin{longtable}[]{@{}
  >{\raggedright\arraybackslash}p{(\linewidth - 2\tabcolsep) * \real{0.5000}}
  >{\raggedright\arraybackslash}p{(\linewidth - 2\tabcolsep) * \real{0.5000}}@{}}
\toprule\noalign{}
\begin{minipage}[b]{\linewidth}\raggedright
Variable Name
\end{minipage} & \begin{minipage}[b]{\linewidth}\raggedright
Definition
\end{minipage} \\
\midrule\noalign{}
\endhead
\bottomrule\noalign{}
\endlastfoot
\textbf{acclimated-salinity} \(S_{agent}\) & The salinity level the agent is currently acclimated to. \\
\textbf{ionregulatory-stress} \(I_{stress}\) & The level of stress an agent experiences when regulating ion balance due to osmotic difference. \\
\textbf{chloride-density-min} \(C_{min}\) & Minimum level of chloride cells, present even in low-stress conditions. \\
\textbf{chloride-density-max} \(C_{max}\) & Maximum level of chloride cells at high stress. \\
\textbf{chloride-cell-density} \(C\) & The current number of chloride cells expressed by the agent. \\
\textbf{chloride-max-proliferation} \(R_{proliferation}\) & The max number of chloride cells that can be expressed per time step. \\
\textbf{chloride-cells-this-tick} \(C_{tick}\) & The number of chloride cells created (or destroyed) in the current time step. \\
\textbf{acclimation-rate} \(\alpha\) & The rate at which chloride cell density increases over time. \\
\textbf{C-mid} \(C_{mid}\) & The chloride cell density at which stress buffering is 50\% effective. \\
\textbf{time-since-last-osmoregulation} \(t_{osmo}\) & The time elapsed since the last chloride cell regulation event. \\
\textbf{Energy} \(E_{agent}\) & The agent's total available energy for physiological functions. \\
\textbf{E-osmo} \(E_{osmo}\) & Total energy used for ion regulation (osmoregulation). \\
\textbf{E-base} \(E_{base}\) & The base energy cost per chloride cell. \\
\textbf{E-creation} \(E_{creation}\) & The energy cost for producing new chloride cells. \\
\textbf{metabolic-max} \(Met_{max}\) & Maximum metabolic cost for chloride cell creation. \\
\end{longtable}

\section{Process Overview and Scheduling}\label{process-overview-and-scheduling}

\begin{enumerate}
\def\labelenumi{\arabic{enumi}.}
\item
  Compute osmotic stress based on difference between \(S_{patch}\) and \(S_{agent}\).
\item
  Adjust chloride cell density depending on time since last osmoregulation.
\item
  Compute energy cost of osmoregulation.
\item
  Deduct energy expenditure from agent's energy pool.
\end{enumerate}

\section{Design Concepts}\label{design-concepts}

\textbf{Basic Principles:} The model is based on principles of physiological ecology and osmoregulatory energetics in teleost and apterygian species. It draws from empirical findings (e.g., Allen et al., 2009; Little et al., 2023) and includes size scaling, stress buffering, and energy constraints. These principles are implemented at the submodel level (e.g., chloride proliferation, stress calculation) to simulate realistic physiological feedbacks to changes in environmental salinity.

\textbf{Emergence:} Ion-regulatory stress, chloride cell expression, and energy expenditure emerge from an agent's interaction with temporally and spatially variable salinity environments. These patterns are not pre-specified but arise dynamically through adaptive physiological responses.

\textbf{Adaptation}: Agents respond to osmotic stress by adjusting chloride cell density, a trait that buffers stress. This process allows individuals to reduce internal-external salinity gradients and maintain ion homeostasis.

\textbf{Objectives:} Agents seek to support survival by reducing stress and avoiding excessive energy loss through regulating chloride cell expression.

\textbf{Sensing}: Agents sense local salinity (\(S_{patch}\)) and compare it with their acclimated salinity (\(S_{agent}\)). They also track their own energy state and time since last osmoregulation.

\textbf{Stochasticity}: Acclimation may vary with \(\alpha\), which can be drawn from a defined range per individual to reflect physiological variation across the population.

\textbf{Observation:} Outputs include \(I_{stress}\), \(C\), \(E_{osmo}\), and \(E_{agent}\), all tracked per individual and exportable for analysis or visualization.

\section{Initialization}\label{initialization}

\begin{longtable}[]{@{}
  >{\centering\arraybackslash}p{(\linewidth - 4\tabcolsep) * \real{0.1034}}
  >{\centering\arraybackslash}p{(\linewidth - 4\tabcolsep) * \real{0.2069}}
  >{\centering\arraybackslash}p{(\linewidth - 4\tabcolsep) * \real{0.6828}}@{}}
\toprule\noalign{}
\begin{minipage}[b]{\linewidth}\centering
Variable
\end{minipage} & \begin{minipage}[b]{\linewidth}\centering
Initialized Value
\end{minipage} & \begin{minipage}[b]{\linewidth}\centering
Justification
\end{minipage} \\
\midrule\noalign{}
\endhead
\bottomrule\noalign{}
\endlastfoot
\(S_{patch}\) & user-defined for data input & This input can be user-defined realistic data values or known spatial data. \\
\(S_{agent}\) & 35 (psu) & Assumes agents start acclimated to marine environment. \\
\(I_{stress}\) & 1 & Acclimated agents have minimal stress levels. \\
\(C\) & 50\% & Starts with partial cell density, allowing for regulation depending on environmental conditions. \\
\(C_{min}\) & 10\% & A baseline level of chloride cells is necessary for basic osmoregulatory functions. \\
\(C_{max}\) & 100\% & Agents can't express more than 100\% of cells. \\
\(\alpha\) & 0.0017 - 0.002 & Osmolarity stabilization from Figure 3. in (Allen et al., 2009). \\
\(C_{mid}\) & 50\% & When cells are 50\% density, stress buffering is 50\% effective (Allen et al., 2009). \\
\(E_{agent}\) & 100\% & Agent starts with limited energy before migration. \\
\(E_{base}\) & Teleost (4\%)

Aptoerygian () & Based on the \textbf{branchial cost} (Little et al., 2023; Kirschner, 1993). \\
\(Met_{max}\) & Teleost (3.5\%) & Based on the intestinal and renal cost \& size of agent (Little et al., 2023; Kirschner, 1993). \\
\(k\) & -0.75 & Scaling component for body mass is negative (Kirschner, 1993) and follows Kleiber's Law. \\
\end{longtable}

\section{Submodels}\label{submodels}

\subsection{Osmotic Stress}\label{osmotic-stress}

Ion-regulatory stress (\(I_{stress}\)) is calculated based on the difference between an agent's acclimated salinity and the ambient patch salinity, adjusted by the chloride cell buffering effect:

\[
I_{stress} = \frac{\log_{10}(1 + |S_{agent} - S_{patch}|) \cdot 10}{1 + e^{-2 \cdot (C / C_{mid})}}
\]

Stress is capped within the range {[}1, 10{]}, and may be reduced slightly over time if salinity remains stable and chloride density is sufficient:

\[
I_{stress} = I_{stress} \cdot 0.98 \quad \text{if conditions are stable and } C > C_{min}
\]

Agents also slowly shift their acclimated salinity toward ambient salinity when conditions have been stable for several time steps:

\[
S_{agent} = S_{agent} + (S_{patch} - S_{agent}) \cdot 0.02
\]

Where:

\begin{itemize}
\tightlist
\item
  \(I_{stress}\) is ion-regulatory (osmotic) stress, scaled between 1 and 10.
\item
  \(S_{agent}\) is the agent's acclimated salinity.
\item
  \(S_{patch}\) is the environmental salinity at the current patch.
\item
  \(C\) is the chloride cell density (percent of maximum).
\item
  \(C_{mid}\) is the density at which buffering is 50\% effective.
\end{itemize}

\subsection{Chloride Cell Density}\label{chloride-cell-density}

Chloride cell proliferation is driven by the level of ion-regulatory stress the agent experiences when encountering a difference in salinity. The greater the stress, the higher the target chloride density the agent attempts to reach, up to a maximum threshold. Agents adjust their chloride cell density based on their current ion-regulatory stress and acclimation status. Chloride cells are not adjusted unless the agent's energy exceeds 25\%.

The chloride cell density is based on stress:

\[
C_{target} = C_{min} + (C_{max} - C_{min}) \cdot \left(\frac{I_{stress}}{10}\right)
\]

If salinity conditions have remained stable for an extended period (e.g., 288 ticks), \textbackslash(C\_\{target\}\textbackslash) is slightly reduced to reflect partial downregulation of chloride cells due to long-term acclimation:

\[
C_{target} = C_{target} \cdot 0.99 \quad \text{if stable}
\]

Chloride cell density then approaches the target using a double-rate adjustment and capped maximum rate of change:

\[
\Delta C = \left(C_{target} - C_{current}\right) \cdot \left(2 \cdot R_{proliferation}\right)
\]

\[
\Delta C = \max\left(-R_{max}, \min(R_{max}, \Delta C)\right)
\]

If the agent has low energy (\(\leq 50\%\)), the adjustment rate is halved:

\[
\Delta C = \Delta C \cdot 0.5 \quad \text{if energy is low}
\]

Finally, the chloride cell density is updated and constrained between \(C_{min}\) and \(C_{max}\):

\[
C_{new} = \max(C_{min}, \min(C_{max}, C_{current} + \Delta C))
\]

This ensures that the agent does not overshoot the physiologically realistic limit of chloride cell density, while still responding to osmotic stress.

Chloride density is only recalculated after a given acclimation interval:

\[
t_{osmo} \geq \alpha^{-1}
\]

After updating, the acclimation timer is reset:

\[
t_{osmo} = 0
\]

This prevents agents from recalculating chloride density every time step and allows for controlled, realistic responses to prolonged stress and salinity changes.

Where:

\begin{itemize}
\tightlist
\item
  \(I_{stress}\) is the ion-regulatory stress, scaled from 1 to 10.
\item
  \(C_{target}\) is the desired chloride cell density based on stress level.
\item
  \(C_{min}\) and \(C_{max}\) are the bounds for chloride cell density.
\item
  \(R_{proliferation}\) determines the \textbf{maximum allowable increase} per time step.
\item
  \(\Delta C\) is the rate of change in chloride cell expression.
\item
  \(R_{max} = (C_{max} - C_{min}) \cdot R_{proliferation}\)
\item
  \(C_{new}\) is the percent of new chloride cell expression.
\item
  \(\alpha\) is the acclimation rate constant.
\item
  \(t_{osmo}\) represents time since the last osmoregulation event.
\end{itemize}

\subsection*{Osmoregulation Energy}\label{osmoregulation-energy}
\addcontentsline{toc}{subsection}{Osmoregulation Energy}

Metabolic cost related to size:

\[
E_{creation} = Met_{max} * (\frac{M}{M_{max}})^k
\]

Where:

\begin{itemize}
\item
  \(E_{creation}\) is the energy cost of chloride cell creation
\item
  \(Met_{max}\) is the maximum metabolic cost of the agent
\item
  \(M\) is equal to the agent's size, where smaller fish spend proportionally more energy on osmoregulation (Little et al., 2023)
\item
  \(M_{max}\) is the maximum mass of an agent within the population
\item
  \(k\) follows size-dependent variation in energy allocation, consistent with a negative scaling exponent.
\end{itemize}

Energy required for ion regulation:

\[
E_{osmo} = (E_{base} \cdot C) + (E_{creation} \cdot C_{tick})
\]

Where:

\begin{itemize}
\item
  \(E_{base}\) represents the energy cost per chloride cell for maintenance.
\item
  \(C\) is the current chloride density.
\item
  \(E_{creation}\) represents the cost of producing new chloride cells.
\item
  \(C_{tick}\) is the number of newly created chloride cells in the current time step.
\end{itemize}

\subsection{Energy Balance}\label{energy-balance}

Agents balance energy to osmoregulate with total energy allowance:

\[
E_{agent} = E_{agent} - E_{osmo}
\]

Where:

\begin{itemize}
\item
  \(E_{agent}\) is the total energy of the agent.
\item
  \(E_{osmo}\) is the energy consumed during osmoregulation.
\end{itemize}

\chapter{Thermoregulation Function}\label{thermoregulation-function}

\section{Overview}\label{overview-2}

This model simulates thermoregulation behavior and physiological consequences in migratory fish agents. Each agent experiences temperature-dependent effects on movement, survival, and performance based on its local environment. Agents have defined thermal traits including an optimal temperature, a minimum viable temperature, and a maximum tolerable temperature. These constraints influence swimming speed and determine whether survival is possible in the current patch.

\section{Purpose and Patterns}\label{purpose-and-patterns}

This submodel captures ecologically grounded thermoregulatory behavior in migratory fish by simulating how temperature influences movement, survival, and physiological performance during migration and staging.

\begin{itemize}
\item
  Agents experience thermal stress based on deviations from their optimal temperature, with stress scaling nonlinearly as conditions approach species-specific tolerance thresholds.
\item
  Swimming speed is reduced as environmental temperature deviates from thermal optima, reflecting empirical performance curves and diminished locomotor efficiency at suboptimal temperatures.
\item
  Agents calculate energy expenditure associated with thermoregulation, with increased metabolic cost arising from reduced swimming efficiency under thermal stress.
\item
  When thermal stress exceeds a threshold, agents evaluate neighboring patches and relocate to cooler or warmer areas to minimize physiological strain.
\item
  Agents track cumulative time spent beyond minimum and maximum viable temperature thresholds, and mortality occurs after sustained exposure to lethal conditions.
\item
  This model supports the analysis of thermal niche constraints, temperature-driven habitat selection, and the vulnerability of fish to extreme temperature events in estuarine systems.
\end{itemize}

\section{Entities, State Variables, and Scales}\label{entities-state-variables-and-scales-1}

\subsection{Spatial and Temporal Scales}\label{spatial-and-temporal-scales}

\begin{itemize}
\tightlist
\item
  \textbf{Spatial Unit}: Patch (3 m x 3 m resolution)
\item
  \textbf{Temporal Unit}: 5-minute time steps (\texttt{tick})
\end{itemize}

\subsection{Patch Variables}\label{patch-variables-1}

\begin{longtable}[]{@{}
  >{\raggedright\arraybackslash}p{(\linewidth - 2\tabcolsep) * \real{0.3153}}
  >{\raggedright\arraybackslash}p{(\linewidth - 2\tabcolsep) * \real{0.6847}}@{}}
\toprule\noalign{}
\begin{minipage}[b]{\linewidth}\raggedright
Variable Name
\end{minipage} & \begin{minipage}[b]{\linewidth}\raggedright
Definition
\end{minipage} \\
\midrule\noalign{}
\endhead
\bottomrule\noalign{}
\endlastfoot
\begin{minipage}[t]{\linewidth}\raggedright
\textbf{Temperature}\\
\(T_{patch}\)\strut
\end{minipage} & The temperature of a given patch, derived from hydrodynamic model inputs. \\
\end{longtable}

\subsection{Agent Variables}\label{agent-variables-1}

\begin{longtable}[]{@{}
  >{\raggedright\arraybackslash}p{(\linewidth - 2\tabcolsep) * \real{0.3468}}
  >{\raggedright\arraybackslash}p{(\linewidth - 2\tabcolsep) * \real{0.6532}}@{}}
\toprule\noalign{}
\begin{minipage}[b]{\linewidth}\raggedright
Variable Name
\end{minipage} & \begin{minipage}[b]{\linewidth}\raggedright
Definition
\end{minipage} \\
\midrule\noalign{}
\endhead
\bottomrule\noalign{}
\endlastfoot
\textbf{Optimal Temperature} \(T_{opt}\) & Optimal performance temperature \\
\textbf{Minimum Viable Temperature} \(T_{min}\) & Minimum viable temperature for survival \\
\textbf{Maximum Viable Temperature} \(T_{max}\) & Maximum viable temperature for survival \\
\textbf{swimming speed} \(V_{agent}\) & The speed at which a prey agent is moving \\
\textbf{corrected speed} \(V_{corr}\) & The corrected speed from thermal implications \\
\textbf{time below minimum} \(t_{min}\) & Tracks cumulative cold exposure duration \\
\textbf{time above maximum} \(t_{max}\) & Tracks cumulative heat exposure duration \\
\textbf{energy} \(E_{agent}\) & Metabolic cost associated with thermoregulation \\
\textbf{swimming speed} \(V_{agent}\) & The current swimming speed of the agent. \\
\textbf{acclimation rate} \(\lambda\) & Rate at which agent acclimates to temperature based on size, age, and species. \\
\end{longtable}

\section{Process Overview and Scheduling}\label{process-overview-and-scheduling-1}

\begin{enumerate}
\def\labelenumi{\arabic{enumi}.}
\item
  \textbf{Thermal Stress Calculation}\\
  Each agent calculates thermal stress based on the deviation between the current patch temperature and its individual thermal optimum. Stress is scaled nonlinearly to reflect increasing physiological strain near temperature limits.
\item
  \textbf{Acclimation}\\
  Agents gradually adjust their optimal temperature toward the current patch temperature using a fixed acclimation rate, simulating physiological plasticity over time.
\item
  \textbf{Thermal Mortality Check}\\
  Agents track cumulative exposure to lethal temperatures. If the patch temperature falls below the agent's minimum threshold or exceeds its maximum threshold for more than three consecutive time steps, the agent dies.
\item
  \textbf{Swimming Speed Correction}\\
  Swimming speed is modified using a thermal correction factor derived from the agent's deviation from its optimal temperature. This adjustment affects movement performance and downstream behaviors.
\item
  \textbf{Thermal Avoidance}\\
  If thermal stress exceeds a threshold value (e.g., \$S\_\{thermal\} \textgreater{} 7\$), agents evaluate neighboring patches and move to the patch with the lowest thermal stress.
\item
  \textbf{Energy Cost Adjustment}\\
  Swimming energy cost is scaled based on the ratio of thermally corrected speed to the agent's base swimming speed. This reflects increased metabolic effort under suboptimal thermal conditions.
\end{enumerate}

\section{Design Concepts}\label{design-concepts-1}

\textbf{Basic Principles:} Agents are ectothermic and thermally constrained. Movement and survival depend on physiological performance at environmental temperatures.

\textbf{Emergence:} Spatial thermal gradients produce differential survival, movement, and performance outcomes across the landscape.

\textbf{Adaptation}: Agents modify their optimal temperature slowly to match local conditions, simulating physiological acclimation.

\textbf{Objectives:} Agents attempt to avoid lethal temperatures and maintain performance within their thermal niche while minimizing energy loss.

\textbf{Sensing}: Agents sense local patch temperature and compare it to their thermal limits and optimal range.

\textbf{Interaction}: Agents interact with their environment by relocating to lower-stress patches and adjusting internal states accordingly.

\textbf{Observation:} Thermal stress, adjusted swimming speed, energy cost, cumulative heat/cold exposure, and thermal mortality can be tracked per agent.

\section{Initialization}\label{initialization-1}

\begin{longtable}[]{@{}
  >{\raggedright\arraybackslash}p{(\linewidth - 6\tabcolsep) * \real{0.1714}}
  >{\raggedright\arraybackslash}p{(\linewidth - 6\tabcolsep) * \real{0.2857}}
  >{\raggedright\arraybackslash}p{(\linewidth - 6\tabcolsep) * \real{0.5143}}
  >{\raggedright\arraybackslash}p{(\linewidth - 6\tabcolsep) * \real{0.0143}}@{}}
\toprule\noalign{}
\endhead
\bottomrule\noalign{}
\endlastfoot
\multirow{2}{=}{Variable
:=====================:
\(T_{opt}\)} & \multicolumn{3}{>{\raggedright\arraybackslash}p{(\linewidth - 6\tabcolsep) * \real{0.8143} + 4\tabcolsep}@{}}{%
\multirow{2}{=}{Initialized Value \textbar{} Justification \textbar{}
:=====================================:+:=====================================================================:+
Species- or life stage--specific value \textbar{} Reflects thermal preference; varies by developmental stage or habitat \textbar{}}} \\
 \\
\(T_{min}\) & Species-specific threshold & Based on empirical lower thermal limit for survival & \\
\(T_{max}\) & Species-specific threshold & Based on empirical upper thermal limit for survival & \\
\(T_{patch}\) & Hydrodynamic model input & External environmental condition updated each tick & \\
\(S_{thermal}\) & 1 & No stress at initialization if \$T\_\{patch\} = T\_\{opt\}\$ & \\
\(t_{min}\) & 0 & No prior exposure to lethal cold stress & \\
\(t_{max}\) & 0 & No prior exposure to lethal heat stress & \\
\(V_{agent}\) & Size-scaled base speed & Initial swimming speed assuming full capacity & \\
\(E_{agent}\) & 100 & No thermal deviation at model start & \\
\(E_{swim}\) & Size-scaled base swim cost & Scales with body size; used in movement energy balance & \\
\(E_{swim}^{corr}\) & Equal to \(E_{swim}\) & No thermal penalty at initialization & \\
\(\lambda\) & \multicolumn{3}{>{\raggedright\arraybackslash}p{(\linewidth - 6\tabcolsep) * \real{0.8143} + 4\tabcolsep}@{}}{%
0.01--0.05 (user-defined) \textbar{} Acclimation rate controls speed of physiological adjustment} \\
\end{longtable}

\section{Submodels}\label{submodels-1}

\subsection{Thermal Stress}\label{thermal-stress}

Thermal stress is scaled exponentially between 1 (optimal) and 10 (at or beyond minimum or maximum viable temperatures).

\[
S_{thermal} = 1 + 9 \cdot \left(\frac{|T_{patch} - T_{opt}|}{\max(|T_{opt} - T_{min}|, |T_{max} - T_{opt}|)}\right)^k
\]

Where:

\begin{itemize}
\item
  \(S_{thermal}\) is the thermal stress score.
\item
  \(T_{patch}\) is the current patch temperature.
\item
  \(T_{opt}\), \(T_{min}\), \(T_{max}\) are the agent's thermal preferences.
\item
  \(k\) is an exponent controlling the steepness of the stress curve.
\end{itemize}

\subsection{Acclimation and Plasticity}\label{acclimation-and-plasticity}

\[
T_{opt}(t+1) = T_{opt}(t) + \lambda \cdot (T_{patch} - T_{opt}(t))
\]

Where:

\begin{itemize}
\item
  \(T_{opt}\) gradually shifts toward \(T_{patch}\)
\item
  \(\lambda\) is the acclimation rate (e.g.~0.01-0.05)
\end{itemize}

\subsection{Temperature-Adjusted Swimming Speed (place inside migration function(s))}\label{temperature-adjusted-swimming-speed-place-inside-migration-functions}

\[
V_{corr} = V_{agent} \cdot \left(1 - \left|\frac{T_{patch} - T_{opt}}{T_{opt}}\right|\right)
\]

Where:

\begin{itemize}
\item
  \(V_{agent}\) is the agent's swimming speed.
\item
  \(V_{corr}\) is the thermal corrected speed.
\item
  \(T_{patch}\) is the temperature of the current patch.
\item
  \(T_{opt}\) is the agent's optimal temperature.
\end{itemize}

\subsection{Thermal-Avoidance Movement (place inside migration function(s))}\label{thermal-avoidance-movement-place-inside-migration-functions}

Agents check thermal stress in their current patch. If \(S_{thermal} > 7\), they evaluate neighboring patches. The agent moves to the neighbor with the lowest \(S_{thermal}\) value.

Where:

\begin{itemize}
\tightlist
\item
  \(S_{thermal}\) is the thermal stress score.
\end{itemize}

\subsection{Thermal Mortality Thresholds (place inside migration function(s))}\label{thermal-mortality-thresholds-place-inside-migration-functions}

\(t_{min}\) and \(t_{max}\) are incremented when temperature thresholds are exceeded:

\[
\text{if } T_{patch} < T_{min} \text{ for more than 3 consecutive time steps}, \quad \text{agent dies}
\]

\[
\text{if } T_{patch} > T_{max} \text{ for more than 3 consecutive time steps}, \quad \text{agent dies}
\]

Where:

\begin{itemize}
\item
  \(T_{min}\) is the minimum viable temperature.
\item
  \(T_{max}\) is the maximum tolerable temperature.
\end{itemize}

\subsection{Energy Consumption (place inside migration function(s))}\label{energy-consumption-place-inside-migration-functions}

Acclimation energy cost:

\[
E_{acclimation} = \phi \cdot |T_{patch} - T_{opt}|\]

Where:

\begin{itemize}
\item
  \(E_{acclimation}\) is the energy cost for physiological acclimation.
\item
  \(T_{patch}\) is the current patch temperature.
\item
  \(\phi\) is a scaling constant for acclimation cost (e.g., 0.01--0.05).
\item
  \(T_{opt}\) is the agent's current thermal optimum.
\end{itemize}

This energy consumption corrects the energy used based on the \(V_{corr}\):

\[
E_{swim}^{corr} = E_{swim} \cdot \left( \frac{V_{agent}}{V_{corr}} \right)\]

Where:

\begin{itemize}
\item
  \(E_{swim}^{corr}\) is the corrected energy cost for movement.
\item
  \(E_{swim}\) is swimming-related energetic inefficiency from thermal deviation.
\item
  \(V_{agent}\) is the agent's swimming speed.
\item
  \(V_{corr}\) is the thermal corrected speed.
\end{itemize}

Total energy balance:

\[
E_{agent} = E_{agent} - (E_{swim}^{corr} + E_{acclimation})
\]

Where:

\begin{itemize}
\item
  \(E_{swim}^{corr}\) accounts for temperature-related movement inefficiency.
\item
  \(E_{acclimation}\) accounts for metabolic costs of internal physiological adjustment.
\item
  \(E_{agent}\) is the updated available energy of the agent.
\end{itemize}

\chapter{Mercury Contamination Bioaccumulation Function}\label{mercury-contamination-bioaccumulation-function}

\section{Overview}\label{overview-3}

This function simulates exposure and uptake risk of mercury (Hg) and methylmercury (MeHg) for migratory fish navigating contaminated aquatic environments. The model accounts for spatial and temporal variation in contaminant concentrations and includes physiological modulation based on ion-regulatory stress and suspended particulate matter (SPM).

\section{Purpose}\label{purpose-1}

To evaluate contaminant exposure and bioaccumulation risk in migratory fish due to mercury and methylmercury during migration through estuarine or coastal systems using stress.

\section{Entities, State Variables, and Scales}\label{entities-state-variables-and-scales-2}

\subsection{Global Variables}\label{global-variables}

\begin{longtable}[]{@{}
  >{\centering\arraybackslash}p{(\linewidth - 4\tabcolsep) * \real{0.3333}}
  >{\centering\arraybackslash}p{(\linewidth - 4\tabcolsep) * \real{0.3333}}
  >{\centering\arraybackslash}p{(\linewidth - 4\tabcolsep) * \real{0.3333}}@{}}
\toprule\noalign{}
\begin{minipage}[b]{\linewidth}\centering
Variable
\end{minipage} & \begin{minipage}[b]{\linewidth}\centering
Initialized Value
\end{minipage} & \begin{minipage}[b]{\linewidth}\centering
Justification
\end{minipage} \\
\midrule\noalign{}
\endhead
\bottomrule\noalign{}
\endlastfoot
\textbf{MeHg-Threshold} \(MeHg_{threshold}\) & 15 ug/kg & 10\% of mercury concentration (Gaudet et al., 1995) \\
\textbf{Hg-Threshold} \(Hg_{threshold}\) & 150 ug/kg & (Gaudet et al., 1995) ((NOAA) National Oceanic and Atmospheric Administration, 1990) \\
\end{longtable}

\subsection{Patch Variables}\label{patch-variables-2}

\begin{longtable}[]{@{}
  >{\raggedright\arraybackslash}p{(\linewidth - 2\tabcolsep) * \real{0.5000}}
  >{\raggedright\arraybackslash}p{(\linewidth - 2\tabcolsep) * \real{0.5000}}@{}}
\toprule\noalign{}
\begin{minipage}[b]{\linewidth}\raggedright
Variable Name
\end{minipage} & \begin{minipage}[b]{\linewidth}\raggedright
Definition
\end{minipage} \\
\midrule\noalign{}
\endhead
\bottomrule\noalign{}
\endlastfoot
\textbf{Mercury} \(Hg_{patch}\) & The mercury concentration of a patch. \\
\textbf{Methylmercury} \(MeHg_{patch}\) & The methylmercury concentration of a patch. \\
\textbf{Suspended-particulate-matter} \(SPM_{t}\) & The concentration of suspended particulate matter (SPM) for a given patch, derived from hydrodynamic model inputs, which change temporally. \\
\end{longtable}

\subsection{Agent Variables}\label{agent-variables-2}

\begin{longtable}[]{@{}
  >{\raggedright\arraybackslash}p{(\linewidth - 2\tabcolsep) * \real{0.5000}}
  >{\raggedright\arraybackslash}p{(\linewidth - 2\tabcolsep) * \real{0.5000}}@{}}
\toprule\noalign{}
\begin{minipage}[b]{\linewidth}\raggedright
Variable Name
\end{minipage} & \begin{minipage}[b]{\linewidth}\raggedright
Definition
\end{minipage} \\
\midrule\noalign{}
\endhead
\bottomrule\noalign{}
\endlastfoot
\textbf{stress} \(S\) & The level of stress an agent experiences when moving. \\
\textbf{Hg-exposure-duration} \(Hg_{exp_t}\) & The amount of time an agent is exposed to mercury above healthy levels. \\
\textbf{MeHg-exposure-duration} \(MeHg_{exp_t}\) & The amount of time an agent is exposed to methylmercury above healthy levels. \\
\textbf{Hg-uptake-risk} \(Hg_{risk}\) & The risk associated for uptake of mercury. \\
\textbf{MeHg-uptake-risk} \(MeHg_{risk}\) & The risk associated for uptake of methylmercury. \\
\textbf{Hg-exposure} \(Hg_{t}\) & The amount of mercury exposed during current time step. \\
\textbf{MeHg-exposure} \(MeHg_{t}\) & The amount of methlymercury exposed during current time step. \\
\textbf{Hg-exposure-total} \(Hg_{net}\) & The net sum of mercury exposed to during migration. \\
\textbf{MeHg-exposure-total} \(MeHg_{net}\) & The net sum of methylmercury exposed to during migration. \\
\end{longtable}

\section{Process Overview and Scheduling}\label{process-overview-and-scheduling-2}

\begin{enumerate}
\def\labelenumi{\arabic{enumi}.}
\item
  Evaluate current patch concentrations of mercury and methylmercury.
\item
  Determine whether these exceed defined toxicity thresholds.
\item
  Calculate exposure duration (if thresholds exceeded).
\item
  Compute bioaccumulation risk based on contaminant levels, stress, and SPM.
\item
  Update cumulative exposure totals.
\end{enumerate}

\section{Design Concepts}\label{design-concepts-2}

\textbf{Basic Principles:} this model is grounded in toxicokinetics and ecological exposure theory. It draws on empirical literature (e.g., Gaudet et al.~1995, NOAA 1990) and integrates physiological stress responses with contaminant risk, reflecting a mechanistic understanding of exposure and bioaccumulation dynamics.

\textbf{Emergence:} While exposure durations and patch-level concentrations are direct inputs, the exposure patterns (\(Hg_{t}\), \(MeHg_{t}\)), cumulative exposure totals (\(Hg_{net}\), \(MeHg_{net}\)), and risk profiles (\(Hg_{risk}\), \(MeHg_{risk}\)) emerge from agent movement across spatially and temporally variable environments and their physiological state, which arise from behavioral-environmental interactions over time.

\textbf{Adaptation}: Agents adaptively accumulate risk based on their movement decisions, stress state, and encountered contaminant levels, simulating a physiological feedback process.

\textbf{Objectives:} Agents do not explicitly seek to minimize risk, but their cumulative exposure and risk profiles can be used to evaluate environmental quality and cumulative toxicity risk for migratory fish.

\textbf{Sensing}: Agents sense the local contaminant levels (\(Hg_{patch}\), \(MeHg_{patch}\)), suspended particulate matter (\(SPM_{t}\)), and their own stress state (\(S\)).

\textbf{Stochasticity}: Randomized initial conditions (e.g., Hg and MeHg levels) may introduce variability in exposure patterns.

\textbf{Observation:} Exposure variables (\(Hg_t\), \(MeHg_{t}\)), cumulative exposure (\(Hg_{net}\), \(MeHg_{net}\)), and risk scores (\(Hg_{risk}\), \(MeHg_{risk}\)) are collected per agent and can be exported for analysis.

\section{Initialization}\label{initialization-2}

\begin{longtable}[]{@{}
  >{\centering\arraybackslash}p{(\linewidth - 4\tabcolsep) * \real{0.3333}}
  >{\centering\arraybackslash}p{(\linewidth - 4\tabcolsep) * \real{0.3333}}
  >{\centering\arraybackslash}p{(\linewidth - 4\tabcolsep) * \real{0.3333}}@{}}
\toprule\noalign{}
\begin{minipage}[b]{\linewidth}\centering
Variable
\end{minipage} & \begin{minipage}[b]{\linewidth}\centering
Initialized Value
\end{minipage} & \begin{minipage}[b]{\linewidth}\centering
Justification
\end{minipage} \\
\midrule\noalign{}
\endhead
\bottomrule\noalign{}
\endlastfoot
\(S\) & user-defined stress function & Changes in an agent's environment can induce a stress response, and can be induced by the user or environmental response. \\
\(Hg_{patch}\) & user-defined or data input & This input can be user-defined realistic data values or known spatial data. \\
\(MeHg_{patch}\) & user-defined for data input & This input can be user-defined realistic data values or known spatial data. \\
\end{longtable}

\section{Submodels}\label{submodels-2}

\subsection{Exposure Duration}\label{exposure-duration}

The cumulative number of time steps an agent is exposed to mercury and methylmercury above specified environmental thresholds:

\[
Hg_{exp_t} = Hg_{exp_t} + 1 \quad \text{if } Hg_{patch} > Hg_{threshold}
\]

\[
MeHg_{exp_t} = MeHg_{exp_t} + 1 \quad \text{if } MeHg_{patch} > MeHg_{threshold}
\]

Where:

\begin{itemize}
\tightlist
\item
  \(Hg_{exp_t}\) is the total number of time steps exposed to mercury above threshold.\\
\item
  \(Hg_{patch}\) is the mercury concentration at the current patch location.\\
\item
  \(Hg_{threshold}\) is the defined mercury toxicity threshold.\\
\item
  \(MeHg_{exp_t}\) is the total number of time steps exposed to methylmercury above threshold.\\
\item
  \(MeHg_{patch}\) is the methylmercury concentration at the current patch location.\\
\item
  \(MeHg_{threshold}\) is the defined methylmercury toxicity threshold.
\end{itemize}

\subsection{Bioaccumulation Risk}\label{bioaccumulation-risk}

Estimate the bioaccumulation risk associated with mercury and methylmercury, where risk increases with contaminant concentration, ion-regulatory stress, and suspended particulate matter (SPM):

\[
Hg_{risk} = \sum_{t=1}^T Hg_{normalized} * (1+S) * (1+SPM_{t})
\]

\[
MeHg_{risk} = \sum_{t=1}^T MeHg_{normalized} * (1+S) * (1+SPM_{t})
\]

Where:

\begin{itemize}
\tightlist
\item
  \(Hg_{risk}\) is the instantaneous mercury uptake risk.\\
\item
  \(MeHg_{risk}\) is the instantaneous methylmercury uptake risk.\\
\item
  \(Hg_{normalized}\) is the normalized Hg concentration (scaled 0--1) at each time step.\\
\item
  \(MeHg_{normalized}\) is the normalized MeHg concentration (scaled 0--1).\\
\item
  \(S\) is the stress level of the agent.\\
\item
  \(SPM_{t}\) is the suspended particulate matter concentration at the patch for that time step.
\end{itemize}

\subsection{Exposure to Contamination}\label{exposure-to-contamination}

Agents record exposure to mercury and methylmercury only when concentrations exceed threshold values. These values are stored per time step and accumulated over time to assess total contaminant burden.

\subsubsection*{Mercury:}\label{mercury}
\addcontentsline{toc}{subsubsection}{Mercury:}

\[
Hg_{t} = \begin{cases}Hg_{patch}, & \text{if } Hg_{patch} > Hg_{threshold} \\0, & \text{otherwise}\end{cases} 
\]

\[
Hg_{net} = \sum_{t=1}^{T} Hg_{t} 
\]

Where:

\begin{itemize}
\tightlist
\item
  \(Hg_{t}\) is the mercury exposure in ng/g for current time step
\item
  \(Hg_{net}\) is the cumulative mercury exposure in ng/g
\item
  \(Hg_{patch}\) is the mercury concentration for the agent's current patch
\end{itemize}

\subsubsection*{Methylmercury:}\label{methylmercury}
\addcontentsline{toc}{subsubsection}{Methylmercury:}

\[
MeHg_{t} = \begin{cases}MeHg_{patch}, & \text{if } MeHg_{patch} > MeHg_{threshold} \\0, & \text{otherwise}\end{cases}
\]

\[
MeHg_{net} = \sum_{t=1}^{T} MeHg_{t}
\]

Where:

\begin{itemize}
\tightlist
\item
  \(MeHg_{t}\) is the level of methylmercury exposure in ng/g for current time step
\item
  \(MeHg_{net}\) is the cumulative methylmercury exposure in ng/g
\item
  \(MeHg_{patch}\) is the methylmercury concentration for the agent's current patch
\end{itemize}

\chapter{Landward Migration Behavior}\label{landward-migration-behavior}

\section{Overview}\label{overview-4}

This function simulates landward (upstream) migratory behavior of fish navigating riverine and estuarine systems. Agents face resistance from environmental water velocity and incur energetic costs that scale with flow conditions and their size.

\section{Purpose}\label{purpose-2}

To model how migratory fish respond to varying riverine and tidal velocities by calculating effective swimming speed, difficulty of movement, and the energetic cost associated with upstream migration.

\section{Entities, State Variables, and Scales}\label{entities-state-variables-and-scales-3}

\subsection{Global Variables}\label{global-variables-1}

\begin{longtable}[]{@{}
  >{\centering\arraybackslash}p{(\linewidth - 4\tabcolsep) * \real{0.3333}}
  >{\centering\arraybackslash}p{(\linewidth - 4\tabcolsep) * \real{0.3333}}
  >{\centering\arraybackslash}p{(\linewidth - 4\tabcolsep) * \real{0.3333}}@{}}
\toprule\noalign{}
\begin{minipage}[b]{\linewidth}\centering
Variable
\end{minipage} & \begin{minipage}[b]{\linewidth}\centering
Initialized Value
\end{minipage} & \begin{minipage}[b]{\linewidth}\centering
Justification
\end{minipage} \\
\midrule\noalign{}
\endhead
\bottomrule\noalign{}
\endlastfoot
\textbf{minimum-velocity} \(V_{min}\) & Calculated from \(V_{patch}\) over the simulation period. & Minimum river velocity based on hydrodynamic observations. \\
\textbf{maximum-velocity} \(V_{max}\) & Calculated from \(V_{patch}\) over the simulation period. & Maximum river velocity based on hydrodynamic observations. \\
\end{longtable}

\subsection{Patch Variables}\label{patch-variables-3}

\begin{longtable}[]{@{}
  >{\centering\arraybackslash}p{(\linewidth - 2\tabcolsep) * \real{0.5000}}
  >{\centering\arraybackslash}p{(\linewidth - 2\tabcolsep) * \real{0.5000}}@{}}
\toprule\noalign{}
\begin{minipage}[b]{\linewidth}\centering
Variable Name
\end{minipage} & \begin{minipage}[b]{\linewidth}\centering
Definition
\end{minipage} \\
\midrule\noalign{}
\endhead
\bottomrule\noalign{}
\endlastfoot
\textbf{Velocity} \(V_{patch}\) & The along-channel velocity of a given patch, derived from hydrodynamic model inputs, where positive values are in the landward direction and negative values are in the seaward direction. \\
\end{longtable}

\subsection{Agent Variables}\label{agent-variables-3}

\begin{longtable}[]{@{}
  >{\centering\arraybackslash}p{(\linewidth - 2\tabcolsep) * \real{0.5000}}
  >{\centering\arraybackslash}p{(\linewidth - 2\tabcolsep) * \real{0.5000}}@{}}
\toprule\noalign{}
\begin{minipage}[b]{\linewidth}\centering
Variable Name
\end{minipage} & \begin{minipage}[b]{\linewidth}\centering
Definition
\end{minipage} \\
\midrule\noalign{}
\endhead
\bottomrule\noalign{}
\endlastfoot
\textbf{size} \(M_{agent}\) & The size of an agent. \\
\textbf{M-max} \(M_{max}\) & Maximum size found within the agent's population. \\
\textbf{swimming-speed} \(V_{agent}\) & The current swimming speed of the agent. \\
\textbf{maximum-speed} \(swim_{max}\) & The maximum sustained speed of the agent. \\
\textbf{difficulty-factor} \(D_{f}\) & The level of difficulty an agent experiences when swimming. \\
\textbf{energy} \(E_{agent}\) & The total energy an agent has. \\
\textbf{swimming-energy-cost} \(Swim_{base}\) & The base energy cost of swimming. \\
\textbf{net-swimming-cost} \(E_{swim}\) & The total energy expenditure for swimming. \\
\textbf{heading} \(\hat{u}\) & The direction agent is facing or ``headed towards'' \\
\textbf{Y-position} \(\vec{Y}_t\) & This is the agent's position in the Y plane \\
\end{longtable}

\section{Process Overview and Scheduling}\label{process-overview-and-scheduling-3}

\begin{enumerate}
\def\labelenumi{\arabic{enumi}.}
\item
  Determine effective swimming speed based on flow velocity and agent energy.
\item
  Compute swimming difficulty factor using normalized velocity.
\item
  Calculate movement direction and update position.
\item
  Deduct swimming energy cost from agent's energy pool.
\end{enumerate}

\section{Design Concepts}\label{design-concepts-3}

\textbf{Basic Principles:} This model builds on hydrodynamic constraints and energetic theory. It assumes that swimming against current imposes increased metabolic demands and that movement is energetically limited by individual traits.

\textbf{Emergence:} Movement trajectories (\(\hat{u}\)) and energy (\(E_{agent}\)) depletion emerge from the interaction between local flow conditions, fish traits, and directional behavior.

\textbf{Objectives:} Agents aim to migrate upstream. While they do not explicitly optimize, their movement is shaped by their capacity to overcome current velocity \(V_{patch}\)).

\textbf{Sensing:} Agents detect the local water velocity (\(V_{patch}\)) and use it to update their speed (\(V_{agent}\)) and effort (\(D_{f}\)).

\textbf{Observation:} Agent positions (\(\vec{Y}_t\) )and energy states can be tracked per time step to analyze migration success and efficiency.

\section{Initialization}\label{initialization-3}

\begin{longtable}[]{@{}
  >{\centering\arraybackslash}p{(\linewidth - 4\tabcolsep) * \real{0.3333}}
  >{\centering\arraybackslash}p{(\linewidth - 4\tabcolsep) * \real{0.3333}}
  >{\centering\arraybackslash}p{(\linewidth - 4\tabcolsep) * \real{0.3333}}@{}}
\toprule\noalign{}
\begin{minipage}[b]{\linewidth}\centering
Variable
\end{minipage} & \begin{minipage}[b]{\linewidth}\centering
Initialized Value
\end{minipage} & \begin{minipage}[b]{\linewidth}\centering
Justification
\end{minipage} \\
\midrule\noalign{}
\endhead
\bottomrule\noalign{}
\endlastfoot
\(V_{patch}\) & user-defined for data input & This input can be user-defined realistic data values or known spatial data. \\
\(M_{agent}\) & user-defined and species-specific & Representative body length of a migrating agent. \\
\(M_{max}\) & user-defined and species-specific & Based on the maximum body length in the agent's population. \\
\(V_{agent}\) & \(\frac{V_{max}}{2}\) & Fish begin migration with a moderate swimming speed relative to their maximum capacity. \\
\(swim_{max}\) & \(1.5-3 \frac{body lengths}{sec}\) & Typical value for sustained swimming speed in small pelagic fish (refer to Videler, 1993). \\
\(E_{agent}\) & 100\% & Agent starts migration at 100\% relative energy capacity. \\
\(swim_{base}\) & \(0.02 \cdot \frac{M_{agent}}{M_{max}}^{k}\) & Scales locomotion cost nonlinearly with size; can be calibrated. \\
\(k\) & 0.75 & Energetic scaling component that follows Kleiber's Law. \\
\(\hat{u}\) & \(0^\circ\) & Unit vector in the upstream direction \\
\end{longtable}

\section{Submodels}\label{submodels-3}

\subsection{Swimming Speed}\label{swimming-speed}

Agents calculate swimming speed based on their available energy and hydrodynamic resistance:

\[ V_{agent} = \frac{swim_{max} * E_{agent}}{100} - (-k \cdot |V_{patch}|) \]

\textbf{Where:}

\begin{itemize}
\item
  \(swim_{max}\) is the maximum sustained swimming speed of the agent.
\item
  \(V_{patch}\) is the environmental velocity at the agent's current patch.
\item
  \(V_{agent}\) is the effective swimming speed of the agent.
\item
  \(E_{agent}\) is the agent's available energy percentage (0-100\%).
\item
  \(k\) is a scaling factor that determines how velocity influences swimming effort.
\end{itemize}

\subsection{\texorpdfstring{\textbf{Swimming Difficulty}}{Swimming Difficulty}}\label{swimming-difficulty}

The difficulty factor quantifies the additional energetic burden of swimming against different velocity conditions. In this case, difficulty is calculated using a normalized velocity-based proxy that scales difficulty from 1-10 between observed flow extremes.

\[D_{f} = 1 + 9 \cdot \left(\frac{\left( \frac{|V_{patch}|}{swim_{max} \cdot \left(\frac{M_{agent}}{M_{max}}\right)} \right)^k - V_{min}}{V_{max} - V_{min}}\right)\]

Where:

\begin{itemize}
\tightlist
\item
  \(M_{agent}\) is the size of the agent.
\item
  \(M_{max}\) is the maximum size within the agent's population.
\item
  \(swim_{max}\) is the maximum swimming speed capability of the agent.
\item
  \(V_{max}\) is the maximum depth-averaged water velocity observed within the simulation.
\item
  \(V_{min}\) is the minimum depth-averaged water velocity observed within the simulation.
\item
  \(V_{patch}\) is the depth-averaged water velocity for the agent's current patch.
\item
  \(D_{f}\) is the swimming difficulty factor.
\end{itemize}

\textbf{Biological Justification:}

When \(V_{patch} \approx 0\), difficulty is moderate.

When \(V_{patch} < 0\), difficulty increases because the fish is actively swimming against the current.

When \(V_{patch} > 0\), difficulty is minimal as fish drift with the current.

\subsection{Swimming Movement}\label{swimming-movement}

During landward migration, agents orient upstream and move forward based on their calculated swimming speed:

\[ \vec{Y}_{t+1} = \vec{Y}_t + V_{agent} \cdot \hat{u} \]

Where:

\begin{itemize}
\tightlist
\item
  \(\vec{Y}_t\) is the agent's current spatial position.
\item
  \(\vec{Y}_{t+1}\) is the agent's updated spatial position after one time step.
\item
  \(V_{agent}\) is the swimming speed calculated from energy and difficulty.
\item
  \(\hat{u}\) is the unit vector in the landward direction.
\end{itemize}

\subsection{Swimming Energy}\label{swimming-energy}

Swimming energy cost is determined by the base cost of locomotion scaled by a difficulty factor raised to a scaling factor. This allows energy expenditure to increase non-linearly as flow resistance increases.

\[ E_{swim} = Swim_{base} \cdot D_f^{k} \]

Where:

\begin{itemize}
\tightlist
\item
  \(E_{swim}\) is the energy cost of swimming.
\item
  \(Swim_{base}\) is the base swimming cost based on agent size.
\item
  \(D_f\) is the swimming difficulty factor.
\item
  \(k\) is the scaling factor, reflecting nonlinear energy demand.
\end{itemize}

\subsection{Energy Balance}\label{energy-balance-1}

Agents balance energy to swim with total energy allowance:

\[ E_{agent} = E_{agent} - E_{swim} \]

Where:

\begin{itemize}
\tightlist
\item
  \(E_{agent}\) is the current energy available to the agent.
\item
  \(E_{swim}\) is the energy cost of swimming in this time step.
\end{itemize}

\chapter{Seaward Migration Behavior}\label{seaward-migration-behavior}

\section{Overview}\label{overview-5}

This function simulates seaward (downstream) migratory behavior of fish navigating riverine and estuarine systems. Agents face resistance from tidal flows and benefit from downstream riverine flows. The function models effective swimming speed, the difficulty of movement, and energetic costs during seaward migration.

\section{Purpose}\label{purpose-3}

To model how migratory fish respond to along-channel velocity when traveling seaward by determining swimming speed, hydrodynamic difficulty, and the energetic cost of downstream migration.

\section{Entities, State Variables, and Scales}\label{entities-state-variables-and-scales-4}

\subsection{Global Variables}\label{global-variables-2}

\begin{longtable}[]{@{}
  >{\centering\arraybackslash}p{(\linewidth - 4\tabcolsep) * \real{0.3333}}
  >{\centering\arraybackslash}p{(\linewidth - 4\tabcolsep) * \real{0.3333}}
  >{\centering\arraybackslash}p{(\linewidth - 4\tabcolsep) * \real{0.3333}}@{}}
\toprule\noalign{}
\begin{minipage}[b]{\linewidth}\centering
Variable
\end{minipage} & \begin{minipage}[b]{\linewidth}\centering
Initialized Value
\end{minipage} & \begin{minipage}[b]{\linewidth}\centering
Justification
\end{minipage} \\
\midrule\noalign{}
\endhead
\bottomrule\noalign{}
\endlastfoot
\textbf{minimum-velocity} \(V_{min}\) & Calculated from \(V_{patch}\) over the simulation period. & Minimum river velocity based on hydrodynamic observations. \\
\textbf{maximum-velocity} \(V_{max}\) & Calculated from \(V_{patch}\) over the simulation period. & Maximum river velocity based on hydrodynamic observations. \\
\end{longtable}

\subsection{Patch Variables}\label{patch-variables-4}

\begin{longtable}[]{@{}
  >{\centering\arraybackslash}p{(\linewidth - 2\tabcolsep) * \real{0.5000}}
  >{\centering\arraybackslash}p{(\linewidth - 2\tabcolsep) * \real{0.5000}}@{}}
\toprule\noalign{}
\begin{minipage}[b]{\linewidth}\centering
Variable Name
\end{minipage} & \begin{minipage}[b]{\linewidth}\centering
Definition
\end{minipage} \\
\midrule\noalign{}
\endhead
\bottomrule\noalign{}
\endlastfoot
\textbf{Velocity} \(V_{patch}\) & The along-channel velocity of a given patch, derived from hydrodynamic model inputs, where positive values are in the landward direction and negative values are in the seaward direction. \\
\end{longtable}

\subsection{Agent Variables}\label{agent-variables-4}

\begin{longtable}[]{@{}
  >{\centering\arraybackslash}p{(\linewidth - 2\tabcolsep) * \real{0.5000}}
  >{\centering\arraybackslash}p{(\linewidth - 2\tabcolsep) * \real{0.5000}}@{}}
\toprule\noalign{}
\begin{minipage}[b]{\linewidth}\centering
Variable Name
\end{minipage} & \begin{minipage}[b]{\linewidth}\centering
Definition
\end{minipage} \\
\midrule\noalign{}
\endhead
\bottomrule\noalign{}
\endlastfoot
\textbf{size} \(M_{agent}\) & The size of an agent. \\
\textbf{M-max} \(M_{max}\) & Maximum size found within the agent's population. \\
\textbf{swimming-speed} \(V_{agent}\) & The current swimming speed of the agent. \\
\textbf{maximum-speed} \(swim_{max}\) & The maximum sustained speed of the agent. \\
\textbf{difficulty-factor} \(D_{f}\) & The level of difficulty an agent experiences when swimming. \\
\textbf{energy} \(E_{agent}\) & The total energy an agent has. \\
\textbf{swimming-energy-cost} \(Swim_{base}\) & The base energy cost of swimming. \\
\textbf{net-swimming-cost} \(E_{swim}\) & The total energy expenditure for swimming. \\
\textbf{heading} \(\hat{u}\) & The direction agent is facing or ``headed towards'' \\
\textbf{Y-position} \(\vec{Y}_t\) & This is the agent's position in the Y plane \\
\end{longtable}

\section{Process Overview and Scheduling}\label{process-overview-and-scheduling-4}

\begin{enumerate}
\def\labelenumi{\arabic{enumi}.}
\item
  Determine effective swimming speed based on flow velocity and agent energy.
\item
  Compute swimming difficulty factor using normalized velocity.
\item
  Calculate movement direction and update position.
\item
  Deduct swimming energy cost from agent's energy pool.
\end{enumerate}

\section{Design Concepts}\label{design-concepts-4}

\textbf{Basic Principles:} This model builds on hydrodynamic constraints and energetic theory. It assumes that swimming against current imposes increased metabolic demands and that movement is energetically limited by individual traits.

\textbf{Emergence:} Movement trajectories (\(\hat{u}\)) and energy (\(E_{agent}\)) depletion emerge from the interaction between local flow conditions, fish traits, and directional behavior.

\textbf{Objectives:} Agents aim to migrate downstream. While they do not explicitly optimize, their movement is shaped by their capacity to overcome current velocity \(V_{patch}\)).

\textbf{Sensing:} Agents detect the local water velocity (\(V_{patch}\)) and use it to update their speed (\(V_{agent}\)) and effort (\(D_{f}\)).

\textbf{Observation:} Agent positions (\(\vec{Y}_t\) )and energy states can be tracked per time step to analyze migration success and efficiency.

\section{Initialization}\label{initialization-4}

\begin{longtable}[]{@{}
  >{\centering\arraybackslash}p{(\linewidth - 4\tabcolsep) * \real{0.3333}}
  >{\centering\arraybackslash}p{(\linewidth - 4\tabcolsep) * \real{0.3333}}
  >{\centering\arraybackslash}p{(\linewidth - 4\tabcolsep) * \real{0.3333}}@{}}
\toprule\noalign{}
\begin{minipage}[b]{\linewidth}\centering
Variable
\end{minipage} & \begin{minipage}[b]{\linewidth}\centering
Initialized Value
\end{minipage} & \begin{minipage}[b]{\linewidth}\centering
Justification
\end{minipage} \\
\midrule\noalign{}
\endhead
\bottomrule\noalign{}
\endlastfoot
\(V_{patch}\) & user-defined for data input & This input can be user-defined realistic data values or known spatial data. \\
\(M_{agent}\) & user-defined and species-specific & Representative body length of a migrating agent. \\
\(M_{max}\) & user-defined and species-specific & Based on the maximum body length in the agent's population. \\
\(V_{agent}\) & \(\frac{V_{max}}{2}\) & Fish begin migration with a moderate swimming speed relative to their maximum capacity. \\
\(swim_{max}\) & \(1.5-3 \frac{body lengths}{sec}\) & Typical value for sustained swimming speed in small pelagic fish (refer to Videler, 1993). \\
\(E_{agent}\) & 100\% & Agent starts migration at 100\% relative energy capacity. \\
\(swim_{base}\) & \(0.02 \cdot \frac{M_{agent}}{M_{max}}^{k}\) & Scales locomotion cost nonlinearly with size; can be calibrated. \\
\(k\) & 0.75 & Energetic scaling component that follows Kleiber's Law. \\
\(\hat{u}\) & \(180^\circ\) & Unit vector in the downstream direction \\
\end{longtable}

\section{Submodels}\label{submodels-4}

\subsection{Swimming Speed}\label{swimming-speed-1}

\[ V_{agent} = \frac{V_{max} * E_{agent}}{100} - (k \cdot |V_{patch}|) \]

Where:

\begin{itemize}
\item
  \(V_{max}\) is the maximum sustained swimming speed of the agent.
\item
  \(V_{patch}\) is the environmental velocity at the agent's current patch.
\item
  \(V_{agent}\) is the effective swimming speed of the agent.
\item
  \(E_{agent}\) is the agent's available energy percentage (0-100\%).
\item
  \(k\) is a scaling factor that determines how velocity influences swimming effort.
\end{itemize}

\subsection{\texorpdfstring{\textbf{Swimming Difficulty}}{Swimming Difficulty}}\label{swimming-difficulty-1}

The difficulty factor quantifies the additional energetic burden of swimming against different velocity conditions. In this case, difficulty is calculated using a normalized velocity-based proxy that linearly scales difficulty from 1-10 between observed flow extremes.

\[D_{f} = 1 + 9 \cdot \left(\frac{\left( \frac{|V_{patch}|}{V_{max} \cdot \left(\frac{M_{agent}}{M_{max}}\right)} \right)^k - V_{min}}{V_{max} - V_{min}}\right)\]

Where:

\begin{itemize}
\tightlist
\item
  \(M_{agent}\) is the size of the agent.
\item
  \(M_{max}\) is the maximum size within the agent's population.
\item
  \(V_{max}\) is the maximum swimming speed capability of the agent.
\item
  \(V_{max}\) is the maximum depth-averaged water velocity observed within the simulation.
\item
  \(V_{min}\) is the minimum depth-averaged water velocity observed within the simulation.
\item
  \(V_{patch}\) is the depth-averaged water velocity for the agent's current patch.
\item
  \(D_{f}\) is the swimming difficulty factor.
\end{itemize}

\textbf{Biological Justification}

\begin{itemize}
\item
  When \(V_{patch} \approx 0\), difficulty is moderate.
\item
  When \(V_{patch} < 0\), difficulty increases because the fish is actively swimming against the current.
\item
  When \(V_{patch} > 0\), difficulty is minimal as fish drift with the current.
\end{itemize}

\subsection{Swimming Movement}\label{swimming-movement-1}

During landward migration, agents orient upstream and move forward based on their calculated swimming speed:

\[ \vec{Y}_{t+1} = \vec{Y}_t + V_{agent} \cdot \hat{u} \]

Where:

\begin{itemize}
\tightlist
\item
  \(\vec{Y}_t\) is the agent's current spatial position.
\item
  \(\vec{Y}_{t+1}\) is the agent's updated spatial position after one time step.
\item
  \(V_{agent}\) is the swimming speed calculated from energy and difficulty.
\item
  \(\hat{u}\) is the unit vector in the seaward direction (180° heading, downstream).
\end{itemize}

\section{Swimming Energy}\label{swimming-energy-1}

Swimming energy cost is determined by the base cost of locomotion scaled by a difficulty factor raised to a power. This allows energy expenditure to increase non-linearly as flow resistance increases.

\[ E_{swim} = Swim_{base} \cdot D_f^{k} \]

Where:

\begin{itemize}
\tightlist
\item
  \(E_{swim}\) is the energy cost of swimming.
\item
  \(Swim_{base}\) is the base swimming cost based on agent size.
\item
  \(D_f\) is the swimming difficulty factor.
\item
  \(k\) is the scaling exponent, reflecting nonlinear energy demand.
\end{itemize}

\subsection{Energy Balance}\label{energy-balance-2}

Fish allocate energy efficiently, balancing osmoregulation with other survival functions.

\[ E_{agent} = E_{agent} - E_{swim} \]

Where:

\begin{itemize}
\tightlist
\item
  \(E_{agent}\) is the current energy available to the agent.
\item
  \(E_{swim}\) is the energy cost of swimming in this time step.
\end{itemize}

\chapter{Schooling Behavior}\label{schooling-behavior}

\section{Overview}\label{overview-6}

\section{Purpose}\label{purpose-4}

\section{Entities, State Variables, and Scales}\label{entities-state-variables-and-scales-5}

\subsection{Patch Variables}\label{patch-variables-5}

\begin{longtable}[]{@{}ll@{}}
\toprule\noalign{}
Variable Name & Definition \\
\midrule\noalign{}
\endhead
\bottomrule\noalign{}
\endlastfoot
& \\
\end{longtable}

\subsection{Agent Variables}\label{agent-variables-5}

\begin{longtable}[]{@{}ll@{}}
\toprule\noalign{}
Variable Name & Definition \\
\midrule\noalign{}
\endhead
\bottomrule\noalign{}
\endlastfoot
& \\
& \\
& \\
& \\
& \\
& \\
& \\
& \\
& \\
\end{longtable}

\section{Process Overview and Scheduling}\label{process-overview-and-scheduling-5}

\begin{enumerate}
\def\labelenumi{\arabic{enumi}.}
\tightlist
\item
\item
\item
\item
\end{enumerate}

\section{Design Concepts}\label{design-concepts-5}

\textbf{Basic Principles:}

\textbf{Emergence:}

\textbf{Adaptation}:

\textbf{Objectives:}

\textbf{Learning:}

\textbf{Prediction:}

\textbf{Sensing}:

\textbf{Interaction}:

\textbf{Stochasticity}:

\textbf{Collectives:}

\textbf{Observation:}

\section{Initialization}\label{initialization-5}

\begin{longtable}[]{@{}ccc@{}}
\toprule\noalign{}
Variable & Initialized Value & Justification \\
\midrule\noalign{}
\endhead
\bottomrule\noalign{}
\endlastfoot
& & \\
& & \\
& & \\
& & \\
& & \\
\end{longtable}

\section{Submodels}\label{submodels-5}

\chapter{Selective Tidal Stream Transport}\label{selective-tidal-stream-transport}

\section{Overview}\label{overview-7}

Selective Tidal Stream Transport (STST) is a behavioral strategy that enables agents to conserve energy by passively drifting with the current. It is triggered when the along-channel velocity of the patch exceeds the agent's effective swimming speed, and that speed is below a species-specific minimum threshold. Once engaged, agents align with the tidal current and are carried downstream or upstream, depending on flow velocity. STST reduces the metabolic cost of movement by substituting active swimming with passive transport. This behavior persists for a limited duration or until swimming ability improves, after which agents resume directional migration.

\section{Purpose}\label{purpose-5}

To simulate a passive energy-conserving behavior in migratory fish that allows them to use tidal currents to move when swimming capacity is insufficient to overcome flow velocities.

\section{Entities, State Variables, and Scales}\label{entities-state-variables-and-scales-6}

\subsection{Patch Variables}\label{patch-variables-6}

\begin{longtable}[]{@{}
  >{\centering\arraybackslash}p{(\linewidth - 2\tabcolsep) * \real{0.5000}}
  >{\centering\arraybackslash}p{(\linewidth - 2\tabcolsep) * \real{0.5000}}@{}}
\toprule\noalign{}
\begin{minipage}[b]{\linewidth}\centering
Variable Name
\end{minipage} & \begin{minipage}[b]{\linewidth}\centering
Definition
\end{minipage} \\
\midrule\noalign{}
\endhead
\bottomrule\noalign{}
\endlastfoot
\textbf{Velocity} \(V_{patch}\) & The along-channel velocity of a given patch, derived from hydrodynamic model inputs, where positive values are in the landward direction and negative values are in the seaward direction. \\
\textbf{tidal-transport-in-patch} & Count of agents exhibiting tidal stream transport within a patch (for habitat quality analysis). \\
\end{longtable}

\subsection{Agent Variables}\label{agent-variables-6}

\begin{longtable}[]{@{}
  >{\centering\arraybackslash}p{(\linewidth - 2\tabcolsep) * \real{0.5000}}
  >{\centering\arraybackslash}p{(\linewidth - 2\tabcolsep) * \real{0.5000}}@{}}
\toprule\noalign{}
\endhead
\bottomrule\noalign{}
\endlastfoot
\textbf{Variable Name} & \textbf{Definition} \\
\textbf{energy} \(E_{agent}\) & Total energy available to the agent. \\
\textbf{swimming energy} \(E_{swim}\) & Energy expenditure from movement per time step. \\
\textbf{base swim energy} \(swim_{base}\) & Baseline energy cost of movement. \\
\textbf{swimming difficulty} \(D_f\) & Velocity-based proxy representing hydrodynamic resistance. \\
\textbf{in-STST?} \(STST_{?}\) & Boolean value indicating if the agent is actively in STST. \\
\textbf{swimming speed} \(V_{agent}\) & The effective swimming speed of the agent. \\
\textbf{minimum threshold speed} \(Speed_{min}\) & The minimum speed at which an agent will move. \\
\end{longtable}

\section{Process Overview and Scheduling}\label{process-overview-and-scheduling-6}

\begin{enumerate}
\def\labelenumi{\arabic{enumi}.}
\item
  Compare swimming speed (\(V_{agent}\)) with flow speed (\(V_{patch}\)).
\item
  If \(|V_{patch}| > V_{agent}\) and \(V_{agent} \leq Speed_{min}\), enter STST.
\item
  In STST: align with current, update position via drift, apply reduced energy cost.
\item
  If \(V_{agent} > Speed_{min}\), exit STST and resume active swimming.
\end{enumerate}

\section{Design Concepts}\label{design-concepts-6}

\textbf{Basic Principles:} Selective tidal stream transport is based on behavioral ecology and energetics, simulating the tradeoff between active swimming and energy conservation through passive transport.

\textbf{Emergence:} Passive drift behavior and resulting migration paths emerge from agent-flow interactions and individual swimming limitations.

\textbf{Adaptation:} Agents adapt their mode of movement based on their swimming ability relative to environmental flow, dynamically choosing energy-efficient strategies.

\textbf{Objectives:} Agents seek to minimizing energy loss in strong flows.

\textbf{Sensing:} Agents sense their own \(V_{agent}\) and the \(V_{patch}\) to determine whether passive drift is needed.

\textbf{Observation:} Records STST patch events , energy expenditure (\(E_{agent}\)), and displacement are logged to analyze behavior across flow regimes.

\section{Initialization}\label{initialization-6}

\begin{longtable}[]{@{}
  >{\centering\arraybackslash}p{(\linewidth - 4\tabcolsep) * \real{0.3333}}
  >{\centering\arraybackslash}p{(\linewidth - 4\tabcolsep) * \real{0.3333}}
  >{\centering\arraybackslash}p{(\linewidth - 4\tabcolsep) * \real{0.3333}}@{}}
\toprule\noalign{}
\begin{minipage}[b]{\linewidth}\centering
Variable
\end{minipage} & \begin{minipage}[b]{\linewidth}\centering
Initialized Value
\end{minipage} & \begin{minipage}[b]{\linewidth}\centering
Justification
\end{minipage} \\
\midrule\noalign{}
\endhead
\bottomrule\noalign{}
\endlastfoot
\(V_{agent}\) & Based on size, energy, and difficulty factor & Reflects agent's swimming capability based on metabolic limits. \\
\(swim_{min}\) & Species-specific parameter & Represents the minimum sustained swimming velocity of the agent. \\
\(swim_{max}\) & Species-specific parameter & Represents the maximum sustained swimming velocity of the agent. \\
\(E_{agent}\) & 100 & Assumes full energy at the start of simulation or at spawning. \\
\(swim_{base}\) & \(0.02 \cdot \frac{M_{agent}}{M_{max}}^{k}\) & Scales locomotion cost nonlinearly with size; can be calibrated. \\
\end{longtable}

\section{Submodels}\label{submodels-6}

\subsection{Trigger Conditions for STST}\label{trigger-conditions-for-stst}

Agents compare their swimming ability to the flow conditions. If local flow exceeds their capability and their effort is below a defined threshold, they enter STST:

\[
|V_{patch}| > V_{agent} \quad \text{and} \quad V_{agent} \leq Speed_{min}
\]

While in STST:

Heading aligns with the current (drift vector) \& swimming speed is set to:

\[
V_{agent} = |V_{patch}|
\]

Energy is set as:

\[
E_{swim} = swim_{base}
\]

Where:

\begin{itemize}
\item
  \(V_{agent}\) is the current swimming speed of the agent.
\item
  \(V_{patch}\) is the along-channel velocity at the agent's current patch.
\item
  \(Speed_{min}\) is the minimum sustainable swimming speed of the agent.
\item
  \(swim_{base}\) is the base swimming cost based on agent size.
\item
  \(E_{swim}\) is the total energy cost during passive movement.
\end{itemize}

\subsection{Behavior During STST}\label{behavior-during-stst}

While in STST, agents align with the current (either landward or seaward) and are passively transported:

\[
\vec{Y}_{t+1} = \vec{Y}_t + |V_{patch}| \cdot \hat{u}
\]

Swimming speed is overwritten:

\[
V_{agent} = |V_{patch}|
\]

Energy cost is minimized:

\[
E_{swim} = swim_{base}
\]

Where:

\begin{itemize}
\item
  \(\vec{Y}_t\) is the agent's current spatial position.
\item
  \(\vec{Y}_{t+1}\) is the position after drifting.
\item
  \(V_{patch}\) is the along-channel velocity at the agent's current patch.
\item
  \(\hat{u}_{patch}\) is the direction of the patch velocity (unit vector).
\item
  \(swim_{base}\) is the base swimming cost based on agent size.
\item
  \(E_{swim}\) is the total energy cost during passive movement.
\end{itemize}

\subsection{Stop Conditions for STST}\label{stop-conditions-for-stst}

Agents exit STST when they regain sufficient swimming capacity to exceed threshold:

\[
V_{agent} > Speed_{min}
\]

Where:

\begin{itemize}
\item
  \(V_{agent}\) is the current swimming speed of the agent.
\item
  \(Speed_{min}\) is the minimum sustainable swimming speed of the agent.
\end{itemize}

This triggers a return to active migratory movement and deactivates \(STST_{?}\).

\chapter{Staging Behavior}\label{staging-behavior}

\section{Overview}\label{overview-8}

Staging is a behavioral state that allows agents to temporarily halt migration and recover energy or acclimate to dynamic estuary conditions (i.e., temperature, salinity) before continuing upstream (landward) or downstream (seaward) movement. It is triggered when agents experience low energy or high physiological stress and is resolved when recovery thresholds are met.

\section{Purpose}\label{purpose-6}

To simulate the biologically necessary pause in migratory activity used for energy recovery and physiological acclimation, particularly under stressful conditions.

\section{Entities, State Variables, and Scales}\label{entities-state-variables-and-scales-7}

\subsection{Patch Variables}\label{patch-variables-7}

\begin{longtable}[]{@{}
  >{\centering\arraybackslash}p{(\linewidth - 2\tabcolsep) * \real{0.5000}}
  >{\centering\arraybackslash}p{(\linewidth - 2\tabcolsep) * \real{0.5000}}@{}}
\toprule\noalign{}
\begin{minipage}[b]{\linewidth}\centering
Variable Name
\end{minipage} & \begin{minipage}[b]{\linewidth}\centering
Definition
\end{minipage} \\
\midrule\noalign{}
\endhead
\bottomrule\noalign{}
\endlastfoot
\textbf{Velocity} \(V_{patch}\) & The along-channel velocity of a given patch, derived from hydrodynamic model inputs, where positive values are in the landward direction and negative values are in the seaward direction. \\
\textbf{staging-in-patch} & Count of staging agents within a patch (for habitat quality analysis). \\
\end{longtable}

\subsection{Agent Variables}\label{agent-variables-7}

\begin{longtable}[]{@{}
  >{\centering\arraybackslash}p{(\linewidth - 2\tabcolsep) * \real{0.5000}}
  >{\centering\arraybackslash}p{(\linewidth - 2\tabcolsep) * \real{0.5000}}@{}}
\toprule\noalign{}
\begin{minipage}[b]{\linewidth}\centering
Variable Name
\end{minipage} & \begin{minipage}[b]{\linewidth}\centering
Definition
\end{minipage} \\
\midrule\noalign{}
\endhead
\bottomrule\noalign{}
\endlastfoot
\textbf{size} \(M_{agent}\) & The size of an agent. \\
\textbf{M-max} \(M_{max}\) & Maximum size found within the agent's population. \\
\textbf{age} \(A_{agent}\) & The age of an agent. \\
\textbf{A-max} \(A_{max}\) & Maximum age found within the agent's population. \\
\textbf{energy} \(E_{agent}\) & The total energy the agent currently possesses. \\
\textbf{stress} \(S\) & Stress level based on environmental mismatch. \\
\textbf{staging?} \(stage_{?}\) & Boolean flag indicating whether the agent is currently staging. \\
\end{longtable}

\section{Process Overview and Scheduling}\label{process-overview-and-scheduling-7}

\begin{enumerate}
\def\labelenumi{\arabic{enumi}.}
\item
  Evaluate energy and stress levels.
\item
  If energy is ≤ 25\% or stress \textgreater{} 5, agent enters staging behavior.
\item
  During staging, agents stop migrating, form schools, seek calm water, and regain energy.
\item
  If energy ≥ 75\% and stress = 1, staging ends and active migration resumes.
\end{enumerate}

\section{Design Concepts}\label{design-concepts-7}

\textbf{Basic Principles:} Staging is based on physiological ecology principles recognizing the need for energetic and osmoregulatory recovery before continued migration.

\textbf{Emergence:} Collective staging areas and patterns emerge from local environmental conditions and individual agent needs.

\textbf{Adaptation:} Agents adaptively stop migrating when unable to continue due to exhaustion or stress, shifting to a recovery behavior.

\textbf{Sensing:} Agents assess their internal energy and stress state.

\textbf{Stochasticity:} Recovery rate includes random variation to simulate individual differences.

\textbf{Collectives:} Agents may cluster spatially during staging but do not form persistent groups.

\textbf{Observation:} Number of staging agents and energy dynamics can be recorded for habitat analysis.

\section{Initialization}\label{initialization-7}

\begin{longtable}[]{@{}
  >{\centering\arraybackslash}p{(\linewidth - 4\tabcolsep) * \real{0.1603}}
  >{\centering\arraybackslash}p{(\linewidth - 4\tabcolsep) * \real{0.2308}}
  >{\centering\arraybackslash}p{(\linewidth - 4\tabcolsep) * \real{0.6026}}@{}}
\toprule\noalign{}
\begin{minipage}[b]{\linewidth}\centering
Variable
\end{minipage} & \begin{minipage}[b]{\linewidth}\centering
Initialized Value
\end{minipage} & \begin{minipage}[b]{\linewidth}\centering
Justification
\end{minipage} \\
\midrule\noalign{}
\endhead
\bottomrule\noalign{}
\endlastfoot
\textbf{size} \(M_{agent}\) & user-defined and species-specific & Representative body length of a migrating agent. \\
\textbf{M-max} \(M_{max}\) & user-defined and species-specific & Based on the maximum body length in the agent's population. \\
\textbf{age} \(A_{agent}\) & user-defined and species-specific & Representative age length of a migrating agent. \\
\textbf{A-max} \(A_{max}\) & user-defined and species-specific & Based on the maximum age in the agent's population. \\
\textbf{energy} \(E_{agent}\) & \(100 \%\) & Agent starts migration at 100\% relative energy capacity. \\
\textbf{Stress} \(S\) & 1 & Acclimated agents have minimal stress levels. \\
\(k\) & 0.75 & Energetic scaling component that follows Kleiber's Law. \\
\(\alpha\) & -0.3 to 0.5 (user-defined) & \begin{minipage}[t]{\linewidth}\centering
This value can be calibrated based on your biological assumptions:\\
• \(\alpha < 0\): Younger fish recover faster (higher turnover, rapid metabolism)\\
• \(\alpha > 0\): Older fish recover faster (more energy reserves, lower stress sensitivity)\\
• \(\alpha = 0\): Age has no effect on recovery (neutral assumption)\strut
\end{minipage} \\
\end{longtable}

\section{Submodels}\label{submodels-7}

\subsection{Trigger Conditions for Staging}\label{trigger-conditions-for-staging}

Agents will enter the staging state under either of the following conditions:

\[
E_{agent} \leq 25 \%
\]

\[
S > 5
\]

Where:

\begin{itemize}
\item
  \(E_{agent}\) is the current energy available to the agent.
\item
  \(S\) is the current stress level of the agent.
\end{itemize}

These thresholds are designed to prevent migration collapse due to exhaustion or high osmotic stress.

\subsection{Behavior During Staging}\label{behavior-during-staging}

During staging, agents move to the nearest water patch with the lowest absolute velocity to reduce energetic costs. If no such patch is found, a random neighboring water patch is selected.

Position is updated as:

\[
\vec{Y}_{t+1} = \vec{Y}_{target}
\]

Energy recovery occurs at a variable rate:

\[
E_{agent} = E_{agent} + \left((1 + \epsilon) \cdot \left( \frac{M_{agent}}{M_{max}} \right)^k \cdot \left( \frac{A_{agent}}{A_{max}} \right)^\alpha \right) \quad \text{where } \epsilon \sim U(0,1),\ k \leq 1
\]

Patch records presence of staging agents:

\[
\text{staging-in-patch} = \text{staging-in-patch} + 1
\]

Where:

\begin{itemize}
\item
  \(\vec{Y}_{t+1}\) is the position after drifting.
\item
  \(\vec{Y}_{target}\) is the agent's target spatial position.
\item
  \(V_{patch}\) is the along-channel velocity at the agent's current patch.
\item
  \(M_{agent}\) is the size of the agent.
\item
  \(M_{max}\) is the maximum size within the agent's population.
\item
  \(A_{agent}\): the age of the agent.
\item
  \(A_{max}\): the maximum age within the population.
\item
  \(E_{swim}\) is the total energy cost during passive movement.
\item
  \(k\) is the scaling exponent, reflecting nonlinear energy recovery.
\item
  \(\alpha\): an age-scaling exponent
\end{itemize}

\textbf{Suggestion}:\\

• \(\alpha = -0.25\) if modeling faster recovery in younger fish.\\

• \(\alpha = 0.25\) if modeling increased efficiency in older/larger individuals.

\subsection{Stop Conditions for Staging}\label{stop-conditions-for-staging}

Agent will remain in the staging state until both of the following conditions are met:

\[
E_{agent} \geq 75 \%
\]

\[
S = 1
\]

Where:

\begin{itemize}
\item
  \(E_{agent}\) is the current energy available to the agent.
\item
  \(S\) is the current stress level of the agent.
\end{itemize}

\chapter{Homing Behavior}\label{homing-behavior}

\section{Overview}\label{overview-9}

\section{Purpose}\label{purpose-7}

\section{Entities, State Variables, and Scales}\label{entities-state-variables-and-scales-8}

\subsection{Patch Variables}\label{patch-variables-8}

\begin{longtable}[]{@{}ll@{}}
\toprule\noalign{}
Variable Name & Definition \\
\midrule\noalign{}
\endhead
\bottomrule\noalign{}
\endlastfoot
& \\
\end{longtable}

\subsection{Agent Variables}\label{agent-variables-8}

\begin{longtable}[]{@{}ll@{}}
\toprule\noalign{}
Variable Name & Definition \\
\midrule\noalign{}
\endhead
\bottomrule\noalign{}
\endlastfoot
& \\
& \\
& \\
& \\
& \\
& \\
& \\
& \\
& \\
\end{longtable}

\section{Process Overview and Scheduling}\label{process-overview-and-scheduling-8}

\begin{enumerate}
\def\labelenumi{\arabic{enumi}.}
\tightlist
\item
\item
\item
\item
\end{enumerate}

\section{Design Concepts}\label{design-concepts-8}

\textbf{Basic Principles:}

\textbf{Emergence:}

\textbf{Adaptation}:

\textbf{Objectives:}

\textbf{Learning:}

\textbf{Prediction:}

\textbf{Sensing}:

\textbf{Interaction}:

\textbf{Stochasticity}:

\textbf{Collectives:}

\textbf{Observation:}

\section{Initialization}\label{initialization-8}

\begin{longtable}[]{@{}ccc@{}}
\toprule\noalign{}
Variable & Initialized Value & Justification \\
\midrule\noalign{}
\endhead
\bottomrule\noalign{}
\endlastfoot
& & \\
& & \\
& & \\
& & \\
& & \\
\end{longtable}

\section{Submodels}\label{submodels-8}

\chapter{Foraging Behavior}\label{foraging-behavior}

\section{Overview}\label{overview-10}

Foraging is a facultative behavior triggered under specific physiological and environmental conditions. In this model, fish evaluate nearby patches and opportunistically forage only when their energy is low, stress is minimal, and conditions are suitable. This behavior is designed to simulate optimal foraging theory for migratory fish within a spatially explicit, agent-based framework.

\section{Purpose and Patterns}\label{purpose-and-patterns-1}

The purpose of this submodel is to simulate facultative foraging behavior in migratory fish, where agents opportunistically forage during natural pauses in migration when energy is low but physiological stress is minimal. The behavior is shaped by individual condition, environmental suitability, and competition from other fish. This framework allows the model to assess foraging quality, competition, and the ecological trade-offs fish face during migration.

The following ecological patterns are represented as explicit rules in the model:

\begin{itemize}
\tightlist
\item
  \textbf{Conditional foraging triggers}:\\
  Fish forage only when \textbf{all} of the following are true:\\
  \texttt{Staging?\ =\ true}, \texttt{Energy\ \textless{}\ 75\%}, \texttt{Ionregulatory\ stress\ \textless{}\ 3}, \texttt{Thermal\ stress\ \textless{}\ 3}, and \texttt{SPM\ \textless{}\ mean\ SPM}.\\
  \emph{(Reflects behavioral ecology studies showing fish avoid feeding during stress and turbidity but will opportunistically forage when paused.)}
\item
  \textbf{Habitat-based foraging selection}:\\
  Fish score nearby patches based on salinity, temperature, depth, velocity, and turbidity (SPM), and choose the one with the highest score.\\
  \emph{(Represents optimal foraging theory and empirical habitat preference studies in estuarine fish.)}
\item
  \textbf{Intraspecific competition}:\\
  Fish can forage alongside others of the same species, but energy gain is reduced when the patch is crowded.\\
  \emph{(Captures resource sharing and diminishing returns observed in schooling behavior.)}
\item
  \textbf{Interspecific competition}:\\
  Fish will not forage in patches where larger individuals of a different species are already present. They will search again for the next best patch.\\
  \emph{(Reflects asymmetric, size-structured competition documented in estuarine communities.)}
\item
  \textbf{Size-based exclusion}:\\
  Fish avoid patches where the \textbf{combined size} of all different-species foragers is greater than their own.\\
  \emph{(Simulates energetic trade-offs in contested foraging zones.)}
\end{itemize}

\section{Entities, State Variables, and Scales}\label{entities-state-variables-and-scales-9}

\subsection{Patch Variables}\label{patch-variables-9}

\begin{longtable}[]{@{}
  >{\raggedright\arraybackslash}p{(\linewidth - 2\tabcolsep) * \real{0.5000}}
  >{\raggedright\arraybackslash}p{(\linewidth - 2\tabcolsep) * \real{0.5000}}@{}}
\toprule\noalign{}
\begin{minipage}[b]{\linewidth}\raggedright
Variable Name
\end{minipage} & \begin{minipage}[b]{\linewidth}\raggedright
Definition
\end{minipage} \\
\midrule\noalign{}
\endhead
\bottomrule\noalign{}
\endlastfoot
\textbf{Salinity} \(S_{patch}\) & The current salinity at the patch location \\
\textbf{Temperature} \(T_{patch}\) & The current temperature at the patch location \\
\textbf{Velocity} \(V_{patch}\) & The along-channel velocity at the patch \\
\textbf{Depth} \(d_{patch}\) & The water depth at the patch \\
\textbf{SPM} \(SPM\) & Suspended particulate matter concentration \\
\textbf{Forage Visits} \(forage-visits\) & The number of times a patch has been visited for foraging \\
\textbf{Forage Species} \(forage-species\) & List of species that have foraged on this patch \\
\textbf{Forager Count} \(forager-counts\) & Number of agents currently foraging on patch \\
\end{longtable}

\subsection{Agent Variables}\label{agent-variables-9}

\begin{longtable}[]{@{}
  >{\raggedright\arraybackslash}p{(\linewidth - 2\tabcolsep) * \real{0.5000}}
  >{\raggedright\arraybackslash}p{(\linewidth - 2\tabcolsep) * \real{0.5000}}@{}}
\toprule\noalign{}
\endhead
\bottomrule\noalign{}
\endlastfoot
\textbf{Variable Name} & \textbf{Definition} \\
\textbf{energy} & Current energy level of the agent \\
\textbf{foraging?} & Boolean indicating if the agent is currently foraging \\
\textbf{acclimated-salinity} & Current internal salt balance acclimated to \\
\textbf{optimal-temperature} & Preferred temperature \\
\textbf{optimal-depth} & Preferred depth \\
\textbf{optimal-velocity} & Preferred current velocity \\
\textbf{ionregulatory-stress} & Stress level based on osmoregulatory load \\
\textbf{thermal-stress} & Stress level based on thermal tolerance \\
\textbf{staging?} & Boolean for staging behavior \\
\textbf{breed} & Species classification (used to exclude non-foragers) \\
\end{longtable}

\section{Process Overview and Scheduling}\label{process-overview-and-scheduling-9}

\begin{enumerate}
\def\labelenumi{\arabic{enumi}.}
\item
  Fish evaluate local conditions only when \texttt{Staging?\ =\ true}, \texttt{Energy\ \textless{}\ 75\%}, \texttt{Ionregulatory\ stress\ \textless{}\ 3}, \texttt{Thermal\ stress\ \textless{}\ 3}, and \texttt{SPM\ \textless{}\ mean\ SPM.}
\item
  The agent assesses neighboring patches for suitability.
\item
  The best patch is selected only if no larger individual of a different species is already present. If out-competed, the agent will move to the second best patch and try again.
\item
  Energy gain is scaled down by the number of agents present on that patch.
\item
  Patch visitation and species are tracked for foraging density and competition analysis.
\end{enumerate}

\section{Design Concepts}\label{design-concepts-9}

\textbf{Basic Principles}:\\
Fish only forage during rest periods along their migration route. They do not interrupt active migration solely to seek food. Foraging occurs facultatively when conditions are favorable and the fish have temporary reprieve from other energetically costly behaviors. This reflects optimal foraging theory, where agents balance energy needs with the opportunity to feed without compromising migration.

\textbf{Emergence}:\\
Spatial foraging hotspots emerge from a combination of individual preferences, environmental suitability, and dynamic competition from other foragers.

\textbf{Adaptation}:\\
Agents decide whether to forage based on their internal energy level, physiological stress, and the presence of competitors in surrounding patches.

\textbf{Sensing}:\\
Agents sense local environmental conditions like salinity, temperature, depth, velocity, and SPM, along with, social cues like the size and species of other foragers. They avoid moving into patches occupied by larger fish or where the total biomass exceeds their own size.

\textbf{Stochasticity}:\\
Foraging success is not deterministic; energy gain includes a random multiplier to account for variability in foraging efficiency and prey availability.

\textbf{Interaction}:\\
Competition is modeled through species- and size-based exclusion. Agents will not forage in a patch if a larger individual of a different species is already present. However, they may forage alongside larger individuals of their own species. If out-competed, the agent moves to a random neighboring patch. Energy gain is shared among all foragers on the patch, simulating diminishing returns under crowding.

\textbf{Observation}:\\
The model records foraging events, patch visitation frequency, species identity, and total forager biomass per patch to support analyses of habitat quality and interspecific competition.

\section{Initialization}\label{initialization-9}

\begin{longtable}[]{@{}lll@{}}
\toprule\noalign{}
\endhead
\bottomrule\noalign{}
\endlastfoot
\textbf{Variable} & \textbf{Initialized Value} & \textbf{Justification} \\
\textbf{energy} & 100 & Represents full energy at migration onset \\
\textbf{foraging?} & false & Starts false; updated based on conditions \\
\textbf{ionregulatory-stress} & 1 & Minimal stress at initialization \\
\textbf{thermal-stress} & N/A & Calculated from environmental conditions \\
\textbf{forage-visits} & 0 & No patch has been visited at setup \\
\textbf{forage-species} & empty list & No species recorded at setup \\
\textbf{forager-count} & 0 & No agents foraging initially \\
\end{longtable}

\section{Submodels}\label{submodels-9}

\subsection{Trigger Conditions for Foraging}\label{trigger-conditions-for-foraging}

Fish initiate foraging if:

\[
staging? = \text{true} \quad \text{and} \quad E_{agent} < 75 \quad \text{and} \quad S_{ion} < 3 \quad \text{and} \quad S_{thermal} < 3 \quad \text{and} \quad SPM < \overline{SPM}\]

\begin{itemize}
\tightlist
\item
  \texttt{staging?\ =\ true} (i.e., the fish is not actively migrating)
\end{itemize}

\begin{itemize}
\tightlist
\item
  \texttt{energy\ \textless{}\ 75\%} (low energy state)
\end{itemize}

\begin{itemize}
\tightlist
\item
  \texttt{ionregulatory-stress\ \textless{}\ 3} (low osmoregulatory stress)
\end{itemize}

\begin{itemize}
\tightlist
\item
  \texttt{thermal-stress\ \textless{}\ 3} (within thermal tolerance)
\end{itemize}

\begin{itemize}
\tightlist
\item
  \texttt{SPM\ \textless{}\ mean\ SPM} (low turbidity)
\end{itemize}

\subsection{Patch Evaluation}\label{patch-evaluation}

Each agent evaluates neighboring patches for optimal foraging conditions:

\[
Score = \left(0.2 \cdot Sal_{opt} + 0.3 \cdot temp_{opt} + 0.2 \cdot vel_{opt} + 0.2 \cdot depth_{opt} + 0.1 \cdot spm_{penalty} \right) \cdot (\text{random}(0,3.0))
\]

Where:

\begin{itemize}
\item
  \(sal_{opt} = 1 - \frac{|salinity - acclimated\ salinity|}{10}\)
\item
  \(temp_{opt} = 1 - \frac{|temperature - optimal\ temperature|}{10}\)
\item
  \(vel_{opt} = 1 - |velocity - optimal\ velocity|\)
\item
  \(depth_{opt} = 1 - \frac{|depth - optimal\ depth|}{2.5}\)
\item
  \(spm_{penalty} = 1 - \frac{SPM - mean\ SPM}{mean\ SPM}\) (bounded to {[}0,1{]})
\end{itemize}

\subsection{Species and Size-Based Competition Constraints}\label{species-and-size-based-competition-constraints}

Agents will only forage in a patch if either (1) the patch is unoccupied, (2) the patch contains only members of the same species, or (3) all other individuals on the patch are smaller if they are of a different species. This behavior simulates asymmetric, size-based competition, where larger individuals of different species out-compete smaller fish for access to high-quality foraging habitat.

Then foraging is allowed only if:

\[
b_{\text{agent}} \in \mathcal{B}_{\text{others}} \quad \text{or} \quad \left( S_{\text{sum}} < S_{\text{agent}} \text{ and } b_{\text{agent}} \notin \mathcal{B}_{\text{others}} \right)
\]

Where:

\begin{itemize}
\tightlist
\item
  \(S_{\text{agent}}\): size of the focal fish\\
\item
  \(S_{\text{sum}} = \sum_{i} S_{\text{agent}_{\mathcal{b}_{\text{others}}\, i}}\)\\
\item
  \(\mathcal{b}_{\text{others}}\): set of breeds (species) of other foragers on the patch\\
\item
  \(b_{\text{agent}}\): the breed of the focal fish
\end{itemize}

If a patch is occupied by a larger fish of a different species, the agent is considered out-competed and will seek the next most suitable patch among its neighbors. This ensures that foraging attempts are adaptive to interspecific competition.

\subsection{Energy Gain}\label{energy-gain}

Energy gain is then calculated:

\[
E_{\text{gain}} = \left( \frac{\text{Score}_i}{\sum_{j=1}^{N} size_j} \right) \cdot size_i\]

Where:

\begin{itemize}
\item
  \(Score_{i}\) is the individual fish's foraging score (bounded between 0 and 3)
\item
  \(Size_{j}\) is the sum of sizes for all \(N\) foraging fish on that patch
\item
  \(\sum_{j=1}^{N} size_{j}\) is the sum of sizes for all \(N\) foraging fish on that patch
\end{itemize}

\subsection{Foraging Patch Tracking}\label{foraging-patch-tracking}

Each time a fish forages on a patch, the following patch-level variables are updated:

\begin{itemize}
\item
  \texttt{forage-visits}: Increments by 1 with each foraging event, enabling spatial analysis of foraging intensity.
\item
  \texttt{forager-count}: Tracks the number of agents currently foraging on the patch, reflecting competition density.
\item
  \texttt{forage-species}: Stores a list of species that have foraged on the patch, supporting interspecific competition analysis.
\end{itemize}

\chapter{Predation and Fleeing Behavior}\label{predation-and-fleeing-behavior}

\section{Overview}\label{overview-11}

This module simulates predator--prey interactions between migratory fish species, specifically predators (i.e., striped bass) and prey (i.e., alewife), during key migratory and staging periods. Predators detect, pursue, and consume prey based on visibility conditions and prey size constraints. Prey detect approaching predators through visual and social cues and respond with directional fleeing behaviors, which vary based on their size, age, energy level, and water clarity. These behaviors reflect biologically observed trade-offs between escape speed, reaction time, and group-based vigilance. The model also tracks predator energy gain and prey energy depletion and recovery, allowing for analysis of individual fitness and risk exposure under different environmental and behavioral conditions.

\section{Purpose and Patterns}\label{purpose-and-patterns-2}

This submodel captures ecologically grounded predator--prey dynamics by simulating behaviorally realistic rules of pursuit and evasion observed in estuarine fishes.

\begin{itemize}
\item
  Prey initiate fleeing based on predator proximity, visual detection, and turbidity-adjusted reaction time, reflecting sensory limitations in murky estuarine water.
\item
  Predators only pursue prey that are within their gape limit and provide a favorable energy return, consistent with optimal foraging theory and size-selective predation.
\item
  Fleeing speed is scaled by energy and body size, capturing the biologically observed trade-off between fatigue and escape performance.
\item
  Directional fleeing (left, right, up, or down) is based on the predator's relative position, aligning with empirical studies on spatial escape responses.
\item
  Patch-level alarm cues propagate fleeing behavior across prey groups, mimicking the benefits of schooling and collective vigilance in fish.
\item
  Energetic constraints influence both predator foraging frequency and prey escape success, contributing to variability in encounter outcomes.
\item
  Spatial tracking of predation events allows identification of high-risk zones, offering insight into how environmental conditions shape survival landscapes.
\end{itemize}

\section{Entities, State Variables, and Scales}\label{entities-state-variables-and-scales-10}

\subsection{Spatial and Temporal Scales}\label{spatial-and-temporal-scales-1}

\begin{itemize}
\tightlist
\item
  \textbf{Spatial Unit}: Patch (3 m x 3 m resolution)
\end{itemize}

\begin{itemize}
\tightlist
\item
  \textbf{Temporal Unit}: 5-minute time steps (\texttt{tick})
\end{itemize}

\subsection{Patch Variables}\label{patch-variables-10}

\begin{longtable}[]{@{}
  >{\raggedright\arraybackslash}p{(\linewidth - 2\tabcolsep) * \real{0.5000}}
  >{\raggedright\arraybackslash}p{(\linewidth - 2\tabcolsep) * \real{0.5000}}@{}}
\toprule\noalign{}
\begin{minipage}[b]{\linewidth}\raggedright
Variable Name
\end{minipage} & \begin{minipage}[b]{\linewidth}\raggedright
Definition
\end{minipage} \\
\midrule\noalign{}
\endhead
\bottomrule\noalign{}
\endlastfoot
\textbf{Suspended-particulate-matter} \(SPM_{t}\) & Level of suspended material in patch (exponential difficulty increase between min and max observed in system) worse vision at max, best vision at min \\
\textbf{Prey-Eaten-in-Patch} & Counts how much prey are consumed in a patch \\
\textbf{Prey-in-Patch} \(prey-in-patch_{t}\) & The amount of prey currently in the patch \\
\textbf{prey-alarmed?} & Boolean variable set to true if any prey in the patch initiates fleeing behavior, triggering collective escape. \\
\end{longtable}

\subsection{Prey: Agent Variables}\label{prey-agent-variables}

\begin{longtable}[]{@{}
  >{\raggedright\arraybackslash}p{(\linewidth - 2\tabcolsep) * \real{0.5000}}
  >{\raggedright\arraybackslash}p{(\linewidth - 2\tabcolsep) * \real{0.5000}}@{}}
\toprule\noalign{}
\begin{minipage}[b]{\linewidth}\raggedright
Variable Name
\end{minipage} & \begin{minipage}[b]{\linewidth}\raggedright
Definition
\end{minipage} \\
\midrule\noalign{}
\endhead
\bottomrule\noalign{}
\endlastfoot
vision & Radius (in patches) within which the prey can detect predators. Decreases with high SPM. \\
fleeing? & Boolean indicator of whether the prey is actively escaping from a predator. \\
energy & Energy levels of an agent \\
swimming speed & The speed at which a prey agent is moving \\
max-speed & Maximum normal swim speed (not in a flee state). \\
max-flee & Maximum achievable fleeing speed, calculated as a function of max-speed, swimming-speed, size, and max-rate-of-speed-change. \\
reaction-time & Time delay in response to predator presence. Increases with smaller size, higher SPM, larger/faster predators, and lower flee-ability. \\
size & Body length (size) of the prey; influences visibility and escape ability. \\
age & Agent's age class; mid-range age groups tend to have better escape success (a normal distribution across age classes). \\
flee-ability & Learning metric based on age; moderate-aged fish have best reflexes and performance. \\
\end{longtable}

\subsection{Predator: Agent Variables}\label{predator-agent-variables}

\begin{longtable}[]{@{}
  >{\raggedright\arraybackslash}p{(\linewidth - 2\tabcolsep) * \real{0.5000}}
  >{\raggedright\arraybackslash}p{(\linewidth - 2\tabcolsep) * \real{0.5000}}@{}}
\toprule\noalign{}
\begin{minipage}[b]{\linewidth}\raggedright
Variable Name
\end{minipage} & \begin{minipage}[b]{\linewidth}\raggedright
Definition
\end{minipage} \\
\midrule\noalign{}
\endhead
\bottomrule\noalign{}
\endlastfoot
vision & Radius (in patches) within which predator can visually detect prey. Decreases with high SPM or low prey size. \\
swimming-speed & The speed at which a predator agent is moving \\
max-speed & Maximum normal swim speed (not in a burst state). \\
bursting? & Boolean indicating whether the predator is currently in a high-speed chase or attack behavior. \\
prey-in-vision & Agentset of prey within vision range \\
daytime-prey-eaten & Count of prey eaten that day \\
time-since-full & Time since last feeding (when will predator be hungry again?) \\
reaction-time & Delay between prey detection and initiation of chase. Influenced by predator size/age, prey density, prey speed, water clarity (SPM), and predation ability. \\
limit-daily-prey-allowance & Max prey allowed per day (predator becomes full) \\
gape-limit & Maximum size of prey the predator can successfully capture and ingest. \\
handling-effort & Time and energy cost required to subdue and consume a prey item. Depends on distance, prey size, and prey density. Used in patch and prey selection decisions. \\
size & Body length (size) of the prey; influences visibility and escape ability. \\
age & Age class of the predator agent. Predation success follows a normal distribution across age classes. \\
predation-ability & Agent's age class; mid-range age groups tend to have better predation success (a normal distribution across age classes). \\
prey-species-eaten & Species identifiers of consumed prey; enables diet tracking and multispecies modeling. \\
total-prey-eaten & Cumulative prey consumed across all days or scenarios. \\
digestion-time & Time it takes to digest prey \\
\end{longtable}

\section{Process Overview and Scheduling}\label{process-overview-and-scheduling-10}

\subsection{Prey Behavior (per agent)}\label{prey-behavior-per-agent}

\begin{enumerate}
\def\labelenumi{\arabic{enumi}.}
\item
  \textbf{Fleeing Trigger}\\
  Each prey evaluates whether a predator is in its visual cone or whether \texttt{prey-alarmed?\ =\ true} in the patch. If so, the prey calculates reaction time based on its own size, age, flee-ability, local SPM, and the predator's size and speed.
\item
  \textbf{Directional Escape}\\
  If the reaction time threshold is met, the prey executes a directional fleeing behavior (\texttt{scare-left}, \texttt{scare-right}, \texttt{scare-down}, or \texttt{scare-up}), and sets the patch to \texttt{prey-alarmed?\ =\ true}.
\item
  \textbf{Schooling Response}\\
  All other prey in the same patch automatically begin fleeing when \texttt{prey-alarmed?\ =\ true}, simulating social alarm propagation.
\item
  \textbf{Speed and Energy Update}\\
  If the agent is fleeing, its speed increases toward \texttt{max-flee}, and energy is reduced based on movement cost. If not fleeing, the agent moves normally and may recover energy depending on its resting status.
\end{enumerate}

\subsection{Predator Behavior (per agent)}\label{predator-behavior-per-agent}

\begin{enumerate}
\def\labelenumi{\arabic{enumi}.}
\item
  \textbf{Visual Detection}\\
  The predator identifies prey within its in-cone vision, adjusted based on local SPM concentration, prey size, and predator size.
\item
  \textbf{Prey Filtering}\\
  Prey that exceed the predator's gape limit or whose handling effort exceeds the predator's available energy are excluded.
\item
  \textbf{Prey Selection}\\
  Among remaining prey, the predator evaluates candidates using an optimal foraging approach that considers size, handling effort, and net energy gain. One prey is selected.
\item
  \textbf{Pursuit Decision}\\
  If a prey is selected, the predator enters a bursting state, increasing its speed toward max-speed. Reaction time is computed based on predator age, size, SPM, and prey proximity.
\item
  \textbf{Scare Prey}\\
  The predator triggers fleeing in visible prey by calling the \texttt{scare-prey} procedure. If any prey flees, the patch is flagged as \texttt{prey-alarmed?\ =\ true}.
\item
  \textbf{Consume Prey}\\
  If within striking range, the predator consumes the prey and updates variables: \texttt{daytime-prey-eaten}, \texttt{prey-species-eaten}, \texttt{total-prey-eaten}, and \texttt{Prey-Eaten-in-Patch}.
\item
  \textbf{Digestion}\\
  If no prey are eaten, \texttt{time-since-eaten} increases. If the predator has reached its daily feeding limit, it enters a digestive phase. Digestion ends when both \texttt{time-since-eaten} and \texttt{time-since-full} exceed the digestion time
\end{enumerate}

\section{Design Concepts}\label{design-concepts-10}

\textbf{Basic Principles:} Simulates predator and prey interactions in a spatial estuarine system. Behaviors such as chasing, fleeing, and feeding are determined by individual traits, patch conditions, and environmental clarity due to suspended particulate matter.

\textbf{Emergence:} Predation hotspots develop in areas with higher prey density and better visibility. Prey movement and group alarm responses create shifting patterns of refuge and risk across the landscape.

\textbf{Adaptation}: Prey modify their direction and intensity of escape based on the predator's location and their own size, age, and escape ability. Predators filter out prey that are too large to consume and shift to resting behavior after reaching their feeding limit.

\textbf{Objectives:} Predators aim to select prey that provide the highest net energy gain while minimizing pursuit and handling costs. Prey aim to avoid detection or flee successfully using both individual detection and social alarm cues.

\textbf{Learning:} Learning is represented through Mid-aged agents generally perform better than very young or very old individuals.

\textbf{Prediction:}

\textbf{Sensing}: Agents rely on in-radius and in-cone vision to detect others. Detection range is influenced by water clarity, body size, and prey movement. Predators exclude prey that are too large to handle even if they fall within the visible field.

\textbf{Interaction}: Predators and prey interact through visual detection, spatial proximity, and energy transfer through feeding. Social interactions occur through alarm cue propagation at the patch level and school level.

\textbf{Stochasticity}: Prey selection, fleeing direction, and reaction timing include probabilistic variation. Randomness is used to reflect natural variability in movement and perception.

\textbf{Collectives:} When one prey detects a predator and flees, all prey in the same patch respond. This collective behavior improves predator detection and increases the likelihood of survival through schooling.

\textbf{Observation:} Key outputs include prey consumption, alarm responses, spatial predation distribution, and energy use of individual agents.

\section{Initialization}\label{initialization-10}

\begin{longtable}[]{@{}
  >{\centering\arraybackslash}p{(\linewidth - 4\tabcolsep) * \real{0.3333}}
  >{\centering\arraybackslash}p{(\linewidth - 4\tabcolsep) * \real{0.3333}}
  >{\centering\arraybackslash}p{(\linewidth - 4\tabcolsep) * \real{0.3333}}@{}}
\toprule\noalign{}
\begin{minipage}[b]{\linewidth}\centering
Variable
\end{minipage} & \begin{minipage}[b]{\linewidth}\centering
Initialized Value
\end{minipage} & \begin{minipage}[b]{\linewidth}\centering
Justification
\end{minipage} \\
\midrule\noalign{}
\endhead
\bottomrule\noalign{}
\endlastfoot
energy & 100 & Prey begin fully energized to allow immediate fleeing or normal movement \\
fleeing? & false & Prey are not actively escaping at the start \\
prey-alarmed? & false & Alarm cue inactive at start of simulation \\
daytime-prey-eaten & 0 & daily counter of prey eaten starts at zero \\
time-since-full & 0 & predator begins simulation hungry \\
gape-limit & function of size & Large predators have wider gape limits \\
handling-effort & calculated per encounter & Depends on prey traits and local patch conditions \\
reaction-time & size-, age-, and SPM-based & Prey and predator values dynamically computed each tick \\
flee-ability & based on age & Represents individual escape competence across age classes \\
predation-ability & based on age & Mid-age predators have fastest and most accurate prey responses \\
max-speed & 1.5-3.0x size & \\
\end{longtable}

\section{Submodels}\label{submodels-10}

\subsection{Prey Detection Probability}\label{prey-detection-probability}

Visual detection of prey by predators is scaled by local water clarity, prey size, and prey movement. The effective detection radius decreases as SPM increases and increases as prey size increases.

\[
P_{detect} = \left( \frac{S_{prey}}{S_{max}} \right) \cdot e^{-SPM_t / \tau}
\]

Where:

\begin{itemize}
\item
  \(P_{detect}\) is the probability of detecting a prey agent.
\item
  \(S_{prey}\) is the size of the prey agent.
\item
  \(S_{max}\) is the maximum size of any prey in the system.
\item
  \(SPM_t\) is the suspended particulate matter concentration in the current patch.
\item
  \(\tau\) is a turbidity scaling constant.
\end{itemize}

Biological Justification: Larger prey are more visible, but high turbidity (SPM) reduces contrast and visual range. The exponential function reflects rapid degradation of visibility with increased turbidity.

\subsection{Reaction Time}\label{reaction-time}

Reaction time determines how quickly an agent initiates a behavior in response to a threat (for prey) or opportunity (for predators). It is influenced by SPM, size, age, and ability.

\[
RT_{agent} = \left( \frac{1}{F_{ability}} \cdot \frac{S_{opp}}{S_{agent}} \right) \cdot (1 + \alpha \cdot SPM_t)
\]

Where:

\begin{itemize}
\item
  \(RT_{agent}\) is the reaction time of the prey or predator.
\item
  \(F_{ability}\) is flee-ability (for prey) or predation-ability (for predators).
\item
  \(S_{opp}\) is the size of the opposing agent (predator or prey).
\item
  \(S_{agent}\) is the size of the focal agent.
\item
  \(SPM_t\) is the suspended particulate matter in the patch.
\item
  \(\alpha\) is the turbidity penalty factor.
\end{itemize}

Biological Justification: Smaller or younger fish tend to react slower. High turbidity delays response time. Age-related ability improves reaction time in middle-aged agents.

\subsection{Alarm Response Propagation}\label{alarm-response-propagation}

\[
P_{alarm} = 1 - e^{\frac{-n_{flex}}{k}}
\]

Where:

\begin{itemize}
\item
  \(P_{alarm}\) is the probability that non-detecting prey will flee.
\item
  \(n_{flee}\) is the number of fleeing fish in the patch.
\item
  \(\kappa\) is a sensitivity constant governing social responsiveness.
\end{itemize}

\textbf{Biological Justification:} Prey benefit from collective vigilance. The more individuals that flee in a patch, the more likely others will follow.

\subsection{Fleeing Speed}\label{fleeing-speed}

The maximum speed a prey agent can reach when fleeing is scaled by body size, energy, and physical acceleration limits.

\[
V_{flee} = V_{max} \cdot \left( \frac{E_{agent}}{100} \right) + \delta \cdot S_{agent}
\]

Where:

\begin{itemize}
\item
  \(V_{flee}\) is the maximum fleeing speed.
\item
  \(V_{max}\) is the prey's maximum speed.
\item
  \(E_{agent}\) is the energy level of the prey agent.
\item
  \(S_{agent}\) is the size of the prey.
\item
  \(\delta\) is a scaling factor for size-based speed increase.
\end{itemize}

Biological Justification: Larger prey generally swim faster, but energy limits how much of that speed can be used. Exhausted fish flee more slowly.

\subsection{Handling Effort}\label{handling-effort}

Predators calculate the cost of pursuing and consuming prey using the prey's size and density, and the distance to the target.

\[
H_{effort} = \gamma \cdot S_{prey} \cdot D_{patch} \cdot \left(1 + \frac{1}{\rho_{patch}} \right)
\]

Where:

\begin{itemize}
\item
  \(H_{effort}\) is the handling effort for the predator.
\item
  \(S_{prey}\) is the size of the prey.
\item
  \(D_{patch}\) is the distance between predator and prey.
\item
  \(\rho_{patch}\) is the prey density in the patch.
\item
  \(\gamma\) is a handling cost coefficient.
\end{itemize}

Biological Justification: Prey that are farther away or more dispersed are harder to catch. Denser prey clusters reduce handling time by enabling rapid repeat captures.

\subsection{Gape Filtering}\label{gape-filtering}

Predators apply a binary filter before pursuit, excluding prey that exceed their maximum ingestible size.

\[
P_{pursue} =\begin{cases}1 & \text{if } S_{prey} \leq G_{pred} \\0 & \text{otherwise}\end{cases}
\]

Where:

\begin{itemize}
\item
  \(P_{pursue}\) is the pursuit decision (1 = pursue, 0 = ignore).
\item
  \(S_{prey}\) is the size of the prey.
\item
  \(G_{pred}\) is the predator's gape limit.
\end{itemize}

Biological Justification: Predators cannot capture or ingest prey that are too large, so these individuals are ignored even if they are detected.

\subsection{Energy Gain by Predators}\label{energy-gain-by-predators}

\[
E_{agent} = E_{agent} - E_{burst}
\]

Where:

\begin{itemize}
\item
  \(E_{agent}\) is the current energy of the predator.
\item
  \(E_{burst}\) is the energy cost per unit of bursting behavior.
\end{itemize}

\textbf{Biological Justification:} Bursting reduces available energy. Energy cost is a limiting factor for predation.

\subsection{Energy Depletion by Prey}\label{energy-depletion-by-prey}

\[
E_{agent} = E_{agent} - E_{flee}
\]

Where:

\begin{itemize}
\item
  \(E_{agent}\) is the current energy of the prey.
\item
  \(E_{flee}\) is the energy cost per unit of fleeing behavior.
\end{itemize}

\textbf{Biological Justification:} Fleeing reduces available energy and increases recovery time. Energy loss is a limiting factor for repeated escape attempts.

(no need to account for swimming energy in addition if you are using the migration functions)

\chapter{Spawning Behavior}\label{spawning-behavior}

\section{Overview}\label{overview-12}

This submodel simulates the conditions, triggers, and behavioral outcomes associated with spawning in anadromous fish. It integrates physiological thresholds, habitat preferences, reproductive strategies, and energy-based constraints to identify spawning events and their consequences during migration.

\section{Purpose and Patterns}\label{purpose-and-patterns-3}

This submodel captures ecologically grounded reproductive strategies in anadromous fishes by simulating individual-based spawning behavior as a function of habitat, physiological readiness, and reproductive strategy.

\begin{itemize}
\item
  Spawning is initiated when agents meet combined salinity and thermal stress thresholds and have sufficient energy reserves, reflecting empirically observed constraints on reproductive viability in migratory fishes.
\item
  Broadcast spawners initiate spawning in freshwater patches with suitable depth and flow conditions, consistent with the behavior of species like alewife and rainbow smelt. Fertilization success is probabilistically scaled by the local density of mature males, reflecting external gamete mixing dynamics in flowing water.
\item
  Pairwise spawners, such as sturgeon and striped bass, require physical co-location of a reproductively ready male and female in suitable habitat. This reflects species that engage in close-contact courtship and synchronize gamete release based on microhabitat cues.
\item
  Spawning events result in an immediate energy cost and update internal states for both male and female agents. Iteroparous species may spawn more than once if conditions remain favorable, while semelparous species are subject to mortality following reproductive exhaustion.
\item
  Energetic depletion after spawning triggers resting or staging behavior before resuming seaward migration. Overwintering probability is dynamically calculated based on energy levels, spawning timing, age, and size, reflecting seasonal decisions observed in species like Atlantic salmon and shortnose sturgeon.
\item
  Spatial overlap of spawning-ready agents in high-quality patches allows emergent identification of reproductive hotspots and helps model the intersection of life history traits and estuarine hydrodynamics.
\end{itemize}

\section{Entities, State Variables, and Scales}\label{entities-state-variables-and-scales-11}

\subsection{Spatial and Temporal Scales}\label{spatial-and-temporal-scales-2}

\begin{itemize}
\tightlist
\item
  \textbf{Spatial Unit}: Patch (3 m x 3 m resolution)
\item
  \textbf{Temporal Unit}: 5-minute time steps (\texttt{tick})
\end{itemize}

\subsection{Patch Variables}\label{patch-variables-11}

\begin{longtable}[]{@{}
  >{\raggedright\arraybackslash}p{(\linewidth - 2\tabcolsep) * \real{0.5000}}
  >{\raggedright\arraybackslash}p{(\linewidth - 2\tabcolsep) * \real{0.5000}}@{}}
\toprule\noalign{}
\begin{minipage}[b]{\linewidth}\raggedright
Variable Name
\end{minipage} & \begin{minipage}[b]{\linewidth}\raggedright
Definition
\end{minipage} \\
\midrule\noalign{}
\endhead
\bottomrule\noalign{}
\endlastfoot
Salinity \(S_{patch}\) & anadromous fish are known to typically spawn in freshwater reaches of the estuary \\
Temperature \(T_{patch}\) & Patch-level temperature from hydrodynamic model \\
Depth \(d_{patch}\) & Patch-level water column depth \\
Velocity \(V_{patch}\) & Patch-level water velocity \\
spawning-in-patch & Boolean flag indicating active spawning event \\
\end{longtable}

\subsection{Agent Variables}\label{agent-variables-10}

\begin{longtable}[]{@{}
  >{\raggedright\arraybackslash}p{(\linewidth - 2\tabcolsep) * \real{0.5000}}
  >{\raggedright\arraybackslash}p{(\linewidth - 2\tabcolsep) * \real{0.5000}}@{}}
\toprule\noalign{}
\begin{minipage}[b]{\linewidth}\raggedright
Variable Name
\end{minipage} & \begin{minipage}[b]{\linewidth}\raggedright
Definition
\end{minipage} \\
\midrule\noalign{}
\endhead
\bottomrule\noalign{}
\endlastfoot
size \(M_{agent}\) & Individual fish body size \\
age \(A_{agent}\) & Individual fish age \\
sex \(s_{agent}\) & Sex of agent \\
energy \(E_{agent}\) & Current metabolic energy level \\
spawning energy \(E_{spawn}\) & Energy cost to spawn \\
spawning encounters \(spawn_{encounters}\) & Count of fertilization events related to agent's spawn \\
spawns \(spawn_{n}\) & Number of individual spawning encounters \\
spawning limit \(lim_{spawn}\) & Maximum spawns allowed for individual \\
spawning strategy & Reproductive mode (broadcast or pairwise) \\
Salinity Stress \(S_{stress}\) & Osmotic stress response \\
Thermal Stress \(T_{stress}\) & Deviation from thermal optima \\
Staging? & Boolean for if fish is staging \\
STST? & Boolean for if agent is using Selective Tidal Stream Transport \\
\end{longtable}

\section{Process Overview and Scheduling}\label{process-overview-and-scheduling-11}

\begin{enumerate}
\def\labelenumi{\arabic{enumi}.}
\item
  \textbf{Spawning Readiness}\\
  Each agent evaluates whether it is spawning-ready based on salinity stress, thermal stress, energy levels, and whether it has exceeded its spawning limit. If all conditions are met, the agent is considered ready to spawn.
\item
  \textbf{Spawning Strategy}\\
  For broadcast spawners, the female initiates spawning if the patch meets the suitability criteria. Fertilization probability is calculated based on the number of males within the local radius. For pairwise spawners, spawning occurs only if one male and one female are both in the same patch, both are spawning-ready, and habitat conditions meet the required criteria.
\item
  \textbf{Spawning Energy}\\
  After spawning, the agent's energy is reduced by the spawning energy cost.
\item
  \textbf{Post-Spawning Behavior}\\
  If the agent's energy is sufficient, it switches to seaward migration. If energy is below the threshold, the agent enters a resting/staging state until its energy is restored.
\item
  \textbf{Mortality Risk}\\
  For semelparous species, the mortality risk increases after spawning, with the probability increasing based on age. For iteroparous species, the mortality risk is lower and also age-dependent.
\item
  \textbf{Overwintering Probability}\\
  The overwintering probability is calculated based on the agent's energy, age, size, and days since the last spawning. Overwintering is more likely for older fish, those with low energy, or those that spawned recently.
\end{enumerate}

\section{Design Concepts}\label{design-concepts-11}

\textbf{Basic Principles:} Spawning behavior in anadromous fish is influenced by a combination of environmental conditions (salinity, temperature, depth, and velocity) and physiological readiness. Fish assess these conditions and their internal energy reserves to decide when and where to spawn.

\textbf{Emergence:} High-quality spawning patches with favorable environmental conditions emerge, and spawning encounters increase in these areas. Overlap of spawning-ready agents in these patches leads to emergent patterns of reproductive success.

\textbf{Adaptation:} Fish respond to varying environmental conditions by adjusting their spawning behavior. Broadcast spawners will select the highest probability patch based on the presence of males, while pairwise spawners rely on physical proximity for spawning. Iteroparous species may spawn multiple times in a season if conditions remain favorable, whereas semelparous species spawn once and die.

\textbf{Objectives:} The primary objective of spawning behavior is to maximize reproductive success. This is achieved by ensuring optimal environmental conditions for gamete fertilization while minimizing the energetic cost of spawning and migration.

\textbf{Prediction:} Broadcast spawners do not predict future conditions but react to local male density and environmental suitability. Pairwise spawners predict spawning success based on habitat conditions and the availability of a mate.

\textbf{Sensing:} Agents sense local patch temperature, salinity, depth, and velocity. They also assess the presence of other agents (both male and female) to determine readiness for spawning.

\textbf{Interaction:} Male and female agents interact through proximity in pairwise spawning, while broadcast spawners interact with males via fertilization probability calculations. Both strategies involve the exchange of gametes, with varying degrees of synchronization and spatial organization.

\textbf{Stochasticity:} The model includes probabilistic factors in fertilization success and mortality. For broadcast spawners, fertilization success is probabilistically scaled by the number of males, while mortality is age-dependent.

\textbf{Observation:} The model tracks spawning encounters, spawning energy expenditures, energy recovery, mortality rates, and overwintering probabilities.

\section{Initialization}\label{initialization-11}

\begin{longtable}[]{@{}
  >{\raggedright\arraybackslash}p{(\linewidth - 4\tabcolsep) * \real{0.2500}}
  >{\raggedright\arraybackslash}p{(\linewidth - 4\tabcolsep) * \real{0.2500}}
  >{\raggedright\arraybackslash}p{(\linewidth - 4\tabcolsep) * \real{0.5000}}@{}}
\toprule\noalign{}
\begin{minipage}[b]{\linewidth}\raggedright
Variable
\end{minipage} & \begin{minipage}[b]{\linewidth}\raggedright
Initialized Value
\end{minipage} & \begin{minipage}[b]{\linewidth}\raggedright
Justification
\end{minipage} \\
\midrule\noalign{}
\endhead
\bottomrule\noalign{}
\endlastfoot
\(E_{agent}\) & 100 & Assumes full energy at initialization \\
\(E_{spawn}\) & species-specific & Threshold to permit spawning \\
\(spawned-this-tick?\) & false & Agent has not spawned yet \\
\(A_{agent}\) & population distribution & Age assigned from population parameters \\
\(M_{agent}\) & size-at-age curve & Species-specific size growth \\
\(spawning-strategy\) & broadcast or pairwise & Species-level trait \\
\(lim_{spawn}\) & dependent on sex and species & How many times a fish is capable of spawning in a single migration \\
\(S_{stress}\) & 0 & No stress at initialization \\
\(T_{stress}\) & 0 & No stress at initialization \\
\(S_{overwinter}\) & 0 or 1 & Binary species trait \\
\(t_{spawn}\) & 0 & Time since last spawning \\
\(E_{recover}\) & 75 & Required energy to resume post-spawning migration \\
\end{longtable}

\section{Submodels}\label{submodels-11}

\subsection{Spawning Readiness}\label{spawning-readiness}

Agents are considered spawning-ready if they meet the following conditions:

\[
\text{SpawningReady} = \begin{cases}\text{true}, & \text{if } S_{stress} < 3 \text{ and } T_{stress} < 3 \text{ and } E_{agent} > E_{spawn} \text{ and } spawn_{n} < lim_{spawn}  \\\text{false}, & \text{otherwise}\end{cases}
\]

\begin{itemize}
\tightlist
\item
  \(S_{stress}\) = salinity stress
\item
  \(T_{stress}\) = thermal stress
\item
  \(E_{agent}\) = current energy level of agent
\item
  \(E_{spawn}\) = minimum energy required for spawning (species-specific)
\item
  \(spawn_{n}\) how many spawns agent has performed
\end{itemize}

\subsection{\texorpdfstring{\textbf{Spawning Strategy}}{Spawning Strategy}}\label{spawning-strategy}

\paragraph{\texorpdfstring{If \texttt{spawning-strategy\ =\ broadcast}:}{If spawning-strategy = broadcast:}}\label{if-spawning-strategy-broadcast}

\begin{enumerate}
\def\labelenumi{\arabic{enumi}.}
\tightlist
\item
  Female agent initiates spawning if the patch meets habitat criteria.

  \begin{itemize}
  \tightlist
  \item
    \(S_{patch} <= 0.5\)
  \item
    \(d_{patch} <= 3\)
  \item
    \(staging? = false\)
  \item
    \(STST? = false\)
  \end{itemize}
\item
  The agent evaluates \textbf{neighboring patches} for the \textbf{fertilization probability} (\(P_{fertilize}\)) sexually mature males in the local radius (\(n_{males}\))\hspace{0pt}.
\end{enumerate}

The fertilization probability is calculated as:

\[
P_{fertilize} = 1 - e^{- \alpha \cdot n_{males}}
\]

Where:

\begin{itemize}
\item
  \(P_{fertilize}\) fertilization success probability
\item
  \(\alpha\) = proximity sensitivity constant
\item
  \(n_{males}\) = number of sexually mature males within a defined radius
\end{itemize}

If no suitable patch with \(P_{fertilize} > 0.5\) is found, the female moves to the next patch and repeats the evaluation process until a suitable patch is identified.

Alternatively, if the female encounters a male in the same patch, the fertilization probability is considered high (near 1), and the female spawns immediately without evaluating neighboring patches.

Spawning encounters are recorded as:

\[
spawn_{n} = spawn_{n} + 1
\]

\[
Spawn_{encounters} = \sum_{i=1}^{N_f} P_{fertilize,i}
\]

Where:

\begin{itemize}
\item
  \(spawn_{encounters}\) is the number of spawning encounters
\item
  \(P_{fertilize}\) is the probability of fertilization.
\item
  \(N_{f}\) is the number of spawning-ready females in suitable patches.
\end{itemize}

\paragraph{\texorpdfstring{If \texttt{spawning-strategy\ =\ pairwise}:}{If spawning-strategy = pairwise:}}\label{if-spawning-strategy-pairwise}

\begin{enumerate}
\def\labelenumi{\arabic{enumi}.}
\tightlist
\item
  Once a female is spawning-ready, she locates the best spawning habitat:
\end{enumerate}

\begin{verbatim}
-   $S_{patch} <= 0.5$
-   $d_{patch} <= 3$
-   $staging? = false$
-   $STST? = false$
\end{verbatim}

\begin{enumerate}
\def\labelenumi{\arabic{enumi}.}
\setcounter{enumi}{1}
\item
  Spawning-ready males actively search for the closest spawning-ready female. This search is based on both visual cues (proximity) and pheromone-based cues (chemical signals from the female). Sawning only occurs if all of the following conditions are met:

  \begin{itemize}
  \item
    One male and one female are in the same patch
  \item
    Both are \texttt{spawning-ready}
  \end{itemize}
\end{enumerate}

Then, the total number of spawning events in the patch is:

\[ Spawn_{encounters} = \min(N_{\text{males}}, N_{\text{females}}) \]

\[ spawn_{n} = spawn_{encounters} \]

\begin{itemize}
\item
  \(spawn_{encounters}\) is the number of spawning encounters
\item
  \(spawn_{n}\) is the number of times an agent has spawned
\item
  \(N_{\text{males}}\) be the number of spawning-ready males in the patch.
\item
  \(N_{\text{females}}\) be the number of spawning-ready femalesi n the patch.
\end{itemize}

\subsection{Spawning energy}\label{spawning-energy}

\[
E_{agent} = E_{agent}-E_{spawn}
\]

Where:

\begin{itemize}
\item
  \(E_{agent}\)
\item
  \(E_{spawn}\)
\end{itemize}

\subsection{Post-Spawning Behavior}\label{post-spawning-behavior}

\[
\text{if } E_{\text{agent}} \geq E_{\text{recover}}, \quad \text{then } \texttt{landward_migration?} = \texttt{false}, \quad \texttt{seaward_migration?} = \texttt{true}, \quad \texttt{spawned-this-tick?} = \texttt{true}
\]

Where:

\begin{itemize}
\item
  \(E_{agent}\)
\item
  \(E_{recover}\)
\end{itemize}

\subsubsection{\texorpdfstring{\textbf{Mortality Risk}}{Mortality Risk}}\label{mortality-risk}

\[
\text{if } \text{random}(0,1) < P_{\text{mortality}}, \quad \text{then die}
\]

\[
P_{\text{mortality}} = \begin{cases}0.9 + 0.1 \cdot \frac{A_{\text{agent}}}{A_{\text{max}}}, & \text{if semelparous} \\0.01 + 0.05 \cdot \frac{A_{\text{agent}}}{A_{\text{max}}}, & \text{if iteroparous}\end{cases}
\]

Where:

\begin{itemize}
\item
  \(A_{\text{agent}}\) = agent's current age
\item
  \(A_{\text{max}}\) = maximum lifespan for the species
\end{itemize}

\texttt{semelparous} = species that spawn once and die

\texttt{iteroparous} = species that can spawn multiple times

\subsubsection{\texorpdfstring{\textbf{Resting Period}}{Resting Period}}\label{resting-period}

Agents with low energy enter a temporary staging state until recover:

\[
\text{if } E_{\text{agent}} < E_{\text{recover}}, \quad \texttt{staging?} = \texttt{true}
\]

Where:

\begin{itemize}
\item
  \(E_{recover}\) energy threshold to resume migration (e.g., 75)
\item
  \(E_{agent}\) is the current energy level of the agents
\end{itemize}

\subsubsection{Overwintering Probability}\label{overwintering-probability}

\[
P_{\text{overwinter}} = S_{\text{overwinter}} \cdot \left[  \alpha \cdot \left(1 - \frac{E_{\text{agent}}}{100} \right) +  \beta \cdot \left( \frac{A_{\text{agent}}}{A_{\text{max}}} \right) +  \gamma \cdot \left( \frac{M_{\text{agent}}}{M_{\text{max}}} \right) +  \delta \cdot \left( \frac{t_{\text{spawn}}}{t_{\text{max}}} \right)\right]
\]

Where:

\begin{itemize}
\tightlist
\item
  \(P_{\text{overwinter}}\) = agent's probability of overwintering
\item
  \(S_{\text{overwinter}}\) = species-level binary (1 = overwinter-capable, 0 = not)
\item
  \(E_{\text{agent}}\) = energy level (0--100\%)
\item
  \(A_{\text{agent}}\) = agent age
\item
  \(A_{\text{max}}\) = species-specific maximum age
\item
  \(M_{\text{agent}}\) = agent body size
\item
  \(M_{\text{max}}\) = maximum body size in population
\item
  \(t_{\text{spawn}}\) = days since last spawn
\item
  \(t_{\text{max}}\) = max time allowed for overwinter decision (e.g., 60 days)
\item
  \(\alpha, \beta, \gamma, \delta\) = tunable weights (e.g., 0.25 each)
\end{itemize}

\textbf{Interpretation}:

\begin{itemize}
\item
  Lower energy → more likely to overwinter
\item
  Older or larger fish → more likely to overwinter
\item
  Fish that spawned recently → higher likelihood (if within seasonal window)
\item
  Species must be capable of overwintering for this to activate
\end{itemize}

\textbf{\hfill\break
}

\chapter{Model Building Tutorial: Staging \& Schooling}\label{model-building-tutorial-staging-schooling}

\subsection{Module Integration}\label{module-integration}

\begin{itemize}
\tightlist
\item
  Brief explanation of how functions (e.g., staging, schooling) interact.
\item
  Clarify temporal structure (e.g., tick-based sequence) and spatial scale.
\item
  Describe coupling logic
\end{itemize}

\subsection{Function Dependencies}\label{function-dependencies}

\begin{itemize}
\item
  What variables are required as inputs for each function?
\item
  What functions must be called before/after? (e.g., must calculate stress before checking staging triggers)
\item
  Dependency table showing variable flow between submodels.
\end{itemize}

\begin{longtable}[]{@{}
  >{\raggedright\arraybackslash}p{(\linewidth - 6\tabcolsep) * \real{0.1375}}
  >{\raggedright\arraybackslash}p{(\linewidth - 6\tabcolsep) * \real{0.2875}}
  >{\raggedright\arraybackslash}p{(\linewidth - 6\tabcolsep) * \real{0.3500}}
  >{\raggedright\arraybackslash}p{(\linewidth - 6\tabcolsep) * \real{0.2250}}@{}}
\toprule\noalign{}
\begin{minipage}[b]{\linewidth}\raggedright
Function
\end{minipage} & \begin{minipage}[b]{\linewidth}\raggedright
Required Inputs
\end{minipage} & \begin{minipage}[b]{\linewidth}\raggedright
Output Variables
\end{minipage} & \begin{minipage}[b]{\linewidth}\raggedright
Dependent On
\end{minipage} \\
\midrule\noalign{}
\endhead
\bottomrule\noalign{}
\endlastfoot
\texttt{staging} & \texttt{E\_agent}, \texttt{I\_stress} & \texttt{E\_agent}, \texttt{patch\ records} & \texttt{osmoregulation} \\
schooling & & & \\
\end{longtable}

\section{Implementation in NetLogo:}\label{implementation-in-netlogo}

Programming details

example code

\section{Implementation in R:}\label{implementation-in-r}

Programming details

example code

\chapter{Model Building Tutorial: Landward Migration \& Selective Tidal Stream Transport}\label{model-building-tutorial-landward-migration-selective-tidal-stream-transport}

\subsection{Module Integration}\label{module-integration-1}

\begin{itemize}
\tightlist
\item
  Brief explanation of how functions (e.g., staging, schooling) interact.
\item
  Clarify temporal structure (e.g., tick-based sequence) and spatial scale.
\item
  Describe coupling logic
\end{itemize}

\subsection{Function Dependencies}\label{function-dependencies-1}

\begin{itemize}
\item
  What variables are required as inputs for each function?
\item
  What functions must be called before/after? (e.g., must calculate stress before checking staging triggers)
\item
  Dependency table showing variable flow between submodels.
\end{itemize}

\begin{longtable}[]{@{}
  >{\raggedright\arraybackslash}p{(\linewidth - 6\tabcolsep) * \real{0.2500}}
  >{\raggedright\arraybackslash}p{(\linewidth - 6\tabcolsep) * \real{0.2500}}
  >{\raggedright\arraybackslash}p{(\linewidth - 6\tabcolsep) * \real{0.2500}}
  >{\raggedright\arraybackslash}p{(\linewidth - 6\tabcolsep) * \real{0.2500}}@{}}
\toprule\noalign{}
\begin{minipage}[b]{\linewidth}\raggedright
Function
\end{minipage} & \begin{minipage}[b]{\linewidth}\raggedright
Required Inputs
\end{minipage} & \begin{minipage}[b]{\linewidth}\raggedright
Output Variables
\end{minipage} & \begin{minipage}[b]{\linewidth}\raggedright
Dependent On
\end{minipage} \\
\midrule\noalign{}
\endhead
\bottomrule\noalign{}
\endlastfoot
\texttt{osmoregulation} & \texttt{S\_patch}, \texttt{S\_agent}, \texttt{C} & \texttt{I\_stress}, \texttt{C\_new}, \texttt{E\_osmo} & \texttt{environmental-sensing} \\
\texttt{staging} & \texttt{E\_agent}, \texttt{I\_stress} & \texttt{E\_agent}, \texttt{patch\ records} & \texttt{osmoregulation} \\
\texttt{migration} & \texttt{E\_agent}, \texttt{V\_patch} & \texttt{Y\_t}, \texttt{E\_agent} & \texttt{staging}, \texttt{STST} \\
\texttt{STST} & \texttt{V\_patch}, \texttt{V\_agent} & \texttt{Y\_t+1}, \texttt{E\_agent} & \texttt{migration\ logic} \\
\end{longtable}

\section{Implementation in NetLogo:}\label{implementation-in-netlogo-1}

Programming details

example code

\section{Implementation in R:}\label{implementation-in-r-1}

Programming details

example code

\chapter{Complex Model Building Tutorial: One-Way Migration}\label{complex-model-building-tutorial-one-way-migration}

\subsection{Module Integration}\label{module-integration-2}

\begin{itemize}
\tightlist
\item
  Brief explanation of how functions (e.g., staging, schooling) interact.
\item
  Clarify temporal structure (e.g., tick-based sequence) and spatial scale.
\item
  Describe coupling logic
\end{itemize}

\subsection{Function Dependencies}\label{function-dependencies-2}

\begin{itemize}
\item
  What variables are required as inputs for each function?
\item
  What functions must be called before/after? (e.g., must calculate stress before checking staging triggers)
\item
  Dependency table showing variable flow between submodels.
\end{itemize}

\section{Implementation in NetLogo:}\label{implementation-in-netlogo-2}

Programming details

example code

\section{Implementation in R:}\label{implementation-in-r-2}

Programming details

example code

\chapter{Modeling Toolkit}\label{modeling-toolkit}

\section{Learning Resources}\label{learning-resources}

\begin{itemize}
\tightlist
\item
  \textbf{NetLogo Library}: Brief note on structure (e.g., model categories), how you used it (e.g., behavior ideas, calibration).
\item
  \textbf{NetLogo User Manual}: Link to official documentation.
\item
  \textbf{NetLogo Modeling Commons}: Peer-contributed models, code sharing, and idea sourcing.
\item
  \textbf{NetLogo Forum}: Where to get help or search issues.
\item
  \textbf{Book References}:
\end{itemize}

\begin{verbatim}
-    Railsback & Grimm (2019) *Agent-Based and Individual-Based Modeling*

-    Grimm & Railsback (2005) *Individual-Based Modeling and Ecology*

-    Add others specific to modeling.
\end{verbatim}

\section{Best Practices}\label{best-practices}

\begin{itemize}
\item
  \textbf{ODD Protocol (Overview, Design concepts, Details)}: Follow for transparent and structured documentation of agent-based models. Ensures clarity across entities, processes, and assumptions.
\item
  \textbf{Modular Design}: Structure behavioral functions into separate procedures (e.g., swimming, osmoregulation, staging) to support testing, integration, and reuse.
\item
  \textbf{Version Control}: Use GitHub or other tools to track changes and link code to research outputs.
\item
  \textbf{Model File structure}:
\item
  \textbf{Journal References}:
\end{itemize}

\section{Code Repositories \& Examples}\label{code-repositories-examples}

\begin{longtable}[]{@{}
  >{\raggedright\arraybackslash}p{(\linewidth - 4\tabcolsep) * \real{0.3333}}
  >{\raggedright\arraybackslash}p{(\linewidth - 4\tabcolsep) * \real{0.3333}}
  >{\raggedright\arraybackslash}p{(\linewidth - 4\tabcolsep) * \real{0.3333}}@{}}
\toprule\noalign{}
\begin{minipage}[b]{\linewidth}\raggedright
Project
\end{minipage} & \begin{minipage}[b]{\linewidth}\raggedright
Repository
\end{minipage} & \begin{minipage}[b]{\linewidth}\raggedright
Description
\end{minipage} \\
\midrule\noalign{}
\endhead
\bottomrule\noalign{}
\endlastfoot
\textbf{Penobscot River} & GitHub link & Simulates agent-based landward \& seaward migration using STST. \\
\textbf{Martha's Vineyard} & GitHub link & Chloride cell regulation and stress-energy tradeoffs. \\
\end{longtable}

\chapter*{References}\label{references}
\addcontentsline{toc}{chapter}{References}

\bibliography{book.bib,packages.bib}

\end{document}
